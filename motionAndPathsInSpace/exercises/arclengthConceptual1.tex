\documentclass{ximera}

%\usepackage{todonotes}

\newcommand{\todo}{}

\usepackage{esint} % for \oiint
\ifxake%%https://math.meta.stackexchange.com/questions/9973/how-do-you-render-a-closed-surface-double-integral
\renewcommand{\oiint}{{\large\bigcirc}\kern-1.56em\iint}
\fi


\graphicspath{
  {./}
  {ximeraTutorial/}
  {basicPhilosophy/}
  {functionsOfSeveralVariables/}
  {normalVectors/}
  {lagrangeMultipliers/}
  {vectorFields/}
  {greensTheorem/}
  {shapeOfThingsToCome/}
  {dotProducts/}
  {../productAndQuotientRules/exercises/}
  {../normalVectors/exercisesParametricPlots/}
  {../continuityOfFunctionsOfSeveralVariables/exercises/}
  {../partialDerivatives/exercises/}
  {../chainRuleForFunctionsOfSeveralVariables/exercises/}
  {../commonCoordinates/exercisesCylindricalCoordinates/}
  {../commonCoordinates/exercisesSphericalCoordinates/}
  {../greensTheorem/exercisesCurlAndLineIntegrals/}
  {../greensTheorem/exercisesDivergenceAndLineIntegrals/}
  {../shapeOfThingsToCome/exercisesDivergenceTheorem/}
  {../greensTheorem/}
  {../shapeOfThingsToCome/}
}

\newcommand{\mooculus}{\textsf{\textbf{MOOC}\textnormal{\textsf{ULUS}}}}

\usepackage{tkz-euclide}\usepackage{tikz}
\usepackage{tikz-cd}
\usetikzlibrary{arrows}
\tikzset{>=stealth,commutative diagrams/.cd,
  arrow style=tikz,diagrams={>=stealth}} %% cool arrow head
\tikzset{shorten <>/.style={ shorten >=#1, shorten <=#1 } } %% allows shorter vectors

\usetikzlibrary{backgrounds} %% for boxes around graphs
\usetikzlibrary{shapes,positioning}  %% Clouds and stars
\usetikzlibrary{matrix} %% for matrix
\usepgfplotslibrary{polar} %% for polar plots
\usepgfplotslibrary{fillbetween} %% to shade area between curves in TikZ
\usetkzobj{all}
%\usepackage[makeroom]{cancel} %% for strike outs
%\usepackage{mathtools} %% for pretty underbrace % Breaks Ximera
%\usepackage{multicol}
\usepackage{pgffor} %% required for integral for loops



%% http://tex.stackexchange.com/questions/66490/drawing-a-tikz-arc-specifying-the-center
%% Draws beach ball
\tikzset{pics/carc/.style args={#1:#2:#3}{code={\draw[pic actions] (#1:#3) arc(#1:#2:#3);}}}



\usepackage{array}
\setlength{\extrarowheight}{+.1cm}   
\newdimen\digitwidth
\settowidth\digitwidth{9}
\def\divrule#1#2{
\noalign{\moveright#1\digitwidth
\vbox{\hrule width#2\digitwidth}}}





\newcommand{\RR}{\mathbb R}
\newcommand{\R}{\mathbb R}
\newcommand{\N}{\mathbb N}
\newcommand{\Z}{\mathbb Z}

\newcommand{\sagemath}{\textsf{SageMath}}


%\renewcommand{\d}{\,d\!}
\renewcommand{\d}{\mathop{}\!d}
\newcommand{\dd}[2][]{\frac{\d #1}{\d #2}}
\newcommand{\pp}[2][]{\frac{\partial #1}{\partial #2}}
\renewcommand{\l}{\ell}
\newcommand{\ddx}{\frac{d}{\d x}}

\newcommand{\zeroOverZero}{\ensuremath{\boldsymbol{\tfrac{0}{0}}}}
\newcommand{\inftyOverInfty}{\ensuremath{\boldsymbol{\tfrac{\infty}{\infty}}}}
\newcommand{\zeroOverInfty}{\ensuremath{\boldsymbol{\tfrac{0}{\infty}}}}
\newcommand{\zeroTimesInfty}{\ensuremath{\small\boldsymbol{0\cdot \infty}}}
\newcommand{\inftyMinusInfty}{\ensuremath{\small\boldsymbol{\infty - \infty}}}
\newcommand{\oneToInfty}{\ensuremath{\boldsymbol{1^\infty}}}
\newcommand{\zeroToZero}{\ensuremath{\boldsymbol{0^0}}}
\newcommand{\inftyToZero}{\ensuremath{\boldsymbol{\infty^0}}}



\newcommand{\numOverZero}{\ensuremath{\boldsymbol{\tfrac{\#}{0}}}}
\newcommand{\dfn}{\textbf}
%\newcommand{\unit}{\,\mathrm}
\newcommand{\unit}{\mathop{}\!\mathrm}
\newcommand{\eval}[1]{\bigg[ #1 \bigg]}
\newcommand{\seq}[1]{\left( #1 \right)}
\renewcommand{\epsilon}{\varepsilon}
\renewcommand{\phi}{\varphi}


\renewcommand{\iff}{\Leftrightarrow}

\DeclareMathOperator{\arccot}{arccot}
\DeclareMathOperator{\arcsec}{arcsec}
\DeclareMathOperator{\arccsc}{arccsc}
\DeclareMathOperator{\si}{Si}
\DeclareMathOperator{\scal}{scal}
\DeclareMathOperator{\sign}{sign}


%% \newcommand{\tightoverset}[2]{% for arrow vec
%%   \mathop{#2}\limits^{\vbox to -.5ex{\kern-0.75ex\hbox{$#1$}\vss}}}
\newcommand{\arrowvec}[1]{{\overset{\rightharpoonup}{#1}}}
%\renewcommand{\vec}[1]{\arrowvec{\mathbf{#1}}}
\renewcommand{\vec}[1]{{\overset{\boldsymbol{\rightharpoonup}}{\mathbf{#1}}}}
\DeclareMathOperator{\proj}{\vec{proj}}
\newcommand{\veci}{{\boldsymbol{\hat{\imath}}}}
\newcommand{\vecj}{{\boldsymbol{\hat{\jmath}}}}
\newcommand{\veck}{{\boldsymbol{\hat{k}}}}
\newcommand{\vecl}{\vec{\boldsymbol{\l}}}
\newcommand{\uvec}[1]{\mathbf{\hat{#1}}}
\newcommand{\utan}{\mathbf{\hat{t}}}
\newcommand{\unormal}{\mathbf{\hat{n}}}
\newcommand{\ubinormal}{\mathbf{\hat{b}}}

\newcommand{\dotp}{\bullet}
\newcommand{\cross}{\boldsymbol\times}
\newcommand{\grad}{\boldsymbol\nabla}
\newcommand{\divergence}{\grad\dotp}
\newcommand{\curl}{\grad\cross}
%\DeclareMathOperator{\divergence}{divergence}
%\DeclareMathOperator{\curl}[1]{\grad\cross #1}
\newcommand{\lto}{\mathop{\longrightarrow\,}\limits}

\renewcommand{\bar}{\overline}

\colorlet{textColor}{black} 
\colorlet{background}{white}
\colorlet{penColor}{blue!50!black} % Color of a curve in a plot
\colorlet{penColor2}{red!50!black}% Color of a curve in a plot
\colorlet{penColor3}{red!50!blue} % Color of a curve in a plot
\colorlet{penColor4}{green!50!black} % Color of a curve in a plot
\colorlet{penColor5}{orange!80!black} % Color of a curve in a plot
\colorlet{penColor6}{yellow!70!black} % Color of a curve in a plot
\colorlet{fill1}{penColor!20} % Color of fill in a plot
\colorlet{fill2}{penColor2!20} % Color of fill in a plot
\colorlet{fillp}{fill1} % Color of positive area
\colorlet{filln}{penColor2!20} % Color of negative area
\colorlet{fill3}{penColor3!20} % Fill
\colorlet{fill4}{penColor4!20} % Fill
\colorlet{fill5}{penColor5!20} % Fill
\colorlet{gridColor}{gray!50} % Color of grid in a plot

\newcommand{\surfaceColor}{violet}
\newcommand{\surfaceColorTwo}{redyellow}
\newcommand{\sliceColor}{greenyellow}




\pgfmathdeclarefunction{gauss}{2}{% gives gaussian
  \pgfmathparse{1/(#2*sqrt(2*pi))*exp(-((x-#1)^2)/(2*#2^2))}%
}


%%%%%%%%%%%%%
%% Vectors
%%%%%%%%%%%%%

%% Simple horiz vectors
\renewcommand{\vector}[1]{\left\langle #1\right\rangle}


%% %% Complex Horiz Vectors with angle brackets
%% \makeatletter
%% \renewcommand{\vector}[2][ , ]{\left\langle%
%%   \def\nextitem{\def\nextitem{#1}}%
%%   \@for \el:=#2\do{\nextitem\el}\right\rangle%
%% }
%% \makeatother

%% %% Vertical Vectors
%% \def\vector#1{\begin{bmatrix}\vecListA#1,,\end{bmatrix}}
%% \def\vecListA#1,{\if,#1,\else #1\cr \expandafter \vecListA \fi}

%%%%%%%%%%%%%
%% End of vectors
%%%%%%%%%%%%%

%\newcommand{\fullwidth}{}
%\newcommand{\normalwidth}{}



%% makes a snazzy t-chart for evaluating functions
%\newenvironment{tchart}{\rowcolors{2}{}{background!90!textColor}\array}{\endarray}

%%This is to help with formatting on future title pages.
\newenvironment{sectionOutcomes}{}{} 



%% Flowchart stuff
%\tikzstyle{startstop} = [rectangle, rounded corners, minimum width=3cm, minimum height=1cm,text centered, draw=black]
%\tikzstyle{question} = [rectangle, minimum width=3cm, minimum height=1cm, text centered, draw=black]
%\tikzstyle{decision} = [trapezium, trapezium left angle=70, trapezium right angle=110, minimum width=3cm, minimum height=1cm, text centered, draw=black]
%\tikzstyle{question} = [rectangle, rounded corners, minimum width=3cm, minimum height=1cm,text centered, draw=black]
%\tikzstyle{process} = [rectangle, minimum width=3cm, minimum height=1cm, text centered, draw=black]
%\tikzstyle{decision} = [trapezium, trapezium left angle=70, trapezium right angle=110, minimum width=3cm, minimum height=1cm, text centered, draw=black]


\author{Jim Talamo}

\outcome{Answer conceptual questions about arc length.}

\begin{document}
\begin{exercise}

Select all of the following statements that are true.

\begin{selectAll}
\choice[correct]{If $\vec{p}(t)$ uses arc length as a parameter and $\uvec{t}(t)$ denotes the \emph{unit} tangent vector, then $\uvec{t}(t) = \vec{p}'(t)$.}
\choice[correct]{If $\vec{l}(t) = \vec{v}t+\vec{P}_0$ is an arc length parameterization of a line, then $\vec{v}$ \emph{must} be a unit vector.}
\choice{If the curve $C$ is traced out by $\vec{r}(t) = \vector{t^2,\frac{1}{2}t}, -\infty < t < \infty$, then the curve is parameterized by arc length since \[ |\vec{r}'(t)| = \sqrt{4t^2+\frac{1}{4}} = 1 \textrm{  when } t = \frac{\sqrt{3}}{4}. \]}
\choice{If the curve $C$ is parameterized by $\vec{p}(t)$, and it is known that $\vec{p}(0) = \vector{0,0}$ and $\vec{p}(3) = \vector{2,4}$, it is possible that $\vec{p}(t)$ is an arc length parameterization.}
\choice[correct]{If the curve $C$ is parameterized by $\vec{p}(t)$, and it is known that $\vec{p}(0) = \vector{0,0}$ and $\vec{p}(8) = \vector{2,4}$, it is possible that $\vec{p}(t)$ is an arc length parameterization.}
\end{selectAll}
(Use the hint to see the logic behind each choice)

\begin{hint}


\begin{question}
For the statement

\begin{quote}
True or False: If $\vec{p}(t)$ uses arc length as a parameter and $\uvec{t}(t)$ denotes the \emph{unit} tangent vector, then $\uvec{t}(t) = \vec{p}'(t)$
\end{quote}
do you want to see an explanation?

\begin{multipleChoice}
\choice[correct]{Yes.}
\choice{No}
\end{multipleChoice}

\begin{feedback}[correct]
In general, the unit tangent vector is given by

\[
\uvec{t}(t) = \frac{\vec{p}'(t)}{\left|\vec{p}'(t)\right|}.
\]
Since $\vec{p}(t)$ uses arc length as a parameter, we have that $\left|\vec{p}'(t)\right| = \answer{1}$.

\end{feedback}
\end{question}
%%%%%%%%%%%%%%%%%%%%%%%%%%%%%%%%%%%%%%%%%%%%%%%%%%%%


\begin{question}
For the statement

\begin{quote}
True or False: If $\vec{l}(t) = \vec{v}t+\vec{P}_0$ is an arc length parameterization of a line, then $\vec{v}$ \emph{must} be a unit vector.
\end{quote}
do you want to see an explanation?

\begin{multipleChoice}
\choice[correct]{Yes.}
\choice{No}
\end{multipleChoice}

\begin{feedback}[correct]
If $\vec{l}(t)$ is an arc length parameterization, $\left|\vec{l}'(t)\right| = \answer{1}$.  Since the $\vec{v}$ and $\vec{P}_0$ are constant vectors, 

\[
\vec{l}'(t) = \vec{v}.
\]

Hence, $\left|\vec{l}'(t)\right| = \left|\vec{v}\right|$ for all $t$.

\end{feedback}
\end{question}
%%%%%%%%%%%%%%%%%%%%%%%%%%%%%%%%%%%%%%%%%%%%%%%%%%%%


\begin{question}
For the statement

\begin{quote}
True or False: If the curve $C$ is traced out by $\vec{r}(t) = \vector{t^2,\frac{1}{2}t}, -\infty < t < \infty$, then the curve is parameterized by arc length since 

\[ |\vec{r}'(t)| = \sqrt{4t^2+\frac{1}{4}} = 1 \textrm{  when } t = \frac{\sqrt{3}}{4}. \]
\end{quote}
do you want to see an explanation?

\begin{multipleChoice}
\choice[correct]{Yes.}
\choice{No}
\end{multipleChoice}

\begin{feedback}[correct]
The calculation is correct; since $\vec{r}(t) = \vector{t^2,\frac{1}{2}t}$, we have

\[
\vec{r}'(t) = \vector{2t,\frac{1}{2}}.
\]
Thus, $|\vec{r}'(t)| = \sqrt{(2t)^2+\left(\frac{1}{2}\right)^2} = \sqrt{4t^2+\frac{1}{4}}$.

However, the curve is parameterized by arc length if and only if $|\vec{r}'(t)| =1$ for \emph{all} $t$, not just a specific $t$-value.
\end{feedback}
\end{question}
%%%%%%%%%%%%%%%%%%%%%%%%%%%%%%%%%%%%%%%%%%%%%%%%%%%%


\begin{question}
For the statement

\begin{quote}
True or False: If the curve $C$ is parameterized by $\vec{p}(t)$, and it is known that $\vec{p}(0) = \vector{0,0}$ and $\vec{p}(3) = \vector{2,4}$, it is possible that $\vec{p}(t)$ is an arc length parameterization.
\end{quote}
do you want to see an explanation?

\begin{multipleChoice}
\choice[correct]{Yes.}
\choice{No}
\end{multipleChoice}

\begin{feedback}[correct]
If the curve uses arc length as a parameter, $\vec{p}(3)$ should give us the position $(x,y)$ after we have travelled $3$ units along the curve.  The question is, can we travel from the points associated to $\vec{p}(0)$ and $\vec{p}(3)$ by going exactly $3$ units?

Note, the shortest distance between the points associated to $\vec{p}(0)$ and $\vec{p}(3)$ is the length of the line segment between them, and this line segment has length

\[
\sqrt{(2-0)^2+(4-0)^2} = \sqrt{20} >3.
\]

\end{feedback}
\end{question}
%%%%%%%%%%%%%%%%%%%%%%%%%%%%%%%%%%%%%%%%%%%%%%%%%%%%



\begin{question}
For the statement

\begin{quote}
True or False: If the curve $C$ is parameterized by $\vec{p}(t)$, and it is known that $\vec{p}(0) = \vector{0,0}$ and $\vec{p}(8) = \vector{2,4}$, it is possible that $\vec{p}(t)$ is an arc length parameterization.
\end{quote}
do you want to see an explanation?

\begin{multipleChoice}
\choice[correct]{Yes.}
\choice{No}
\end{multipleChoice}

\begin{feedback}[correct]
If the curve uses arc length as a parameter, $\vec{p}(8)$ should give us the position $(x,y)$ after we have travelled $8$ units along the curve.  The question is, can we travel from the points associated to $\vec{p}(0)$ and $\vec{p}(8)$ by going exactly $8$ units?

Note, the shortest distance between the points associated to $\vec{p}(0)$ and $\vec{p}(8)$ is the length of the line segment between them, and this line segment has length

\[
\sqrt{(2-0)^2+(4-0)^2} = \sqrt{20} <8.
\]

It thus should be possible.  

In fact, an explicit example is obtained geometrically by tracing along the segment below at a rate of one unit of distance per unit of time, so we trace out a border of each block each unit of time.

\begin{image}
    \begin{tikzpicture}
      \begin{axis}%
        [
	  xmin=-1.5,xmax=4.5,
          ymin=-1.5,ymax=4.5,
          xlabel=$x$,ylabel=$y$,
          axis lines=center,
          every axis y label/.style={at=(current axis.above origin),anchor=south},
          every axis x label/.style={at=(current axis.right of origin),anchor=west},
          clip=false,
	  grid =major,
          width=8cm,
          height=5cm,
          xtick={-1,...,4},
          ytick={-1,...,4},
	]
        \addplot[line join =bevel,black,ultra thick,->] coordinates{
          (0,0)(0,-.7)
        };
                \addplot[line join =bevel,black,ultra thick,->] coordinates{
          (0,-1)(1,-1)
        };
                \addplot[line join =bevel,black,ultra thick,->] coordinates{
          (2,-1) (2,2)
        };
                \addplot[line join =bevel,black,ultra thick] coordinates{
          (0,0)(0,-1)(2,-1) (2,4)
        };
        \addplot[color=black,fill=black,only marks,mark=*] coordinates{(0,0)};  %% closed hole
        \addplot[color=black,fill=black,only marks,mark=*] coordinates{(2,4)};  %% closed hole
        \node[black,above right] at (axis cs: 0,0) {$\vec{p}(0)$};
        \node[black,right] at (axis cs: 2,4) {$\vec{p}(8)$};
      \end{axis}
    \end{tikzpicture}
\end{image}

It is of course possible to construct other examples as well.
\end{feedback}
\end{question}
%%%%%%%%%%%%%%%%%%%%%%%%%%%%%%%%%%%%%%%%%%%%%%%%%%%%


\end{hint}

\end{exercise}
\end{document}
