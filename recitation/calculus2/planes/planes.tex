\documentclass[noauthor]{ximera}
%handout:  for handout version with no solutions or instructor notes
%handout,instructornotes:  for instructor version with just problems and notes, no solutions
%noinstructornotes:  shows only problem and solutions

%% handout
%% space
%% newpage
%% numbers
%% nooutcomes

%I added the commands here so that I would't have to keep looking them up
%\newcommand{\RR}{\mathbb R}
%\renewcommand{\d}{\,d}
%\newcommand{\dd}[2][]{\frac{d #1}{d #2}}
%\renewcommand{\l}{\ell}
%\newcommand{\ddx}{\frac{d}{dx}}
%\everymath{\displaystyle}
%\newcommand{\dfn}{\textbf}
%\newcommand{\eval}[1]{\bigg[ #1 \bigg]}

%\begin{image}
%\includegraphics[trim= 170 420 250 180]{Figure1.pdf}
%\end{image}

%add a ``.'' below when used in a specific directory.

%%\usepackage{todonotes}

\newcommand{\todo}{}

\usepackage{esint} % for \oiint
\ifxake%%https://math.meta.stackexchange.com/questions/9973/how-do-you-render-a-closed-surface-double-integral
\renewcommand{\oiint}{{\large\bigcirc}\kern-1.56em\iint}
\fi


\graphicspath{
  {./}
  {ximeraTutorial/}
  {basicPhilosophy/}
  {functionsOfSeveralVariables/}
  {normalVectors/}
  {lagrangeMultipliers/}
  {vectorFields/}
  {greensTheorem/}
  {shapeOfThingsToCome/}
  {dotProducts/}
  {../productAndQuotientRules/exercises/}
  {../normalVectors/exercisesParametricPlots/}
  {../continuityOfFunctionsOfSeveralVariables/exercises/}
  {../partialDerivatives/exercises/}
  {../chainRuleForFunctionsOfSeveralVariables/exercises/}
  {../commonCoordinates/exercisesCylindricalCoordinates/}
  {../commonCoordinates/exercisesSphericalCoordinates/}
  {../greensTheorem/exercisesCurlAndLineIntegrals/}
  {../greensTheorem/exercisesDivergenceAndLineIntegrals/}
  {../shapeOfThingsToCome/exercisesDivergenceTheorem/}
  {../greensTheorem/}
  {../shapeOfThingsToCome/}
}

\newcommand{\mooculus}{\textsf{\textbf{MOOC}\textnormal{\textsf{ULUS}}}}

\usepackage{tkz-euclide}\usepackage{tikz}
\usepackage{tikz-cd}
\usetikzlibrary{arrows}
\tikzset{>=stealth,commutative diagrams/.cd,
  arrow style=tikz,diagrams={>=stealth}} %% cool arrow head
\tikzset{shorten <>/.style={ shorten >=#1, shorten <=#1 } } %% allows shorter vectors

\usetikzlibrary{backgrounds} %% for boxes around graphs
\usetikzlibrary{shapes,positioning}  %% Clouds and stars
\usetikzlibrary{matrix} %% for matrix
\usepgfplotslibrary{polar} %% for polar plots
\usepgfplotslibrary{fillbetween} %% to shade area between curves in TikZ
\usetkzobj{all}
%\usepackage[makeroom]{cancel} %% for strike outs
%\usepackage{mathtools} %% for pretty underbrace % Breaks Ximera
%\usepackage{multicol}
\usepackage{pgffor} %% required for integral for loops



%% http://tex.stackexchange.com/questions/66490/drawing-a-tikz-arc-specifying-the-center
%% Draws beach ball
\tikzset{pics/carc/.style args={#1:#2:#3}{code={\draw[pic actions] (#1:#3) arc(#1:#2:#3);}}}



\usepackage{array}
\setlength{\extrarowheight}{+.1cm}   
\newdimen\digitwidth
\settowidth\digitwidth{9}
\def\divrule#1#2{
\noalign{\moveright#1\digitwidth
\vbox{\hrule width#2\digitwidth}}}





\newcommand{\RR}{\mathbb R}
\newcommand{\R}{\mathbb R}
\newcommand{\N}{\mathbb N}
\newcommand{\Z}{\mathbb Z}

\newcommand{\sagemath}{\textsf{SageMath}}


%\renewcommand{\d}{\,d\!}
\renewcommand{\d}{\mathop{}\!d}
\newcommand{\dd}[2][]{\frac{\d #1}{\d #2}}
\newcommand{\pp}[2][]{\frac{\partial #1}{\partial #2}}
\renewcommand{\l}{\ell}
\newcommand{\ddx}{\frac{d}{\d x}}

\newcommand{\zeroOverZero}{\ensuremath{\boldsymbol{\tfrac{0}{0}}}}
\newcommand{\inftyOverInfty}{\ensuremath{\boldsymbol{\tfrac{\infty}{\infty}}}}
\newcommand{\zeroOverInfty}{\ensuremath{\boldsymbol{\tfrac{0}{\infty}}}}
\newcommand{\zeroTimesInfty}{\ensuremath{\small\boldsymbol{0\cdot \infty}}}
\newcommand{\inftyMinusInfty}{\ensuremath{\small\boldsymbol{\infty - \infty}}}
\newcommand{\oneToInfty}{\ensuremath{\boldsymbol{1^\infty}}}
\newcommand{\zeroToZero}{\ensuremath{\boldsymbol{0^0}}}
\newcommand{\inftyToZero}{\ensuremath{\boldsymbol{\infty^0}}}



\newcommand{\numOverZero}{\ensuremath{\boldsymbol{\tfrac{\#}{0}}}}
\newcommand{\dfn}{\textbf}
%\newcommand{\unit}{\,\mathrm}
\newcommand{\unit}{\mathop{}\!\mathrm}
\newcommand{\eval}[1]{\bigg[ #1 \bigg]}
\newcommand{\seq}[1]{\left( #1 \right)}
\renewcommand{\epsilon}{\varepsilon}
\renewcommand{\phi}{\varphi}


\renewcommand{\iff}{\Leftrightarrow}

\DeclareMathOperator{\arccot}{arccot}
\DeclareMathOperator{\arcsec}{arcsec}
\DeclareMathOperator{\arccsc}{arccsc}
\DeclareMathOperator{\si}{Si}
\DeclareMathOperator{\scal}{scal}
\DeclareMathOperator{\sign}{sign}


%% \newcommand{\tightoverset}[2]{% for arrow vec
%%   \mathop{#2}\limits^{\vbox to -.5ex{\kern-0.75ex\hbox{$#1$}\vss}}}
\newcommand{\arrowvec}[1]{{\overset{\rightharpoonup}{#1}}}
%\renewcommand{\vec}[1]{\arrowvec{\mathbf{#1}}}
\renewcommand{\vec}[1]{{\overset{\boldsymbol{\rightharpoonup}}{\mathbf{#1}}}}
\DeclareMathOperator{\proj}{\vec{proj}}
\newcommand{\veci}{{\boldsymbol{\hat{\imath}}}}
\newcommand{\vecj}{{\boldsymbol{\hat{\jmath}}}}
\newcommand{\veck}{{\boldsymbol{\hat{k}}}}
\newcommand{\vecl}{\vec{\boldsymbol{\l}}}
\newcommand{\uvec}[1]{\mathbf{\hat{#1}}}
\newcommand{\utan}{\mathbf{\hat{t}}}
\newcommand{\unormal}{\mathbf{\hat{n}}}
\newcommand{\ubinormal}{\mathbf{\hat{b}}}

\newcommand{\dotp}{\bullet}
\newcommand{\cross}{\boldsymbol\times}
\newcommand{\grad}{\boldsymbol\nabla}
\newcommand{\divergence}{\grad\dotp}
\newcommand{\curl}{\grad\cross}
%\DeclareMathOperator{\divergence}{divergence}
%\DeclareMathOperator{\curl}[1]{\grad\cross #1}
\newcommand{\lto}{\mathop{\longrightarrow\,}\limits}

\renewcommand{\bar}{\overline}

\colorlet{textColor}{black} 
\colorlet{background}{white}
\colorlet{penColor}{blue!50!black} % Color of a curve in a plot
\colorlet{penColor2}{red!50!black}% Color of a curve in a plot
\colorlet{penColor3}{red!50!blue} % Color of a curve in a plot
\colorlet{penColor4}{green!50!black} % Color of a curve in a plot
\colorlet{penColor5}{orange!80!black} % Color of a curve in a plot
\colorlet{penColor6}{yellow!70!black} % Color of a curve in a plot
\colorlet{fill1}{penColor!20} % Color of fill in a plot
\colorlet{fill2}{penColor2!20} % Color of fill in a plot
\colorlet{fillp}{fill1} % Color of positive area
\colorlet{filln}{penColor2!20} % Color of negative area
\colorlet{fill3}{penColor3!20} % Fill
\colorlet{fill4}{penColor4!20} % Fill
\colorlet{fill5}{penColor5!20} % Fill
\colorlet{gridColor}{gray!50} % Color of grid in a plot

\newcommand{\surfaceColor}{violet}
\newcommand{\surfaceColorTwo}{redyellow}
\newcommand{\sliceColor}{greenyellow}




\pgfmathdeclarefunction{gauss}{2}{% gives gaussian
  \pgfmathparse{1/(#2*sqrt(2*pi))*exp(-((x-#1)^2)/(2*#2^2))}%
}


%%%%%%%%%%%%%
%% Vectors
%%%%%%%%%%%%%

%% Simple horiz vectors
\renewcommand{\vector}[1]{\left\langle #1\right\rangle}


%% %% Complex Horiz Vectors with angle brackets
%% \makeatletter
%% \renewcommand{\vector}[2][ , ]{\left\langle%
%%   \def\nextitem{\def\nextitem{#1}}%
%%   \@for \el:=#2\do{\nextitem\el}\right\rangle%
%% }
%% \makeatother

%% %% Vertical Vectors
%% \def\vector#1{\begin{bmatrix}\vecListA#1,,\end{bmatrix}}
%% \def\vecListA#1,{\if,#1,\else #1\cr \expandafter \vecListA \fi}

%%%%%%%%%%%%%
%% End of vectors
%%%%%%%%%%%%%

%\newcommand{\fullwidth}{}
%\newcommand{\normalwidth}{}



%% makes a snazzy t-chart for evaluating functions
%\newenvironment{tchart}{\rowcolors{2}{}{background!90!textColor}\array}{\endarray}

%%This is to help with formatting on future title pages.
\newenvironment{sectionOutcomes}{}{} 



%% Flowchart stuff
%\tikzstyle{startstop} = [rectangle, rounded corners, minimum width=3cm, minimum height=1cm,text centered, draw=black]
%\tikzstyle{question} = [rectangle, minimum width=3cm, minimum height=1cm, text centered, draw=black]
%\tikzstyle{decision} = [trapezium, trapezium left angle=70, trapezium right angle=110, minimum width=3cm, minimum height=1cm, text centered, draw=black]
%\tikzstyle{question} = [rectangle, rounded corners, minimum width=3cm, minimum height=1cm,text centered, draw=black]
%\tikzstyle{process} = [rectangle, minimum width=3cm, minimum height=1cm, text centered, draw=black]
%\tikzstyle{decision} = [trapezium, trapezium left angle=70, trapezium right angle=110, minimum width=3cm, minimum height=1cm, text centered, draw=black]

%\usepackage{todonotes}

\newcommand{\todo}{}

\usepackage{esint} % for \oiint
\ifxake%%https://math.meta.stackexchange.com/questions/9973/how-do-you-render-a-closed-surface-double-integral
\renewcommand{\oiint}{{\large\bigcirc}\kern-1.56em\iint}
\fi


\graphicspath{
  {./}
  {ximeraTutorial/}
  {basicPhilosophy/}
  {functionsOfSeveralVariables/}
  {normalVectors/}
  {lagrangeMultipliers/}
  {vectorFields/}
  {greensTheorem/}
  {shapeOfThingsToCome/}
  {dotProducts/}
  {../productAndQuotientRules/exercises/}
  {../normalVectors/exercisesParametricPlots/}
  {../continuityOfFunctionsOfSeveralVariables/exercises/}
  {../partialDerivatives/exercises/}
  {../chainRuleForFunctionsOfSeveralVariables/exercises/}
  {../commonCoordinates/exercisesCylindricalCoordinates/}
  {../commonCoordinates/exercisesSphericalCoordinates/}
  {../greensTheorem/exercisesCurlAndLineIntegrals/}
  {../greensTheorem/exercisesDivergenceAndLineIntegrals/}
  {../shapeOfThingsToCome/exercisesDivergenceTheorem/}
  {../greensTheorem/}
  {../shapeOfThingsToCome/}
}

\newcommand{\mooculus}{\textsf{\textbf{MOOC}\textnormal{\textsf{ULUS}}}}

\usepackage{tkz-euclide}\usepackage{tikz}
\usepackage{tikz-cd}
\usetikzlibrary{arrows}
\tikzset{>=stealth,commutative diagrams/.cd,
  arrow style=tikz,diagrams={>=stealth}} %% cool arrow head
\tikzset{shorten <>/.style={ shorten >=#1, shorten <=#1 } } %% allows shorter vectors

\usetikzlibrary{backgrounds} %% for boxes around graphs
\usetikzlibrary{shapes,positioning}  %% Clouds and stars
\usetikzlibrary{matrix} %% for matrix
\usepgfplotslibrary{polar} %% for polar plots
\usepgfplotslibrary{fillbetween} %% to shade area between curves in TikZ
\usetkzobj{all}
%\usepackage[makeroom]{cancel} %% for strike outs
%\usepackage{mathtools} %% for pretty underbrace % Breaks Ximera
%\usepackage{multicol}
\usepackage{pgffor} %% required for integral for loops



%% http://tex.stackexchange.com/questions/66490/drawing-a-tikz-arc-specifying-the-center
%% Draws beach ball
\tikzset{pics/carc/.style args={#1:#2:#3}{code={\draw[pic actions] (#1:#3) arc(#1:#2:#3);}}}



\usepackage{array}
\setlength{\extrarowheight}{+.1cm}   
\newdimen\digitwidth
\settowidth\digitwidth{9}
\def\divrule#1#2{
\noalign{\moveright#1\digitwidth
\vbox{\hrule width#2\digitwidth}}}





\newcommand{\RR}{\mathbb R}
\newcommand{\R}{\mathbb R}
\newcommand{\N}{\mathbb N}
\newcommand{\Z}{\mathbb Z}

\newcommand{\sagemath}{\textsf{SageMath}}


%\renewcommand{\d}{\,d\!}
\renewcommand{\d}{\mathop{}\!d}
\newcommand{\dd}[2][]{\frac{\d #1}{\d #2}}
\newcommand{\pp}[2][]{\frac{\partial #1}{\partial #2}}
\renewcommand{\l}{\ell}
\newcommand{\ddx}{\frac{d}{\d x}}

\newcommand{\zeroOverZero}{\ensuremath{\boldsymbol{\tfrac{0}{0}}}}
\newcommand{\inftyOverInfty}{\ensuremath{\boldsymbol{\tfrac{\infty}{\infty}}}}
\newcommand{\zeroOverInfty}{\ensuremath{\boldsymbol{\tfrac{0}{\infty}}}}
\newcommand{\zeroTimesInfty}{\ensuremath{\small\boldsymbol{0\cdot \infty}}}
\newcommand{\inftyMinusInfty}{\ensuremath{\small\boldsymbol{\infty - \infty}}}
\newcommand{\oneToInfty}{\ensuremath{\boldsymbol{1^\infty}}}
\newcommand{\zeroToZero}{\ensuremath{\boldsymbol{0^0}}}
\newcommand{\inftyToZero}{\ensuremath{\boldsymbol{\infty^0}}}



\newcommand{\numOverZero}{\ensuremath{\boldsymbol{\tfrac{\#}{0}}}}
\newcommand{\dfn}{\textbf}
%\newcommand{\unit}{\,\mathrm}
\newcommand{\unit}{\mathop{}\!\mathrm}
\newcommand{\eval}[1]{\bigg[ #1 \bigg]}
\newcommand{\seq}[1]{\left( #1 \right)}
\renewcommand{\epsilon}{\varepsilon}
\renewcommand{\phi}{\varphi}


\renewcommand{\iff}{\Leftrightarrow}

\DeclareMathOperator{\arccot}{arccot}
\DeclareMathOperator{\arcsec}{arcsec}
\DeclareMathOperator{\arccsc}{arccsc}
\DeclareMathOperator{\si}{Si}
\DeclareMathOperator{\scal}{scal}
\DeclareMathOperator{\sign}{sign}


%% \newcommand{\tightoverset}[2]{% for arrow vec
%%   \mathop{#2}\limits^{\vbox to -.5ex{\kern-0.75ex\hbox{$#1$}\vss}}}
\newcommand{\arrowvec}[1]{{\overset{\rightharpoonup}{#1}}}
%\renewcommand{\vec}[1]{\arrowvec{\mathbf{#1}}}
\renewcommand{\vec}[1]{{\overset{\boldsymbol{\rightharpoonup}}{\mathbf{#1}}}}
\DeclareMathOperator{\proj}{\vec{proj}}
\newcommand{\veci}{{\boldsymbol{\hat{\imath}}}}
\newcommand{\vecj}{{\boldsymbol{\hat{\jmath}}}}
\newcommand{\veck}{{\boldsymbol{\hat{k}}}}
\newcommand{\vecl}{\vec{\boldsymbol{\l}}}
\newcommand{\uvec}[1]{\mathbf{\hat{#1}}}
\newcommand{\utan}{\mathbf{\hat{t}}}
\newcommand{\unormal}{\mathbf{\hat{n}}}
\newcommand{\ubinormal}{\mathbf{\hat{b}}}

\newcommand{\dotp}{\bullet}
\newcommand{\cross}{\boldsymbol\times}
\newcommand{\grad}{\boldsymbol\nabla}
\newcommand{\divergence}{\grad\dotp}
\newcommand{\curl}{\grad\cross}
%\DeclareMathOperator{\divergence}{divergence}
%\DeclareMathOperator{\curl}[1]{\grad\cross #1}
\newcommand{\lto}{\mathop{\longrightarrow\,}\limits}

\renewcommand{\bar}{\overline}

\colorlet{textColor}{black} 
\colorlet{background}{white}
\colorlet{penColor}{blue!50!black} % Color of a curve in a plot
\colorlet{penColor2}{red!50!black}% Color of a curve in a plot
\colorlet{penColor3}{red!50!blue} % Color of a curve in a plot
\colorlet{penColor4}{green!50!black} % Color of a curve in a plot
\colorlet{penColor5}{orange!80!black} % Color of a curve in a plot
\colorlet{penColor6}{yellow!70!black} % Color of a curve in a plot
\colorlet{fill1}{penColor!20} % Color of fill in a plot
\colorlet{fill2}{penColor2!20} % Color of fill in a plot
\colorlet{fillp}{fill1} % Color of positive area
\colorlet{filln}{penColor2!20} % Color of negative area
\colorlet{fill3}{penColor3!20} % Fill
\colorlet{fill4}{penColor4!20} % Fill
\colorlet{fill5}{penColor5!20} % Fill
\colorlet{gridColor}{gray!50} % Color of grid in a plot

\newcommand{\surfaceColor}{violet}
\newcommand{\surfaceColorTwo}{redyellow}
\newcommand{\sliceColor}{greenyellow}




\pgfmathdeclarefunction{gauss}{2}{% gives gaussian
  \pgfmathparse{1/(#2*sqrt(2*pi))*exp(-((x-#1)^2)/(2*#2^2))}%
}


%%%%%%%%%%%%%
%% Vectors
%%%%%%%%%%%%%

%% Simple horiz vectors
\renewcommand{\vector}[1]{\left\langle #1\right\rangle}


%% %% Complex Horiz Vectors with angle brackets
%% \makeatletter
%% \renewcommand{\vector}[2][ , ]{\left\langle%
%%   \def\nextitem{\def\nextitem{#1}}%
%%   \@for \el:=#2\do{\nextitem\el}\right\rangle%
%% }
%% \makeatother

%% %% Vertical Vectors
%% \def\vector#1{\begin{bmatrix}\vecListA#1,,\end{bmatrix}}
%% \def\vecListA#1,{\if,#1,\else #1\cr \expandafter \vecListA \fi}

%%%%%%%%%%%%%
%% End of vectors
%%%%%%%%%%%%%

%\newcommand{\fullwidth}{}
%\newcommand{\normalwidth}{}



%% makes a snazzy t-chart for evaluating functions
%\newenvironment{tchart}{\rowcolors{2}{}{background!90!textColor}\array}{\endarray}

%%This is to help with formatting on future title pages.
\newenvironment{sectionOutcomes}{}{} 



%% Flowchart stuff
%\tikzstyle{startstop} = [rectangle, rounded corners, minimum width=3cm, minimum height=1cm,text centered, draw=black]
%\tikzstyle{question} = [rectangle, minimum width=3cm, minimum height=1cm, text centered, draw=black]
%\tikzstyle{decision} = [trapezium, trapezium left angle=70, trapezium right angle=110, minimum width=3cm, minimum height=1cm, text centered, draw=black]
%\tikzstyle{question} = [rectangle, rounded corners, minimum width=3cm, minimum height=1cm,text centered, draw=black]
%\tikzstyle{process} = [rectangle, minimum width=3cm, minimum height=1cm, text centered, draw=black]
%\tikzstyle{decision} = [trapezium, trapezium left angle=70, trapezium right angle=110, minimum width=3cm, minimum height=1cm, text centered, draw=black]




\author{Jim Talamo}

\outcome{Find the equation of a plane.}
\outcome{Determine if planes are parallel, perpendicular, or neither.}
\outcome{Explore geometric concepts and planes.}

\title[Collaborate:]{Planes}

\begin{document}
\begin{abstract}
\end{abstract}
\maketitle

\section{Discussion Questions}

\begin{problem}
A student is asked to find a description of a line that is parallel to the plane $2x+3y-4z=5$ and passes through the point $(2,4,3)$.
The student says that the answer is $2x+3y-4z = 4$.  Is the student correct?  If not, explain why the student is incorrect then provide a correct response.

\begin{freeResponse}
The student is incorrect.  The student was asked to find the description of a line and gave the equation of a plane.  It is true that the given plane is parallel to the plane the student provided since each plane has the same normal vector.  

There are many lines that are parallel to the given plane through the plane in question.  In order to find one, we can find a vector parallel to the plane by finding two points on it.  This can be done by specifying two of $x,y$ and $z$, and using the equation of the plane to find the third.

\begin{itemize}
\item If $x=0$ and $y=0$, the equation of the plane $2x+3y-4z = 4$ ensures that $z=-1$.
\item If $y=0$ and $z=0$, the equation of the plane $2x+3y-4z = 4$ ensures that $x=2$.
\end{itemize}

Thus, a vector parallel to the plane is $\vector{2,0,0} - \vector{0,0,-1} = \vector{2,0,1}$, and a parametric description of a line parallel to the plane that also passes through $(2,4,3)$ is

\[
\vec{l}(t) = \vector{2,0,1}t+\vector{2,4,3} = \vector{2t-2,4,t+3}.
\]

Note that many other lines can be found that meet the criteria specified in this problem.

\end{freeResponse}
\end{problem}

%%%%%%%%%%%%%%%%%%%%%%%%%%%%%%%%%%%%%%%%%%%%%%%%%
\begin{problem}
Give a normal vector to the plane $2x-7z=17$.

\begin{freeResponse}
A normal vector can be found using the coefficients; one option is

\[
\vec{n} = \vector{2,0,-7}.
\]

\begin{remark}
To establish that $\vec{n}$ is normal to the plane, we must establish that $\vec{n}$ is orthogonal to \emph{any} vector that is parallel to the plane. To do this, pick two points $(x_1,y_1,z_1)$ and $(x_2,y_2,z_2)$ that lie on the plane.

A vector that is parallel to the plane that starts at $(x_1,y_1,z_1)$ and ends at $(x_2,y_2,z_2)$ is $\vec{v} = \vector{x_2-x_1,y_2-y_1,z_2-z_1}$.

Now, 

\[\vec{n} \dotp \vec{v} = \vector{x_2-x_1,y_2-y_1,z_2-z_1} \dotp \vector{2,0,-7} = 2(x_2-x_1) -7(z_2-z_1),\] and we must establish that this is $0$.

We can rearrange the equation.

\begin{align*}
2(x_2-x_1) -7(z_2-z_1) &= 2x_2-2x_1-7z_2+7z_1 \\
&= \left[2x_2 -7z_2\right]-\left[2x_1 -7z_1\right]
\end{align*}

Now, since $(x_1,y_1,z_1)$ and ends at $(x_2,y_2,z_2)$  lie on the plane, $2x_2 -7z_2 = 17$ and $2x_1 -7z_1 = 17$, so 

\[
\vec{n} \dotp \vec{v} = \left[2x_2 --7z_2\right]-\left[2x_1 -7z_1\right] = 17-17 = 0.
\]

Thus, $\vec{n}$ is orthogonal to \emph{any} arbitrary vector that is parallel to the plane, and thus $\vec{n}$ is a normal vector to the plane.





\end{remark}
\end{freeResponse}
\end{problem}

%%%%%%%%%%%%%%%%%%%%%%%%%%%%%%%%%%%%%%%%%%%%%%%%%
\begin{problem}

In all of the following statements, the objects described are taken to be in $\R^3$.  Determine if the following are true or false and explain your reasoning.  Try to think about/visualize each statement geometrically.

I. If two distinct lines intersect, then there is a plane that contains both lines.

II. If two distinct lines do not intersect, then there is not a plane that contains both lines.

III. Given two points, there is a unique plane that passes through both points.

IV. Given three points, there is a unique plane that passes through all three points.

V. Given three points, it is possible that there is not a plane that passes through all of them.

VI. Given four points, there is no plane that passes through all four points.

\begin{freeResponse}
Think about each statement geometrically.

I. True; if both lines are distinct and intersect, there must be a plane that contains both of them.  

II. False; take a plane, and draw two parallel lines on it.  For an explicit example, you can verify that the lines traced out by the vector-valued functions

\begin{align*}
\vec{l}_1(t) = \vector{1,0,0}t+ \vector{0,0,1} = \vector{t,0,1} \\
\vec{l}_2(t) = \vector{1,0,0}t+ \vector{0,0,2} = \vector{t,0,2} \\
\end{align*}

 are distinct parallel lines, but both lie on the plane $y=0$.
 
III. False; from a geometric standpoint, there are many planes that contains the line that passes through the two given points.  From an algebraic standpoint, the equation of a plane can be written as 

\[
ax+by+cz=0 \textrm{ or } ax+by+cz=1,
\]
since if $d \neq 0$, we can divide both sides of $ax+by+cz=d$ by $d$.

The two points will give us two conditions to find the three unknown constants $a$, $b$, and $c$, so we must have freedom to choose one of them and there is thus not a single plane that passes through the two given points.
 
 IV. False; if the three points all lie on the same line, there are infinitely many planes that contain them.
 
 V. False; try to visualize this scenario geometrically.  Also, suppose that the points do not all lie on the same line; if they do, there are many planes that contain all three.  Take one of the points and form a vector that starts at it and ends at the second point.  You can also find a vector that extends from the first point to the third one.  These can be used to find a normal vector for the plane.  Since we can find a normal vector and have a choice of three points, we can find the equation of the plane (see problem 5 for details in the context of a specific example).
 
 VI. False; if three of the points lie on a common line, but the fourth does not lie on the line, there will be a unique plane that passes through them all.
 
\end{freeResponse}
\end{problem}

%%%%%%%%%%%%%%%%%%%%%%%%%%%%%%%%%%%%%%%%%%%%%%%%%

\section{Group Work}

\begin{problem}
Find a parameterization of the line that is perpendicular to the plane $x-4y+z=5$ that passes through the point $(1,-3,2)$.

\begin{freeResponse}
A vector parallel to the line will be parallel to the normal vector for the plane.  Since an option for the normal vector for the plane is $\vector{1,-4,1}$, a parameterization for the line is

\[
\vec{l}(t) = \vector{1,-4,1}t+\vector{1,-3,2} = \vector{t+1,-4t-3,t+2}.
\]

\end{freeResponse}
\end{problem}
%%%%%%%%%%%%%%%%%%%%%%%%%%%%%%%%%%%%%%%%%%%%%%%%%
\begin{problem}
Find the equation of a plane that passes through $(0,1,2)$, $(-1,3,1)$, and $(5,0,2)$.  Express your final answer in the form $ax+by+cz=d$.

\begin{freeResponse}
We need to find a normal vector for the plane.  Notice the following.

\begin{itemize}
\item The vector that extends from $(0,1,2)$ to $(5,0,2)$ is parallel to the plane.  This vector is $\vector{0-5, 1- 0, 2-2} = \vector{-5,1,0}$.
\item The vector that extends from $(0,1,2)$ to $(-1,3,1)$ is parallel to the plane.  This vector is $\vector{0-(-1), 1- 3, 2-1} = \vector{1,-2,1}$.
\end{itemize}

A normal vector for the plane is thus $\vector{-5,1,0} \cross \vector{1,-2,1} = \vector{1,5,9}$, and the equation of the plane can be found from

\[
a(x-x_0)+b(y-y_0)+c(z-z_0) = 0,
\]
where $\vec{n} = \vector{a,b,c}$ and $(x_0,y_0,z_0)$ is a point on the plane.  Using $(0,1,2)$ as this point gives

\begin{align*}
1(x-0)+5(y-1)+9(z-2) &= 0 \\
x+5y+9z &= 23.
\end{align*}

\begin{remark}
Note that we could have chosen either of the three points as the base point for our vector (instead of $(0,1,2)$) and used any of the three points as $(x_0,y_0,z_0)$.  This will lead to different choice for the constants in the equation $a(x-x_0)+b(y-y_0)+c(z-z_0) = 0$, but all of these will yield $x+5y+9z = 23$ after some algebra.
\end{remark}
\end{freeResponse}
\end{problem}
%%%%%%%%%%%%%%%%%%%%%%%%%%%%%%%%%%%%%%%%%%%%%%%%%
\begin{problem}
Show that the lines given by the parametric equations $\vecl_1(t)=\vector{2t+1,t-4,3t}$ and $\vecl_2(T)=\vector{T-1,-3,2T-5}$ intersect.
Then, find the equation of the plane that contains them.  Express your final answer in the form $ax+by+cz=d$.

\begin{freeResponse}
The lines will intersect if there is a $t$-value and a $T$-value for which $x(t)=x(T)$, $y(t)=y(T)$, and $z(t)=z(T)$.  

\begin{itemize}
\item The condition for $y$ requires that $t-4=-3$ or $t=1$.  
\item The condition for $x$ requires that $2t+1=T-1$.  Since $t=1$, we have $T=4$.
\item The lines will intersect if the above choices for $t$ and $T$ yield a consistent requirement in $z$; that is, we must check whether
\[
3t=2T-5
\]
for our choice $t=1$ and $T=4$. Since this is the case, the lines will intersect.
\end{itemize}

Now, to find the equation of the plane that contains them we need a normal vector for the plane and a point through which the plane passes.  We have just shown that the lines intersect at $(3,-3,3)$, so let's use that as our point.

To find a normal vector, note that a vector parallel to the first line is $\vector{2,1,3}$ and a vector parallel to the second line is $\vector{1,0,2}$.  A choice for the normal vector is thus 

\[
\vec{n} = \vector{2,1,3} \cross \vector{1,0,2} = \vector{2,-1,-1}.
\]
and the equation of the plane is thus
\begin{align*}
a(x-x_0)+b(y-y_0)+c(z-z_0) &= 0 \\
2(x-3)-(y-(-3))-(z-3) &= 0 \\
2x-y-z &= 6.
\end{align*}

\begin{remark}
We can check our work by verifying that the original lines lie on this plane.  For instance, for the first line $\vecl_1(t)=\vector{2t+1,t-4,3t}$, we have $x(t) = 2t+1$, $y(t) =t-4$ and $z(t) =3t$.  We note

\[
2[x(t)]-[y(t)]-[z(t)] = 2[2t+1]-[t-4]-[3t] = 4t+2-t+4-3t = 6.
\]
Since this holds for all $t$, the first line lies on the plane.  We leave it to the reader to verify the same occurs for the second line.
\end{remark}

\end{freeResponse}
\end{problem}
%%%%%%%%%%%%%%%%%%%%%%%%%%%%%%%%%%%%%%%%%%%%%%%%%
\begin{problem}
The planes $P_1$, $P_2$, and $P_3$ are defined from the equations below.  Determine which planes are parallel to each other and which planes are orthogonal to each other.

\[
\begin{array}{lrc}
P_1 : &  x-3y+2z  = & 5 \\
P_2 : &  2x-6y+4z  =& 7 \\
P_3 : &  x+y-4z =& 10 \\
\end{array}
\]

\begin{freeResponse}
Two planes are parallel if and only if their normal vectors are parallel and orthogonal if and only if their normal vectors are orthogonal.  

\begin{itemize}
\item A normal vector for $P_1$ is $\vec{n}_1 = \vector{1,-3,2}$.
\item A normal vector for $P_2$ is $\vec{n}_2 = \vector{2,-6,4}$.
\item A normal vector for $P_3$ is $\vec{n}_3 = \vector{1,1,-4}$.
\end{itemize}

Now let's look at each pair of planes.

\begin{itemize}
\item For $P_1$ and $P_2$, note that since $\vec{n}_2 = 2 \vec{n}_1$, $P_1$ and $P_2$ are parallel planes.
\item For $P_1$ and $P_3$, note that $\vec{n}_1$ and $\vec{n}_3$ are not scalar multiples of each other, so the planes are not parallel.  Also, $\vec{n}_1 \dotp \vec{n}_3 = -10 \neq 0$, so they are not orthogonal either.
\item For $P_2$ and $P_3$, note that $\vec{n}_1$ and $\vec{n}_3$ are not scalar multiples of each other, so the planes are not parallel.  Also, $\vec{n}_1 \dotp \vec{n}_3 = -20 \neq 0$, so they are not orthogonal either.
\end{itemize}


\end{freeResponse}
\end{problem}
%%%%%%%%%%%%%%%%%%%%%%%%%%%%%%%%%%%%%%%%%%%%%%%%%
\begin{problem}
The planes $P_1$ and $P_2$ are defined from the equations below.  

\begin{align*}
P_1 : x+ay+3z=8 \\
P_2 : 3x-9y+4z=3 \\
\end{align*}

I. Find a value for $a$ so the planes are parallel to each other or explain why no such $a$-value exists.

II. Find a value for $a$ so the planes are orthogonal to each other or explain why no such $a$-value exists.

\begin{freeResponse}
Recall that two planes are parallel if and only if their normal vectors are parallel and orthogonal if and only if their normal vectors are orthogonal.  

\begin{itemize}
\item A normal vector for $P_1$ is $\vec{n}_1 = \vector{1,a,3}$.
\item A normal vector for $P_2$ is $\vec{n}_2 = \vector{3,-9,4}$.
\end{itemize}

I. From the $x$ components, if $\vec{n}_1$ and $\vec{n}_2$ are parallel, we must have $\vec{n}_2 =3 \vec{n}_1$.  However, the $z$ components of these vectors makes this an impossibility.  Hence, there is no choice for $a$ for which the planes will be parallel.

II. We find that $\vec{n}_1 \dotp \vec{n}_2 = 10-9a$, so the planes will be orthogonal for $a=\frac{10}{9}$.
\end{freeResponse}
\end{problem}
\end{document}
