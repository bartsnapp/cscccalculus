\documentclass[noauthor, handout]{ximera}
%handout:  for handout version with no solutions or instructor notes
%handout,instructornotes:  for instructor version with just problems and notes, no solutions
%noinstructornotes:  shows only problem and solutions

%% handout
%% space
%% newpage
%% numbers
%% nooutcomes

%I added the commands here so that I would't have to keep looking them up
%\newcommand{\RR}{\mathbb R}
%\renewcommand{\d}{\,d}
%\newcommand{\dd}[2][]{\frac{d #1}{d #2}}
%\renewcommand{\l}{\ell}
%\newcommand{\ddx}{\frac{d}{dx}}
%\everymath{\displaystyle}
%\newcommand{\dfn}{\textbf}
%\newcommand{\eval}[1]{\bigg[ #1 \bigg]}

%\begin{image}
%\includegraphics[trim= 170 420 250 180]{Figure1.pdf}
%\end{image}

%add a ``.'' below when used in a specific directory.

%\usepackage{todonotes}

\newcommand{\todo}{}

\usepackage{esint} % for \oiint
\ifxake%%https://math.meta.stackexchange.com/questions/9973/how-do-you-render-a-closed-surface-double-integral
\renewcommand{\oiint}{{\large\bigcirc}\kern-1.56em\iint}
\fi


\graphicspath{
  {./}
  {ximeraTutorial/}
  {basicPhilosophy/}
  {functionsOfSeveralVariables/}
  {normalVectors/}
  {lagrangeMultipliers/}
  {vectorFields/}
  {greensTheorem/}
  {shapeOfThingsToCome/}
  {dotProducts/}
  {../productAndQuotientRules/exercises/}
  {../normalVectors/exercisesParametricPlots/}
  {../continuityOfFunctionsOfSeveralVariables/exercises/}
  {../partialDerivatives/exercises/}
  {../chainRuleForFunctionsOfSeveralVariables/exercises/}
  {../commonCoordinates/exercisesCylindricalCoordinates/}
  {../commonCoordinates/exercisesSphericalCoordinates/}
  {../greensTheorem/exercisesCurlAndLineIntegrals/}
  {../greensTheorem/exercisesDivergenceAndLineIntegrals/}
  {../shapeOfThingsToCome/exercisesDivergenceTheorem/}
  {../greensTheorem/}
  {../shapeOfThingsToCome/}
}

\newcommand{\mooculus}{\textsf{\textbf{MOOC}\textnormal{\textsf{ULUS}}}}

\usepackage{tkz-euclide}\usepackage{tikz}
\usepackage{tikz-cd}
\usetikzlibrary{arrows}
\tikzset{>=stealth,commutative diagrams/.cd,
  arrow style=tikz,diagrams={>=stealth}} %% cool arrow head
\tikzset{shorten <>/.style={ shorten >=#1, shorten <=#1 } } %% allows shorter vectors

\usetikzlibrary{backgrounds} %% for boxes around graphs
\usetikzlibrary{shapes,positioning}  %% Clouds and stars
\usetikzlibrary{matrix} %% for matrix
\usepgfplotslibrary{polar} %% for polar plots
\usepgfplotslibrary{fillbetween} %% to shade area between curves in TikZ
\usetkzobj{all}
%\usepackage[makeroom]{cancel} %% for strike outs
%\usepackage{mathtools} %% for pretty underbrace % Breaks Ximera
%\usepackage{multicol}
\usepackage{pgffor} %% required for integral for loops



%% http://tex.stackexchange.com/questions/66490/drawing-a-tikz-arc-specifying-the-center
%% Draws beach ball
\tikzset{pics/carc/.style args={#1:#2:#3}{code={\draw[pic actions] (#1:#3) arc(#1:#2:#3);}}}



\usepackage{array}
\setlength{\extrarowheight}{+.1cm}   
\newdimen\digitwidth
\settowidth\digitwidth{9}
\def\divrule#1#2{
\noalign{\moveright#1\digitwidth
\vbox{\hrule width#2\digitwidth}}}





\newcommand{\RR}{\mathbb R}
\newcommand{\R}{\mathbb R}
\newcommand{\N}{\mathbb N}
\newcommand{\Z}{\mathbb Z}

\newcommand{\sagemath}{\textsf{SageMath}}


%\renewcommand{\d}{\,d\!}
\renewcommand{\d}{\mathop{}\!d}
\newcommand{\dd}[2][]{\frac{\d #1}{\d #2}}
\newcommand{\pp}[2][]{\frac{\partial #1}{\partial #2}}
\renewcommand{\l}{\ell}
\newcommand{\ddx}{\frac{d}{\d x}}

\newcommand{\zeroOverZero}{\ensuremath{\boldsymbol{\tfrac{0}{0}}}}
\newcommand{\inftyOverInfty}{\ensuremath{\boldsymbol{\tfrac{\infty}{\infty}}}}
\newcommand{\zeroOverInfty}{\ensuremath{\boldsymbol{\tfrac{0}{\infty}}}}
\newcommand{\zeroTimesInfty}{\ensuremath{\small\boldsymbol{0\cdot \infty}}}
\newcommand{\inftyMinusInfty}{\ensuremath{\small\boldsymbol{\infty - \infty}}}
\newcommand{\oneToInfty}{\ensuremath{\boldsymbol{1^\infty}}}
\newcommand{\zeroToZero}{\ensuremath{\boldsymbol{0^0}}}
\newcommand{\inftyToZero}{\ensuremath{\boldsymbol{\infty^0}}}



\newcommand{\numOverZero}{\ensuremath{\boldsymbol{\tfrac{\#}{0}}}}
\newcommand{\dfn}{\textbf}
%\newcommand{\unit}{\,\mathrm}
\newcommand{\unit}{\mathop{}\!\mathrm}
\newcommand{\eval}[1]{\bigg[ #1 \bigg]}
\newcommand{\seq}[1]{\left( #1 \right)}
\renewcommand{\epsilon}{\varepsilon}
\renewcommand{\phi}{\varphi}


\renewcommand{\iff}{\Leftrightarrow}

\DeclareMathOperator{\arccot}{arccot}
\DeclareMathOperator{\arcsec}{arcsec}
\DeclareMathOperator{\arccsc}{arccsc}
\DeclareMathOperator{\si}{Si}
\DeclareMathOperator{\scal}{scal}
\DeclareMathOperator{\sign}{sign}


%% \newcommand{\tightoverset}[2]{% for arrow vec
%%   \mathop{#2}\limits^{\vbox to -.5ex{\kern-0.75ex\hbox{$#1$}\vss}}}
\newcommand{\arrowvec}[1]{{\overset{\rightharpoonup}{#1}}}
%\renewcommand{\vec}[1]{\arrowvec{\mathbf{#1}}}
\renewcommand{\vec}[1]{{\overset{\boldsymbol{\rightharpoonup}}{\mathbf{#1}}}}
\DeclareMathOperator{\proj}{\vec{proj}}
\newcommand{\veci}{{\boldsymbol{\hat{\imath}}}}
\newcommand{\vecj}{{\boldsymbol{\hat{\jmath}}}}
\newcommand{\veck}{{\boldsymbol{\hat{k}}}}
\newcommand{\vecl}{\vec{\boldsymbol{\l}}}
\newcommand{\uvec}[1]{\mathbf{\hat{#1}}}
\newcommand{\utan}{\mathbf{\hat{t}}}
\newcommand{\unormal}{\mathbf{\hat{n}}}
\newcommand{\ubinormal}{\mathbf{\hat{b}}}

\newcommand{\dotp}{\bullet}
\newcommand{\cross}{\boldsymbol\times}
\newcommand{\grad}{\boldsymbol\nabla}
\newcommand{\divergence}{\grad\dotp}
\newcommand{\curl}{\grad\cross}
%\DeclareMathOperator{\divergence}{divergence}
%\DeclareMathOperator{\curl}[1]{\grad\cross #1}
\newcommand{\lto}{\mathop{\longrightarrow\,}\limits}

\renewcommand{\bar}{\overline}

\colorlet{textColor}{black} 
\colorlet{background}{white}
\colorlet{penColor}{blue!50!black} % Color of a curve in a plot
\colorlet{penColor2}{red!50!black}% Color of a curve in a plot
\colorlet{penColor3}{red!50!blue} % Color of a curve in a plot
\colorlet{penColor4}{green!50!black} % Color of a curve in a plot
\colorlet{penColor5}{orange!80!black} % Color of a curve in a plot
\colorlet{penColor6}{yellow!70!black} % Color of a curve in a plot
\colorlet{fill1}{penColor!20} % Color of fill in a plot
\colorlet{fill2}{penColor2!20} % Color of fill in a plot
\colorlet{fillp}{fill1} % Color of positive area
\colorlet{filln}{penColor2!20} % Color of negative area
\colorlet{fill3}{penColor3!20} % Fill
\colorlet{fill4}{penColor4!20} % Fill
\colorlet{fill5}{penColor5!20} % Fill
\colorlet{gridColor}{gray!50} % Color of grid in a plot

\newcommand{\surfaceColor}{violet}
\newcommand{\surfaceColorTwo}{redyellow}
\newcommand{\sliceColor}{greenyellow}




\pgfmathdeclarefunction{gauss}{2}{% gives gaussian
  \pgfmathparse{1/(#2*sqrt(2*pi))*exp(-((x-#1)^2)/(2*#2^2))}%
}


%%%%%%%%%%%%%
%% Vectors
%%%%%%%%%%%%%

%% Simple horiz vectors
\renewcommand{\vector}[1]{\left\langle #1\right\rangle}


%% %% Complex Horiz Vectors with angle brackets
%% \makeatletter
%% \renewcommand{\vector}[2][ , ]{\left\langle%
%%   \def\nextitem{\def\nextitem{#1}}%
%%   \@for \el:=#2\do{\nextitem\el}\right\rangle%
%% }
%% \makeatother

%% %% Vertical Vectors
%% \def\vector#1{\begin{bmatrix}\vecListA#1,,\end{bmatrix}}
%% \def\vecListA#1,{\if,#1,\else #1\cr \expandafter \vecListA \fi}

%%%%%%%%%%%%%
%% End of vectors
%%%%%%%%%%%%%

%\newcommand{\fullwidth}{}
%\newcommand{\normalwidth}{}



%% makes a snazzy t-chart for evaluating functions
%\newenvironment{tchart}{\rowcolors{2}{}{background!90!textColor}\array}{\endarray}

%%This is to help with formatting on future title pages.
\newenvironment{sectionOutcomes}{}{} 



%% Flowchart stuff
%\tikzstyle{startstop} = [rectangle, rounded corners, minimum width=3cm, minimum height=1cm,text centered, draw=black]
%\tikzstyle{question} = [rectangle, minimum width=3cm, minimum height=1cm, text centered, draw=black]
%\tikzstyle{decision} = [trapezium, trapezium left angle=70, trapezium right angle=110, minimum width=3cm, minimum height=1cm, text centered, draw=black]
%\tikzstyle{question} = [rectangle, rounded corners, minimum width=3cm, minimum height=1cm,text centered, draw=black]
%\tikzstyle{process} = [rectangle, minimum width=3cm, minimum height=1cm, text centered, draw=black]
%\tikzstyle{decision} = [trapezium, trapezium left angle=70, trapezium right angle=110, minimum width=3cm, minimum height=1cm, text centered, draw=black]


\author{Jim Talamo}

\outcome{Understand terminology associated with ODEs.}
\outcome{Classify ODE as linear or nonlinear.}
\outcome{Determine the order of an ODE.}
\outcome{Verify if a given function is a solution to an ODE.}
\outcome{Verify if a given function is a solution to an IVP.}
\outcome{Find parameters so a function satisfies an ODE/IVP or explain why no such parameters exist.}

\title{Introduction To Differential Equations}

\begin{document}
\begin{abstract}
\end{abstract}
\maketitle

\vspace{-0.9in}

\section{Discussion Questions}

\begin{problem} 

Give the order of the following ordinary differential equations (ODEs). Then, determine if they are linear or nonlinear.

\begin{tabular}{lll}
I. $\frac{d^2y}{dx^2}+3x \left(\frac{dy}{dx}\right)^4  = y^3$ \qquad & II. $\frac{d^3y}{dx^3} + 2xy\frac{dy}{dx} = x^2$  \qquad & III. $\frac{d^4y}{dx^4}+\frac{dy}{dx} -x^2y=e^{3x}$
\end{tabular}

\begin{freeResponse}
I. Writing 
$$
\frac{d^2y}{dx^2}  =  y^3 - 3x \left(\frac{dy}{dx}\right)^4,
$$
we see that this is a second-order differential equation. Since the right-hand side is not linear in $y$ or $\frac{\d y}{\d x}$ due to the exponents on the terms $y^3$ and $\left(\frac{dy}{dx}\right)^4$, this differential equation is nonlinear.

II. The equation
$$
\frac{d^3y}{dx^3} = - 2xy\frac{dy}{dx} + x^2
$$
is third-order and nonlinear, due to the presence of the product $y \frac{\d y}{\d x}$. 

III. Rewriting the equation as 
$$
\frac{d^4y}{dx^4} = -\frac{dy}{dx} + x^2y + e^{3x},
$$
we see that it is fourth-order and linear, since it is linear in $y$ and $\frac{\d y}{\d x}$. 
\end{freeResponse}
\end{problem}


\begin{problem} 
For which of the following differential equations is $y=5e^{2x}$ a solution?

\begin{tabular}{lll}
I. $y'=5e^{2x}$ \qquad \qquad \qquad & II. $y''-3y'-y =0$ \qquad \qquad \qquad & III. $y^2\frac{dy}{dx}=50e^{2x}$
\end{tabular}

\begin{freeResponse}
First note that 
$$
y' = 10 e^{2x}
$$
and
$$
y'' = 20 e^{2x}.
$$

I. Since 
$$
y' = 10 e^{2x} \neq 5 e^{2x},
$$
$y$ is not a solution of this equation.

II. Substituting $y$ and its derivatives into the equation, we have
$$
y''-3y'-y = 20e^{2x}-3 \cdot 10e^{2x} - 5e^{2x} = -15 e^{2x} \neq 0,
$$
so $y$ is not a solution.

III. We have 
$$
y^2 \frac{dy}{dx} = \left(5e^{2x}\right)^2 \cdot 10e^{2x} = 250 e^{6x} \neq 50 e^{2x},
$$
so $y$ is not a solution of the equation.
\end{freeResponse}
\end{problem}

\section{Group Work}
%%%%%%%%%%%%%%%%%%%%%%%%%%%%%%%%
\begin{problem}
Explain whether $y= \sqrt{1-x^2}$ is a solution to the initial value problem (IVP)  below. $$\frac{dy}{dx} +\frac{2x}{y} =0, \quad y(0)=1$$

\begin{freeResponse}
We first calculate
$$
\frac{dy}{dx} = \frac{1}{2} (1-x^2)^{-1/2} \cdot (-2x) = -\frac{x}{\sqrt{1-x^2}}.
$$
Substituting this into the equation, we have
$$
\frac{dy}{dx} +\frac{2x}{y} = -\frac{x}{\sqrt{1-x^2}} + \frac{2x}{\sqrt{1-x^2}} = \frac{x}{\sqrt{1-x^2}} \neq 0.
$$
Therefore $y$ is not a solution of the IVP.
\end{freeResponse}
\end{problem}

\begin{problem} 
Consider the IVP: $$\frac{dy}{dx} +xy =0, \quad y(0)=1.$$  

\begin{itemize}
\item[I.]  Let $y=f(x)$ be a solution of the initial value problem.  Find the equation of the tangent line to $y=f(x)$ at $x=0$.
\item[II.] Find  $y''(0)$.
%Jim's Note: They WILL forget product rule.  please emphasize they do not have to solve the ODE to to this and point out PR is necessary.
\end{itemize}

\begin{freeResponse}
One approach to solving these types of problems would be to solve the IVP for $f(x)$, then answer the questions directly using the function. On the other hand, we can use the IVP to derive the necessary properties of $f(x)$ without solving for it explicitly. This is the approach we will take.

I. We have $f(0) = y(0) = 1$ and 
$$
f'(0) = \left.\frac{dy}{dx}\right|_{x=0} = \left.-xy\right|_{x=0} = 0.
$$
Therefore the tangent line to $y=f(x)$ at $x=0$ is the horizontal line $y=1$. 

II. From the differential equation, we have
$$
y' = -xy,
$$
and it follows that
$$
y'' = \frac{d}{dx}(-xy) = -y - xy'
$$
(remember that the product rule must be applied to take the derivative). Evaluating at $x=0$, we have
$$
y''(0) = -y(0)-0 \cdot y'(0) = -1.
$$
\end{freeResponse}
\end{problem}


\begin{problem} 
Find a function $f(x)$ so $y=\sin(x^2)$ is a solution to the ODE $\frac{dy}{dx}+4xy = f(x)$.

\begin{freeResponse}
The derivative of $y$ is 
$$
\frac{dy}{dx} = \cos(x^2)\cdot 2x.
$$
Substituting this into the left hand side of the ODE, we have
$$
2x \cos(x^2)  + 4x \sin(x^2) = f(x).
$$
\end{freeResponse}
\end{problem}


\begin{problem} 
Consider the differential equation $y''-4y'-5y = 0$.  Find all real values for the parameter $a$ such that $y=Ce^{ax}$ is a solution to the differential equation or explain why no such values exist.

\begin{freeResponse}
The derivatives of $y$ are 
$$
y' = aC e^{ax}
$$
and 
$$
y'' = a^2 C e^{ax}.
$$
Substituting this into the differential equation, we have
$$
y'' - 4y' -5y = a^2 C e^{ax} - 4 aC e^{ax} - 5Ce^{ax} = C e^{ax}\left(a^2 - 4a - 5\right) = 0.
$$
Therefore it must be that $C e^{ax} = 0$ or $a^2 - 4a - 5 = 0$. The first equation holds if and only if $C = 0$. We solve the second equation by factoring
$$
(a-5)(a+1) = 0,
$$ 
so that $a=-1$ or $a=5$. We conclude that if $C=0$ then all values of $a$ give a solution (but in this case $y=Ce^{ax} = 0$), and if $C \neq 0$, then $a=-1$ or $5$. 
\end{freeResponse}
\end{problem}


\begin{problem} 
Find values of $a_1$ and $a_2$ for which $y(x) = a _1 e^{-3x}+a_2$ is a solution to the initial value problem below or explain why no such values of $a_1$ and $a_2$ exist.  $$y'(x) +3y(x) = 9  , \qquad y(0) = -1 $$

\begin{freeResponse}
The derivative of $y$ is 
$$
y' = -3 a_1 e^{-3x}.
$$
For $y$ to solve the IVP, we need
$$
y' + 3y = -3a_1 e^{-3x} + 3a_1 e^{-3x} + 3 a_2 = 3 a_2 = 9.
$$
Therefore $a_2 = 3$. The IVP also requires 
$$
y(0) = a_1 e^{-3 \cdot 0} + a_2 = a_1 + 3 = -1.
$$
Therefore $a_1 = -4$. 
\end{freeResponse}
\end{problem}



\end{document}
