\documentclass[noauthor]{ximera}
%handout:  for handout version with no solutions or instructor notes
%handout,instructornotes:  for instructor version with just problems and notes, no solutions
%noinstructornotes:  shows only problem and solutions

%% handout
%% space
%% newpage
%% numbers
%% nooutcomes

%I added the commands here so that I would't have to keep looking them up
%\newcommand{\RR}{\mathbb R}
%\renewcommand{\d}{\,d}
%\newcommand{\dd}[2][]{\frac{d #1}{d #2}}
%\renewcommand{\l}{\ell}
%\newcommand{\ddx}{\frac{d}{dx}}
%\everymath{\displaystyle}
%\newcommand{\dfn}{\textbf}
%\newcommand{\eval}[1]{\bigg[ #1 \bigg]}

%\begin{image}
%\includegraphics[trim= 170 420 250 180]{Figure1.pdf}
%\end{image}

%add a ``.'' below when used in a specific directory.


%\usepackage{todonotes}

\newcommand{\todo}{}

\usepackage{esint} % for \oiint
\ifxake%%https://math.meta.stackexchange.com/questions/9973/how-do-you-render-a-closed-surface-double-integral
\renewcommand{\oiint}{{\large\bigcirc}\kern-1.56em\iint}
\fi


\graphicspath{
  {./}
  {ximeraTutorial/}
  {basicPhilosophy/}
  {functionsOfSeveralVariables/}
  {normalVectors/}
  {lagrangeMultipliers/}
  {vectorFields/}
  {greensTheorem/}
  {shapeOfThingsToCome/}
  {dotProducts/}
  {../productAndQuotientRules/exercises/}
  {../normalVectors/exercisesParametricPlots/}
  {../continuityOfFunctionsOfSeveralVariables/exercises/}
  {../partialDerivatives/exercises/}
  {../chainRuleForFunctionsOfSeveralVariables/exercises/}
  {../commonCoordinates/exercisesCylindricalCoordinates/}
  {../commonCoordinates/exercisesSphericalCoordinates/}
  {../greensTheorem/exercisesCurlAndLineIntegrals/}
  {../greensTheorem/exercisesDivergenceAndLineIntegrals/}
  {../shapeOfThingsToCome/exercisesDivergenceTheorem/}
  {../greensTheorem/}
  {../shapeOfThingsToCome/}
}

\newcommand{\mooculus}{\textsf{\textbf{MOOC}\textnormal{\textsf{ULUS}}}}

\usepackage{tkz-euclide}\usepackage{tikz}
\usepackage{tikz-cd}
\usetikzlibrary{arrows}
\tikzset{>=stealth,commutative diagrams/.cd,
  arrow style=tikz,diagrams={>=stealth}} %% cool arrow head
\tikzset{shorten <>/.style={ shorten >=#1, shorten <=#1 } } %% allows shorter vectors

\usetikzlibrary{backgrounds} %% for boxes around graphs
\usetikzlibrary{shapes,positioning}  %% Clouds and stars
\usetikzlibrary{matrix} %% for matrix
\usepgfplotslibrary{polar} %% for polar plots
\usepgfplotslibrary{fillbetween} %% to shade area between curves in TikZ
\usetkzobj{all}
%\usepackage[makeroom]{cancel} %% for strike outs
%\usepackage{mathtools} %% for pretty underbrace % Breaks Ximera
%\usepackage{multicol}
\usepackage{pgffor} %% required for integral for loops



%% http://tex.stackexchange.com/questions/66490/drawing-a-tikz-arc-specifying-the-center
%% Draws beach ball
\tikzset{pics/carc/.style args={#1:#2:#3}{code={\draw[pic actions] (#1:#3) arc(#1:#2:#3);}}}



\usepackage{array}
\setlength{\extrarowheight}{+.1cm}   
\newdimen\digitwidth
\settowidth\digitwidth{9}
\def\divrule#1#2{
\noalign{\moveright#1\digitwidth
\vbox{\hrule width#2\digitwidth}}}





\newcommand{\RR}{\mathbb R}
\newcommand{\R}{\mathbb R}
\newcommand{\N}{\mathbb N}
\newcommand{\Z}{\mathbb Z}

\newcommand{\sagemath}{\textsf{SageMath}}


%\renewcommand{\d}{\,d\!}
\renewcommand{\d}{\mathop{}\!d}
\newcommand{\dd}[2][]{\frac{\d #1}{\d #2}}
\newcommand{\pp}[2][]{\frac{\partial #1}{\partial #2}}
\renewcommand{\l}{\ell}
\newcommand{\ddx}{\frac{d}{\d x}}

\newcommand{\zeroOverZero}{\ensuremath{\boldsymbol{\tfrac{0}{0}}}}
\newcommand{\inftyOverInfty}{\ensuremath{\boldsymbol{\tfrac{\infty}{\infty}}}}
\newcommand{\zeroOverInfty}{\ensuremath{\boldsymbol{\tfrac{0}{\infty}}}}
\newcommand{\zeroTimesInfty}{\ensuremath{\small\boldsymbol{0\cdot \infty}}}
\newcommand{\inftyMinusInfty}{\ensuremath{\small\boldsymbol{\infty - \infty}}}
\newcommand{\oneToInfty}{\ensuremath{\boldsymbol{1^\infty}}}
\newcommand{\zeroToZero}{\ensuremath{\boldsymbol{0^0}}}
\newcommand{\inftyToZero}{\ensuremath{\boldsymbol{\infty^0}}}



\newcommand{\numOverZero}{\ensuremath{\boldsymbol{\tfrac{\#}{0}}}}
\newcommand{\dfn}{\textbf}
%\newcommand{\unit}{\,\mathrm}
\newcommand{\unit}{\mathop{}\!\mathrm}
\newcommand{\eval}[1]{\bigg[ #1 \bigg]}
\newcommand{\seq}[1]{\left( #1 \right)}
\renewcommand{\epsilon}{\varepsilon}
\renewcommand{\phi}{\varphi}


\renewcommand{\iff}{\Leftrightarrow}

\DeclareMathOperator{\arccot}{arccot}
\DeclareMathOperator{\arcsec}{arcsec}
\DeclareMathOperator{\arccsc}{arccsc}
\DeclareMathOperator{\si}{Si}
\DeclareMathOperator{\scal}{scal}
\DeclareMathOperator{\sign}{sign}


%% \newcommand{\tightoverset}[2]{% for arrow vec
%%   \mathop{#2}\limits^{\vbox to -.5ex{\kern-0.75ex\hbox{$#1$}\vss}}}
\newcommand{\arrowvec}[1]{{\overset{\rightharpoonup}{#1}}}
%\renewcommand{\vec}[1]{\arrowvec{\mathbf{#1}}}
\renewcommand{\vec}[1]{{\overset{\boldsymbol{\rightharpoonup}}{\mathbf{#1}}}}
\DeclareMathOperator{\proj}{\vec{proj}}
\newcommand{\veci}{{\boldsymbol{\hat{\imath}}}}
\newcommand{\vecj}{{\boldsymbol{\hat{\jmath}}}}
\newcommand{\veck}{{\boldsymbol{\hat{k}}}}
\newcommand{\vecl}{\vec{\boldsymbol{\l}}}
\newcommand{\uvec}[1]{\mathbf{\hat{#1}}}
\newcommand{\utan}{\mathbf{\hat{t}}}
\newcommand{\unormal}{\mathbf{\hat{n}}}
\newcommand{\ubinormal}{\mathbf{\hat{b}}}

\newcommand{\dotp}{\bullet}
\newcommand{\cross}{\boldsymbol\times}
\newcommand{\grad}{\boldsymbol\nabla}
\newcommand{\divergence}{\grad\dotp}
\newcommand{\curl}{\grad\cross}
%\DeclareMathOperator{\divergence}{divergence}
%\DeclareMathOperator{\curl}[1]{\grad\cross #1}
\newcommand{\lto}{\mathop{\longrightarrow\,}\limits}

\renewcommand{\bar}{\overline}

\colorlet{textColor}{black} 
\colorlet{background}{white}
\colorlet{penColor}{blue!50!black} % Color of a curve in a plot
\colorlet{penColor2}{red!50!black}% Color of a curve in a plot
\colorlet{penColor3}{red!50!blue} % Color of a curve in a plot
\colorlet{penColor4}{green!50!black} % Color of a curve in a plot
\colorlet{penColor5}{orange!80!black} % Color of a curve in a plot
\colorlet{penColor6}{yellow!70!black} % Color of a curve in a plot
\colorlet{fill1}{penColor!20} % Color of fill in a plot
\colorlet{fill2}{penColor2!20} % Color of fill in a plot
\colorlet{fillp}{fill1} % Color of positive area
\colorlet{filln}{penColor2!20} % Color of negative area
\colorlet{fill3}{penColor3!20} % Fill
\colorlet{fill4}{penColor4!20} % Fill
\colorlet{fill5}{penColor5!20} % Fill
\colorlet{gridColor}{gray!50} % Color of grid in a plot

\newcommand{\surfaceColor}{violet}
\newcommand{\surfaceColorTwo}{redyellow}
\newcommand{\sliceColor}{greenyellow}




\pgfmathdeclarefunction{gauss}{2}{% gives gaussian
  \pgfmathparse{1/(#2*sqrt(2*pi))*exp(-((x-#1)^2)/(2*#2^2))}%
}


%%%%%%%%%%%%%
%% Vectors
%%%%%%%%%%%%%

%% Simple horiz vectors
\renewcommand{\vector}[1]{\left\langle #1\right\rangle}


%% %% Complex Horiz Vectors with angle brackets
%% \makeatletter
%% \renewcommand{\vector}[2][ , ]{\left\langle%
%%   \def\nextitem{\def\nextitem{#1}}%
%%   \@for \el:=#2\do{\nextitem\el}\right\rangle%
%% }
%% \makeatother

%% %% Vertical Vectors
%% \def\vector#1{\begin{bmatrix}\vecListA#1,,\end{bmatrix}}
%% \def\vecListA#1,{\if,#1,\else #1\cr \expandafter \vecListA \fi}

%%%%%%%%%%%%%
%% End of vectors
%%%%%%%%%%%%%

%\newcommand{\fullwidth}{}
%\newcommand{\normalwidth}{}



%% makes a snazzy t-chart for evaluating functions
%\newenvironment{tchart}{\rowcolors{2}{}{background!90!textColor}\array}{\endarray}

%%This is to help with formatting on future title pages.
\newenvironment{sectionOutcomes}{}{} 



%% Flowchart stuff
%\tikzstyle{startstop} = [rectangle, rounded corners, minimum width=3cm, minimum height=1cm,text centered, draw=black]
%\tikzstyle{question} = [rectangle, minimum width=3cm, minimum height=1cm, text centered, draw=black]
%\tikzstyle{decision} = [trapezium, trapezium left angle=70, trapezium right angle=110, minimum width=3cm, minimum height=1cm, text centered, draw=black]
%\tikzstyle{question} = [rectangle, rounded corners, minimum width=3cm, minimum height=1cm,text centered, draw=black]
%\tikzstyle{process} = [rectangle, minimum width=3cm, minimum height=1cm, text centered, draw=black]
%\tikzstyle{decision} = [trapezium, trapezium left angle=70, trapezium right angle=110, minimum width=3cm, minimum height=1cm, text centered, draw=black]




\author{Tom Needham}

\outcome{Identify alternating series.}
\outcome{Apply the alternating series test to determine that a series converges.}

\title[Collaborate:]{The Alternating Series Test}

\begin{document}
\begin{abstract}
\end{abstract}
\maketitle

\vspace{-.4in}

\section{Discussion Questions}

\begin{problem}
Determine whether the following statements are true or false and justify your answers.

I. If $\{a_n\}$ is a nonincreasing sequence of positive numbers and $\lim_{n\rightarrow \infty} a_n = 0$, then the series $\sum (-1)^n a_n$ converges.

II. If the series $\sum (-1)^n a_n$ converges then $\lim_{n \rightarrow \infty} a_n = 0$.

III. If the series $\sum (-1)^n a_n$ converges, $\lim_{n\rightarrow \infty} a_n = 0$ and $a_n > 0$ for all $n$, then the sequence $\{a_n\}$ is nonincreasing.

IV. If the series $\sum (-1)^n a_n$ diverges and the sequence $\{a_n\}$ is nonincreasing, then $\lim_{n \rightarrow \infty} a_n \neq 0$.

\begin{freeResponse}
I. This is true; it is the Alternating Series Test.

II. This is true, because it follows from the Divergence Test. If the series converges, then $\lim_{n\rightarrow \infty} (-1)^n a_n = 0$, and this in turn implies that $\lim_{n\rightarrow \infty} a_n = 0$.

III. This statement is false. As a simple counterexample, consider the sequence defined by
$$
a_n = \left\{\begin{array}{cc}
n & n \leq 10 \\
\frac{1}{n} & n > 10. \end{array}\right.
$$
Then all of the hypotheses hold, but $a_n$ is not nonincreasing.

IV. This statement is true, but the logic is a bit tricky. The Alternating Series Test implies that if $\sum (-1)^n a_n$ diverges and $\{a_n\}$ is nonincreasing, then $\lim_{n \rightarrow \infty} a_n \neq 0$ \emph{or} the sequence $\{a_n\}$ is not strictly positive. A counterexample to the statement would therefore require a nonincreasing sequence $\{a_n\}$ containing negative terms with $\lim_{n\rightarrow \infty} a_n = 0$, and this is impossible.
\end{freeResponse}
\end{problem}

\begin{problem}
Determine which of the following series satisfy the hypotheses of the alternating series test.
\begin{center}
\begin{tabular}{lll}
I. $\sum_{n=1}^\infty (-1)^{n+1} \frac{1}{n}$ \hspace{.1in}&  II. $\sum_{n=1}^\infty \frac{\cos(n)}{n^3}$ \hspace{.1in}& III. $\sum_{k=1}^\infty a_k$, where $a_k = \left\{\begin{array}{cc}
\frac{1}{k^2} & \mbox{$k$ is even} \\
-\frac{1}{k} & \mbox{$k$ is odd}. \end{array}\right.$
\end{tabular}
\end{center}
\begin{freeResponse}
I. This series satisfies the hypotheses. This series is the famous \emph{alternating harmonic series}.

II. This series does not satisfy the hypotheses. This is a consequence of the fact that $\{\cos(n)\}$ does not alternate. Indeed, writing out the first few terms of the series, we have
$$
\sum_{n=1}^\infty \frac{\cos(n)}{n^3} = \frac{\cos(1)}{1^3} + \frac{\cos(2)}{2^3} + \frac{\cos(3)}{3^3} + \cdots \approx \frac{0.54}{1} + \frac{-0.42}{2^3} + \frac{-0.99}{3^3} + \cdots,
$$
and this is enough to show that the series is not alternating.

III. This series does not satisfy the hyptheses, because the sequence $\{|a_k|\}$ of absolute values of its terms is not nonincreasing. This is verified by writing out the first few terms:
\begin{align*}
|a_1| &= 1\\
|a_2| &= 1/4 \\
|a_3| &= 1/3 \\
|a_4| &= 1/16 \\
&\vdots
\end{align*}
\end{freeResponse}
\end{problem}

\section{Group Work}

\begin{problem}
Determine whether the following series converge or diverge. Justify your answer.
\begin{center}
\begin{tabular}{lll}
I. $\sum_{n=1}^\infty (-1)^n \frac{n^3+e^2}{n^{7/2}+n^2}$ \hspace{.2in} II. $\sum_{k=1}^\infty (-1)^k \frac{\ln(k/2)}{k}$ \hspace{.2in} III. $\sum_{n=1}^\infty (-1)^{n-1} \frac{5^n}{3^{2n}}$ 
\end{tabular}
\end{center}

\begin{freeResponse}
I. This series satisfies the hypotheses of the Alternating Series Test. Since 
$$
\lim_{n\rightarrow \infty}  \frac{n^3+e^2}{n^{7/2}+n^2} = 0
$$
by comparing growth rates of the dominant terms ($n^3$ in the numerator and $n^{7/2}$ in the denominator), this series converges.

II. This series does not quite satisfy the hypotheses of the Alternating Series Test, since $\ln(1/2) < 0$, $\ln(2/2) = 0$ and $\ln(k/2) > 0$ for all $k \geq 3$. However, the series
$$
\sum_{k=3}^\infty (-1)^k \frac{\ln(k/2)}{k}
$$
does satisfy the hypotheses. Since $\lim_{k \rightarrow \infty} \ln(k/2)/k = 0$, by growth rates comparison, this new series converges by the Alternating Series Test. The new series differs from the original one by finitely many terms, so this implies that the original series also converges.

III. We can solve this two ways.  By noting that

\[
(-1)^{n-1} \frac{5^n}{3^{2n}} = (-1)^n\cdot(-1)^1 \frac{5^n}{(3^2)^n}=  -\frac{(-1)^n \cdot 5^n}{9^n} =  -\frac{(-5)^n}{9^n}=  -\left(\frac{-5}{9}\right)^n,
\]

we can see that the series is geometric with $r=-\frac{5}{9}$, so it converges.

We can also note that, by doing similar algebra, 
\[
(-1)^{n-1} \frac{5^n}{3^{2n}} = (-1)^{n-1} \left(\frac{5}{9}\right)^n
\]

and use the alternating series test.  Notice that $\left(\frac{5}{9}\right)^n$ is decreasing and $\lim_{n \to \infty} \left(\frac{5}{9}\right)^n=0$ . Therefore the series diverges by the Alternating Series Test.

The advantage to recognizing this as a geometric series is that we can determine the value to which it converges; the alternating series test only tells us that the series converges but gives no information about its value.
\end{freeResponse}
\end{problem}

\begin{problem}
Suppose that $\{a_n\}_{n=1}$ is a sequence and let $s_n = \sum_{k=1}^n a_k$.  Suppose that the explicit formula for $s_n$ is known to be
$$
s_n = \frac{(-1)^n}{2n+1}.
$$
\begin{itemize}
\item[I.] Determine if $\sum_{k=1}^{\infty} a_k$ converges or diverges.  If it converges, can you determine its value?
\item[II.] Determine if $\sum_{k=1}^{\infty} s_k$ converges or diverges.
\end{itemize}

\begin{freeResponse}
First note that 
$$
\lim_{n\rightarrow \infty} \frac{(-1)^n}{2n+1} = \lim_{n\rightarrow \infty} \frac{1}{2n+1} = 0.
$$
We also see that the sequence $\{1/(2n+1)\}$ is strictly positive and nonincreasing. We therefore conclude that the series $\sum a_k$ converges to $0$, by the definition of series convergence, and that the series $\sum s_n$ converges (to some value we are unable to determine) by the Alternating Series Test.
\end{freeResponse}
\end{problem}

\begin{problem}
Suppose that $\{a_n\}_{n=1}$ is a sequence for which $a_n = \frac{(-1)^n}{\sqrt{n}}$ and let $s_n = \sum_{k=1}^n a_k$.  
\begin{itemize}
\item[I.] Is the sequence $\{a_n\}_{n=1}$ monotonic?  Is it bounded?
\item[II.] Is the sequence $\{s_n\}_{n=1}$ monotonic?  Is it bounded?
\end{itemize}

\begin{freeResponse}
I. Notice that since each term successively changes sign, $\{a_n\}$ will not be monotonic.  However, it is bounded since $\left|\frac{1}{\sqrt{n}} <1\right|$ for all $n \geq 1$, so $-1 \leq  \frac{(-1)^n}{\sqrt{n}} \leq 1$ (You could also say that since $\lim_{n \to \infty} \frac{(-1)^n}{\sqrt{n}} =0$, that $\{a_n\}_{n=1}$ is bounded).

II. $\{s_n\}_{n=1}$ is not monotonic since $s_{n+1} =s_n +a_{n+1}$ and the terms in the sequence $\{a_n\}$ alternate in sign.  $\{s_n\}$ is bounded though since $\sum_{k=1}^{\infty} \frac{(-1)^k}{\sqrt{k}}$ converges by the alternating series test, $\lim_{n \to \infty} s_n$ exists.  If the limit of a sequence exists, then the sequence must be bounded.


\end{freeResponse}
\end{problem}
\end{document}
