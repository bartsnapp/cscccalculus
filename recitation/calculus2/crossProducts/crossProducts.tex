\documentclass[noauthor, handout]{ximera}
%handout:  for handout version with no freeResponses or instructor notes
%handout,instructornotes:  for instructor version with just problems and notes, no freeResponses
%noinstructornotes:  shows only problem and freeResponses

%% handout
%% space
%% newpage
%% numbers
%% nooutcomes

%I added the commands here so that I would't have to keep looking them up
%\newcommand{\RR}{\mathbb R}
%\renewcommand{\d}{\,d}
%\newcommand{\dd}[2][]{\frac{d #1}{d #2}}
%\renewcommand{\l}{\ell}
%\newcommand{\ddx}{\frac{d}{dx}}
%\everymath{\displaystyle}
%\newcommand{\dfn}{\textbf}
%\newcommand{\eval}[1]{\bigg[ #1 \bigg]}

%\begin{image}
%\includegraphics[trim= 170 420 250 180]{Figure1.pdf}
%\end{image}

%add a ``.'' below when used in a specific directory.

%\usepackage{todonotes}

\newcommand{\todo}{}

\usepackage{esint} % for \oiint
\ifxake%%https://math.meta.stackexchange.com/questions/9973/how-do-you-render-a-closed-surface-double-integral
\renewcommand{\oiint}{{\large\bigcirc}\kern-1.56em\iint}
\fi


\graphicspath{
  {./}
  {ximeraTutorial/}
  {basicPhilosophy/}
  {functionsOfSeveralVariables/}
  {normalVectors/}
  {lagrangeMultipliers/}
  {vectorFields/}
  {greensTheorem/}
  {shapeOfThingsToCome/}
  {dotProducts/}
  {../productAndQuotientRules/exercises/}
  {../normalVectors/exercisesParametricPlots/}
  {../continuityOfFunctionsOfSeveralVariables/exercises/}
  {../partialDerivatives/exercises/}
  {../chainRuleForFunctionsOfSeveralVariables/exercises/}
  {../commonCoordinates/exercisesCylindricalCoordinates/}
  {../commonCoordinates/exercisesSphericalCoordinates/}
  {../greensTheorem/exercisesCurlAndLineIntegrals/}
  {../greensTheorem/exercisesDivergenceAndLineIntegrals/}
  {../shapeOfThingsToCome/exercisesDivergenceTheorem/}
  {../greensTheorem/}
  {../shapeOfThingsToCome/}
}

\newcommand{\mooculus}{\textsf{\textbf{MOOC}\textnormal{\textsf{ULUS}}}}

\usepackage{tkz-euclide}\usepackage{tikz}
\usepackage{tikz-cd}
\usetikzlibrary{arrows}
\tikzset{>=stealth,commutative diagrams/.cd,
  arrow style=tikz,diagrams={>=stealth}} %% cool arrow head
\tikzset{shorten <>/.style={ shorten >=#1, shorten <=#1 } } %% allows shorter vectors

\usetikzlibrary{backgrounds} %% for boxes around graphs
\usetikzlibrary{shapes,positioning}  %% Clouds and stars
\usetikzlibrary{matrix} %% for matrix
\usepgfplotslibrary{polar} %% for polar plots
\usepgfplotslibrary{fillbetween} %% to shade area between curves in TikZ
\usetkzobj{all}
%\usepackage[makeroom]{cancel} %% for strike outs
%\usepackage{mathtools} %% for pretty underbrace % Breaks Ximera
%\usepackage{multicol}
\usepackage{pgffor} %% required for integral for loops



%% http://tex.stackexchange.com/questions/66490/drawing-a-tikz-arc-specifying-the-center
%% Draws beach ball
\tikzset{pics/carc/.style args={#1:#2:#3}{code={\draw[pic actions] (#1:#3) arc(#1:#2:#3);}}}



\usepackage{array}
\setlength{\extrarowheight}{+.1cm}   
\newdimen\digitwidth
\settowidth\digitwidth{9}
\def\divrule#1#2{
\noalign{\moveright#1\digitwidth
\vbox{\hrule width#2\digitwidth}}}





\newcommand{\RR}{\mathbb R}
\newcommand{\R}{\mathbb R}
\newcommand{\N}{\mathbb N}
\newcommand{\Z}{\mathbb Z}

\newcommand{\sagemath}{\textsf{SageMath}}


%\renewcommand{\d}{\,d\!}
\renewcommand{\d}{\mathop{}\!d}
\newcommand{\dd}[2][]{\frac{\d #1}{\d #2}}
\newcommand{\pp}[2][]{\frac{\partial #1}{\partial #2}}
\renewcommand{\l}{\ell}
\newcommand{\ddx}{\frac{d}{\d x}}

\newcommand{\zeroOverZero}{\ensuremath{\boldsymbol{\tfrac{0}{0}}}}
\newcommand{\inftyOverInfty}{\ensuremath{\boldsymbol{\tfrac{\infty}{\infty}}}}
\newcommand{\zeroOverInfty}{\ensuremath{\boldsymbol{\tfrac{0}{\infty}}}}
\newcommand{\zeroTimesInfty}{\ensuremath{\small\boldsymbol{0\cdot \infty}}}
\newcommand{\inftyMinusInfty}{\ensuremath{\small\boldsymbol{\infty - \infty}}}
\newcommand{\oneToInfty}{\ensuremath{\boldsymbol{1^\infty}}}
\newcommand{\zeroToZero}{\ensuremath{\boldsymbol{0^0}}}
\newcommand{\inftyToZero}{\ensuremath{\boldsymbol{\infty^0}}}



\newcommand{\numOverZero}{\ensuremath{\boldsymbol{\tfrac{\#}{0}}}}
\newcommand{\dfn}{\textbf}
%\newcommand{\unit}{\,\mathrm}
\newcommand{\unit}{\mathop{}\!\mathrm}
\newcommand{\eval}[1]{\bigg[ #1 \bigg]}
\newcommand{\seq}[1]{\left( #1 \right)}
\renewcommand{\epsilon}{\varepsilon}
\renewcommand{\phi}{\varphi}


\renewcommand{\iff}{\Leftrightarrow}

\DeclareMathOperator{\arccot}{arccot}
\DeclareMathOperator{\arcsec}{arcsec}
\DeclareMathOperator{\arccsc}{arccsc}
\DeclareMathOperator{\si}{Si}
\DeclareMathOperator{\scal}{scal}
\DeclareMathOperator{\sign}{sign}


%% \newcommand{\tightoverset}[2]{% for arrow vec
%%   \mathop{#2}\limits^{\vbox to -.5ex{\kern-0.75ex\hbox{$#1$}\vss}}}
\newcommand{\arrowvec}[1]{{\overset{\rightharpoonup}{#1}}}
%\renewcommand{\vec}[1]{\arrowvec{\mathbf{#1}}}
\renewcommand{\vec}[1]{{\overset{\boldsymbol{\rightharpoonup}}{\mathbf{#1}}}}
\DeclareMathOperator{\proj}{\vec{proj}}
\newcommand{\veci}{{\boldsymbol{\hat{\imath}}}}
\newcommand{\vecj}{{\boldsymbol{\hat{\jmath}}}}
\newcommand{\veck}{{\boldsymbol{\hat{k}}}}
\newcommand{\vecl}{\vec{\boldsymbol{\l}}}
\newcommand{\uvec}[1]{\mathbf{\hat{#1}}}
\newcommand{\utan}{\mathbf{\hat{t}}}
\newcommand{\unormal}{\mathbf{\hat{n}}}
\newcommand{\ubinormal}{\mathbf{\hat{b}}}

\newcommand{\dotp}{\bullet}
\newcommand{\cross}{\boldsymbol\times}
\newcommand{\grad}{\boldsymbol\nabla}
\newcommand{\divergence}{\grad\dotp}
\newcommand{\curl}{\grad\cross}
%\DeclareMathOperator{\divergence}{divergence}
%\DeclareMathOperator{\curl}[1]{\grad\cross #1}
\newcommand{\lto}{\mathop{\longrightarrow\,}\limits}

\renewcommand{\bar}{\overline}

\colorlet{textColor}{black} 
\colorlet{background}{white}
\colorlet{penColor}{blue!50!black} % Color of a curve in a plot
\colorlet{penColor2}{red!50!black}% Color of a curve in a plot
\colorlet{penColor3}{red!50!blue} % Color of a curve in a plot
\colorlet{penColor4}{green!50!black} % Color of a curve in a plot
\colorlet{penColor5}{orange!80!black} % Color of a curve in a plot
\colorlet{penColor6}{yellow!70!black} % Color of a curve in a plot
\colorlet{fill1}{penColor!20} % Color of fill in a plot
\colorlet{fill2}{penColor2!20} % Color of fill in a plot
\colorlet{fillp}{fill1} % Color of positive area
\colorlet{filln}{penColor2!20} % Color of negative area
\colorlet{fill3}{penColor3!20} % Fill
\colorlet{fill4}{penColor4!20} % Fill
\colorlet{fill5}{penColor5!20} % Fill
\colorlet{gridColor}{gray!50} % Color of grid in a plot

\newcommand{\surfaceColor}{violet}
\newcommand{\surfaceColorTwo}{redyellow}
\newcommand{\sliceColor}{greenyellow}




\pgfmathdeclarefunction{gauss}{2}{% gives gaussian
  \pgfmathparse{1/(#2*sqrt(2*pi))*exp(-((x-#1)^2)/(2*#2^2))}%
}


%%%%%%%%%%%%%
%% Vectors
%%%%%%%%%%%%%

%% Simple horiz vectors
\renewcommand{\vector}[1]{\left\langle #1\right\rangle}


%% %% Complex Horiz Vectors with angle brackets
%% \makeatletter
%% \renewcommand{\vector}[2][ , ]{\left\langle%
%%   \def\nextitem{\def\nextitem{#1}}%
%%   \@for \el:=#2\do{\nextitem\el}\right\rangle%
%% }
%% \makeatother

%% %% Vertical Vectors
%% \def\vector#1{\begin{bmatrix}\vecListA#1,,\end{bmatrix}}
%% \def\vecListA#1,{\if,#1,\else #1\cr \expandafter \vecListA \fi}

%%%%%%%%%%%%%
%% End of vectors
%%%%%%%%%%%%%

%\newcommand{\fullwidth}{}
%\newcommand{\normalwidth}{}



%% makes a snazzy t-chart for evaluating functions
%\newenvironment{tchart}{\rowcolors{2}{}{background!90!textColor}\array}{\endarray}

%%This is to help with formatting on future title pages.
\newenvironment{sectionOutcomes}{}{} 



%% Flowchart stuff
%\tikzstyle{startstop} = [rectangle, rounded corners, minimum width=3cm, minimum height=1cm,text centered, draw=black]
%\tikzstyle{question} = [rectangle, minimum width=3cm, minimum height=1cm, text centered, draw=black]
%\tikzstyle{decision} = [trapezium, trapezium left angle=70, trapezium right angle=110, minimum width=3cm, minimum height=1cm, text centered, draw=black]
%\tikzstyle{question} = [rectangle, rounded corners, minimum width=3cm, minimum height=1cm,text centered, draw=black]
%\tikzstyle{process} = [rectangle, minimum width=3cm, minimum height=1cm, text centered, draw=black]
%\tikzstyle{decision} = [trapezium, trapezium left angle=70, trapezium right angle=110, minimum width=3cm, minimum height=1cm, text centered, draw=black]






\author{Jim Talamo and Tom Needham}

\outcome{Compute cross products between $2$ and $3$-dimensional vectors.}
\outcome{Answer conceptual questions about dot products, cross products, and projections.}

\title[Collaborate:]{Cross Products}

\begin{document}
\begin{abstract}
\end{abstract}
\maketitle

\section{Discussion Questions}

\begin{problem}
Suppose that $\vec{u}$ and $\vec{v}$ are three dimensional, nonzero vectors and let $\dotp$ denote the vector dot product and $\cross$ denote the vector cross product.  Determine whether the following quantities are scalars, vectors, or undefined. 

Suppose that $\vec{u} = \vector{1,-1,0}$ and $\vec{v} = \vector{0,2,1}$.

\begin{tabular}{lll}
I. $(\vec{u} \dotp \vec{v})+\vec{u}$ \qquad \qquad \qquad & II. $(proj_{\vec{u}} \vec{v}) \cross \vec{v}$.   \qquad \qquad  & III. $(\vec{u} \cross \vec{v}) \dotp \vec{u}$\\[2ex]
IV. $(scal_{\vec{v}} \vec{u}) \dotp \vec{v}$ & V. $\frac{\vec{u}}{\vec{v}} \cross \vec{v}$ & VI. $proj_{\vec{u} \dotp \vec{v}} \vec{u}$
\end{tabular}

\begin{freeResponse}
I. Since $\vec{u} \dotp \vec{v}$ is a scalar and $\vec{u}$ is a vector, this quantity is undefined.

II. This quantity is a vector.

III. This quantity is a scalar.

IV. Since $scal_{\vec{v}} \vec{u}$ is a scalar, this quantity is undefined. 

V. The object $\frac{\vec{u}}{\vec{v}}$ is undefined, so the quantity is also undefined. 

VI. Since $\vec{u} \dotp \vec{v}$ is a scalar, this quantity is undefined. 
\end{freeResponse}

\end{problem}


\begin{problem}
Suppose that $\vec{u} = \vector{1,-1,0}$ and $\vec{v} = \vector{0,2,1}$.

\begin{itemize}
\item[I.] Find $\vec{u} \cross \vec{v}$.
\item[II.] Check whether $\vec{u} \cross \vec{v}$ is orthogonal to $\vec{u}$.  Is it orthogonal to $\vec{v}$?  
\end{itemize}

\begin{freeResponse}
I. We compute
$$
\vec{u} \cross \vec{v} = \left<-1 \cdot 1 - 0 \cdot 2, -(1 \cdot 1- 0 \cdot 0), 1 \cdot 2 - (-1) \cdot 0\right> = \left<-1,-1,2\right>.
$$

II. Taking the dot product of our answer to part I. with $\vec{u}$, we obtain
$$
(\vec{u} \cross \vec{v}) \dotp \vec{u} = \left<-1,-1,2\right> \dotp \left<1,-1,0\right> = -1 \cdot 1 + (-1) \cdot (-1) + 2 \cdot 0 = 0.
$$
Therefore $\vec{u} \cross \vec{v}$ is orthogonal to $\vec{u}$. Similarly, it is orthogonal to $\vec{v}$. 
\end{freeResponse}

\end{problem}



\section{Group Work}

\begin{problem}
Let $\vec{u} = \vector{0,4,5}$ and $\vec{v} =\vector{-1,0,2}$. 
\begin{itemize}
\item[I.] Find a vector $\vec{w}$ with magnitude 4 that is \emph{orthogonal} to $\vec{u}$.
\item[II.] Find a vector $\vec{w}$ with magnitude 4 that is \emph{parallel} to $\vec{u}$.
%Jim's Note: Many of them tried to do this by setting \vec{w} = <a,b,c> and using cross products on AU 17 Midterm 3.  Please indicate that we have 3 ways to determine if vectors are parallel: (1) Compute the angle using dot products (or note that in this case, u \dotp v = \pm |u||v|, (2) show corss product is 0 (but only works for 3d vectors) or (3) using the definition that u = c v for c \neq 0.

\end{itemize}
\begin{freeResponse}
I. Since $\vec{u}$ lies in the $yz$-plane, any vector of the form $\vector{a,0,0}$ is orthogonal to it. In particular, the vector $\vector{4,0,0}$ has magnitude $4$. 

II. Any vector of the form $a \vec{u}$ is parallel to $\vec{u}$. In particular, the vector
$$
\frac{4}{\left|\vec{u}\right|} \vec{u} = \frac{4}{\sqrt{4^2 + 5^2}} \vector{0,4,5} = \frac{4}{41}\vector{0,4,5}
$$
has magnitude $4$. 
\end{freeResponse}
\end{problem}

\begin{problem}
Find the area of the parallelogram below.

\begin{center}
\resizebox {5cm} {!} {\begin{tikzpicture}

\begin{axis}
	[
	domain=0:6, ymax=3.5,xmax=3.5, ymin=-.25, xmin=-.25,
	axis lines=center, xlabel=$x$, ylabel=$y$,
	xtick={1,2,3},
	ytick={1,2,3},
	every axis y label/.style={at=(current axis.above origin),anchor=south},
	every axis x label/.style={at=(current axis.right of origin),anchor=west},
	axis on top,
	typeset ticklabels with strut,
	]

	\addplot [draw=penColor,very thick, smooth,domain=0:2] {.5*x};
	\addplot [draw=penColor,very thick, smooth,domain=2:3] {2*x-3};
	\addplot [draw=penColor,very thick, smooth,domain=0:1] {2*x};
	\addplot [draw=penColor,very thick, smooth,domain=1:3] {.5*x+1.5};
	
	\addplot [name path=A,domain=0:2,draw=none] {.5*x};  
	\addplot [name path=B,domain=2:3,draw=none]  {2*x-3};  
	\addplot [name path=C,domain=0:1,draw=none] {2*x}; 
	\addplot [name path=D,domain=1:3,draw=none] {.5*x+1.5}; 
	\addplot [fillp] fill between[of=A and C];
	\addplot [fillp] fill between[of=A and D];
	\addplot [fillp] fill between[of=B and D];

\end{axis}

\end{tikzpicture}}
\end{center}

\begin{freeResponse}
The parallelogram is spanned by $\vector{2,1}$ and $\vector{1,2}$, so its area is given by
$$
\left|\vector{2,1,0} \cross \vector{1,2,0}\right| = 3.
$$
\end{freeResponse}
\end{problem}

\begin{problem}
Let $\vec{u} = \vector{-2,5,0}$ and $\vec{v} = \vector{0,-1,2}$. 
\begin{itemize}
\item[I.] Find a vector $\vec{p}$ parallel to $\vec{v}$ and a vector $\vec{n}$ orthogonal to $\vec{v}$ so that $\vec{u} = \vec{p} + \vec{n}$.

\item[II.]  Find a vector orthogonal to both $\vec{p}$ and $\vec{n}$.

\item[III.] Let $\vec{w} = \vector{-6,2,-4}$. Find constants $a$, $b$, and $c$ so $\vec{w} = a\vec{p}+b\vec{n}+c\vec{p} \times \vec{n}$.
%Jim: If you compute w = u x v, you should get <10,4,2>.  Then, a=2,b=1/2, c=-1/2 if I didn't do something silly.  Please write down both the system of equations you get by equating components, comment that it is annoying to solve.  then, use the fact that we have an orthognal basis so we can find a by taking the dot product of both sides of $\vec{w} = a\vec{p}+b\vec{n}+c\vec{w}$ with p, etc.

\end{itemize}

\begin{freeResponse}
I. Let 
$$
\vec{p} = proj_\vec{v} \vec{u} = \frac{\vec{u}\dotp \vec{v}}{\left|\vec{v}\right|^2} \vec{v} = -\frac{5}{5} \vector{0,-1,2} = \vector{0,1,-2}.
$$
Then $\vec{p}$ is parallel to $\vec{v}$, 
$$
\vec{n} = \vec{u} - \vec{p} = \vector{-2,5,0} - \vector{0,1,-2} = \vector{-2,4,2}.
$$
is orthogonal to $\vec{v}$, and $\vec{u} = \vec{p} + \vec{n}$.

II. A vector which is orthogonal to both $\vec{p}$ and $\vec{n}$ is given by
$$
\vec{p} \times \vec{n} = \vector{10,4,2}.
$$

III. We wish to find constants $a$, $b$ and $c$ such that 
$$
\vector{-6,2,-4} = a \vector{0,1,-2} + b \vector{-2,4,2} + c \vector{10,4,2}.
$$
This can be written as a system of three linear equations in three unknowns:
\begin{align*}
-6 &= -2b + 10c \\
2 &= a + 4b + 4c \\
-4 &= -2a + 2b + 2c.
\end{align*}
This could be solved via the usual methods, although it would require some tedious algebra. On the other hand, a more clever solution is to realize that $\vec{p}$, $\vec{n}$ and $\vec{p} \times \vec{n}$ are mutually orthogonal. This means we can solve for the constants by taking dot products; that is, dotting both sides of the equation with $\vec{p}$ yields
$$
\vec{w} \dotp \vec{p} = (a\vec{p}+b\vec{n}+c\vec{p} \times \vec{n})\dotp \vec{p} = a \left|\vec{p}\right|^2.
$$
Therefore
$$
\vector{-6,2,-4} \dotp \vector{0,1,-2} = a \vec{p} \dotp \vec{p} + b \vec{n} \dotp \vec{p} + c (\vec{p} \times \vec{n})\dotp \vec{p} = a \left|\vec{p}\right|^2 + b \cdot 0 + c \cdot 0,
$$
or $10 = a \cdot 5$. Therefore $a=2$. 

Similarly, we dot with $\vec{n}$ to obtain
$$
\vec{w} \dotp \vec{n} = b \left|\vec{n}\right|^2 \Rightarrow 12 = b \cdot 24,
$$
hence $b = 1/2$. Finally, dotting with $\vec{p} \times \vec{n}$ yields
$$
\vec{w} \dotp (\vec{p} \times \vec{n}) = c \cdot \left|\vec{p} \times \vec{n} \right|^2 \Rightarrow -60 = c \cdot 120,
$$
so that $c = -1/2$. 
\end{freeResponse}
\end{problem}

\begin{problem}
Suppose that $\vec{u}$ and $\vec{v}$ are nonzero three dimensional vectors.  Show that $\vec{u}$ and $\vec{v}$ are parallel if and only if $\vec{u} \cross (\vec{u} \cross \vec{v}) = \vec{0}$. 

\begin{freeResponse}
Recall that, in general, $\vec{u} \times \vec{v} = \vec{0}$ if and only if $\vec{u}$ and $\vec{v}$ are parallel. Therefore $\vec{u} \cross (\vec{u} \cross \vec{v}) = \vec{0}$ if and only if $\vec{u}$ is parallel to $\vec{u} \cross \vec{v}$. On the other hand, $\vec{u} \cross \vec{v}$ and $\vec{u}$ are always perpendicular. It follows that $\vec{u} \cross \vec{v}$ is both perpendicular and parallel to the same vector, and it must therefore be the case that $\vec{u} \cross \vec{v} = \vec{0}$. By the first comment, this is the case if and only if $\vec{u}$ and $\vec{v}$ are parallel. 
\end{freeResponse}
\end{problem}


\begin{problem}
Let $\vec{u}$ and $\vec{v}$ be nonzero $3$-dimensional vectors. Determine whether the following statements are true or false. 

\begin{enumerate}[label=(\alph*)]
\item $|\vec{u} \cross \vec{v}| = |\vec{v} \cross \vec{u}|$
\item $proj_{\vec{v}} (\vec{u} \cross \vec{v}) = \vec{0}$
\item If $\hat{u}$ and $\hat{v}$ are unit vectors, then $\hat{u} \cross \hat{v}$ is a unit vector. 
\item $(proj_{\vec{v}} \vec{u}) \cross \vec{u} = \vec{0}$
\item If $|\vec{u} \cross \vec{v}| = |\vec{u}||\vec{v}|$, then $\vec{u}$ and $\vec{v}$ are orthogonal. 
\end{enumerate}


\begin{freeResponse}
\begin{enumerate}[label=(\alph*)]
\item Switching the order in a cross product only changes the sign of the result, so this statement is true.
\item This statement is true, since $\vec{u} \times \vec{v}$ is perpendicular to $\vec{v}$. 
\item This is false, since (in general) $\left|\vec{u} \times \vec{v}\right| = \left|\vec{u}\right| \cdot \left|\vec{v} \right| \cdot \sin \theta$, where $\theta$ is the interior angle between the vectors. The claim therefore fails if $\hat{u}$ and $\hat{v}$ are not orthogonal.
\item This statement is false; for example, it fails when $\vec{u} = \vector{1,1,0}$ and $\vec{v} = \vector{1,0,0}$. 
\item This statement is true. If $|\vec{u} \cross \vec{v}| = |\vec{u}||\vec{v}|$, then the interior angle $\theta$ between $\vec{u}$ and $\vec{v}$ must be $\theta = \pi/2$, in which case $\vec{u}$ and $\vec{v}$ are orthogonal. 
\end{enumerate}
\end{freeResponse}
\end{problem}


\end{document}
