\documentclass[noauthor]{ximera}
%handout:  for handout version with no solutions or instructor notes
%handout,instructornotes:  for instructor version with just problems and notes, no solutions
%noinstructornotes:  shows only problem and solutions

%% handout
%% space
%% newpage
%% numbers
%% nooutcomes

%I added the commands here so that I would't have to keep looking them up
%\newcommand{\RR}{\mathbb R}
%\renewcommand{\d}{\,d}
%\newcommand{\dd}[2][]{\frac{d #1}{d #2}}
%\renewcommand{\l}{\ell}
%\newcommand{\ddx}{\frac{d}{dx}}
%\everymath{\displaystyle}
%\newcommand{\dfn}{\textbf}
%\newcommand{\eval}[1]{\bigg[ #1 \bigg]}

%\begin{image}
%\includegraphics[trim= 170 420 250 180]{Figure1.pdf}
%\end{image}

%add a ``.'' below when used in a specific directory.

%%\usepackage{todonotes}

\newcommand{\todo}{}

\usepackage{esint} % for \oiint
\ifxake%%https://math.meta.stackexchange.com/questions/9973/how-do-you-render-a-closed-surface-double-integral
\renewcommand{\oiint}{{\large\bigcirc}\kern-1.56em\iint}
\fi


\graphicspath{
  {./}
  {ximeraTutorial/}
  {basicPhilosophy/}
  {functionsOfSeveralVariables/}
  {normalVectors/}
  {lagrangeMultipliers/}
  {vectorFields/}
  {greensTheorem/}
  {shapeOfThingsToCome/}
  {dotProducts/}
  {../productAndQuotientRules/exercises/}
  {../normalVectors/exercisesParametricPlots/}
  {../continuityOfFunctionsOfSeveralVariables/exercises/}
  {../partialDerivatives/exercises/}
  {../chainRuleForFunctionsOfSeveralVariables/exercises/}
  {../commonCoordinates/exercisesCylindricalCoordinates/}
  {../commonCoordinates/exercisesSphericalCoordinates/}
  {../greensTheorem/exercisesCurlAndLineIntegrals/}
  {../greensTheorem/exercisesDivergenceAndLineIntegrals/}
  {../shapeOfThingsToCome/exercisesDivergenceTheorem/}
  {../greensTheorem/}
  {../shapeOfThingsToCome/}
}

\newcommand{\mooculus}{\textsf{\textbf{MOOC}\textnormal{\textsf{ULUS}}}}

\usepackage{tkz-euclide}\usepackage{tikz}
\usepackage{tikz-cd}
\usetikzlibrary{arrows}
\tikzset{>=stealth,commutative diagrams/.cd,
  arrow style=tikz,diagrams={>=stealth}} %% cool arrow head
\tikzset{shorten <>/.style={ shorten >=#1, shorten <=#1 } } %% allows shorter vectors

\usetikzlibrary{backgrounds} %% for boxes around graphs
\usetikzlibrary{shapes,positioning}  %% Clouds and stars
\usetikzlibrary{matrix} %% for matrix
\usepgfplotslibrary{polar} %% for polar plots
\usepgfplotslibrary{fillbetween} %% to shade area between curves in TikZ
\usetkzobj{all}
%\usepackage[makeroom]{cancel} %% for strike outs
%\usepackage{mathtools} %% for pretty underbrace % Breaks Ximera
%\usepackage{multicol}
\usepackage{pgffor} %% required for integral for loops



%% http://tex.stackexchange.com/questions/66490/drawing-a-tikz-arc-specifying-the-center
%% Draws beach ball
\tikzset{pics/carc/.style args={#1:#2:#3}{code={\draw[pic actions] (#1:#3) arc(#1:#2:#3);}}}



\usepackage{array}
\setlength{\extrarowheight}{+.1cm}   
\newdimen\digitwidth
\settowidth\digitwidth{9}
\def\divrule#1#2{
\noalign{\moveright#1\digitwidth
\vbox{\hrule width#2\digitwidth}}}





\newcommand{\RR}{\mathbb R}
\newcommand{\R}{\mathbb R}
\newcommand{\N}{\mathbb N}
\newcommand{\Z}{\mathbb Z}

\newcommand{\sagemath}{\textsf{SageMath}}


%\renewcommand{\d}{\,d\!}
\renewcommand{\d}{\mathop{}\!d}
\newcommand{\dd}[2][]{\frac{\d #1}{\d #2}}
\newcommand{\pp}[2][]{\frac{\partial #1}{\partial #2}}
\renewcommand{\l}{\ell}
\newcommand{\ddx}{\frac{d}{\d x}}

\newcommand{\zeroOverZero}{\ensuremath{\boldsymbol{\tfrac{0}{0}}}}
\newcommand{\inftyOverInfty}{\ensuremath{\boldsymbol{\tfrac{\infty}{\infty}}}}
\newcommand{\zeroOverInfty}{\ensuremath{\boldsymbol{\tfrac{0}{\infty}}}}
\newcommand{\zeroTimesInfty}{\ensuremath{\small\boldsymbol{0\cdot \infty}}}
\newcommand{\inftyMinusInfty}{\ensuremath{\small\boldsymbol{\infty - \infty}}}
\newcommand{\oneToInfty}{\ensuremath{\boldsymbol{1^\infty}}}
\newcommand{\zeroToZero}{\ensuremath{\boldsymbol{0^0}}}
\newcommand{\inftyToZero}{\ensuremath{\boldsymbol{\infty^0}}}



\newcommand{\numOverZero}{\ensuremath{\boldsymbol{\tfrac{\#}{0}}}}
\newcommand{\dfn}{\textbf}
%\newcommand{\unit}{\,\mathrm}
\newcommand{\unit}{\mathop{}\!\mathrm}
\newcommand{\eval}[1]{\bigg[ #1 \bigg]}
\newcommand{\seq}[1]{\left( #1 \right)}
\renewcommand{\epsilon}{\varepsilon}
\renewcommand{\phi}{\varphi}


\renewcommand{\iff}{\Leftrightarrow}

\DeclareMathOperator{\arccot}{arccot}
\DeclareMathOperator{\arcsec}{arcsec}
\DeclareMathOperator{\arccsc}{arccsc}
\DeclareMathOperator{\si}{Si}
\DeclareMathOperator{\scal}{scal}
\DeclareMathOperator{\sign}{sign}


%% \newcommand{\tightoverset}[2]{% for arrow vec
%%   \mathop{#2}\limits^{\vbox to -.5ex{\kern-0.75ex\hbox{$#1$}\vss}}}
\newcommand{\arrowvec}[1]{{\overset{\rightharpoonup}{#1}}}
%\renewcommand{\vec}[1]{\arrowvec{\mathbf{#1}}}
\renewcommand{\vec}[1]{{\overset{\boldsymbol{\rightharpoonup}}{\mathbf{#1}}}}
\DeclareMathOperator{\proj}{\vec{proj}}
\newcommand{\veci}{{\boldsymbol{\hat{\imath}}}}
\newcommand{\vecj}{{\boldsymbol{\hat{\jmath}}}}
\newcommand{\veck}{{\boldsymbol{\hat{k}}}}
\newcommand{\vecl}{\vec{\boldsymbol{\l}}}
\newcommand{\uvec}[1]{\mathbf{\hat{#1}}}
\newcommand{\utan}{\mathbf{\hat{t}}}
\newcommand{\unormal}{\mathbf{\hat{n}}}
\newcommand{\ubinormal}{\mathbf{\hat{b}}}

\newcommand{\dotp}{\bullet}
\newcommand{\cross}{\boldsymbol\times}
\newcommand{\grad}{\boldsymbol\nabla}
\newcommand{\divergence}{\grad\dotp}
\newcommand{\curl}{\grad\cross}
%\DeclareMathOperator{\divergence}{divergence}
%\DeclareMathOperator{\curl}[1]{\grad\cross #1}
\newcommand{\lto}{\mathop{\longrightarrow\,}\limits}

\renewcommand{\bar}{\overline}

\colorlet{textColor}{black} 
\colorlet{background}{white}
\colorlet{penColor}{blue!50!black} % Color of a curve in a plot
\colorlet{penColor2}{red!50!black}% Color of a curve in a plot
\colorlet{penColor3}{red!50!blue} % Color of a curve in a plot
\colorlet{penColor4}{green!50!black} % Color of a curve in a plot
\colorlet{penColor5}{orange!80!black} % Color of a curve in a plot
\colorlet{penColor6}{yellow!70!black} % Color of a curve in a plot
\colorlet{fill1}{penColor!20} % Color of fill in a plot
\colorlet{fill2}{penColor2!20} % Color of fill in a plot
\colorlet{fillp}{fill1} % Color of positive area
\colorlet{filln}{penColor2!20} % Color of negative area
\colorlet{fill3}{penColor3!20} % Fill
\colorlet{fill4}{penColor4!20} % Fill
\colorlet{fill5}{penColor5!20} % Fill
\colorlet{gridColor}{gray!50} % Color of grid in a plot

\newcommand{\surfaceColor}{violet}
\newcommand{\surfaceColorTwo}{redyellow}
\newcommand{\sliceColor}{greenyellow}




\pgfmathdeclarefunction{gauss}{2}{% gives gaussian
  \pgfmathparse{1/(#2*sqrt(2*pi))*exp(-((x-#1)^2)/(2*#2^2))}%
}


%%%%%%%%%%%%%
%% Vectors
%%%%%%%%%%%%%

%% Simple horiz vectors
\renewcommand{\vector}[1]{\left\langle #1\right\rangle}


%% %% Complex Horiz Vectors with angle brackets
%% \makeatletter
%% \renewcommand{\vector}[2][ , ]{\left\langle%
%%   \def\nextitem{\def\nextitem{#1}}%
%%   \@for \el:=#2\do{\nextitem\el}\right\rangle%
%% }
%% \makeatother

%% %% Vertical Vectors
%% \def\vector#1{\begin{bmatrix}\vecListA#1,,\end{bmatrix}}
%% \def\vecListA#1,{\if,#1,\else #1\cr \expandafter \vecListA \fi}

%%%%%%%%%%%%%
%% End of vectors
%%%%%%%%%%%%%

%\newcommand{\fullwidth}{}
%\newcommand{\normalwidth}{}



%% makes a snazzy t-chart for evaluating functions
%\newenvironment{tchart}{\rowcolors{2}{}{background!90!textColor}\array}{\endarray}

%%This is to help with formatting on future title pages.
\newenvironment{sectionOutcomes}{}{} 



%% Flowchart stuff
%\tikzstyle{startstop} = [rectangle, rounded corners, minimum width=3cm, minimum height=1cm,text centered, draw=black]
%\tikzstyle{question} = [rectangle, minimum width=3cm, minimum height=1cm, text centered, draw=black]
%\tikzstyle{decision} = [trapezium, trapezium left angle=70, trapezium right angle=110, minimum width=3cm, minimum height=1cm, text centered, draw=black]
%\tikzstyle{question} = [rectangle, rounded corners, minimum width=3cm, minimum height=1cm,text centered, draw=black]
%\tikzstyle{process} = [rectangle, minimum width=3cm, minimum height=1cm, text centered, draw=black]
%\tikzstyle{decision} = [trapezium, trapezium left angle=70, trapezium right angle=110, minimum width=3cm, minimum height=1cm, text centered, draw=black]

%\usepackage{todonotes}

\newcommand{\todo}{}

\usepackage{esint} % for \oiint
\ifxake%%https://math.meta.stackexchange.com/questions/9973/how-do-you-render-a-closed-surface-double-integral
\renewcommand{\oiint}{{\large\bigcirc}\kern-1.56em\iint}
\fi


\graphicspath{
  {./}
  {ximeraTutorial/}
  {basicPhilosophy/}
  {functionsOfSeveralVariables/}
  {normalVectors/}
  {lagrangeMultipliers/}
  {vectorFields/}
  {greensTheorem/}
  {shapeOfThingsToCome/}
  {dotProducts/}
  {../productAndQuotientRules/exercises/}
  {../normalVectors/exercisesParametricPlots/}
  {../continuityOfFunctionsOfSeveralVariables/exercises/}
  {../partialDerivatives/exercises/}
  {../chainRuleForFunctionsOfSeveralVariables/exercises/}
  {../commonCoordinates/exercisesCylindricalCoordinates/}
  {../commonCoordinates/exercisesSphericalCoordinates/}
  {../greensTheorem/exercisesCurlAndLineIntegrals/}
  {../greensTheorem/exercisesDivergenceAndLineIntegrals/}
  {../shapeOfThingsToCome/exercisesDivergenceTheorem/}
  {../greensTheorem/}
  {../shapeOfThingsToCome/}
}

\newcommand{\mooculus}{\textsf{\textbf{MOOC}\textnormal{\textsf{ULUS}}}}

\usepackage{tkz-euclide}\usepackage{tikz}
\usepackage{tikz-cd}
\usetikzlibrary{arrows}
\tikzset{>=stealth,commutative diagrams/.cd,
  arrow style=tikz,diagrams={>=stealth}} %% cool arrow head
\tikzset{shorten <>/.style={ shorten >=#1, shorten <=#1 } } %% allows shorter vectors

\usetikzlibrary{backgrounds} %% for boxes around graphs
\usetikzlibrary{shapes,positioning}  %% Clouds and stars
\usetikzlibrary{matrix} %% for matrix
\usepgfplotslibrary{polar} %% for polar plots
\usepgfplotslibrary{fillbetween} %% to shade area between curves in TikZ
\usetkzobj{all}
%\usepackage[makeroom]{cancel} %% for strike outs
%\usepackage{mathtools} %% for pretty underbrace % Breaks Ximera
%\usepackage{multicol}
\usepackage{pgffor} %% required for integral for loops



%% http://tex.stackexchange.com/questions/66490/drawing-a-tikz-arc-specifying-the-center
%% Draws beach ball
\tikzset{pics/carc/.style args={#1:#2:#3}{code={\draw[pic actions] (#1:#3) arc(#1:#2:#3);}}}



\usepackage{array}
\setlength{\extrarowheight}{+.1cm}   
\newdimen\digitwidth
\settowidth\digitwidth{9}
\def\divrule#1#2{
\noalign{\moveright#1\digitwidth
\vbox{\hrule width#2\digitwidth}}}





\newcommand{\RR}{\mathbb R}
\newcommand{\R}{\mathbb R}
\newcommand{\N}{\mathbb N}
\newcommand{\Z}{\mathbb Z}

\newcommand{\sagemath}{\textsf{SageMath}}


%\renewcommand{\d}{\,d\!}
\renewcommand{\d}{\mathop{}\!d}
\newcommand{\dd}[2][]{\frac{\d #1}{\d #2}}
\newcommand{\pp}[2][]{\frac{\partial #1}{\partial #2}}
\renewcommand{\l}{\ell}
\newcommand{\ddx}{\frac{d}{\d x}}

\newcommand{\zeroOverZero}{\ensuremath{\boldsymbol{\tfrac{0}{0}}}}
\newcommand{\inftyOverInfty}{\ensuremath{\boldsymbol{\tfrac{\infty}{\infty}}}}
\newcommand{\zeroOverInfty}{\ensuremath{\boldsymbol{\tfrac{0}{\infty}}}}
\newcommand{\zeroTimesInfty}{\ensuremath{\small\boldsymbol{0\cdot \infty}}}
\newcommand{\inftyMinusInfty}{\ensuremath{\small\boldsymbol{\infty - \infty}}}
\newcommand{\oneToInfty}{\ensuremath{\boldsymbol{1^\infty}}}
\newcommand{\zeroToZero}{\ensuremath{\boldsymbol{0^0}}}
\newcommand{\inftyToZero}{\ensuremath{\boldsymbol{\infty^0}}}



\newcommand{\numOverZero}{\ensuremath{\boldsymbol{\tfrac{\#}{0}}}}
\newcommand{\dfn}{\textbf}
%\newcommand{\unit}{\,\mathrm}
\newcommand{\unit}{\mathop{}\!\mathrm}
\newcommand{\eval}[1]{\bigg[ #1 \bigg]}
\newcommand{\seq}[1]{\left( #1 \right)}
\renewcommand{\epsilon}{\varepsilon}
\renewcommand{\phi}{\varphi}


\renewcommand{\iff}{\Leftrightarrow}

\DeclareMathOperator{\arccot}{arccot}
\DeclareMathOperator{\arcsec}{arcsec}
\DeclareMathOperator{\arccsc}{arccsc}
\DeclareMathOperator{\si}{Si}
\DeclareMathOperator{\scal}{scal}
\DeclareMathOperator{\sign}{sign}


%% \newcommand{\tightoverset}[2]{% for arrow vec
%%   \mathop{#2}\limits^{\vbox to -.5ex{\kern-0.75ex\hbox{$#1$}\vss}}}
\newcommand{\arrowvec}[1]{{\overset{\rightharpoonup}{#1}}}
%\renewcommand{\vec}[1]{\arrowvec{\mathbf{#1}}}
\renewcommand{\vec}[1]{{\overset{\boldsymbol{\rightharpoonup}}{\mathbf{#1}}}}
\DeclareMathOperator{\proj}{\vec{proj}}
\newcommand{\veci}{{\boldsymbol{\hat{\imath}}}}
\newcommand{\vecj}{{\boldsymbol{\hat{\jmath}}}}
\newcommand{\veck}{{\boldsymbol{\hat{k}}}}
\newcommand{\vecl}{\vec{\boldsymbol{\l}}}
\newcommand{\uvec}[1]{\mathbf{\hat{#1}}}
\newcommand{\utan}{\mathbf{\hat{t}}}
\newcommand{\unormal}{\mathbf{\hat{n}}}
\newcommand{\ubinormal}{\mathbf{\hat{b}}}

\newcommand{\dotp}{\bullet}
\newcommand{\cross}{\boldsymbol\times}
\newcommand{\grad}{\boldsymbol\nabla}
\newcommand{\divergence}{\grad\dotp}
\newcommand{\curl}{\grad\cross}
%\DeclareMathOperator{\divergence}{divergence}
%\DeclareMathOperator{\curl}[1]{\grad\cross #1}
\newcommand{\lto}{\mathop{\longrightarrow\,}\limits}

\renewcommand{\bar}{\overline}

\colorlet{textColor}{black} 
\colorlet{background}{white}
\colorlet{penColor}{blue!50!black} % Color of a curve in a plot
\colorlet{penColor2}{red!50!black}% Color of a curve in a plot
\colorlet{penColor3}{red!50!blue} % Color of a curve in a plot
\colorlet{penColor4}{green!50!black} % Color of a curve in a plot
\colorlet{penColor5}{orange!80!black} % Color of a curve in a plot
\colorlet{penColor6}{yellow!70!black} % Color of a curve in a plot
\colorlet{fill1}{penColor!20} % Color of fill in a plot
\colorlet{fill2}{penColor2!20} % Color of fill in a plot
\colorlet{fillp}{fill1} % Color of positive area
\colorlet{filln}{penColor2!20} % Color of negative area
\colorlet{fill3}{penColor3!20} % Fill
\colorlet{fill4}{penColor4!20} % Fill
\colorlet{fill5}{penColor5!20} % Fill
\colorlet{gridColor}{gray!50} % Color of grid in a plot

\newcommand{\surfaceColor}{violet}
\newcommand{\surfaceColorTwo}{redyellow}
\newcommand{\sliceColor}{greenyellow}




\pgfmathdeclarefunction{gauss}{2}{% gives gaussian
  \pgfmathparse{1/(#2*sqrt(2*pi))*exp(-((x-#1)^2)/(2*#2^2))}%
}


%%%%%%%%%%%%%
%% Vectors
%%%%%%%%%%%%%

%% Simple horiz vectors
\renewcommand{\vector}[1]{\left\langle #1\right\rangle}


%% %% Complex Horiz Vectors with angle brackets
%% \makeatletter
%% \renewcommand{\vector}[2][ , ]{\left\langle%
%%   \def\nextitem{\def\nextitem{#1}}%
%%   \@for \el:=#2\do{\nextitem\el}\right\rangle%
%% }
%% \makeatother

%% %% Vertical Vectors
%% \def\vector#1{\begin{bmatrix}\vecListA#1,,\end{bmatrix}}
%% \def\vecListA#1,{\if,#1,\else #1\cr \expandafter \vecListA \fi}

%%%%%%%%%%%%%
%% End of vectors
%%%%%%%%%%%%%

%\newcommand{\fullwidth}{}
%\newcommand{\normalwidth}{}



%% makes a snazzy t-chart for evaluating functions
%\newenvironment{tchart}{\rowcolors{2}{}{background!90!textColor}\array}{\endarray}

%%This is to help with formatting on future title pages.
\newenvironment{sectionOutcomes}{}{} 



%% Flowchart stuff
%\tikzstyle{startstop} = [rectangle, rounded corners, minimum width=3cm, minimum height=1cm,text centered, draw=black]
%\tikzstyle{question} = [rectangle, minimum width=3cm, minimum height=1cm, text centered, draw=black]
%\tikzstyle{decision} = [trapezium, trapezium left angle=70, trapezium right angle=110, minimum width=3cm, minimum height=1cm, text centered, draw=black]
%\tikzstyle{question} = [rectangle, rounded corners, minimum width=3cm, minimum height=1cm,text centered, draw=black]
%\tikzstyle{process} = [rectangle, minimum width=3cm, minimum height=1cm, text centered, draw=black]
%\tikzstyle{decision} = [trapezium, trapezium left angle=70, trapezium right angle=110, minimum width=3cm, minimum height=1cm, text centered, draw=black]




\author{Jim Talamo}

\outcome{State the domain and range of a function of two variables.}
\outcome{Identify level curves for a function of several variables.}
\outcome{Identify whether a curve lies on a level curve of a function.}
\outcome{Give parameterizations of level curves of a function and curves on the associated surface.}
\outcome{Give a parametric description of a curve on a surface given the preimage of the curve.}

\title[Collaborate:]{Functions of Several Variables}

\newcommand{\Set}[1]{ \left\{ (x,y) \in R^2 ~ \big| ~ #1 \right\} }

\begin{document}
\begin{abstract}
\end{abstract}
\maketitle

\section{Discussion Questions}

\begin{problem}
State the domain and range of the following functions.  
\begin{center}
\begin{tabular}{lll}
I. $f(x,y) = \ln(2x+4y^2)$  &II. $f(x,y) =e^{x^2+2y^4-1}$  &III. $f(x,y) = \sqrt{16- 4x^4y^2} $ \hspace{10mm} 
\end{tabular}
\end{center}

\begin{freeResponse}

Recall that unless otherwise specified, the domain of a function will be taken to be the set of all $(x,y) \in \R^2$ for which the formula that describes the function is defined.

Also, the domain of a function $f:\R^2 \to \R$ will be a subset of $\R^2$, whereas the range will be a subset of $\R$.

I. For the natural logarithm to be defined, its argument must be positive. This requires that $2x+4y^2 >0$.  The range of the logarithm is $\R$, with the logarithm tending to $-\infty$ as its argument tends to $0$ and $\infty$ as its argument becomes arbitrarily large.  Since $2x+4y^2$ can approach $0$ along a path in the domain of the function (take $y=0$, $x \to 0^+$) and become arbitrarily large (let $x$ and $y$ become arbitrarily large and positive), the range will be $\R$.

\[
\textrm{Domain:} \Set{2x+4y^2 >0} \qquad \textrm{ and } \qquad \textrm{ Range: } \R
\]
 
II. The exponential is defined for all $(x,y) \in \R^2$.   Since the exponential is an increasing function of its input, and $x^2+2y^4-1 \geq -1$, with equality when $(x,y) = (0,0)$, the range is $[e^{-1},\infty)$.

\[
\textrm{Domain: } \R^2 \qquad \textrm{ and } \qquad \textrm{ Range: } [e^{-1},\infty)
\]

III. The square root is defined when its argument is nonnegative.   The square root always returns nonnegative values, and achieves its minimum when its argument is $0$.  Since $(x,y) = (2,0)$ is in the domain and $\sqrt{16- 4x^4y^2}=0$ there.  Also, $16-4x^4y^2 \leq 16$, with equality when $(x,y)=(0,0)$, so $\sqrt{16- 4x^4y^2} \leq \sqrt{16} =4$.
\[
\textrm{Domain: } \Set{16-4x^4y^2 \geq 0} \qquad \textrm{ and } \qquad \textrm{ Range: } [0,4]
\]

\end{freeResponse}
\end{problem}

%%%%%%%%%%%%%%%%FIX THIS TO HAVE NICER NUMBERS LATER%%%%%%%%%%%%%%%%%%%%%%%%%%%%%%%%%

\begin{problem}
Consider the function $f(x,y) =  \begin{cases} 2x+3y , & x+2y^2 \neq 4 \\ \frac{x-2y}{x^2-y^2} , & x+2y^2=4 \end{cases} $ . 

I. Find $f(1,2)$ and $f(2,1)$.

II. State the domain of $f(x,y)$.

\begin{freeResponse}
We must consider which piece to use.

I. For $f(1,2)$, note that $(1)+2(2)^2 = 9 \neq 4$, so we use the top piece.

\[
f(1,2) =2(1)+3(2) =8.
\]

For $f(2,1)$, note that $(2)+2(1)^2 = 4$, so we use the bottom piece.

\[
f(2,1) = \frac{2-2(1)}{(2)^2-(1)^2} = 0.
\]

II. Note that the top piece is defined for all $(x,y)$, and the bottom is undefined only when $x^2-y^2=0$.  Since the bottom piece is defined only for $(x,y)$ where $x+2y^2=4$, we must find all points $(x,y)$ for which:

\begin{align}
x^2-y^2&=0 \\
x+2y^2&=4.
\end{align}

From the first condition, we have $y^2=x^2$, so substituting this into the second one gives

\begin{align*}
x+2x^2 &=4 \\
2x^2+x-4 &= 0 \\
x &= \frac{-1\pm \sqrt{1-4(2)(-4)}}{2(2)} \\
&=\frac{-1\pm \sqrt{33}}{4} 
\end{align*}

Since $y^2=x^2$, $y= \pm x$, so there are four such points.  Thus, the domain is all $(x,y) \in R^2$ except for  $\left(\frac{-1 - \sqrt{33}}{4},\frac{-1- \sqrt{33}}{4} \right)$, $\left(\frac{-1 - \sqrt{33}}{4},\frac{-1 + \sqrt{33}}{4} \right)$, $\left(\frac{-1 + \sqrt{33}}{4},\frac{-1 - \sqrt{33}}{4} \right)$, and $\left(\frac{-1 + \sqrt{33}}{4},\frac{-1 + \sqrt{33}}{4} \right)$.

\end{freeResponse}
\end{problem}

%%%%%%%%%%%%%%%%%%%%%%%%%%%%%%%%%%%%%%%%%%%%%%%%%

\begin{problem}
Consider the function $f(x,y) = x^2+xy^2+17$.

Which of the following curves in $\R^2$ described below are part of a level curve of $f(x,y)$?

\begin{itemize}
\item[A.] $x=y$
\item[B.] $x^2+xy^2=5$
\item[C.] $\vector{-t^2,t}$
\item[D.] $\vector{0,\sin(2t)}$
\end{itemize}

\begin{freeResponse}
In order to determine which curves lie on level curves, we must determine whether $f(x,y)$ is constant along each.

I. When $y=x$,

\[f(x,y) = f(x,x) =x^2+x^3+17.\]  

This explicitly depends on $x$, so the line $y=x$ is not part of a level curve of $f(x,y)$.

II. When $x^2+xy^2=5$,

\[ f(x,y) = x^2+xy^2+17 = \left[x^2+xy^2\right] +17 = 5+17 = 22.\]

Thus, the curve described by  $x^2+xy^2=5$ is part of a level curve of $f(x,y)$.

III. For the curve parameterized by $\vector{-t^2,t}$, note 

\[f(x,y) = f(-t^2,t) =   (-t^2)^2+(-t^2)(t)^2+17 = 17.\] 

Thus, the curve parameterized by $\vector{-t^2,t}$ is part of a level curve of $f(x,y)$.

IV. For the curve parameterized by $\vector{0,\sin(2t)}$, note 

\[f(x,y) = f(0,\sin(2t)) =   (0)^2+(0)(\sin(2t))^2+17 = 17.\] 

Thus, the curve parameterized by $\vector{0,\sin(2t)}$ is part of a level curve of $f(x,y)$.

\end{freeResponse}
\end{problem}

%%%%%%%%%%%%%%%%%%%%%%%%%%%%%%%%%%%%%%%%%%%%%%%%%

\section{Group Work}


\begin{problem}
What are the level curves of the plane $2x-y+3z=9$?  Find the level curve corresponding to $z=1$ in terms of $x$ and $y$.  On the same set of axes, plot a representation of the level curve and the curve that corresponds to it on the plane.

\begin{freeResponse}
The level curves of $2x-y+3z=9$ are found by setting $z=c$, where $c$ is a constant.  Thus

\begin{align*}
2x-y+3c&=9 \\
2x-y &= 9-3c
\end{align*}

These are lines.  Specifically, when $z=1$, the level curve is the line $2x-y=6$ in the $xy$-plane, which appears a height $6$ above the $xy$-plane on the surface.

\end{freeResponse}
\end{problem}

%%%%%%%%%%%%%%%%%%%%%%%%%%%%%%%%%%%%%%%%%%%%%%%%%

\begin{problem}
Give a parametric description of the curve on the surface \[z=8x^2+3y-2\] associated to the given curve in the $xy$-plane.

\begin{center}
\begin{tabular}{lll}
I. $y=4x^2+1$ \hspace{15mm} & II. $x^2+2xy=4$ \hspace{15mm} & III. $x^2+y^2=9$
\end{tabular}
\end{center}

\begin{freeResponse}
We parameterize the curve in the domain, then use the surface to give a parameterization for the curve on the surface.

I. Let $x(t)=t$, so $y(t) =4t^2+1$.  Since $z=8x^2+3y-2$, we have $z(t) = 8t^2+3(4t^2+1)-2 = 20t^2-1.$  A parameterization is thus

\[
\vec{r}(t) = \vector{ t, 4t^2+1, 20t^2-1}.
\]

II. Note that we can explicitly solve for $y$ in the equation $x^2+2xy=4$ by noting

\begin{align*}
x^2+2xy&=4\\
2xy&=4-x^2 \\
y &= \frac{2}{x}-\frac{x}{2}
\end{align*}
Note that we may divide by $x$ here since there are no points on the curve for which $x=0$.  Thus, we may proceed as before and find that a parameterization of the curve on the surface is

\[
\vec{r}(t) = \vector{ t, \frac{2}{t}-\frac{t}{2}, 8t^2+ \frac{6}{t}-\frac{3t}{2}-2}.
\]

III. When working with circles and ellipses, it is best to take advantage of the Pythagorean trigonometric identities.

\begin{align*}
x(t) &= 3\cos(t) \\
y(t) &= 3\sin(t)
\end{align*}

Note that $[x(t)]^2+[y(t)]^2 = [3\cos(t)]^2+[3\sin(t)]^2 = 9\cos^2(t) + 9\sin^2(t) = 9$, so this gives a parameterization of the circle.

Since $z=8x^2+3y-2$, we have $z(t) = 8[3\cos(t)]^2+3[3\sin(t)]-2 = 24\cos^2(t)-9\sin(t)-2.$
\[
\vec{r}(t) = \vector{3\cos(t), 3\sin(t), 24\cos^2(t)-9\sin(t)-2}.
\]

\end{freeResponse}
\end{problem}

%%%%%%%%%%%%%%%%%%%%%%%%%%%%%%%%%%%%%%%%%%%%%%%%%


\begin{problem}
Consider the function $f(x,y) = 2y+4x^2+5$.

I. For the surface $z=f(x,y)$, find a description of the level curve corresponding to $z=9$ in terms of $x$ and $y$ only.

II. Give a parameterization of the level curve and the corresponding curve that lies on the surface $z=f(x,y)$.

III. Determine if the curve $C$ traced out by $\vec{p}(t) = \vector{t^2,3t}$ is part of a level curve of $f(x,y)$.  Give a parametric description of the curve on the surface $z=f(x,y)$ that corresponds to $C$ .

\begin{freeResponse}
Remember that the desired  level curve is a curve in the $xy$-plane, which is found by setting $z=9$.

I. The level curve is $9 =2y+4x^2+5$.  We can simplify a little, which will be helpful for the next part.

\begin{align*}
2y+4x^2+5 &=9 \\
2y+4x^2 &= 4 \\
y+2x^2 &= 2
\end{align*}

II. Set $x(t)=t$ so $y(t) =2-2t^2$. Since this parameterizes the level curve along which $z=9$, we have a parameterization for the curve on the surface.

\[
\vec{r}(t) = \vector{t,2-2t^2,9}
\]

\begin{remark}
We may check our work by explicitly substituting $x(t)=t$ and $y(t) = 2-2t^2$ into the equation that defines the surface.  

\[z(t) = 2\big[2-2t^2\big] +4 \big[t\big]^2+5 = 9.\]

\end{remark}

III. To determine if the curve traced out by $\vec{p}(t) = \vector{t^2,3t}$ is part of a level curve, we note that along it

\[
z=f(x,y)=f(t^2,3t) = 2\big[3t\big]+4\big[ t^2 \big]^2+5 = 6t+4t^4+5.
\]

This is not a constant since it explicitly depends on $t$, so $\vec{p}(t)$ does not trace out part of a level curve.  However, a parameterization $\vec{P}(t)$ of the curve on the surface is found from the above calculation easily.

\[
\vec{P}(t) = \vector{t^2,3t,4t^4+6t+5}
\]
\end{freeResponse}
\end{problem}

%%%%%%%%%%%%%%%%%%%%%%%%%%%%%%%%%%%%%%%%%%%%%%%%%


\end{document}