\documentclass[]{ximera}
%handout:  for handout version with no solutions or instructor notes
%handout,instructornotes:  for instructor version with just problems and notes, no solutions
%noinstructornotes:  shows only problem and solutions

%% handout
%% space
%% newpage
%% numbers
%% nooutcomes

%I added the commands here so that I would't have to keep looking them up
%\newcommand{\RR}{\mathbb R}
%\renewcommand{\d}{\,d}
%\newcommand{\dd}[2][]{\frac{d #1}{d #2}}
%\renewcommand{\l}{\ell}
%\newcommand{\ddx}{\frac{d}{dx}}
%\everymath{\displaystyle}
%\newcommand{\dfn}{\textbf}
%\newcommand{\eval}[1]{\bigg[ #1 \bigg]}

%\begin{image}
%\includegraphics[trim= 170 420 250 180]{Figure1.pdf}
%\end{image}

%add a ``.'' below when used in a specific directory.

%\usepackage{todonotes}

\newcommand{\todo}{}

\usepackage{esint} % for \oiint
\ifxake%%https://math.meta.stackexchange.com/questions/9973/how-do-you-render-a-closed-surface-double-integral
\renewcommand{\oiint}{{\large\bigcirc}\kern-1.56em\iint}
\fi


\graphicspath{
  {./}
  {ximeraTutorial/}
  {basicPhilosophy/}
  {functionsOfSeveralVariables/}
  {normalVectors/}
  {lagrangeMultipliers/}
  {vectorFields/}
  {greensTheorem/}
  {shapeOfThingsToCome/}
  {dotProducts/}
  {../productAndQuotientRules/exercises/}
  {../normalVectors/exercisesParametricPlots/}
  {../continuityOfFunctionsOfSeveralVariables/exercises/}
  {../partialDerivatives/exercises/}
  {../chainRuleForFunctionsOfSeveralVariables/exercises/}
  {../commonCoordinates/exercisesCylindricalCoordinates/}
  {../commonCoordinates/exercisesSphericalCoordinates/}
  {../greensTheorem/exercisesCurlAndLineIntegrals/}
  {../greensTheorem/exercisesDivergenceAndLineIntegrals/}
  {../shapeOfThingsToCome/exercisesDivergenceTheorem/}
  {../greensTheorem/}
  {../shapeOfThingsToCome/}
}

\newcommand{\mooculus}{\textsf{\textbf{MOOC}\textnormal{\textsf{ULUS}}}}

\usepackage{tkz-euclide}\usepackage{tikz}
\usepackage{tikz-cd}
\usetikzlibrary{arrows}
\tikzset{>=stealth,commutative diagrams/.cd,
  arrow style=tikz,diagrams={>=stealth}} %% cool arrow head
\tikzset{shorten <>/.style={ shorten >=#1, shorten <=#1 } } %% allows shorter vectors

\usetikzlibrary{backgrounds} %% for boxes around graphs
\usetikzlibrary{shapes,positioning}  %% Clouds and stars
\usetikzlibrary{matrix} %% for matrix
\usepgfplotslibrary{polar} %% for polar plots
\usepgfplotslibrary{fillbetween} %% to shade area between curves in TikZ
\usetkzobj{all}
%\usepackage[makeroom]{cancel} %% for strike outs
%\usepackage{mathtools} %% for pretty underbrace % Breaks Ximera
%\usepackage{multicol}
\usepackage{pgffor} %% required for integral for loops



%% http://tex.stackexchange.com/questions/66490/drawing-a-tikz-arc-specifying-the-center
%% Draws beach ball
\tikzset{pics/carc/.style args={#1:#2:#3}{code={\draw[pic actions] (#1:#3) arc(#1:#2:#3);}}}



\usepackage{array}
\setlength{\extrarowheight}{+.1cm}   
\newdimen\digitwidth
\settowidth\digitwidth{9}
\def\divrule#1#2{
\noalign{\moveright#1\digitwidth
\vbox{\hrule width#2\digitwidth}}}





\newcommand{\RR}{\mathbb R}
\newcommand{\R}{\mathbb R}
\newcommand{\N}{\mathbb N}
\newcommand{\Z}{\mathbb Z}

\newcommand{\sagemath}{\textsf{SageMath}}


%\renewcommand{\d}{\,d\!}
\renewcommand{\d}{\mathop{}\!d}
\newcommand{\dd}[2][]{\frac{\d #1}{\d #2}}
\newcommand{\pp}[2][]{\frac{\partial #1}{\partial #2}}
\renewcommand{\l}{\ell}
\newcommand{\ddx}{\frac{d}{\d x}}

\newcommand{\zeroOverZero}{\ensuremath{\boldsymbol{\tfrac{0}{0}}}}
\newcommand{\inftyOverInfty}{\ensuremath{\boldsymbol{\tfrac{\infty}{\infty}}}}
\newcommand{\zeroOverInfty}{\ensuremath{\boldsymbol{\tfrac{0}{\infty}}}}
\newcommand{\zeroTimesInfty}{\ensuremath{\small\boldsymbol{0\cdot \infty}}}
\newcommand{\inftyMinusInfty}{\ensuremath{\small\boldsymbol{\infty - \infty}}}
\newcommand{\oneToInfty}{\ensuremath{\boldsymbol{1^\infty}}}
\newcommand{\zeroToZero}{\ensuremath{\boldsymbol{0^0}}}
\newcommand{\inftyToZero}{\ensuremath{\boldsymbol{\infty^0}}}



\newcommand{\numOverZero}{\ensuremath{\boldsymbol{\tfrac{\#}{0}}}}
\newcommand{\dfn}{\textbf}
%\newcommand{\unit}{\,\mathrm}
\newcommand{\unit}{\mathop{}\!\mathrm}
\newcommand{\eval}[1]{\bigg[ #1 \bigg]}
\newcommand{\seq}[1]{\left( #1 \right)}
\renewcommand{\epsilon}{\varepsilon}
\renewcommand{\phi}{\varphi}


\renewcommand{\iff}{\Leftrightarrow}

\DeclareMathOperator{\arccot}{arccot}
\DeclareMathOperator{\arcsec}{arcsec}
\DeclareMathOperator{\arccsc}{arccsc}
\DeclareMathOperator{\si}{Si}
\DeclareMathOperator{\scal}{scal}
\DeclareMathOperator{\sign}{sign}


%% \newcommand{\tightoverset}[2]{% for arrow vec
%%   \mathop{#2}\limits^{\vbox to -.5ex{\kern-0.75ex\hbox{$#1$}\vss}}}
\newcommand{\arrowvec}[1]{{\overset{\rightharpoonup}{#1}}}
%\renewcommand{\vec}[1]{\arrowvec{\mathbf{#1}}}
\renewcommand{\vec}[1]{{\overset{\boldsymbol{\rightharpoonup}}{\mathbf{#1}}}}
\DeclareMathOperator{\proj}{\vec{proj}}
\newcommand{\veci}{{\boldsymbol{\hat{\imath}}}}
\newcommand{\vecj}{{\boldsymbol{\hat{\jmath}}}}
\newcommand{\veck}{{\boldsymbol{\hat{k}}}}
\newcommand{\vecl}{\vec{\boldsymbol{\l}}}
\newcommand{\uvec}[1]{\mathbf{\hat{#1}}}
\newcommand{\utan}{\mathbf{\hat{t}}}
\newcommand{\unormal}{\mathbf{\hat{n}}}
\newcommand{\ubinormal}{\mathbf{\hat{b}}}

\newcommand{\dotp}{\bullet}
\newcommand{\cross}{\boldsymbol\times}
\newcommand{\grad}{\boldsymbol\nabla}
\newcommand{\divergence}{\grad\dotp}
\newcommand{\curl}{\grad\cross}
%\DeclareMathOperator{\divergence}{divergence}
%\DeclareMathOperator{\curl}[1]{\grad\cross #1}
\newcommand{\lto}{\mathop{\longrightarrow\,}\limits}

\renewcommand{\bar}{\overline}

\colorlet{textColor}{black} 
\colorlet{background}{white}
\colorlet{penColor}{blue!50!black} % Color of a curve in a plot
\colorlet{penColor2}{red!50!black}% Color of a curve in a plot
\colorlet{penColor3}{red!50!blue} % Color of a curve in a plot
\colorlet{penColor4}{green!50!black} % Color of a curve in a plot
\colorlet{penColor5}{orange!80!black} % Color of a curve in a plot
\colorlet{penColor6}{yellow!70!black} % Color of a curve in a plot
\colorlet{fill1}{penColor!20} % Color of fill in a plot
\colorlet{fill2}{penColor2!20} % Color of fill in a plot
\colorlet{fillp}{fill1} % Color of positive area
\colorlet{filln}{penColor2!20} % Color of negative area
\colorlet{fill3}{penColor3!20} % Fill
\colorlet{fill4}{penColor4!20} % Fill
\colorlet{fill5}{penColor5!20} % Fill
\colorlet{gridColor}{gray!50} % Color of grid in a plot

\newcommand{\surfaceColor}{violet}
\newcommand{\surfaceColorTwo}{redyellow}
\newcommand{\sliceColor}{greenyellow}




\pgfmathdeclarefunction{gauss}{2}{% gives gaussian
  \pgfmathparse{1/(#2*sqrt(2*pi))*exp(-((x-#1)^2)/(2*#2^2))}%
}


%%%%%%%%%%%%%
%% Vectors
%%%%%%%%%%%%%

%% Simple horiz vectors
\renewcommand{\vector}[1]{\left\langle #1\right\rangle}


%% %% Complex Horiz Vectors with angle brackets
%% \makeatletter
%% \renewcommand{\vector}[2][ , ]{\left\langle%
%%   \def\nextitem{\def\nextitem{#1}}%
%%   \@for \el:=#2\do{\nextitem\el}\right\rangle%
%% }
%% \makeatother

%% %% Vertical Vectors
%% \def\vector#1{\begin{bmatrix}\vecListA#1,,\end{bmatrix}}
%% \def\vecListA#1,{\if,#1,\else #1\cr \expandafter \vecListA \fi}

%%%%%%%%%%%%%
%% End of vectors
%%%%%%%%%%%%%

%\newcommand{\fullwidth}{}
%\newcommand{\normalwidth}{}



%% makes a snazzy t-chart for evaluating functions
%\newenvironment{tchart}{\rowcolors{2}{}{background!90!textColor}\array}{\endarray}

%%This is to help with formatting on future title pages.
\newenvironment{sectionOutcomes}{}{} 



%% Flowchart stuff
%\tikzstyle{startstop} = [rectangle, rounded corners, minimum width=3cm, minimum height=1cm,text centered, draw=black]
%\tikzstyle{question} = [rectangle, minimum width=3cm, minimum height=1cm, text centered, draw=black]
%\tikzstyle{decision} = [trapezium, trapezium left angle=70, trapezium right angle=110, minimum width=3cm, minimum height=1cm, text centered, draw=black]
%\tikzstyle{question} = [rectangle, rounded corners, minimum width=3cm, minimum height=1cm,text centered, draw=black]
%\tikzstyle{process} = [rectangle, minimum width=3cm, minimum height=1cm, text centered, draw=black]
%\tikzstyle{decision} = [trapezium, trapezium left angle=70, trapezium right angle=110, minimum width=3cm, minimum height=1cm, text centered, draw=black]




\author{Tom Needham and Jim Talamo}

\outcome{Recall basic rules and techniques of integration.}

\title[]{A Review of Integration}

\begin{document}
\begin{abstract}
\end{abstract}
\maketitle

\vspace{-0.9in}

\section{Discussion Questions}

\begin{problem}
Classify each of the following expressions as a number, a function of $t$, a function of $x$ or none of the above.


\begin{tabular}{llll}
I. $\frac{\d}{\d t} \int_0^t e^{x^2} \d x$ \hspace{0.2in} & II. $\int e^{x^2} \d x$ \hspace{0.2in} & III. $\int_0^2 e^{x^2} \d x$ \hspace{0.2in}
IV. $\frac{d}{dx} \int_0^9 e^{x^2} \d x$
\end{tabular}

\end{problem}

\begin{freeResponse}
 I. By the Fundamental Theorem of Calculus, 
$$
\frac{\d}{\d t} \int_0^t e^{x^2} \d x = e^{t^2},
$$
so this expression represents a function of $t$.

II. The expression $\int e^{x^2} \d x$ denotes the indefinite integral of the function $f(x)=e^{x^2}$. The indefinite integral of a function the family of \textit{all} antiderivatives of the function. Therefore $\int e^{x^2} \d x$ is a family of functions of $x$, and the answer to the question is ``none of the above".

III. The definite integral $\int_0^2 e^{x^2} \d x$ is a number. In particular, it is the signed area abounded by the curve $y=e^{x^2}$ and the $x$-axis over the interval $[0,2]$. 

IV. The expression $\frac{d}{dx} \int_0^9 e^{x^2} \d x$ also represents a number. In particular, it is equal to zero, since $\int_0^9 e^{x^2} \d x$ is itself a number, and the derivative of a constant function is always zero.
\end{freeResponse}

\begin{problem}
Being able to compute antiderivatives of functions quickly is an important skill to develop.  Compute the following indefinite integrals by modifying the basic antiderivative rules.  Try not to use a $u$-substitution.

\begin{center}
\begin{tabular}{lll} 
I. $\int \cos(2x) \d x$ \hspace{.7in} & II. $\int e^{x/2} \d x$ \hspace{.7in} & III. $\int (2x+1)^3 \d x$
\end{tabular}
\end{center}

\end{problem}

\begin{freeResponse} By slight modifications of basic antiderivative rules for trigonometric functions, exponential functions, and polynomial functions, respectively, we get the answers:

\begin{tabular}{lll} 
I. $\frac{1}{2} \sin(2x) + C$ \hspace{.7in} & II. $2 e^{x/2} + C$ \hspace{.7in} & III. $\frac{1}{8} (2x+1)^4 + C$
\end{tabular}
\end{freeResponse}

\begin{problem}
Consider the indefinite integral $\int \frac{1}{3x} \d x$. Student A calculates the integral using basic antiderivative rules as
$$
\int \frac{1}{3x} \d x = \frac{1}{3} \ln |3x| + C.
$$
Student B calculates the integral by moving the constant to the front as 
$$
\int \frac{1}{3x} \d x = \frac{1}{3} \int \frac{1}{x} \d x = \frac{1}{3} \ln |x| + C.
$$
Is either student incorrect?  If so, explain the faulty logic in the incorrect student's solution.  If both students are correct, how can their answers be rectified?

\end{problem}

\begin{freeResponse} Both students are correct. To prove this, we can calculate derivatives to get
$$
\frac{\d}{\d x} \left(\frac{1}{3} \ln |3x| + C\right) = \frac{1}{3} \cdot  \frac{3}{3x} = \frac{1}{3x}
$$
and 
$$
\frac{\d}{\d x} \left(\frac{1}{3} \ln |x| + C\right) = \frac{1}{3} \cdot \frac{1}{x} = \frac{1}{3x},
$$
respectively. To reconcile the apparently different solutions, note that Student A's answer can be simplified as
$$
\frac{1}{3} \ln |3x| + C = \frac{1}{3} \left(\ln 3 + \ln |x|\right) + C = \frac{1}{3} \ln |x| + \frac{\ln 3}{3} + C.
$$
It follows that the answers given by Student A and Student B only differ by the addition of a constant number. Since $C$ represents an \textit{arbitrary} constant in each solution, the students' responses are actually equivalent.
\end{freeResponse}

\section{Group Work}

\begin{problem}
Consider the integral $\int_0^4  6x\sqrt{x^2+9} \d x$.  There are common approaches to evaluating this.

\begin{itemize}
\item[I.] Use a substitution to show that:

\[\int  6x\sqrt{x^2+9} \d x = 2 (x^2 + 9)^{3/2} + C.\]  
Then, use the Fundamental Theorem on Calculus to evaluate $\int_0^4  6x\sqrt{x^2+9} \d x.$ 

\item[II.] Make the substitution $u=x^2+9$ and show that:

\[\int_{x=0}^{x=4}  6x\sqrt{x^2+9} \d x = \int_{u=9}^{u=25} 3\sqrt{u}.\]  Then, evaluate the integral in $u$ by using the Fundamental Theorem of Calculus.
\end{itemize}
\end{problem}

\begin{freeResponse} I. Taking $u=x^2 + 9$, we have $\d u = 2 x \d x$, and it follows that
$$
\int  6x\sqrt{x^2+9} \d x = 3 \int \sqrt{u} \d u = 3 \frac{u^{3/2}}{3/2} + C = 2 (x^2 + 9)^{3/2} + C. 
$$
The integral can then be evaluated as 
$$
\int_0^4  6x\sqrt{x^2+9} \d x = \eval{2 (x^2 + 9)^{3/2}}_0^4 = 2 \cdot 25^{3/2} - 2 \cdot 9^{3/2} = 250 - 54 = 196.
$$

II. We once again use the substitution $u=x^2 + 9$, but we now change the limits of integration to $u=0^2+9=9$ and $u=4^2+9 = 25$, respectively, to obtain
$$
\int_0^4  6x\sqrt{x^2+9} \d x = 3 \int_9^{25} \sqrt{u} \d u = \eval{3 \frac{2}{3} u^{3/2}}_9^{25} = 2 \cdot 25^{3/2} - 2 \cdot 9^{3/2} = 196.
$$
\end{freeResponse}

\begin{problem}
Calculate the following integrals.

\begin{tabular}{lll}
I.  $\int \left(\sqrt{y}+1\right)^2 \d y$ \hspace{.5in} & II. $\int_0^{\sqrt{\ln(2)}} x e^{x^2} \d x$ \hspace{.5in} & III. $\int \frac{2x+1}{x^2+4} \d x$ \hspace{.05in}
\end{tabular}

\end{problem}

\begin{freeResponse} I. First using algebra to simplify, we have
$$
\int \left(\sqrt{y}+1\right)^2 \d y = \int y + 2 \sqrt{y} + 1 \d y = \frac{1}{2}y^2 + \frac{4}{3} y^{3/2} + y + C.
$$

II. Let $u = x^2$, so that $\d u = 2 x \d x$ and 
$$
\int_0^{\sqrt{\ln(2)}} x e^{x^2} \d x = \frac{1}{2} \int_0 ^{\ln 2} e^u \d u = \eval{\frac{1}{2} e^u}_0^{\ln 2} = 1- \frac{1}{2} = \frac{1}{2}.
$$

III. We first split into two integrals to obtain 
$$
\int \frac{2x+1}{x^2+4} \d x = \int \frac{2x}{x^2 + 4} \d x + \int \frac{1}{x^2 + 4} \d x.
$$
The first integral is easily handled by the substitution $u = x^2 + 4$, whence $\d u = 2 x \d x$ and 
$$
\int \frac{2x}{x^2 + 4} \d x = \int \frac{1}{u} \d u = \ln |u| + C = \ln (x^2 + 4) + C.
$$
The second integral can be computed by basic antiderivative rules as 
$$
\int \frac{1}{x^2 + 4} \d x = \frac{1}{2} \arctan \left(\frac{x}{2}\right) + C.
$$
We then add integrals to get the final answer
$$
\int \frac{2x+1}{x^2+4} \d x = \ln (x^2 + 4) + \frac{1}{2} \arctan \left(\frac{x}{2}\right) + C.
$$
Note that, because $C$ represents an arbitrary constant, we only need to include it once in the final answer.
\end{freeResponse}

\begin{problem}
Find a function $f(x)$ satisfying each of the given condition(s).

$$
\begin{array}{lll}
\mathrm{I.} \; \int f(x) \d x = 2x \sqrt{1+ x^2} + C & \hspace{.2in} & \mathrm{II.} \; \int xf(x) \d x = x e^x + C \\
\mathrm{III.} \; \frac{\d}{\d t} \int_0^t f(x) \d x = t e^t & \hspace{.2in} & \mathrm{IV.} \; f'(x) = 3x^2 \mbox{ and } f(0)=2
\end{array}
$$
\end{problem}

\begin{freeResponse} I. By definition, we have
$$
f(x) = \frac{\d}{\d x}  2x \sqrt{1+ x^2}  = 2 \sqrt{1+x^2} + x  (1+x^2)^{-1/2}.
$$

II. Once again, using the definition of an indefinite integral, 
$$
x f(x) = \frac{\d}{\d x} x e^x = e^x + x e^x = (1+x)e^x,
$$
so that 
$$
f(x) = \frac{1+x}{x} e^x.
$$

III. Combining the required condition with the Fundamental Theorem of Calculus, we have
$$
te^t = \frac{\d}{\d t} \int_0^t f(x) \d x = f(t).
$$
It follows that $f(x) = x e^x$.

IV. Any $f(x)$ satisfying $f'(x) = 3x^2$ must be of the form $f(x) = x^3 + C$, for some constant $C$. The second condition yields $f(0)=0^3 + C = 2$, so that the particular function we are looking for is $f(x) = x^3 + 2$. 
\end{freeResponse}

\begin{problem}
A student calculates an indefinite integral as follows:
$$
\int \frac{5}{x^2 + 9} \d x = 5\ln (x^2 + 9) + C.
$$
Is the student correct? First explain your answer without actually calculating the antiderivative. Then, if the student is incorrect, calculate the correct antiderivative.
\end{problem}

\begin{freeResponse} The student is not correct, because 
$$
\frac{\d}{\d x} 5 \ln (x^2 + 9) = \frac{5}{x^2 + 9} \cdot 2x,
$$
by the Chain Rule. To see that this function is not equal to the integrand (i.e., it's not possible to algebraically manipulate one to obtain the other), just evaluate both at $x=1$. We have 
$$
\eval{\frac{5}{x^2 + 9}}_{x=1} = \frac{1}{2},
$$
while 
$$
\eval{\frac{5}{x^2 + 9} \cdot 2x}_{x=1} = 1.
$$
Therefore the functions cannot be equal.

The correct answer is obtained by basic integration rules as
$$
\int \frac{5}{x^2 + 9} \d x = 5 \int \frac{1}{x^2 + 9} \d x = \frac{5}{3} \arctan \left(\frac{x}{3}\right) + C.
$$
\end{freeResponse}

\begin{problem}
The figure below shows a plot of a function $y = f(x)$. On the axes below, draw representations of 


\begin{tabular}{ll}
I. $\int f(x) \d x$ \hspace{0.4in} & II. $\int_{-2}^2 f(x) \d x$\\
\resizebox {6cm} {!} { 
          \begin{tikzpicture}
	    \begin{axis}[
            domain=-3:3,
            xmin=-3, xmax=3,
            ymin=-2, ymax=6
         ,
            axis lines =middle, xlabel=$x$, ylabel=$y$,
            every axis y label/.style={at=(current axis.above origin),anchor=south},
            every axis x label/.style={at=(current axis.right of origin),anchor=west},
          ]
	  \addplot [very thick, penColor, smooth] {x^2};
        
        \end{axis}
\end{tikzpicture}
%% \caption{A plot of $f(x)=x^2$ and $f^{-1}(x) = \sqrt{x}$. While
%%   $f(x)=x^2$ is not one-to-one on $\RR$, it is one-to-one on
%%   $[0,\infty)$.}
%% \label{plot:fxn and inverse x^2}
}
  &
\resizebox {6cm} {!} { 
          \begin{tikzpicture}
	    \begin{axis}[
            domain=-3:3,
            xmin=-3, xmax=3,
            ymin=-2, ymax=6
         ,
            axis lines =middle, xlabel=$x$, ylabel=$y$,
            every axis y label/.style={at=(current axis.above origin),anchor=south},
            every axis x label/.style={at=(current axis.right of origin),anchor=west},
          ]
	  \addplot [very thick, penColor, smooth] {x^2};
        
        \end{axis}
\end{tikzpicture}
%% \caption{A plot of $f(x)=x^2$ and $f^{-1}(x) = \sqrt{x}$. While
%%   $f(x)=x^2$ is not one-to-one on $\RR$, it is one-to-one on
%%   $[0,\infty)$.}
%% \label{plot:fxn and inverse x^2}
}

\end{tabular}

\end{problem}

\begin{freeResponse}

I. The expression $\int f(x) \d x$ represents the family of all antiderivatives of the function $f(x)$. Any two antiderivatives of $f(x)$ differ by addition of a constant, which corresponds graphically to a vertical shift. Several antiderivatives of $f(x)$ are drawn in the figure below.

\begin{center}
\resizebox {6cm} {!} { 
          \begin{tikzpicture}
	    \begin{axis}[
            domain=-3:3,
            xmin=-3, xmax=3,
            ymin=-2, ymax=6
         ,
            axis lines =middle, xlabel=$x$, ylabel=$y$,
            every axis y label/.style={at=(current axis.above origin),anchor=south},
            every axis x label/.style={at=(current axis.right of origin),anchor=west},
          ]
	  \addplot [very thick, penColor, smooth] {x^2};
	  \addplot [very thick, penColor2, smooth] {x^3/3};
	  \addplot [very thick, penColor2, smooth] {x^3/3 + 1.5};
	  \addplot [very thick, penColor2, smooth] {x^3/3+3};
	   \addplot [very thick, penColor2, smooth] {x^3/3+4.5};
	    \addplot [very thick, penColor2, smooth] {x^3/3+6};
	  \addplot [very thick, penColor2, smooth] {x^3/3-1.5};
        
        \end{axis}
\end{tikzpicture}
%% \caption{A plot of $f(x)=x^2$ and $f^{-1}(x) = \sqrt{x}$. While
%%   $f(x)=x^2$ is not one-to-one on $\RR$, it is one-to-one on
%%   $[0,\infty)$.}
%% \label{plot:fxn and inverse x^2}
}
\end{center}

II. The expression $\int_{-2}^2 f(x) \d x$ represents the area of the shaded region in the figure below. 

\begin{center}
\resizebox {6cm} {!} { 
          \begin{tikzpicture}
          
	    \begin{axis}[
            domain=-3:3,
            xmin=-3, xmax=3,
            ymin=-2, ymax=6
         ,
            axis lines =middle, xlabel=$x$, ylabel=$y$,
            every axis y label/.style={at=(current axis.above origin),anchor=south},
            every axis x label/.style={at=(current axis.right of origin),anchor=west},
          ]
	  \addplot [draw=none,fill=fillp,domain=-2:2, smooth] {x^2} \closedcycle;
	  \addplot [very thick, penColor, smooth] {x^2};
	  
        
        \end{axis}
\end{tikzpicture}
%% \caption{A plot of $f(x)=x^2$ and $f^{-1}(x) = \sqrt{x}$. While
%%   $f(x)=x^2$ is not one-to-one on $\RR$, it is one-to-one on
%%   $[0,\infty)$.}
%% \label{plot:fxn and inverse x^2}
}
\end{center}
\end{freeResponse}


\begin{problem}
(Multiselect)

Circle all of the statements below that are true.  Make sure you understand the reasoning behind each response.

\begin{itemize}
\item[I.] $\displaystyle \int \dfrac{2}{3+4x} \, dx = 2\ln|3+4x|+C$

\item[II.] $\displaystyle \int 4xe^{2x} \, dx = 2xe^{2x}-e^{2x}+C$

\item[III.] The Fundamental Theorem of Calculus can be applied directly to find $ \int_0^2 \frac{2x}{x^2-1} \d x$.

\item[IV.] To evaluate $ \int_0^1 2x(x^2+1)^2 \d x$, set $u=x^2+1$.  Then. $\d u = 2x \d x$ so:
\[ \int_0^1 2x(x^2+1)^2 \d x = \int_0^1 u^2 \d u = \eval{\frac{1}{3} u^3}_0^1 = \frac{1}{3}\]
\end{itemize}

\end{problem}

\begin{freeResponse} I. This is false, because
$$
\frac{\d}{\d x} 2 \ln |3+4x| = \frac{2}{3+4x} \cdot 4 = \frac{8}{3+4x},
$$
which is not equal to $\frac{2}{3+4x}$. 

II. This is true, because
$$
\frac{\d}{\d x} \left( 2xe^{2x}-e^{2x}\right) = 2 e^{2x} + 4x e^{2x} - 2 e^{2x} = 4x e^{2x}.
$$

III. This is false, because the Fundamental Theorem of Calculus applies to functions which are continuous over the domain of integration. The integrand in this case is not even defined at $x=1$. 

IV. This is false. To find the correct solution, we would either need to convert the limits of integration to be in terms of $u$, or convert back to the the original variable $x$ before evaluation (see Problem 4). For example, a correct solution using the substitution $u = x^2 +1 $ would be 
$$
\int_0^1 2x(x^2+1)^2 \d x = \int_0^2 u^2 \d u = \eval{\frac{1}{3} u^3}_0^2 = \frac{8}{3}.
$$
\end{freeResponse}
\end{document}
