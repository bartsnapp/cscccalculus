\documentclass[noauthor]{ximera}
%handout:  for handout version with no solutions or instructor notes
%handout,instructornotes:  for instructor version with just problems and notes, no solutions
%noinstructornotes:  shows only problem and solutions

%% handout
%% space
%% newpage
%% numbers
%% nooutcomes

%I added the commands here so that I would't have to keep looking them up
%\newcommand{\RR}{\mathbb R}
%\renewcommand{\d}{\,d}
%\newcommand{\dd}[2][]{\frac{d #1}{d #2}}
%\renewcommand{\l}{\ell}
%\newcommand{\ddx}{\frac{d}{dx}}
%\everymath{\displaystyle}
%\newcommand{\dfn}{\textbf}
%\newcommand{\eval}[1]{\bigg[ #1 \bigg]}

%\begin{image}
%\includegraphics[trim= 170 420 250 180]{Figure1.pdf}
%\end{image}

%add a ``.'' below when used in a specific directory.

%\usepackage{todonotes}

\newcommand{\todo}{}

\usepackage{esint} % for \oiint
\ifxake%%https://math.meta.stackexchange.com/questions/9973/how-do-you-render-a-closed-surface-double-integral
\renewcommand{\oiint}{{\large\bigcirc}\kern-1.56em\iint}
\fi


\graphicspath{
  {./}
  {ximeraTutorial/}
  {basicPhilosophy/}
  {functionsOfSeveralVariables/}
  {normalVectors/}
  {lagrangeMultipliers/}
  {vectorFields/}
  {greensTheorem/}
  {shapeOfThingsToCome/}
  {dotProducts/}
  {../productAndQuotientRules/exercises/}
  {../normalVectors/exercisesParametricPlots/}
  {../continuityOfFunctionsOfSeveralVariables/exercises/}
  {../partialDerivatives/exercises/}
  {../chainRuleForFunctionsOfSeveralVariables/exercises/}
  {../commonCoordinates/exercisesCylindricalCoordinates/}
  {../commonCoordinates/exercisesSphericalCoordinates/}
  {../greensTheorem/exercisesCurlAndLineIntegrals/}
  {../greensTheorem/exercisesDivergenceAndLineIntegrals/}
  {../shapeOfThingsToCome/exercisesDivergenceTheorem/}
  {../greensTheorem/}
  {../shapeOfThingsToCome/}
}

\newcommand{\mooculus}{\textsf{\textbf{MOOC}\textnormal{\textsf{ULUS}}}}

\usepackage{tkz-euclide}\usepackage{tikz}
\usepackage{tikz-cd}
\usetikzlibrary{arrows}
\tikzset{>=stealth,commutative diagrams/.cd,
  arrow style=tikz,diagrams={>=stealth}} %% cool arrow head
\tikzset{shorten <>/.style={ shorten >=#1, shorten <=#1 } } %% allows shorter vectors

\usetikzlibrary{backgrounds} %% for boxes around graphs
\usetikzlibrary{shapes,positioning}  %% Clouds and stars
\usetikzlibrary{matrix} %% for matrix
\usepgfplotslibrary{polar} %% for polar plots
\usepgfplotslibrary{fillbetween} %% to shade area between curves in TikZ
\usetkzobj{all}
%\usepackage[makeroom]{cancel} %% for strike outs
%\usepackage{mathtools} %% for pretty underbrace % Breaks Ximera
%\usepackage{multicol}
\usepackage{pgffor} %% required for integral for loops



%% http://tex.stackexchange.com/questions/66490/drawing-a-tikz-arc-specifying-the-center
%% Draws beach ball
\tikzset{pics/carc/.style args={#1:#2:#3}{code={\draw[pic actions] (#1:#3) arc(#1:#2:#3);}}}



\usepackage{array}
\setlength{\extrarowheight}{+.1cm}   
\newdimen\digitwidth
\settowidth\digitwidth{9}
\def\divrule#1#2{
\noalign{\moveright#1\digitwidth
\vbox{\hrule width#2\digitwidth}}}





\newcommand{\RR}{\mathbb R}
\newcommand{\R}{\mathbb R}
\newcommand{\N}{\mathbb N}
\newcommand{\Z}{\mathbb Z}

\newcommand{\sagemath}{\textsf{SageMath}}


%\renewcommand{\d}{\,d\!}
\renewcommand{\d}{\mathop{}\!d}
\newcommand{\dd}[2][]{\frac{\d #1}{\d #2}}
\newcommand{\pp}[2][]{\frac{\partial #1}{\partial #2}}
\renewcommand{\l}{\ell}
\newcommand{\ddx}{\frac{d}{\d x}}

\newcommand{\zeroOverZero}{\ensuremath{\boldsymbol{\tfrac{0}{0}}}}
\newcommand{\inftyOverInfty}{\ensuremath{\boldsymbol{\tfrac{\infty}{\infty}}}}
\newcommand{\zeroOverInfty}{\ensuremath{\boldsymbol{\tfrac{0}{\infty}}}}
\newcommand{\zeroTimesInfty}{\ensuremath{\small\boldsymbol{0\cdot \infty}}}
\newcommand{\inftyMinusInfty}{\ensuremath{\small\boldsymbol{\infty - \infty}}}
\newcommand{\oneToInfty}{\ensuremath{\boldsymbol{1^\infty}}}
\newcommand{\zeroToZero}{\ensuremath{\boldsymbol{0^0}}}
\newcommand{\inftyToZero}{\ensuremath{\boldsymbol{\infty^0}}}



\newcommand{\numOverZero}{\ensuremath{\boldsymbol{\tfrac{\#}{0}}}}
\newcommand{\dfn}{\textbf}
%\newcommand{\unit}{\,\mathrm}
\newcommand{\unit}{\mathop{}\!\mathrm}
\newcommand{\eval}[1]{\bigg[ #1 \bigg]}
\newcommand{\seq}[1]{\left( #1 \right)}
\renewcommand{\epsilon}{\varepsilon}
\renewcommand{\phi}{\varphi}


\renewcommand{\iff}{\Leftrightarrow}

\DeclareMathOperator{\arccot}{arccot}
\DeclareMathOperator{\arcsec}{arcsec}
\DeclareMathOperator{\arccsc}{arccsc}
\DeclareMathOperator{\si}{Si}
\DeclareMathOperator{\scal}{scal}
\DeclareMathOperator{\sign}{sign}


%% \newcommand{\tightoverset}[2]{% for arrow vec
%%   \mathop{#2}\limits^{\vbox to -.5ex{\kern-0.75ex\hbox{$#1$}\vss}}}
\newcommand{\arrowvec}[1]{{\overset{\rightharpoonup}{#1}}}
%\renewcommand{\vec}[1]{\arrowvec{\mathbf{#1}}}
\renewcommand{\vec}[1]{{\overset{\boldsymbol{\rightharpoonup}}{\mathbf{#1}}}}
\DeclareMathOperator{\proj}{\vec{proj}}
\newcommand{\veci}{{\boldsymbol{\hat{\imath}}}}
\newcommand{\vecj}{{\boldsymbol{\hat{\jmath}}}}
\newcommand{\veck}{{\boldsymbol{\hat{k}}}}
\newcommand{\vecl}{\vec{\boldsymbol{\l}}}
\newcommand{\uvec}[1]{\mathbf{\hat{#1}}}
\newcommand{\utan}{\mathbf{\hat{t}}}
\newcommand{\unormal}{\mathbf{\hat{n}}}
\newcommand{\ubinormal}{\mathbf{\hat{b}}}

\newcommand{\dotp}{\bullet}
\newcommand{\cross}{\boldsymbol\times}
\newcommand{\grad}{\boldsymbol\nabla}
\newcommand{\divergence}{\grad\dotp}
\newcommand{\curl}{\grad\cross}
%\DeclareMathOperator{\divergence}{divergence}
%\DeclareMathOperator{\curl}[1]{\grad\cross #1}
\newcommand{\lto}{\mathop{\longrightarrow\,}\limits}

\renewcommand{\bar}{\overline}

\colorlet{textColor}{black} 
\colorlet{background}{white}
\colorlet{penColor}{blue!50!black} % Color of a curve in a plot
\colorlet{penColor2}{red!50!black}% Color of a curve in a plot
\colorlet{penColor3}{red!50!blue} % Color of a curve in a plot
\colorlet{penColor4}{green!50!black} % Color of a curve in a plot
\colorlet{penColor5}{orange!80!black} % Color of a curve in a plot
\colorlet{penColor6}{yellow!70!black} % Color of a curve in a plot
\colorlet{fill1}{penColor!20} % Color of fill in a plot
\colorlet{fill2}{penColor2!20} % Color of fill in a plot
\colorlet{fillp}{fill1} % Color of positive area
\colorlet{filln}{penColor2!20} % Color of negative area
\colorlet{fill3}{penColor3!20} % Fill
\colorlet{fill4}{penColor4!20} % Fill
\colorlet{fill5}{penColor5!20} % Fill
\colorlet{gridColor}{gray!50} % Color of grid in a plot

\newcommand{\surfaceColor}{violet}
\newcommand{\surfaceColorTwo}{redyellow}
\newcommand{\sliceColor}{greenyellow}




\pgfmathdeclarefunction{gauss}{2}{% gives gaussian
  \pgfmathparse{1/(#2*sqrt(2*pi))*exp(-((x-#1)^2)/(2*#2^2))}%
}


%%%%%%%%%%%%%
%% Vectors
%%%%%%%%%%%%%

%% Simple horiz vectors
\renewcommand{\vector}[1]{\left\langle #1\right\rangle}


%% %% Complex Horiz Vectors with angle brackets
%% \makeatletter
%% \renewcommand{\vector}[2][ , ]{\left\langle%
%%   \def\nextitem{\def\nextitem{#1}}%
%%   \@for \el:=#2\do{\nextitem\el}\right\rangle%
%% }
%% \makeatother

%% %% Vertical Vectors
%% \def\vector#1{\begin{bmatrix}\vecListA#1,,\end{bmatrix}}
%% \def\vecListA#1,{\if,#1,\else #1\cr \expandafter \vecListA \fi}

%%%%%%%%%%%%%
%% End of vectors
%%%%%%%%%%%%%

%\newcommand{\fullwidth}{}
%\newcommand{\normalwidth}{}



%% makes a snazzy t-chart for evaluating functions
%\newenvironment{tchart}{\rowcolors{2}{}{background!90!textColor}\array}{\endarray}

%%This is to help with formatting on future title pages.
\newenvironment{sectionOutcomes}{}{} 



%% Flowchart stuff
%\tikzstyle{startstop} = [rectangle, rounded corners, minimum width=3cm, minimum height=1cm,text centered, draw=black]
%\tikzstyle{question} = [rectangle, minimum width=3cm, minimum height=1cm, text centered, draw=black]
%\tikzstyle{decision} = [trapezium, trapezium left angle=70, trapezium right angle=110, minimum width=3cm, minimum height=1cm, text centered, draw=black]
%\tikzstyle{question} = [rectangle, rounded corners, minimum width=3cm, minimum height=1cm,text centered, draw=black]
%\tikzstyle{process} = [rectangle, minimum width=3cm, minimum height=1cm, text centered, draw=black]
%\tikzstyle{decision} = [trapezium, trapezium left angle=70, trapezium right angle=110, minimum width=3cm, minimum height=1cm, text centered, draw=black]




\author{Tom Needham and Jim Talamo}

\outcome{Use the technique of trigonometric substitution to compute integrals.}

\title[]{Trigonometric Substitution}

\begin{document}
\begin{abstract}
\end{abstract}
\maketitle

\vspace{-0.45in}

\section{Discussion Questions}

\begin{problem}
Discuss an effective strategy to evaluate each of the following integrals.

\begin{tabular}{lll}
I. $\int x\sqrt{4-x^2} \d x$ \hspace{5mm} & II. $\int \frac{2x^3}{\sqrt{x^2-1}}  \d x$ \hspace{5mm} & III. $\int 2xe^{x}+\frac{1}{(25-x^2)^{3/2}} \d x$ 
\end{tabular}

\end{problem}

\begin{freeResponse}

I. While a trigonometric substitution can be used here, it is easier to use the $u$-substitution $u=4-x^2$ since a multiple of $\d u$ appears in front of the radical.

While you are not asked to perform the integration, the result will be

\[
\int x\sqrt{4-x^2} \d x = -\frac{1}{3}(4-x^2)^{3/2}+C
\]

II. This can be done either with the $u$-substitution $u= x^2-1$ or by the trigonometric substitution $x=\sec(\theta)$.  If you choose the former, you will have to replace $x^2$ with an appropriate function of $u$.

While you are not asked to perform the integration, the result will be

\[
\int  \frac{2x^3}{\sqrt{x^2-1}} \d x = 2\sqrt{x^2-1}+\frac{2}{3}(x^2-1)^{3/2}+C.
\]

III. The integral should be split up as

\[
\int 2xe^{x}+\frac{1}{(25-x^2)^{3/2}} \d x = \int 2xe^{x} \d x+ \int \frac{1}{(25-x^2)^{3/2}} \d x.
\]

The first integral can be done with integration by parts.  The second can be done via the trig substitution $u=5 \sec(\theta)$.

While you are not asked to perform the integration, the result will be

\begin{align*}
\int 2xe^{x}+\frac{1}{(25-x^2)^{3/2}} \d x &= \int 2xe^{x} \d x+ \int \frac{1}{(25-x^2)^{3/2}} \d x \\
&= 2xe^x-2e^x+\frac{x}{25\sqrt{25-x^2}} +C.
\end{align*}

\end{freeResponse}


\begin{problem}
Two students are asked to evaluate $\int \sqrt{x^2-4} \d x$.

\item[I.] One student claims that
\[
\int \sqrt{x^2-4} \d x = \int x-2 \d x = \frac{1}{2}x^2 - 2x + C.
\]
Is this student correct? If not, determine a likely error that the student made in the calculation.

\item[II.] The other student claims that the substitution $x = 2 \sec(\theta)$ will be helpful and notes
\[
\int \sqrt{x^2-4} \d x =\int \sqrt{4 \sec^2(\theta)-4} \d \theta = \int \sqrt{4 \tan^2(\theta)} \d \theta =  \int \tan(\theta) \d \theta
\]
Is this student's solution correct so far? If not, determine a likely error that the student made in the calculation.
\end{problem}

\begin{freeResponse}
I. The first student is not correct. The student's mistake is in the claim that $\sqrt{x^2-4}=x-2$. This claim is clearly false; the two functions do not even have the same domain and they evaluate to different values for most inputs in their common domain.

II. The second student is not correct either.  The substitution $x = 2\sec(\theta)$ is the correct one to make, but the student forgot to write $\d x$ in terms of $\theta$ and $\d \theta$.

\end{freeResponse}

\begin{problem}
Each of the following integrals can be evaluated using trigonometric substitution. For each example, determine which trigonometric substitution should be used.

\begin{center}
\begin{tabular}{lll}
I. $\int x^3 \sqrt{4-x^2} \d x $ \hspace{.2in} II.$\int \frac{x^3}{\sqrt{x^2+16}} \d x$ \hspace{.2in} III. $\int \frac{1}{x^2 \sqrt{x^2-2}} \d x$
\end{tabular}
\end{center}
\end{problem}

\begin{freeResponse}
Trigonometric substitution is based on the Pythagorean identities:
$$
\sin^2(\theta)  = 1 - \cos^2(\theta) \;\; \mbox{ and } \;\;  1 + \tan^2(\theta) = \sec^2(\theta).
$$
To choose the correct substitution, one applies the relevant identity.

I. Let $x = 2 \sin(\theta)$, so that the first Pythagorean identity can be applied.

II. Let $x=4 \tan(\theta)$, so that the second Pythagorean identity can be applied.

III. Let $x=\sqrt{2}\sec(\theta)$, so that the second Pythagorean identity can be applied.
\end{freeResponse}

\begin{problem}
\begin{enumerate}
\item[I.] Given that $\sin( \theta) = \frac{2}{5}$ and that $\theta$ lies in the second quadrant, determine the value of $\cos(\theta)$.
\item[II.] Given that $\tan (\theta) = -  x$, for some value $x$, and that $\theta$ lies in the fourth quadrant, determine the values of $\cos(\theta)$ and $\csc(\theta)$ in terms of $x$.
\end{enumerate}
\end{problem}

\begin{freeResponse}
I. A reference triangle for $\theta$ has hypotenuse $5$ and opposite side $2$. The adjacent side therefore has length $\sqrt{5^2-2^2} = \sqrt{21}$. Since $\theta$ lies in the second quadrant, cosine must be negative, and we conclude that $\cos\theta = -\frac{\sqrt{21}}{5}$.

II. A reference triangle for $\theta$ has opposite side length $x$ and adjacent side length $1$. The hypotenuse length is therefore $\sqrt{x^2+1}$. Since $\theta$ lies in the fourth quadrant, $\cos(\theta)$ is positive and $\csc(\theta)$ is negative. Therefore
$$
\cos(\theta) = \frac{1}{\sqrt{x^2+1}} \;\; \mbox{ and } \;\; \csc(\theta) = -\frac{\sqrt{x^2+1}}{x}.
$$
\end{freeResponse}

\begin{problem}
For $x>2$, the trigonometric substitution $x = 2 \sec(\theta)$ gives the result: $$\int \dfrac{4}{x^2\sqrt{x^2-4}} \, dx = \int  \cos(\theta) \, d \theta.$$
Evaluate $\displaystyle \int \dfrac{4}{x^2\sqrt{x^2-4}} \, dx$.

\vspace{2mm}
\begin{tabular}{lll}
A.  $\dfrac{\sqrt{x^2-4}}{x}+C$    \hspace{8mm} & B. $-\dfrac{\sqrt{x^2-4}}{x}+C$   \hspace{8mm} & C.  $4 \ln\bigg(x^2\sqrt{x^2-4}\bigg) +C$  \\[3ex]
D. $\dfrac{x}{\sqrt{x^2-4}}+C$   \hspace{8mm} & E.  $-\dfrac{x}{\sqrt{x^2-4}}+C$ \hspace{8mm} & F. None of these\\  [2 ex] 
\end{tabular}
\end{problem}

\begin{freeResponse}
The answer is A. We use the given information to write
\begin{align*}
\displaystyle \int \dfrac{4}{x^2\sqrt{x^2-4}} \d x &=  \int \dfrac{4}{x^2\sqrt{x^2-4}} \d x \\
&=  \int \cos(\theta) \d \theta \\
&=  \sin(\theta) + C \\
&=  \frac{\sqrt{x^2-4}}{x} + C.
\end{align*}
The last expression is found using a reference triangle for $\sec(\theta) = \frac{x}{2}$. 
\end{freeResponse}



\section{Group Work}



\begin{problem}
Evaluate the following integrals using trigonometric substitution.
\begin{center}
\begin{tabular}{lll}
I. $\int_{\sqrt{2}}^2 \frac{1}{t^3 \sqrt{t^2 -1}} \d t$ \hspace{.2in} II. $\int \frac{27 x^2}{(4+9x^2)^{3/2}} \d x$ \hspace{.2in} III. $\int \frac{x^2}{\sqrt{4x-x^2}} \d x$
\end{tabular}
\end{center}
\end{problem}

\begin{freeResponse}
I. The substitution $t = \sec(\theta)$ will be helpful.  Note first that $\d t = \sec(\theta) \tan(\theta) \d \theta$.  When we have definite integrals that require a substitution, there are two ways to proceed.

\paragraph{Way 1:} Compute the antiderivatives $\frac{1}{t^3 \sqrt{t^2 -1}}$ then apply the Fundamental Theorem of Calculus.
\begin{align*}
\int \frac{1}{t^3\sqrt{t^2-1}} \d t &= \int \frac{1}{\sec^3 \theta \tan(\theta)} \sec(\theta) \tan(\theta) \d \theta \\
&= \int \frac{1}{\sec^2(\theta)} \d \theta \\
&= \int \cos^2(\theta) \d \theta \\
&= \int \frac{1}{2} + \frac{1}{2} \cos (2 \theta) \d \theta \\
&= \frac{1}{2}\theta + \frac{1}{4} \sin (2\theta) + C \quad \leftarrow \textrm{ Use } \sin(2\theta) = 2 \sin(\theta) \cos(\theta) \\
&= \frac{1}{2} \theta+\frac{1}{2}\sin(\theta) \cos(\theta) + C \\
&= \frac{1}{2} \arcsec(t) + \frac{\sqrt{t^2-1}}{2t^2} + C,
\end{align*}
where we find the last term using a reference triangle for $\theta$. Since $\sec(\theta) = t$, a reference triangle can be formed with hypotenuse $t$ and adjacent side length $1$, so that the opposite side length is $\sqrt{t^2-1}$. It follows that $\cos(\theta) = \frac{1}{t}$ and $\sin(\theta) = \frac{\sqrt{t^2-1}}{t}$. Therefore
\[
\int_{\sqrt{2}}^2 \frac{1}{t^3 \sqrt{t^2 -1}} \d t = \eval{\frac{1}{2} \sec^{-1}\theta + \frac{\sqrt{t^2-1}}{2t^2}}_{\sqrt{2}}^2 = \frac{\pi}{6} + \frac{\sqrt{3}}{8} -  \frac{\pi}{8} - \frac{1}{4}.
\]

\paragraph{Way 2:} Transform the $t$-limits into $\theta$-limits.

Note that if $t = \sec(\theta)$, when $t=\sqrt{2}$, we have $\sqrt{2} = \sec(\theta)$ so $\theta = \frac{\pi}{4}$ and when $t=2$, we have $2 = \sec(\theta)$ so $\theta=\frac{\pi}{3}$. Then,
\begin{align*}
\int_{t=\sqrt{2}}^{t=2} \frac{1}{t^3\sqrt{t^2-1}} \d t &= \int_{\theta=\pi/4}^{\theta= \pi/3}\frac{1}{\sec^3 \theta \tan(\theta)} \sec(\theta) \tan(\theta) \d \theta \\
&= \int_{\theta=\pi/4}^{\theta= \pi/3} \frac{1}{\sec^2(\theta)} \d \theta \\
&= \int_{\theta=\pi/4}^{\theta= \pi/3} \cos^2(\theta) \d \theta \\
&= \int_{\theta=\pi/4}^{\theta= \pi/3} \frac{1}{2} + \frac{1}{2} \cos (2 \theta) \d \theta \\
&= \eval{\frac{1}{2}\theta + \frac{1}{4} \sin (2\theta)}_{\theta=\pi/4}^{\theta= \pi/3} \\
\end{align*}

We now can evaluate directly without having to draw a triangle to reverse the substitution.

\begin{align*}
\int_{t=\sqrt{2}}^{t=2} \frac{1}{t^3\sqrt{t^2-1}} \d t &= \eval{\frac{1}{2}\theta + \frac{1}{4} \sin (2\theta)}_{\theta=\pi/4}^{\theta= \pi/3} \\
&= \left[\frac{1}{2}\cdot \frac{\pi}{3} + \frac{1}{4} \sin \left(\frac{2\pi}{3}\right) \right] - \left[\frac{1}{2}\cdot \frac{\pi}{4} + \frac{1}{4} \sin \left(\frac{\pi}{2}\right) \right] \\
&= \frac{\pi}{6} +\frac{\sqrt{3}}{8} - \frac{\pi}{8} -\frac{1}{4}
\end{align*}

Which method requires less work?

II. We choose the substitution $ x = \frac{2}{3} \tan\theta$, so that $\d x = \frac{2}{3} \sec^2(\theta) \d \theta$. Making the substitution into the integral gives:
\begin{align*}
\int \frac{27 x^2}{(4+9x^2)^{3/2}} \, dx & =  \int \dfrac{27 \cdot  \frac{4}{9 \tan^2(\theta)}}{(4+9 \cdot \frac{4}{9}\tan^2(\theta))^{3/2}} \, \left[ \dfrac{2}{3} \sec^2(\theta) \d \theta \right] \\ 
&=  \int \dfrac{8 \tan^2(\theta)}{ (4[1+1\tan^2(\theta)])^{3/2}} \,  \cdot \sec^2(\theta) \d \theta  \\
&=  \int \dfrac{8 \tan^2(\theta)}{8 (\sec^2(\theta))^{3/2}} \,  \cdot \sec^2(\theta) \d \theta  \\
&=  \int \dfrac{   \tan^2(\theta)}{\sec^3 \theta} \,  \cdot \sec^2(\theta) \d \theta  \\
&=  \int \dfrac{   \tan^2(\theta)}{\sec(\theta)} \d \theta \\
&=  \int \dfrac{   \sec^2(\theta) + 1}{\sec(\theta)} \,  d\theta  \\
&=  \int \left[\dfrac{   \sec^2(\theta)}{\sec(\theta)} + \dfrac{1}{\sec(\theta)}\right] \,  d\theta  \\
&=  \int ( \sec(\theta) + \cos(\theta) )  \,  d\theta  \\
&=  \ln \left| \sec(\theta) + \tan(\theta) \right| - \sin(\theta) +C  \\
&= \ln \left| \dfrac{\sqrt{4+9x^2}}{2}+\dfrac{3x}{2} \right| - \dfrac{3x}{\sqrt{4x^2+9}} +C.
\end{align*}
To derive the last line, we use a reference triangle for $\theta$, with $\tan(\theta) = \frac{3x}{2}$. This triangle has opposite side length $3x$ and adjacent side length $2$, and therefore has hypotenuse length $\sqrt{9x^2 + 4}$. It follows that $\sec(\theta) = \frac{\sqrt{9x^2+4}}{2}$.

III. We first complete the square under the square root sign, using 
$$
4x-x^2 = -(x^2-4x) = -((x-2)^2-4) = 4-(x-2)^2.
$$
We therfore wish to choose a substitution so that $4-(x-2)^2$ transforms to $4-4\sin^2\theta$. The correct substituion is therefore $x-2 = 2 \sin(\theta)$, or $x= 2 \sin(\theta) + 2$. Then $\d x = 2 \cos(\theta) \d \theta$, and 
\begin{align*}
\int \frac{x^2}{\sqrt{4x-x^2}} \d x & = \int \frac{x^2}{\sqrt{4-(x-2)^2}} \d x \\
&= \int \frac{4 \sin^2(\theta) + 8 \sin(\theta) + 4}{2 \cos(\theta)} 2 \cos(\theta) \d \theta \\
&= \int 4 \sin^2(\theta) + 8 \sin(\theta) + 4 \d \theta \\
&= 4 \int \frac{1}{2} - \frac{\cos (2 \theta)}{2} \d \theta -8 \cos(\theta) + 4 \theta \\
&= 2 \theta - \sin (2 \theta) - 8 \cos(\theta) + 4 \theta + C \\
&= 6 \theta - 2 \sin(\theta) \cos(\theta) - 8 \cos(\theta) + C \\
&= 6 \arcsin \left(\frac{x-2}{2}\right) - (x-2)\frac{\sqrt{4-(x-2)^2}}{2} - 4 \sqrt{4-(x-2)^2} + C.
\end{align*}
The last line follows from a reference triangle. A reference triangle for $\theta$, with $\sin(\theta) = (x-2)/2$, has opposite side length $x-2$ and hypotenuse $2$. It therefore has adjacent side length $\sqrt{4-(x-2)^2}$, so that $\cos(\theta) = \sqrt{4-(x-2)^2}/2$.
\end{freeResponse}

\begin{problem}
Evaluate the integral
$$
\int \frac{x}{x^2-9} \d x
$$
in two ways: first via trigonometric substitution, then via $u$-substitution. Which method takes less work?
\end{problem}

\begin{freeResponse}
Using trigonometric substitution, we take $x = 3 \sec(\theta)$. Then $\d x = 3 \sec(\theta) \tan(\theta) \d \theta$, and 
\begin{align*}
\int \frac{x}{x^2-9} \d x &= \int \frac{3 \sec(\theta)}{9(\sec^2(\theta) - 1)} 3 \sec(\theta) \tan(\theta) \d \theta \\
&= \int \frac{\sec(\theta)}{\tan^2(\theta)} \sec(\theta) \tan(\theta) \d \theta \\
&= \int \frac{\sec^2(\theta)}{\tan(\theta)} \d \theta.
\end{align*}
Now let $u = \tan(\theta)$ so that $\d u = \sec^2(\theta) \d \theta$. The integral becomes
$$
\int \frac{\sec^2(\theta)}{\tan(\theta)} \d \theta = \int \frac{1}{u} \d u = \ln |u| + C = \ln | \tan(\theta) | + C.
$$
Finally, using a reference triangle we conclude that $\tan(\theta) = \frac{\sqrt{x^2-9}}{3}$, so the answer is 
$$
\ln \left| \frac{\sqrt{x^2-9}}{3}\right| + C.
$$

On the other hand, we can immediately make the substitution $u = x^2 - 9$, so that $\d u = 2x \d x$, and 
$$
\int \frac{x}{x^2-9} \d x = \frac{1}{2} \int \frac{1}{u} \d u = \frac{1}{2} \ln |u| + C = \frac{1}{2} \ln |x^2-9| + C.
$$
Note that (by applying log rules) this differs from the previous answer by addition of a constant, so that the two answers are equivalent. Apparently, directly using $u$-substitution takes much less work. 
\end{freeResponse}


\end{document}
