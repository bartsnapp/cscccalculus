\documentclass[noauthor]{ximera}
%handout:  for handout version with no solutions or instructor notes
%handout,instructornotes:  for instructor version with just problems and notes, no solutions
%noinstructornotes:  shows only problem and solutions

%% handout
%% space
%% newpage
%% numbers
%% nooutcomes

%I added the commands here so that I would't have to keep looking them up
%\newcommand{\RR}{\mathbb R}
%\renewcommand{\d}{\,d}
%\newcommand{\dd}[2][]{\frac{d #1}{d #2}}
%\renewcommand{\l}{\ell}
%\newcommand{\ddx}{\frac{d}{dx}}
%\everymath{\displaystyle}
%\newcommand{\dfn}{\textbf}
%\newcommand{\eval}[1]{\bigg[ #1 \bigg]}

%\begin{image}
%\includegraphics[trim= 170 420 250 180]{Figure1.pdf}
%\end{image}

%add a ``.'' below when used in a specific directory.

%\usepackage{todonotes}

\newcommand{\todo}{}

\usepackage{esint} % for \oiint
\ifxake%%https://math.meta.stackexchange.com/questions/9973/how-do-you-render-a-closed-surface-double-integral
\renewcommand{\oiint}{{\large\bigcirc}\kern-1.56em\iint}
\fi


\graphicspath{
  {./}
  {ximeraTutorial/}
  {basicPhilosophy/}
  {functionsOfSeveralVariables/}
  {normalVectors/}
  {lagrangeMultipliers/}
  {vectorFields/}
  {greensTheorem/}
  {shapeOfThingsToCome/}
  {dotProducts/}
  {../productAndQuotientRules/exercises/}
  {../normalVectors/exercisesParametricPlots/}
  {../continuityOfFunctionsOfSeveralVariables/exercises/}
  {../partialDerivatives/exercises/}
  {../chainRuleForFunctionsOfSeveralVariables/exercises/}
  {../commonCoordinates/exercisesCylindricalCoordinates/}
  {../commonCoordinates/exercisesSphericalCoordinates/}
  {../greensTheorem/exercisesCurlAndLineIntegrals/}
  {../greensTheorem/exercisesDivergenceAndLineIntegrals/}
  {../shapeOfThingsToCome/exercisesDivergenceTheorem/}
  {../greensTheorem/}
  {../shapeOfThingsToCome/}
}

\newcommand{\mooculus}{\textsf{\textbf{MOOC}\textnormal{\textsf{ULUS}}}}

\usepackage{tkz-euclide}\usepackage{tikz}
\usepackage{tikz-cd}
\usetikzlibrary{arrows}
\tikzset{>=stealth,commutative diagrams/.cd,
  arrow style=tikz,diagrams={>=stealth}} %% cool arrow head
\tikzset{shorten <>/.style={ shorten >=#1, shorten <=#1 } } %% allows shorter vectors

\usetikzlibrary{backgrounds} %% for boxes around graphs
\usetikzlibrary{shapes,positioning}  %% Clouds and stars
\usetikzlibrary{matrix} %% for matrix
\usepgfplotslibrary{polar} %% for polar plots
\usepgfplotslibrary{fillbetween} %% to shade area between curves in TikZ
\usetkzobj{all}
%\usepackage[makeroom]{cancel} %% for strike outs
%\usepackage{mathtools} %% for pretty underbrace % Breaks Ximera
%\usepackage{multicol}
\usepackage{pgffor} %% required for integral for loops



%% http://tex.stackexchange.com/questions/66490/drawing-a-tikz-arc-specifying-the-center
%% Draws beach ball
\tikzset{pics/carc/.style args={#1:#2:#3}{code={\draw[pic actions] (#1:#3) arc(#1:#2:#3);}}}



\usepackage{array}
\setlength{\extrarowheight}{+.1cm}   
\newdimen\digitwidth
\settowidth\digitwidth{9}
\def\divrule#1#2{
\noalign{\moveright#1\digitwidth
\vbox{\hrule width#2\digitwidth}}}





\newcommand{\RR}{\mathbb R}
\newcommand{\R}{\mathbb R}
\newcommand{\N}{\mathbb N}
\newcommand{\Z}{\mathbb Z}

\newcommand{\sagemath}{\textsf{SageMath}}


%\renewcommand{\d}{\,d\!}
\renewcommand{\d}{\mathop{}\!d}
\newcommand{\dd}[2][]{\frac{\d #1}{\d #2}}
\newcommand{\pp}[2][]{\frac{\partial #1}{\partial #2}}
\renewcommand{\l}{\ell}
\newcommand{\ddx}{\frac{d}{\d x}}

\newcommand{\zeroOverZero}{\ensuremath{\boldsymbol{\tfrac{0}{0}}}}
\newcommand{\inftyOverInfty}{\ensuremath{\boldsymbol{\tfrac{\infty}{\infty}}}}
\newcommand{\zeroOverInfty}{\ensuremath{\boldsymbol{\tfrac{0}{\infty}}}}
\newcommand{\zeroTimesInfty}{\ensuremath{\small\boldsymbol{0\cdot \infty}}}
\newcommand{\inftyMinusInfty}{\ensuremath{\small\boldsymbol{\infty - \infty}}}
\newcommand{\oneToInfty}{\ensuremath{\boldsymbol{1^\infty}}}
\newcommand{\zeroToZero}{\ensuremath{\boldsymbol{0^0}}}
\newcommand{\inftyToZero}{\ensuremath{\boldsymbol{\infty^0}}}



\newcommand{\numOverZero}{\ensuremath{\boldsymbol{\tfrac{\#}{0}}}}
\newcommand{\dfn}{\textbf}
%\newcommand{\unit}{\,\mathrm}
\newcommand{\unit}{\mathop{}\!\mathrm}
\newcommand{\eval}[1]{\bigg[ #1 \bigg]}
\newcommand{\seq}[1]{\left( #1 \right)}
\renewcommand{\epsilon}{\varepsilon}
\renewcommand{\phi}{\varphi}


\renewcommand{\iff}{\Leftrightarrow}

\DeclareMathOperator{\arccot}{arccot}
\DeclareMathOperator{\arcsec}{arcsec}
\DeclareMathOperator{\arccsc}{arccsc}
\DeclareMathOperator{\si}{Si}
\DeclareMathOperator{\scal}{scal}
\DeclareMathOperator{\sign}{sign}


%% \newcommand{\tightoverset}[2]{% for arrow vec
%%   \mathop{#2}\limits^{\vbox to -.5ex{\kern-0.75ex\hbox{$#1$}\vss}}}
\newcommand{\arrowvec}[1]{{\overset{\rightharpoonup}{#1}}}
%\renewcommand{\vec}[1]{\arrowvec{\mathbf{#1}}}
\renewcommand{\vec}[1]{{\overset{\boldsymbol{\rightharpoonup}}{\mathbf{#1}}}}
\DeclareMathOperator{\proj}{\vec{proj}}
\newcommand{\veci}{{\boldsymbol{\hat{\imath}}}}
\newcommand{\vecj}{{\boldsymbol{\hat{\jmath}}}}
\newcommand{\veck}{{\boldsymbol{\hat{k}}}}
\newcommand{\vecl}{\vec{\boldsymbol{\l}}}
\newcommand{\uvec}[1]{\mathbf{\hat{#1}}}
\newcommand{\utan}{\mathbf{\hat{t}}}
\newcommand{\unormal}{\mathbf{\hat{n}}}
\newcommand{\ubinormal}{\mathbf{\hat{b}}}

\newcommand{\dotp}{\bullet}
\newcommand{\cross}{\boldsymbol\times}
\newcommand{\grad}{\boldsymbol\nabla}
\newcommand{\divergence}{\grad\dotp}
\newcommand{\curl}{\grad\cross}
%\DeclareMathOperator{\divergence}{divergence}
%\DeclareMathOperator{\curl}[1]{\grad\cross #1}
\newcommand{\lto}{\mathop{\longrightarrow\,}\limits}

\renewcommand{\bar}{\overline}

\colorlet{textColor}{black} 
\colorlet{background}{white}
\colorlet{penColor}{blue!50!black} % Color of a curve in a plot
\colorlet{penColor2}{red!50!black}% Color of a curve in a plot
\colorlet{penColor3}{red!50!blue} % Color of a curve in a plot
\colorlet{penColor4}{green!50!black} % Color of a curve in a plot
\colorlet{penColor5}{orange!80!black} % Color of a curve in a plot
\colorlet{penColor6}{yellow!70!black} % Color of a curve in a plot
\colorlet{fill1}{penColor!20} % Color of fill in a plot
\colorlet{fill2}{penColor2!20} % Color of fill in a plot
\colorlet{fillp}{fill1} % Color of positive area
\colorlet{filln}{penColor2!20} % Color of negative area
\colorlet{fill3}{penColor3!20} % Fill
\colorlet{fill4}{penColor4!20} % Fill
\colorlet{fill5}{penColor5!20} % Fill
\colorlet{gridColor}{gray!50} % Color of grid in a plot

\newcommand{\surfaceColor}{violet}
\newcommand{\surfaceColorTwo}{redyellow}
\newcommand{\sliceColor}{greenyellow}




\pgfmathdeclarefunction{gauss}{2}{% gives gaussian
  \pgfmathparse{1/(#2*sqrt(2*pi))*exp(-((x-#1)^2)/(2*#2^2))}%
}


%%%%%%%%%%%%%
%% Vectors
%%%%%%%%%%%%%

%% Simple horiz vectors
\renewcommand{\vector}[1]{\left\langle #1\right\rangle}


%% %% Complex Horiz Vectors with angle brackets
%% \makeatletter
%% \renewcommand{\vector}[2][ , ]{\left\langle%
%%   \def\nextitem{\def\nextitem{#1}}%
%%   \@for \el:=#2\do{\nextitem\el}\right\rangle%
%% }
%% \makeatother

%% %% Vertical Vectors
%% \def\vector#1{\begin{bmatrix}\vecListA#1,,\end{bmatrix}}
%% \def\vecListA#1,{\if,#1,\else #1\cr \expandafter \vecListA \fi}

%%%%%%%%%%%%%
%% End of vectors
%%%%%%%%%%%%%

%\newcommand{\fullwidth}{}
%\newcommand{\normalwidth}{}



%% makes a snazzy t-chart for evaluating functions
%\newenvironment{tchart}{\rowcolors{2}{}{background!90!textColor}\array}{\endarray}

%%This is to help with formatting on future title pages.
\newenvironment{sectionOutcomes}{}{} 



%% Flowchart stuff
%\tikzstyle{startstop} = [rectangle, rounded corners, minimum width=3cm, minimum height=1cm,text centered, draw=black]
%\tikzstyle{question} = [rectangle, minimum width=3cm, minimum height=1cm, text centered, draw=black]
%\tikzstyle{decision} = [trapezium, trapezium left angle=70, trapezium right angle=110, minimum width=3cm, minimum height=1cm, text centered, draw=black]
%\tikzstyle{question} = [rectangle, rounded corners, minimum width=3cm, minimum height=1cm,text centered, draw=black]
%\tikzstyle{process} = [rectangle, minimum width=3cm, minimum height=1cm, text centered, draw=black]
%\tikzstyle{decision} = [trapezium, trapezium left angle=70, trapezium right angle=110, minimum width=3cm, minimum height=1cm, text centered, draw=black]




\author{Tom Needham and Jim Talamo}

\outcome{Use the divergence test to conclude that a series diverges.}
\outcome{Use the integral test to determine whether a series converges or diverges.}

\title[]{The Divergence Test}

\begin{document}
\begin{abstract}
\end{abstract}
\maketitle

\vspace{-0.9in}

\section{Discussion Questions}
\begin{problem}
Suppose that  $a_n = \frac{n^2-2n+4}{2n^2+1}$.  Let $s_n = \sum_{k=1}^n a_k$. 

\begin{itemize}
\item[I.] Determine if $\lim_{n \to \infty} a_n$ exists.  If it does, can you determine its value?
\item[II.] Determine if $\sum_{k=1}^{\infty} a_k$ exists.  If it does, can you determine its value?
\item[III.] Determine if $\lim_{n \to \infty} s_n$ exists.  If it does, can you determine its value?
\end{itemize}

\begin{freeResponse}
I. By inspection, $\lim_{n \to \infty} a_n = \frac{1}{2}$.

II. Since $\lim_{n \to \infty} a_n = \frac{1}{2} \neq 0$, $\sum_{k=1}^{\infty} a_k$ diverges by the divergence test.  

Make sure that you realize that information about the \emph{sequence} $\{a_n\}_{n=1}$ here is being used to determine convergence of the \emph{series} (i.e. the attempt to add all of the terms in the sequence together). 

III. The symbols $\sum_{k=1}^{\infty} a_k$ and $\lim_{n \to \infty} s_n$ are equivalent.  Since $\sum_{k=1}^{\infty} a_k$ diverges, $\lim_{n \to \infty} s_n$ does not exist.
\end{freeResponse}
\end{problem}


%%%%%%%%%%%%%%%%%%%%%%%%%%%%%%%%%%%%%%%%%%%%%%%


\begin{problem}
Determine if the following statements are true or false.

I. If the sequence $\{a_n\}$ satisfies $\lim_{n\rightarrow \infty} a_n = 0$, then the series $\sum_{n=1}^\infty a_n$ converges.

II. If the sequence $\{a_n\}$ satisfies $\lim_{n \rightarrow \infty} a_n \neq 0$, then the series $\sum_{n=1}^\infty a_n$ diverges.

III. If the series $\sum_{n=1}^\infty a_n$ converges, then $\lim_{n\rightarrow \infty} a_n = 0$. 

IV. If the series $\sum_{n=1}^\infty a_n$ diverges, then $\lim_{n\rightarrow \infty} a_n \neq 0$. 



\begin{freeResponse}
I. This statement is false. For example, the harmonic series $\sum \frac{1}{n}$ diverges, while $\lim_{n \rightarrow \infty} \frac{1}{n} = 0$. 

II. This is the divergence test, and is therefore true.

III. This is true; in order for a series to converge, the infinitely many terms we are trying to add together must tend to $0$.  More formally, this is the contrapositive of the divergence test, so it is true.

IV. This statement is false. For example, the harmonic series $\sum \frac{1}{n}$ diverges, while $\lim_{n \rightarrow \infty} \frac{1}{n} = 0$. 
\end{freeResponse}
\end{problem}

%%%%%%%%%%%%%%%%%%%%%%%%%%%%%%%%%%%%%%%%%%%%%%%
\begin{problem}
Suppose that the series $\sum_{k=1}^\infty a_k$ has sequence of partial sums $\{s_n\}$ given by the explicit formula
$$
s_n = \frac{2n+1}{4n+3}.
$$
Student A claims that the series $\sum_{k=1}^\infty a_k$ converges because $\lim_{n \rightarrow \infty} s_n = \frac{1}{2}$. Student B claims that $\lim_{n\rightarrow \infty} s_n = \frac{1}{2} \neq 0$, so the series diverges by the divergence test. Which (if either) student is correct?

\begin{freeResponse}
Student A is correct, by the definition of convergence. Student B is incorrect; the divergence test deals with the limit of the sequence of  terms $a_k$, not the limit of the sequence of partial sums.  The divergence test can be used to conclude that since $\lim_{n\rightarrow \infty} s_n = \frac{1}{2} \neq 0$, the series $\sum_{k=1}^{\infty} s_k$ diverges.
\end{freeResponse}

\end{problem}

%%%%%%%%%%%%%%%%%%%%%%%%%%%%%%%%%%%%%%%%%%%%%%%

\begin{problem}
Given that $\sum_{k=1}^n a_k = \frac{3^n}{3^n+\ln^8(n)}$, determine whether $\sum_{k=1}^{\infty} a_k $ converges or diverges.  If it converges, can you determine its value?

\begin{freeResponse}
By now, hopefully it is becoming clear that a lot of information is being stored in the notation used in these problems.  A good strategy before tackling a problem is to take a moment and parse what information is being given.

Note that here, the result of \emph{adding} $a_1$ through $a_n$ is being given, \emph{NOT} an explicit formula for the $n$-th term $a_n$; we are actually told what $s_n$ is!  As such, 

\[
\sum_{k=1}^{\infty} a_k = \lim_{n \to \infty} \left[ \sum_{k=1}^n a_k \right] = \lim_{n \to \infty} \left[ \frac{3^n}{3^n+\ln^8(n)} \right] =1. 
\]

Note that this limit can be computed by inspection using the growth rates results since the dominant term in both the numerator and denominator is $3^n$.
\end{freeResponse}

\end{problem}

%%%%%%%%%%%%%%%%%%%%%%%%%%%%%%%%%%%%%%%%%%%%%%%

\section{Group Work}

\begin{problem}
Let $\{a_n\}_{n=1}$ be a sequence and define $s_n = \sum_{k=1}^n a_k$.  

I. If $\sum_{k=1}^{\infty} a_k =2$, does $\sum_{k=1}^{\infty} s_k$ converge or diverge?  

II. If $\sum_{k=1}^{\infty} a_k =2$, does $\sum_{k=1}^{\infty} (a_k+1)$ converge or diverge?  

III. If $\sum_{k=1}^{\infty} s_k =2$, does $\sum_{k=1}^{\infty} a_k$ converge or diverge?  

If a series above converges, can you determine its value? 

\begin{freeResponse}
Note that a sequence $\{a_n\}$ and its corresponding sequence of partial sums $\{s_n\}$ are related by the equality

\[
\sum_{k=k_0}^{\infty} a_k = \lim_{n \to \infty} s_n.
\] 

I. Since $\sum_{k=1}^{\infty} a_k =2$, we know that $ \lim_{n \to \infty} s_n = 2$.  Since $ \lim_{n \to \infty} s_n \neq 0$, $\sum_{k=1}^{\infty} s_k$ diverges by the divergence test.

II. Recall that if $\sum_{k=k_0}^{\infty} a_k$ converges, then $\lim_{n \to \infty} a_n = 0$; this is a formalization of the heuristic idea that the only chance that an infinite sum of terms has to converge is if the terms tend to $0$.

Since $\sum_{k=1}^{\infty} a_k =2$, we know that $ \lim_{n \to \infty} a_n = 0$, and thus $\lim_{n \to \infty} \left(a_n+1\right)  = 1 \neq 0$.  Since $ \lim_{n \to \infty}\left(a_n+1\right) \neq 0$, $\sum_{k=1}^{\infty} \left(a_n+1\right) $ diverges by the divergence test.

\begin{remark}
The notation here is important; there is a huge difference between the symbols $\sum_{k=1}^{\infty} a_k+1$ and $\sum_{k=1}^{\infty} (a_k+1)$.  

\begin{itemize} 
\item The symbol $\sum_{k=1}^{\infty} a_k+1$ is interpreted as $\left( \sum_{k=1}^{\infty} a_k\right)+1$; that is we add $1$ to the original series.
\item The symbol $\sum_{k=1}^{\infty} (a_k+1)$ requires that we add $1$ to \emph{each} term $a_k$.
\end{itemize}
\end{remark}
\end{freeResponse}

III. Since $\sum_{k=1}^{\infty} s_k =2$, $\sum_{k=1}^{\infty} s_k$ is a convergent series, so we know that $ \lim_{n \to \infty} s_n = 0$.  Since $ \lim_{n \to \infty} s_n \neq 0$, we have by definition that $\sum_{k=1}^{\infty} a_k =0$.
\end{problem}

%%%%%%%%%%%%%%%%%%%%%%%%%%%%%%%%%%%%%%%%%%%%%%%

\begin{problem}
Determine whether the series below converge or diverge.  If a series converges, can you determine its value?

\begin{center}
\begin{tabular}{lll}
I. $\sum_{k=3}^{\infty} \frac{(2k^2+1)^2}{k^4-3k+5}$ \qquad  \qquad II. $\sum_{k=1}^{\infty} \cos\left(\frac{2^k+k^{10}}{3^k}\right)$ \qquad \qquad  III. $\sum_{k=2}^{\infty} \frac{1}{k^2+k}$ 
\end{tabular}
\end{center}


\begin{freeResponse}
I. Comparing the growth rates of the dominant terms in the numerator and denominator, we have 
$$
\lim_{k \rightarrow \infty} \frac{k^3 - \ln(k+10) + k^5}{(2k+100)^5} = \lim_{k\rightarrow \infty} \frac{k^5}{2^5 k^5} = \frac{1}{32} .
$$
Since the limit is nonzero, the series diverges by The divergence test.

II. Note that $\lim_{n \to \infty} \frac{2^n+n^{10}}{3^n} =0$ by growth rates, so 

\[\lim_{n \to \infty} \cos\left(\frac{2^n+n^{10}}{3^n}\right) = \lim_{n \to \infty} \cos(0) =1. \]

Since $\lim_{n \to \infty} \cos\left(\frac{2^n+n^{10}}{3^n}\right) \neq 0$, $\sum_{k=1}^{\infty} \cos\left(\frac{2^k+k^{10}}{3^k}\right)$ diverges by the divergence test.

\end{freeResponse}

\end{problem}

%%%%%%%%%%%%%%%%%%%%%%%%%%%%%%%%%%%%%%%%%%%%%%%


\begin{problem}
For each part, a sequence $\{a_n\}_{n=1}$ is given.  Let $s_n = \sum_{k=1}^n a_k$ give the terms of the sequence $\{s_n\}_{n=1}$ of partial sums. Determine whether each sequence is increasing, decreasing, bounded above, and/or bounded below. 

I. Let $\{a_k\}_{k=1}^\infty$ be the sequence defined by $a_k = \frac{1}{k^3}$ and let $\{s_n\}_{n=1}^\infty$ denote its sequence of partial sums. You may take as given that $\sum_{k=1}^{\infty} \frac{1}{k^3}$ converges.

II. Let $\{a_k\}_{k=1}^\infty$ be the sequence defined by $a_k = \frac{2^k}{2^k +1}$ and let $\{s_n\}_{n=1}^\infty$ denote its sequence of partial sums.  

\begin{freeResponse}
I. The sequence $\{a_k\}$ is monotone decreasing and bounded (above and below), by inspection. Since the terms of $\{a_k\}$ are strictly positive, the sequence $\{s_n\}$ must be monotone increasing. Since $\sum \frac{1}{k^3}$ converges, it follows by the definition of convergence that the sequence $\{s_n\}$ has a finite limit and is therefore bounded.

II. The sequence $\{a_k\}$ is monotone increasing and bounded, since

\[
\frac{2^k}{2^k +1} = \frac{2^k+1 -1}{2^k +1} = \frac{2^k+1}{2^k +1}- \frac{1}{2^k +1} = 1 - \frac{1}{2^k +1}.
\]

Note that $\lim_{k\rightarrow \infty} a_k = 1$, so the divergence test implies that the series $\sum a_k$ diverges. It follows that the sequence $\{s_n\}$ is monotone increasing and bounded below (because the terms of $a_k$ are positive), but not bounded above (because $\lim_{n \rightarrow \infty} s_n = \infty$). 
\end{freeResponse}

\end{problem}

\end{document}
