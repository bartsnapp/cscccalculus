\documentclass[handout,noauthor,nooutcomes]{ximera}

\author{Bart Snapp}

%\usepackage{multicol}

\outcome{}

%\usepackage{todonotes}

\newcommand{\todo}{}

\usepackage{esint} % for \oiint
\ifxake%%https://math.meta.stackexchange.com/questions/9973/how-do-you-render-a-closed-surface-double-integral
\renewcommand{\oiint}{{\large\bigcirc}\kern-1.56em\iint}
\fi


\graphicspath{
  {./}
  {ximeraTutorial/}
  {basicPhilosophy/}
  {functionsOfSeveralVariables/}
  {normalVectors/}
  {lagrangeMultipliers/}
  {vectorFields/}
  {greensTheorem/}
  {shapeOfThingsToCome/}
  {dotProducts/}
  {../productAndQuotientRules/exercises/}
  {../normalVectors/exercisesParametricPlots/}
  {../continuityOfFunctionsOfSeveralVariables/exercises/}
  {../partialDerivatives/exercises/}
  {../chainRuleForFunctionsOfSeveralVariables/exercises/}
  {../commonCoordinates/exercisesCylindricalCoordinates/}
  {../commonCoordinates/exercisesSphericalCoordinates/}
  {../greensTheorem/exercisesCurlAndLineIntegrals/}
  {../greensTheorem/exercisesDivergenceAndLineIntegrals/}
  {../shapeOfThingsToCome/exercisesDivergenceTheorem/}
  {../greensTheorem/}
  {../shapeOfThingsToCome/}
}

\newcommand{\mooculus}{\textsf{\textbf{MOOC}\textnormal{\textsf{ULUS}}}}

\usepackage{tkz-euclide}\usepackage{tikz}
\usepackage{tikz-cd}
\usetikzlibrary{arrows}
\tikzset{>=stealth,commutative diagrams/.cd,
  arrow style=tikz,diagrams={>=stealth}} %% cool arrow head
\tikzset{shorten <>/.style={ shorten >=#1, shorten <=#1 } } %% allows shorter vectors

\usetikzlibrary{backgrounds} %% for boxes around graphs
\usetikzlibrary{shapes,positioning}  %% Clouds and stars
\usetikzlibrary{matrix} %% for matrix
\usepgfplotslibrary{polar} %% for polar plots
\usepgfplotslibrary{fillbetween} %% to shade area between curves in TikZ
\usetkzobj{all}
%\usepackage[makeroom]{cancel} %% for strike outs
%\usepackage{mathtools} %% for pretty underbrace % Breaks Ximera
%\usepackage{multicol}
\usepackage{pgffor} %% required for integral for loops



%% http://tex.stackexchange.com/questions/66490/drawing-a-tikz-arc-specifying-the-center
%% Draws beach ball
\tikzset{pics/carc/.style args={#1:#2:#3}{code={\draw[pic actions] (#1:#3) arc(#1:#2:#3);}}}



\usepackage{array}
\setlength{\extrarowheight}{+.1cm}   
\newdimen\digitwidth
\settowidth\digitwidth{9}
\def\divrule#1#2{
\noalign{\moveright#1\digitwidth
\vbox{\hrule width#2\digitwidth}}}





\newcommand{\RR}{\mathbb R}
\newcommand{\R}{\mathbb R}
\newcommand{\N}{\mathbb N}
\newcommand{\Z}{\mathbb Z}

\newcommand{\sagemath}{\textsf{SageMath}}


%\renewcommand{\d}{\,d\!}
\renewcommand{\d}{\mathop{}\!d}
\newcommand{\dd}[2][]{\frac{\d #1}{\d #2}}
\newcommand{\pp}[2][]{\frac{\partial #1}{\partial #2}}
\renewcommand{\l}{\ell}
\newcommand{\ddx}{\frac{d}{\d x}}

\newcommand{\zeroOverZero}{\ensuremath{\boldsymbol{\tfrac{0}{0}}}}
\newcommand{\inftyOverInfty}{\ensuremath{\boldsymbol{\tfrac{\infty}{\infty}}}}
\newcommand{\zeroOverInfty}{\ensuremath{\boldsymbol{\tfrac{0}{\infty}}}}
\newcommand{\zeroTimesInfty}{\ensuremath{\small\boldsymbol{0\cdot \infty}}}
\newcommand{\inftyMinusInfty}{\ensuremath{\small\boldsymbol{\infty - \infty}}}
\newcommand{\oneToInfty}{\ensuremath{\boldsymbol{1^\infty}}}
\newcommand{\zeroToZero}{\ensuremath{\boldsymbol{0^0}}}
\newcommand{\inftyToZero}{\ensuremath{\boldsymbol{\infty^0}}}



\newcommand{\numOverZero}{\ensuremath{\boldsymbol{\tfrac{\#}{0}}}}
\newcommand{\dfn}{\textbf}
%\newcommand{\unit}{\,\mathrm}
\newcommand{\unit}{\mathop{}\!\mathrm}
\newcommand{\eval}[1]{\bigg[ #1 \bigg]}
\newcommand{\seq}[1]{\left( #1 \right)}
\renewcommand{\epsilon}{\varepsilon}
\renewcommand{\phi}{\varphi}


\renewcommand{\iff}{\Leftrightarrow}

\DeclareMathOperator{\arccot}{arccot}
\DeclareMathOperator{\arcsec}{arcsec}
\DeclareMathOperator{\arccsc}{arccsc}
\DeclareMathOperator{\si}{Si}
\DeclareMathOperator{\scal}{scal}
\DeclareMathOperator{\sign}{sign}


%% \newcommand{\tightoverset}[2]{% for arrow vec
%%   \mathop{#2}\limits^{\vbox to -.5ex{\kern-0.75ex\hbox{$#1$}\vss}}}
\newcommand{\arrowvec}[1]{{\overset{\rightharpoonup}{#1}}}
%\renewcommand{\vec}[1]{\arrowvec{\mathbf{#1}}}
\renewcommand{\vec}[1]{{\overset{\boldsymbol{\rightharpoonup}}{\mathbf{#1}}}}
\DeclareMathOperator{\proj}{\vec{proj}}
\newcommand{\veci}{{\boldsymbol{\hat{\imath}}}}
\newcommand{\vecj}{{\boldsymbol{\hat{\jmath}}}}
\newcommand{\veck}{{\boldsymbol{\hat{k}}}}
\newcommand{\vecl}{\vec{\boldsymbol{\l}}}
\newcommand{\uvec}[1]{\mathbf{\hat{#1}}}
\newcommand{\utan}{\mathbf{\hat{t}}}
\newcommand{\unormal}{\mathbf{\hat{n}}}
\newcommand{\ubinormal}{\mathbf{\hat{b}}}

\newcommand{\dotp}{\bullet}
\newcommand{\cross}{\boldsymbol\times}
\newcommand{\grad}{\boldsymbol\nabla}
\newcommand{\divergence}{\grad\dotp}
\newcommand{\curl}{\grad\cross}
%\DeclareMathOperator{\divergence}{divergence}
%\DeclareMathOperator{\curl}[1]{\grad\cross #1}
\newcommand{\lto}{\mathop{\longrightarrow\,}\limits}

\renewcommand{\bar}{\overline}

\colorlet{textColor}{black} 
\colorlet{background}{white}
\colorlet{penColor}{blue!50!black} % Color of a curve in a plot
\colorlet{penColor2}{red!50!black}% Color of a curve in a plot
\colorlet{penColor3}{red!50!blue} % Color of a curve in a plot
\colorlet{penColor4}{green!50!black} % Color of a curve in a plot
\colorlet{penColor5}{orange!80!black} % Color of a curve in a plot
\colorlet{penColor6}{yellow!70!black} % Color of a curve in a plot
\colorlet{fill1}{penColor!20} % Color of fill in a plot
\colorlet{fill2}{penColor2!20} % Color of fill in a plot
\colorlet{fillp}{fill1} % Color of positive area
\colorlet{filln}{penColor2!20} % Color of negative area
\colorlet{fill3}{penColor3!20} % Fill
\colorlet{fill4}{penColor4!20} % Fill
\colorlet{fill5}{penColor5!20} % Fill
\colorlet{gridColor}{gray!50} % Color of grid in a plot

\newcommand{\surfaceColor}{violet}
\newcommand{\surfaceColorTwo}{redyellow}
\newcommand{\sliceColor}{greenyellow}




\pgfmathdeclarefunction{gauss}{2}{% gives gaussian
  \pgfmathparse{1/(#2*sqrt(2*pi))*exp(-((x-#1)^2)/(2*#2^2))}%
}


%%%%%%%%%%%%%
%% Vectors
%%%%%%%%%%%%%

%% Simple horiz vectors
\renewcommand{\vector}[1]{\left\langle #1\right\rangle}


%% %% Complex Horiz Vectors with angle brackets
%% \makeatletter
%% \renewcommand{\vector}[2][ , ]{\left\langle%
%%   \def\nextitem{\def\nextitem{#1}}%
%%   \@for \el:=#2\do{\nextitem\el}\right\rangle%
%% }
%% \makeatother

%% %% Vertical Vectors
%% \def\vector#1{\begin{bmatrix}\vecListA#1,,\end{bmatrix}}
%% \def\vecListA#1,{\if,#1,\else #1\cr \expandafter \vecListA \fi}

%%%%%%%%%%%%%
%% End of vectors
%%%%%%%%%%%%%

%\newcommand{\fullwidth}{}
%\newcommand{\normalwidth}{}



%% makes a snazzy t-chart for evaluating functions
%\newenvironment{tchart}{\rowcolors{2}{}{background!90!textColor}\array}{\endarray}

%%This is to help with formatting on future title pages.
\newenvironment{sectionOutcomes}{}{} 



%% Flowchart stuff
%\tikzstyle{startstop} = [rectangle, rounded corners, minimum width=3cm, minimum height=1cm,text centered, draw=black]
%\tikzstyle{question} = [rectangle, minimum width=3cm, minimum height=1cm, text centered, draw=black]
%\tikzstyle{decision} = [trapezium, trapezium left angle=70, trapezium right angle=110, minimum width=3cm, minimum height=1cm, text centered, draw=black]
%\tikzstyle{question} = [rectangle, rounded corners, minimum width=3cm, minimum height=1cm,text centered, draw=black]
%\tikzstyle{process} = [rectangle, minimum width=3cm, minimum height=1cm, text centered, draw=black]
%\tikzstyle{decision} = [trapezium, trapezium left angle=70, trapezium right angle=110, minimum width=3cm, minimum height=1cm, text centered, draw=black]


\title[Collaborate:]{Mirrored curves}

\begin{document}
\begin{abstract}
  We study how to bounce vectors off of curves.
\end{abstract}
\maketitle

\textbf{Work in groups of 3--4, writing your answers on a separate
  sheet of paper.}

\section{Normal vectors}

\begin{problem}
Consider the following line:
\begin{image}
  \begin{tikzpicture}
    \begin{axis}[
        xmin=-1,xmax=5,ymin=-1,ymax=3,
        clip=true,
        axis lines=center,
        %ticks=none,
        unit vector ratio*=1 1 1,
        xlabel=$x$, ylabel=$y$,
        %ytick={-2,-1,...,7},
	%xtick={-2,-1,...,10},
	grid = major,
        every axis y label/.style={at=(current axis.above origin),anchor=south},
        every axis x label/.style={at=(current axis.right of origin),anchor=west},
      ]
      \addplot[ultra thick,penColor] {-x/2+2};
      %\addplot[very thick,penColor2,->] plot coordinates {(2,1) (3,3)};
    \end{axis}
  \end{tikzpicture}
\end{image}
Find a vector normal to this line. Explain your reasoning.
\end{problem}

\begin{problem}
  Now consider the line $ax+by = c$ in $\R^2$. Find a vector normal to
  this line. Explain your reasoning.
\end{problem}

\begin{problem}
  Consider the equation:
  \[
  \vec{n}\dotp (\vec{x}-\vec{p}) = 0
  \]
  Explain how this connects to finding normal vectors to lines in
  $\R^2$ of the form:
  \[
  ax + by = c
  \]
  In particular, you should explain what $\vec{n}$, $\vec{x}$, and
  $\vec{p}$ represent.
\end{problem}


\begin{problem}
  Quick! Tell me normal vectors for the following lines:
  %\begin{multicols}{2}
  \begin{enumerate}
  \item $-3x+7y=11$
  \item $4y =8$
  \item $x=y$
  \item $y=-4x+1$
  \end{enumerate}
  %\end{multicols}
\end{problem}


\section{Reflecting off of lines}


Now we will explore how mirrors reflects light.

\begin{fact}[Law of Reflection]
  Light is reflected at the same angle as it arrived, as
  measured from a line perpendicular to the mirror. Draw a picture
  illustrating this fact. 
\end{fact}

Let's see if we can explain why the Law of Reflection is true. We'll
address this in the next several problems.


\begin{problem}
  Consider the following diagram:
  \begin{image}
    \begin{tikzpicture}
      \coordinate (A) at (0,2);
      \coordinate (B) at (0,5);
      \coordinate (C) at (8,1);
      \coordinate (E) at (8,4);
      \coordinate (D) at (4,3);
      
      \draw[ultra thick,penColor] (A)--(E)--(D);
      \draw[thick,penColor3,->] (B)--(D);
      \draw[dashed] (D)--(C);
      \tkzMarkAngle[size=0.7cm,thin](B,D,A)
      \tkzLabelAngle[pos = -0.4](B,D,A){$\alpha$}
            
      \tkzMarkAngle[size=0.9cm,thin](C,D,E)
      \tkzLabelAngle[pos = 0.6](C,D,E){$\beta$}
            
      %\draw[step=.5cm] (0,0) grid (10,5);
    \end{tikzpicture}
  \end{image}
  Explain why the opposite angles $\alpha$ and $\beta$ must be
  equal.
  \begin{hint}
    Label more angles in your picture. Some pairs of angles will sum
    to $180^\circ$. Use this to conclude that $\alpha = \beta$.
  \end{hint}
\end{problem}

Since a mirror simply reflects light, we see that the initial light
beam makes the same angle with the line as the reflected light beam:
\begin{image}
  \begin{tikzpicture}
      \coordinate (A) at (0,2);
      \coordinate (B) at (0,5);
      \coordinate (C) at (8,1);
      \coordinate (E) at (8,4);
      \coordinate (D) at (4,3);
      \coordinate (CR) at (6.59,6.65);
      
      \draw[ultra thick,penColor] (A)--(E)--(D);
      \draw[thick,penColor3,->] (B)--(D);
      \draw[thick,penColor3,->,dashed] (D)--(CR);
      \draw[dashed] (D)--(C);
      \tkzMarkAngle[size=0.7cm,thin](B,D,A)
      \tkzLabelAngle[pos = -0.4](B,D,A){$\alpha$}

      \tkzMarkAngle[size=0.9cm,thin](C,D,E)
      \tkzLabelAngle[pos = 0.6](C,D,E){$\alpha$}
      
      \tkzMarkAngle[size=0.7cm,thin](E,D,CR)
      \tkzLabelAngle[pos = 0.4](E,D,CR){$\alpha$}
            
      
      %\draw[step=.5cm] (0,0) grid (10,5);
  \end{tikzpicture}
\end{image}

\begin{problem}
  Adding a normal vector to the diagram above:
  \begin{image}
    \begin{tikzpicture}
      \coordinate (A) at (0,2);
      \coordinate (B) at (0,5);
      \coordinate (C) at (8,1);
      \coordinate (E) at (8,4);
      \coordinate (D) at (4,3);
      \coordinate (CR) at (6.59,6.65);
      \coordinate (F) at (3,7);
      
      \draw[ultra thick,penColor] (A)--(E)--(D);
      \draw[thick,penColor3,->] (B)--(D);
      \draw[thick,penColor3,->,dashed] (D)--(CR);
      \draw[thick,penColor2,->] (D)--(F);
      %% \tkzMarkAngle[size=0.7cm,thin](B,D,A)
      %% \tkzLabelAngle[pos = -0.4](B,D,A){$\alpha$}

      %\tkzMarkAngle[size=0.9cm,thin](C,D,E)
      %\tkzLabelAngle[pos = 0.6](C,D,E){$\alpha$}
      
      %% \tkzMarkAngle[size=0.7cm,thin](E,D,CR)
      %% \tkzLabelAngle[pos = 0.4](E,D,CR){$\alpha$}

      \tkzMarkAngle[size=0.7cm,thin](CR,D,F)
      \tkzLabelAngle[pos = 0.4](CR,D,F){$\theta$}
      
      \tkzMarkAngle[size=0.8cm,thin](F,D,B)
      \tkzLabelAngle[pos = 0.5](F,D,B){$\varphi$}
            
            
      %\draw[step=.5cm] (0,0) grid (10,5);
    \end{tikzpicture}
  \end{image}
  Explain why $\theta$ and $\varphi$ are equal.
\end{problem}


\begin{problem}
  Someone says that if $\vec{n}$ is a normal vector to a mirror, and
  $\vec{v}$ represents a light beam, then the reflected light beam is
  given by:
  \[
  \vec{r}=\vec{v} - 2\proj_{\vec{n}}(\vec{v})
  \]
  Give \textbf{two} explanations verifying the formula above:
  \begin{itemize}
  \item An explanation where you sketch $\vec{r}$ and show that it is
    a reasonable answer.
  \item An explanation where you show (via a computation) that the
    angle which light is reflected is the same angle as it arrived, as
    measured from a line perpendicular to the mirror.
  \end{itemize}
\end{problem}




\begin{problem}
  Now imagine the line below as a mirror and imagine vector $\vec{v}$
  as a light-ray that strikes the mirror. Find the vector representing
  the reflection of $\vec{v}$.
  \begin{image}
  \begin{tikzpicture}
    \begin{axis}[
        xmin=-1,xmax=5,ymin=-1,ymax=3,
        clip=true,
        axis lines=center,
        %ticks=none,
        unit vector ratio*=1 1 1,
        xlabel=$x$, ylabel=$y$,
        %ytick={-2,-1,...,7},
	%xtick={-2,-1,...,10},
	grid = major,
        every axis y label/.style={at=(current axis.above origin),anchor=south},
        every axis x label/.style={at=(current axis.right of origin),anchor=west},
      ]
      \addplot[ultra thick,penColor] {-x/2+2};
      %\addplot[very thick,penColor2,->] plot coordinates {(2,1) (3,3)};
      \addplot[ultra thick,penColor3,->] plot coordinates {(1,3) (2,1)};
      %\addplot[very thick,dashed,penColor3,->] plot coordinates {(2,1) (4,1)};
    \end{axis}
  \end{tikzpicture}
\end{image}
\end{problem}


\section{Reflecting off of curves}

Reflecting off of mirrored surfaces doesn't require calculus unless
the mirror is curved. 

\begin{problem}
  Let
  \[
  \vec{p}(t) = \vector{t,t^2}
  \]
  We claim that $\vec{n} = \vector{2t,-1}$ is a normal vector for this
  curve for any given $t$. Confirm or deny this claim.
\end{problem}

\begin{problem}
  Suppose light rays $\vec{v}=\vector{0,-1}$ are hitting a mirrored
  surface described by:
  \[
  \vec{p}(t) = \vector{t,t^2}
  \]
  Find a formula for the vector reflected off the surface for any
  given value of $t$. Call this vector $\vec{r}(t)$. Explain your
  reasoning.
\end{problem}


\begin{problem}
  For any given value of $t$, $\vector{t,t^2}$ is a point on our
  curve. In the previous problem you found a vector $\vec{r}(t)$
  describing a reflected light beams $\vec{v}=\vector{0,-1}$. Consider
  the line:
  \[
  \vecl(s) = \vector{t,t^2} + s \cdot \vec{r}(t)
  \]
  \begin{itemize}
  \item Find $a$ so that $\vecl(a)$ intersects the $y$-axis.
  \item What is the $y$-value of $\vecl(a)$?
  \item Do the questions directly above tell us? What does this have
    to do with telescopes, space-heaters, vanity mirrors, and
    eavesdropping devices?
  \end{itemize}
\end{problem}

\end{document}
