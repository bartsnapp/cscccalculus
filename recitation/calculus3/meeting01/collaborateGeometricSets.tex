\documentclass[handout,hints,nooutcomes,noauthor]{ximera}

%\usepackage{todonotes}

\newcommand{\todo}{}

\usepackage{esint} % for \oiint
\ifxake%%https://math.meta.stackexchange.com/questions/9973/how-do-you-render-a-closed-surface-double-integral
\renewcommand{\oiint}{{\large\bigcirc}\kern-1.56em\iint}
\fi


\graphicspath{
  {./}
  {ximeraTutorial/}
  {basicPhilosophy/}
  {functionsOfSeveralVariables/}
  {normalVectors/}
  {lagrangeMultipliers/}
  {vectorFields/}
  {greensTheorem/}
  {shapeOfThingsToCome/}
  {dotProducts/}
  {../productAndQuotientRules/exercises/}
  {../normalVectors/exercisesParametricPlots/}
  {../continuityOfFunctionsOfSeveralVariables/exercises/}
  {../partialDerivatives/exercises/}
  {../chainRuleForFunctionsOfSeveralVariables/exercises/}
  {../commonCoordinates/exercisesCylindricalCoordinates/}
  {../commonCoordinates/exercisesSphericalCoordinates/}
  {../greensTheorem/exercisesCurlAndLineIntegrals/}
  {../greensTheorem/exercisesDivergenceAndLineIntegrals/}
  {../shapeOfThingsToCome/exercisesDivergenceTheorem/}
  {../greensTheorem/}
  {../shapeOfThingsToCome/}
}

\newcommand{\mooculus}{\textsf{\textbf{MOOC}\textnormal{\textsf{ULUS}}}}

\usepackage{tkz-euclide}\usepackage{tikz}
\usepackage{tikz-cd}
\usetikzlibrary{arrows}
\tikzset{>=stealth,commutative diagrams/.cd,
  arrow style=tikz,diagrams={>=stealth}} %% cool arrow head
\tikzset{shorten <>/.style={ shorten >=#1, shorten <=#1 } } %% allows shorter vectors

\usetikzlibrary{backgrounds} %% for boxes around graphs
\usetikzlibrary{shapes,positioning}  %% Clouds and stars
\usetikzlibrary{matrix} %% for matrix
\usepgfplotslibrary{polar} %% for polar plots
\usepgfplotslibrary{fillbetween} %% to shade area between curves in TikZ
\usetkzobj{all}
%\usepackage[makeroom]{cancel} %% for strike outs
%\usepackage{mathtools} %% for pretty underbrace % Breaks Ximera
%\usepackage{multicol}
\usepackage{pgffor} %% required for integral for loops



%% http://tex.stackexchange.com/questions/66490/drawing-a-tikz-arc-specifying-the-center
%% Draws beach ball
\tikzset{pics/carc/.style args={#1:#2:#3}{code={\draw[pic actions] (#1:#3) arc(#1:#2:#3);}}}



\usepackage{array}
\setlength{\extrarowheight}{+.1cm}   
\newdimen\digitwidth
\settowidth\digitwidth{9}
\def\divrule#1#2{
\noalign{\moveright#1\digitwidth
\vbox{\hrule width#2\digitwidth}}}





\newcommand{\RR}{\mathbb R}
\newcommand{\R}{\mathbb R}
\newcommand{\N}{\mathbb N}
\newcommand{\Z}{\mathbb Z}

\newcommand{\sagemath}{\textsf{SageMath}}


%\renewcommand{\d}{\,d\!}
\renewcommand{\d}{\mathop{}\!d}
\newcommand{\dd}[2][]{\frac{\d #1}{\d #2}}
\newcommand{\pp}[2][]{\frac{\partial #1}{\partial #2}}
\renewcommand{\l}{\ell}
\newcommand{\ddx}{\frac{d}{\d x}}

\newcommand{\zeroOverZero}{\ensuremath{\boldsymbol{\tfrac{0}{0}}}}
\newcommand{\inftyOverInfty}{\ensuremath{\boldsymbol{\tfrac{\infty}{\infty}}}}
\newcommand{\zeroOverInfty}{\ensuremath{\boldsymbol{\tfrac{0}{\infty}}}}
\newcommand{\zeroTimesInfty}{\ensuremath{\small\boldsymbol{0\cdot \infty}}}
\newcommand{\inftyMinusInfty}{\ensuremath{\small\boldsymbol{\infty - \infty}}}
\newcommand{\oneToInfty}{\ensuremath{\boldsymbol{1^\infty}}}
\newcommand{\zeroToZero}{\ensuremath{\boldsymbol{0^0}}}
\newcommand{\inftyToZero}{\ensuremath{\boldsymbol{\infty^0}}}



\newcommand{\numOverZero}{\ensuremath{\boldsymbol{\tfrac{\#}{0}}}}
\newcommand{\dfn}{\textbf}
%\newcommand{\unit}{\,\mathrm}
\newcommand{\unit}{\mathop{}\!\mathrm}
\newcommand{\eval}[1]{\bigg[ #1 \bigg]}
\newcommand{\seq}[1]{\left( #1 \right)}
\renewcommand{\epsilon}{\varepsilon}
\renewcommand{\phi}{\varphi}


\renewcommand{\iff}{\Leftrightarrow}

\DeclareMathOperator{\arccot}{arccot}
\DeclareMathOperator{\arcsec}{arcsec}
\DeclareMathOperator{\arccsc}{arccsc}
\DeclareMathOperator{\si}{Si}
\DeclareMathOperator{\scal}{scal}
\DeclareMathOperator{\sign}{sign}


%% \newcommand{\tightoverset}[2]{% for arrow vec
%%   \mathop{#2}\limits^{\vbox to -.5ex{\kern-0.75ex\hbox{$#1$}\vss}}}
\newcommand{\arrowvec}[1]{{\overset{\rightharpoonup}{#1}}}
%\renewcommand{\vec}[1]{\arrowvec{\mathbf{#1}}}
\renewcommand{\vec}[1]{{\overset{\boldsymbol{\rightharpoonup}}{\mathbf{#1}}}}
\DeclareMathOperator{\proj}{\vec{proj}}
\newcommand{\veci}{{\boldsymbol{\hat{\imath}}}}
\newcommand{\vecj}{{\boldsymbol{\hat{\jmath}}}}
\newcommand{\veck}{{\boldsymbol{\hat{k}}}}
\newcommand{\vecl}{\vec{\boldsymbol{\l}}}
\newcommand{\uvec}[1]{\mathbf{\hat{#1}}}
\newcommand{\utan}{\mathbf{\hat{t}}}
\newcommand{\unormal}{\mathbf{\hat{n}}}
\newcommand{\ubinormal}{\mathbf{\hat{b}}}

\newcommand{\dotp}{\bullet}
\newcommand{\cross}{\boldsymbol\times}
\newcommand{\grad}{\boldsymbol\nabla}
\newcommand{\divergence}{\grad\dotp}
\newcommand{\curl}{\grad\cross}
%\DeclareMathOperator{\divergence}{divergence}
%\DeclareMathOperator{\curl}[1]{\grad\cross #1}
\newcommand{\lto}{\mathop{\longrightarrow\,}\limits}

\renewcommand{\bar}{\overline}

\colorlet{textColor}{black} 
\colorlet{background}{white}
\colorlet{penColor}{blue!50!black} % Color of a curve in a plot
\colorlet{penColor2}{red!50!black}% Color of a curve in a plot
\colorlet{penColor3}{red!50!blue} % Color of a curve in a plot
\colorlet{penColor4}{green!50!black} % Color of a curve in a plot
\colorlet{penColor5}{orange!80!black} % Color of a curve in a plot
\colorlet{penColor6}{yellow!70!black} % Color of a curve in a plot
\colorlet{fill1}{penColor!20} % Color of fill in a plot
\colorlet{fill2}{penColor2!20} % Color of fill in a plot
\colorlet{fillp}{fill1} % Color of positive area
\colorlet{filln}{penColor2!20} % Color of negative area
\colorlet{fill3}{penColor3!20} % Fill
\colorlet{fill4}{penColor4!20} % Fill
\colorlet{fill5}{penColor5!20} % Fill
\colorlet{gridColor}{gray!50} % Color of grid in a plot

\newcommand{\surfaceColor}{violet}
\newcommand{\surfaceColorTwo}{redyellow}
\newcommand{\sliceColor}{greenyellow}




\pgfmathdeclarefunction{gauss}{2}{% gives gaussian
  \pgfmathparse{1/(#2*sqrt(2*pi))*exp(-((x-#1)^2)/(2*#2^2))}%
}


%%%%%%%%%%%%%
%% Vectors
%%%%%%%%%%%%%

%% Simple horiz vectors
\renewcommand{\vector}[1]{\left\langle #1\right\rangle}


%% %% Complex Horiz Vectors with angle brackets
%% \makeatletter
%% \renewcommand{\vector}[2][ , ]{\left\langle%
%%   \def\nextitem{\def\nextitem{#1}}%
%%   \@for \el:=#2\do{\nextitem\el}\right\rangle%
%% }
%% \makeatother

%% %% Vertical Vectors
%% \def\vector#1{\begin{bmatrix}\vecListA#1,,\end{bmatrix}}
%% \def\vecListA#1,{\if,#1,\else #1\cr \expandafter \vecListA \fi}

%%%%%%%%%%%%%
%% End of vectors
%%%%%%%%%%%%%

%\newcommand{\fullwidth}{}
%\newcommand{\normalwidth}{}



%% makes a snazzy t-chart for evaluating functions
%\newenvironment{tchart}{\rowcolors{2}{}{background!90!textColor}\array}{\endarray}

%%This is to help with formatting on future title pages.
\newenvironment{sectionOutcomes}{}{} 



%% Flowchart stuff
%\tikzstyle{startstop} = [rectangle, rounded corners, minimum width=3cm, minimum height=1cm,text centered, draw=black]
%\tikzstyle{question} = [rectangle, minimum width=3cm, minimum height=1cm, text centered, draw=black]
%\tikzstyle{decision} = [trapezium, trapezium left angle=70, trapezium right angle=110, minimum width=3cm, minimum height=1cm, text centered, draw=black]
%\tikzstyle{question} = [rectangle, rounded corners, minimum width=3cm, minimum height=1cm,text centered, draw=black]
%\tikzstyle{process} = [rectangle, minimum width=3cm, minimum height=1cm, text centered, draw=black]
%\tikzstyle{decision} = [trapezium, trapezium left angle=70, trapezium right angle=110, minimum width=3cm, minimum height=1cm, text centered, draw=black]


\author{Bart Snapp}

\outcome{Work in three-dimensional space.}
\outcome{Interpert implicit equations as geometric sets.}

\title[Collaborate:]{Geometric sets}

\begin{document}
\begin{abstract}
  We investigate geometric sets determined by implicit equations.
\end{abstract}
\maketitle

\textbf{Work in groups of 3--4, writing your answers on a separate
  sheet of paper.}

%% \begin{problem}
%%   Working in $\R^2$, give an implicit equation for a circle of radius
%%   $2$ centered at the origin.
%% \end{problem}


%% \begin{problem}
%%   Working in $\R^3$, give an implicit equation for a sphere of radius
%%   $2$ centered at the origin.
%% \end{problem}


%% \begin{problem}
%%   Explain the connection from your answers for the previous questions
%%   to the distance formula in $\R^n$.
%% \end{problem}


%% \begin{problem}
%%   Working in $\R^3$, give an implicit equation for a cylinder of
%%   infinite height whose intersection with the plane $z= -3$ is a circle
%%   of radius $2$ centered at the point $(-1,2,-3)$.
%% \end{problem}

%% \begin{problem}
%%   What sort of geometric object does one expect to get when looking at
%%   the solutions to a single equation in $\R^2$? Explain your reasoning.
%% \end{problem}

%% \begin{problem}
%%   What sort of geometric object does one expect to get when looking at
%%   the solutions to a single equation in $\R^3$? Explain your reasoning.
%% \end{problem}


%% \begin{problem}
%%   What sort of geometric object does one expect to get when looking at
%%   the solutions to a single equation in $\R^n$? Explain your reasoning.
%% \end{problem}


%% \begin{problem}
%%   What sort of geometric object does one expect to get when looking at
%%   the simultaneous solutions to two equations in $\R^2$? What about
%%   $\R^3$? What about $\R^n$? Explain your reasoning.
%% \end{problem}

%% \begin{problem}
%%   We claim that the equations
%%   \[
%%   \vec{p}(t)=\begin{cases}
%%     x(t) = 1+3\cos(t)\\
%%     y(t) = 2+3\sin(t)
%%   \end{cases}
%%   \]
%%   parameterize a circle of radius $3$ centered at the point $(1,2)$ as
%%   $t$ runs from $0$ to $2\pi$.
%%   \begin{enumerate}
%%     \item Write the implicit equation of a circle of radius $3$
%%       centered at $(1,2)$.
%%     \item Use your equation to confirm our claim that $\vec{p}(t)$
%%       parameterizes the desired circle.
%%     \item Let
%%       \[
%%       \vec{q}(t)=\begin{cases}
%%       x(t) = 1+3\cos(2t)\\
%%       y(t) = 2+3\sin(2t)
%%       \end{cases}
%%       \]
%%       and let $t$ run from $0$ to $\pi$. Compare and contrast
%%       $\vec{p}(t)$ and $\vec{q}(t)$.
%%   \end{enumerate}
%% \end{problem}


%% \begin{problem}
%%   We claim that the equations
%%   \[
%%   \vec{P}(\theta,\varphi)=\begin{cases}
%%     x(\theta,\varphi) = 3\cos(\theta)\sin(\varphi)\\
%%     y(\theta,\varphi) = 3\sin(\theta)\sin(\varphi)\\
%%     z(\theta,\varphi) = 3 \cos(\varphi)
%%   \end{cases}
%%   \]
%%   parameterize a sphere of radius $3$ centered at the origin as
%%   $\theta$ runs from $0$ to $2\pi$ and $\varphi$ runs from $0$ to $\pi$.
%%   \begin{enumerate}
%%     \item Write the implicit equation of a sphere of radius $3$
%%       centered at $(0,0,0)$.
%%     \item Use your equation to confirm our claim that $\vec{P}(\theta,\varphi)$
%%       parameterizes the desired sphere.
%%     \item Give values for $\theta$ and $\varphi$ corresponding to the
%%       following points: $(0,0,3)$, $(0,0,-3)$, $(3,0,0)$, and
%%       $(0,-3,0)$.
%%   \end{enumerate}
%% \end{problem}


\section{Geometry disguised as algebra}

\begin{problem}
  Consider the equations:
  \begin{align*}
    x+y &= 6\\
    x^2+y^2+z^2&=18
  \end{align*}
  Find a solution to these equations.
  \begin{hint}
    Guess-and-check is not a bad method for this problem.
  \end{hint}
\end{problem}

\begin{problem}
  Explain what the equations
  \begin{align*}
    x+y &= 6\\
    x^2+y^2+z^2&=18
  \end{align*}
  describe from a geometric point of view.
\end{problem}

\begin{problem}
  Use geometry to explain why the equations
  \begin{align*}
    x+y &= 6\\
    x^2+y^2+z^2&=18
  \end{align*}
  have exactly one solution where $x$, $y$, and $z$ are real numbers.
\end{problem}

%% \begin{problem}
%%   Consider the following parametric formula:
%%   \[
%%   \vec{p}(t)=\begin{cases}
%%     x(t) = \frac{\cos(t)+\sin(t)}{2}\\
%%     y(t) = \frac{-\cos(t)-\sin(t)}{2}\\
%%     z(t) = \frac{\cos(t)-\sin(t)}{\sqrt{2}}
%%   \end{cases}
%%   \]
%%   Without using any devices other than a pencil, paper, and your
%%   brain:
%%   \begin{enumerate}
%%   \item Does this plot a curve or a surface?
%%   \item Show this plot lives ``inside'' the set of points satisfying
%%     \[
%%     x+y=0
%%     \]
%%   \item Make your own sketch of this curve to the best of your
%%     ability. Try it a couple times. Draw it carefully.
%%   \item Give a mathematical explanation as to how you know your plot
%%     is correct.
%%   \end{enumerate}
%%   \begin{hint}
%%     Use either the distance formula or a familiar surface to help you. 
%%   \end{hint}
%% \end{problem}



\section{Thinking about a tetrahedron}


\begin{problem}
  Sketch the triangle in $\R^3$ whose vertices are the intersections
  of the plane
  \[
  20x + 15y + 12z = 60
  \]
  and the coordinate axes.
\end{problem}



\begin{problem}
  Compute the volume of the tetrahedron (triangular-based pyramid) of
  in $\R^3$ bounded by the planes $x=0$, $y=0$, $z=0$, and the plane
  $20x + 15y + 12z = 60$.
  \begin{hint}
    Recall that the volume of a cone (pyramid!) is given by:
    \[
    V = \left(\frac{1}{3}\right) (\text{area of base})(\text{height})
    \]
  \end{hint}
\end{problem}



\begin{problem}
  Compute the area of the triangle in $\R^3$ whose vertices are the intersections of the plane $20x + 15y + 12z = 60$ and the coordinate axes.
  \begin{hint}
    For now, use your old friend:
    \[
    A = \frac{bh}{2}
    \]
    and use calculus to minimize the distance between the line
    connecting $(3,0,0)$ to $(0,4,0)$, and the point $(0,0,5)$.
  \end{hint}
\end{problem}




\end{document}
