\documentclass{ximera}

%\usepackage{todonotes}

\newcommand{\todo}{}

\usepackage{esint} % for \oiint
\ifxake%%https://math.meta.stackexchange.com/questions/9973/how-do-you-render-a-closed-surface-double-integral
\renewcommand{\oiint}{{\large\bigcirc}\kern-1.56em\iint}
\fi


\graphicspath{
  {./}
  {ximeraTutorial/}
  {basicPhilosophy/}
  {functionsOfSeveralVariables/}
  {normalVectors/}
  {lagrangeMultipliers/}
  {vectorFields/}
  {greensTheorem/}
  {shapeOfThingsToCome/}
  {dotProducts/}
  {../productAndQuotientRules/exercises/}
  {../normalVectors/exercisesParametricPlots/}
  {../continuityOfFunctionsOfSeveralVariables/exercises/}
  {../partialDerivatives/exercises/}
  {../chainRuleForFunctionsOfSeveralVariables/exercises/}
  {../commonCoordinates/exercisesCylindricalCoordinates/}
  {../commonCoordinates/exercisesSphericalCoordinates/}
  {../greensTheorem/exercisesCurlAndLineIntegrals/}
  {../greensTheorem/exercisesDivergenceAndLineIntegrals/}
  {../shapeOfThingsToCome/exercisesDivergenceTheorem/}
  {../greensTheorem/}
  {../shapeOfThingsToCome/}
}

\newcommand{\mooculus}{\textsf{\textbf{MOOC}\textnormal{\textsf{ULUS}}}}

\usepackage{tkz-euclide}\usepackage{tikz}
\usepackage{tikz-cd}
\usetikzlibrary{arrows}
\tikzset{>=stealth,commutative diagrams/.cd,
  arrow style=tikz,diagrams={>=stealth}} %% cool arrow head
\tikzset{shorten <>/.style={ shorten >=#1, shorten <=#1 } } %% allows shorter vectors

\usetikzlibrary{backgrounds} %% for boxes around graphs
\usetikzlibrary{shapes,positioning}  %% Clouds and stars
\usetikzlibrary{matrix} %% for matrix
\usepgfplotslibrary{polar} %% for polar plots
\usepgfplotslibrary{fillbetween} %% to shade area between curves in TikZ
\usetkzobj{all}
%\usepackage[makeroom]{cancel} %% for strike outs
%\usepackage{mathtools} %% for pretty underbrace % Breaks Ximera
%\usepackage{multicol}
\usepackage{pgffor} %% required for integral for loops



%% http://tex.stackexchange.com/questions/66490/drawing-a-tikz-arc-specifying-the-center
%% Draws beach ball
\tikzset{pics/carc/.style args={#1:#2:#3}{code={\draw[pic actions] (#1:#3) arc(#1:#2:#3);}}}



\usepackage{array}
\setlength{\extrarowheight}{+.1cm}   
\newdimen\digitwidth
\settowidth\digitwidth{9}
\def\divrule#1#2{
\noalign{\moveright#1\digitwidth
\vbox{\hrule width#2\digitwidth}}}





\newcommand{\RR}{\mathbb R}
\newcommand{\R}{\mathbb R}
\newcommand{\N}{\mathbb N}
\newcommand{\Z}{\mathbb Z}

\newcommand{\sagemath}{\textsf{SageMath}}


%\renewcommand{\d}{\,d\!}
\renewcommand{\d}{\mathop{}\!d}
\newcommand{\dd}[2][]{\frac{\d #1}{\d #2}}
\newcommand{\pp}[2][]{\frac{\partial #1}{\partial #2}}
\renewcommand{\l}{\ell}
\newcommand{\ddx}{\frac{d}{\d x}}

\newcommand{\zeroOverZero}{\ensuremath{\boldsymbol{\tfrac{0}{0}}}}
\newcommand{\inftyOverInfty}{\ensuremath{\boldsymbol{\tfrac{\infty}{\infty}}}}
\newcommand{\zeroOverInfty}{\ensuremath{\boldsymbol{\tfrac{0}{\infty}}}}
\newcommand{\zeroTimesInfty}{\ensuremath{\small\boldsymbol{0\cdot \infty}}}
\newcommand{\inftyMinusInfty}{\ensuremath{\small\boldsymbol{\infty - \infty}}}
\newcommand{\oneToInfty}{\ensuremath{\boldsymbol{1^\infty}}}
\newcommand{\zeroToZero}{\ensuremath{\boldsymbol{0^0}}}
\newcommand{\inftyToZero}{\ensuremath{\boldsymbol{\infty^0}}}



\newcommand{\numOverZero}{\ensuremath{\boldsymbol{\tfrac{\#}{0}}}}
\newcommand{\dfn}{\textbf}
%\newcommand{\unit}{\,\mathrm}
\newcommand{\unit}{\mathop{}\!\mathrm}
\newcommand{\eval}[1]{\bigg[ #1 \bigg]}
\newcommand{\seq}[1]{\left( #1 \right)}
\renewcommand{\epsilon}{\varepsilon}
\renewcommand{\phi}{\varphi}


\renewcommand{\iff}{\Leftrightarrow}

\DeclareMathOperator{\arccot}{arccot}
\DeclareMathOperator{\arcsec}{arcsec}
\DeclareMathOperator{\arccsc}{arccsc}
\DeclareMathOperator{\si}{Si}
\DeclareMathOperator{\scal}{scal}
\DeclareMathOperator{\sign}{sign}


%% \newcommand{\tightoverset}[2]{% for arrow vec
%%   \mathop{#2}\limits^{\vbox to -.5ex{\kern-0.75ex\hbox{$#1$}\vss}}}
\newcommand{\arrowvec}[1]{{\overset{\rightharpoonup}{#1}}}
%\renewcommand{\vec}[1]{\arrowvec{\mathbf{#1}}}
\renewcommand{\vec}[1]{{\overset{\boldsymbol{\rightharpoonup}}{\mathbf{#1}}}}
\DeclareMathOperator{\proj}{\vec{proj}}
\newcommand{\veci}{{\boldsymbol{\hat{\imath}}}}
\newcommand{\vecj}{{\boldsymbol{\hat{\jmath}}}}
\newcommand{\veck}{{\boldsymbol{\hat{k}}}}
\newcommand{\vecl}{\vec{\boldsymbol{\l}}}
\newcommand{\uvec}[1]{\mathbf{\hat{#1}}}
\newcommand{\utan}{\mathbf{\hat{t}}}
\newcommand{\unormal}{\mathbf{\hat{n}}}
\newcommand{\ubinormal}{\mathbf{\hat{b}}}

\newcommand{\dotp}{\bullet}
\newcommand{\cross}{\boldsymbol\times}
\newcommand{\grad}{\boldsymbol\nabla}
\newcommand{\divergence}{\grad\dotp}
\newcommand{\curl}{\grad\cross}
%\DeclareMathOperator{\divergence}{divergence}
%\DeclareMathOperator{\curl}[1]{\grad\cross #1}
\newcommand{\lto}{\mathop{\longrightarrow\,}\limits}

\renewcommand{\bar}{\overline}

\colorlet{textColor}{black} 
\colorlet{background}{white}
\colorlet{penColor}{blue!50!black} % Color of a curve in a plot
\colorlet{penColor2}{red!50!black}% Color of a curve in a plot
\colorlet{penColor3}{red!50!blue} % Color of a curve in a plot
\colorlet{penColor4}{green!50!black} % Color of a curve in a plot
\colorlet{penColor5}{orange!80!black} % Color of a curve in a plot
\colorlet{penColor6}{yellow!70!black} % Color of a curve in a plot
\colorlet{fill1}{penColor!20} % Color of fill in a plot
\colorlet{fill2}{penColor2!20} % Color of fill in a plot
\colorlet{fillp}{fill1} % Color of positive area
\colorlet{filln}{penColor2!20} % Color of negative area
\colorlet{fill3}{penColor3!20} % Fill
\colorlet{fill4}{penColor4!20} % Fill
\colorlet{fill5}{penColor5!20} % Fill
\colorlet{gridColor}{gray!50} % Color of grid in a plot

\newcommand{\surfaceColor}{violet}
\newcommand{\surfaceColorTwo}{redyellow}
\newcommand{\sliceColor}{greenyellow}




\pgfmathdeclarefunction{gauss}{2}{% gives gaussian
  \pgfmathparse{1/(#2*sqrt(2*pi))*exp(-((x-#1)^2)/(2*#2^2))}%
}


%%%%%%%%%%%%%
%% Vectors
%%%%%%%%%%%%%

%% Simple horiz vectors
\renewcommand{\vector}[1]{\left\langle #1\right\rangle}


%% %% Complex Horiz Vectors with angle brackets
%% \makeatletter
%% \renewcommand{\vector}[2][ , ]{\left\langle%
%%   \def\nextitem{\def\nextitem{#1}}%
%%   \@for \el:=#2\do{\nextitem\el}\right\rangle%
%% }
%% \makeatother

%% %% Vertical Vectors
%% \def\vector#1{\begin{bmatrix}\vecListA#1,,\end{bmatrix}}
%% \def\vecListA#1,{\if,#1,\else #1\cr \expandafter \vecListA \fi}

%%%%%%%%%%%%%
%% End of vectors
%%%%%%%%%%%%%

%\newcommand{\fullwidth}{}
%\newcommand{\normalwidth}{}



%% makes a snazzy t-chart for evaluating functions
%\newenvironment{tchart}{\rowcolors{2}{}{background!90!textColor}\array}{\endarray}

%%This is to help with formatting on future title pages.
\newenvironment{sectionOutcomes}{}{} 



%% Flowchart stuff
%\tikzstyle{startstop} = [rectangle, rounded corners, minimum width=3cm, minimum height=1cm,text centered, draw=black]
%\tikzstyle{question} = [rectangle, minimum width=3cm, minimum height=1cm, text centered, draw=black]
%\tikzstyle{decision} = [trapezium, trapezium left angle=70, trapezium right angle=110, minimum width=3cm, minimum height=1cm, text centered, draw=black]
%\tikzstyle{question} = [rectangle, rounded corners, minimum width=3cm, minimum height=1cm,text centered, draw=black]
%\tikzstyle{process} = [rectangle, minimum width=3cm, minimum height=1cm, text centered, draw=black]
%\tikzstyle{decision} = [trapezium, trapezium left angle=70, trapezium right angle=110, minimum width=3cm, minimum height=1cm, text centered, draw=black]


\author{Jim Talamo and Alex Beckwith}
\license{Creative Commons 3.0 By-NC}


\outcome{Set up an integral that gives the length of a curve segment and evaluate it.}

\begin{document}
\begin{exercise}

An inverted conical tank has base with diameter $6\unit{m}$ and height
$9\unit{m}$. Suppose the tank is filled to a height of $6\unit{m}$
with acetone ($\rho=785 \unit{kg}/\unit{m}^3$). Find the work required
to pump the acetone out of the tank (use $g=9.8\unit{m}/\unit{s}^2$).
\begin{image}
\begin{tikzpicture}
\begin{axis}[
domain=-3:3,
xmin=-3.5, xmax=4.5,
xtick={-3,-2,-1,1,2,3},
ymin=-1, ymax=11,
axis lines =center,
xlabel=$x$, ylabel=$y$, every axis y label/.style={at=(current axis.above origin),anchor=south},
every axis x label/.style={at=(current axis.right of origin),anchor=west},
axis on top,
]

\draw[penColor,very thick,smooth,fill=fill4] (axis cs: 0,9) ellipse (300 and 10);
\draw[penColor,very thick,smooth,fill=blue,opacity=0.25] (axis cs: 0,6) ellipse (200 and 6.67);
\draw[penColor,very thick,smooth] (axis cs:2,6) arc (360:180:200 and 6.67);
\draw[penColor,very thick,dashed] (axis cs:2,6) arc (0:180:200 and 6.67);
\addplot [penColor,very thick,smooth,domain=0:3]	{3*x};
\addplot [penColor,very thick,smooth,domain=-3:0]	{-3*x};

\addplot [name path=A,domain=0:3,draw=none] {3*x};   
\draw [name path=B,draw=none] (axis cs: 3,9) arc (0:180:300 and 10);
\addplot [fill4,opacity=0.5] fill between[of=A and B];

\addplot [name path=C,domain=-3:0.01,draw=none] {-3*x};   
\draw [name path=D,draw=none] (axis cs: -3,9) arc (180:360:300 and 10);
\addplot [fill4,opacity=0.5] fill between[of=C and D];

\addplot [name path=E,domain=-3:0.01,draw=none] {-3*x};   
\draw [name path=F,draw=none]  (axis cs:2,6) arc (0:180:200 and 6.67);
\addplot [blue,opacity=0.25] fill between[of=E and F];

\draw[decoration={brace,raise=.1cm},decorate,thin] (axis cs:2,6) -- (axis cs:2,0);
\node[anchor=west] at (axis cs:2.2,3) {$6\unit{m}$};
\draw[decoration={brace,raise=.1cm},decorate,thin] (axis cs:3,9) -- (axis cs:3,0);
\node[anchor=west] at (axis cs:3.2,4.5) {$9\unit{m}$};
\addplot[thick] plot coordinates {(0,6) (2,6)};
         
\end{axis}
\end{tikzpicture}
\end{image}

\[
W= \int_{y=\answer{0}}^{y=\answer{6}} \answer{7693\pi\left(\frac{y}{3}\right)^2 (9-y)}\d y = \answer[tolerance=10]{276948\pi} \unit{J}
\]

\begin{hint}
Set the height $y=0$ at the base of the tank.  We want to use the formula:

\[ 
W = \int_{y=0}^{y=b} \rho g A(y) d(y) \d y
\]

Since $b$ is the height to which the tank is filled, $b=\answer{6}$.

Since $h$ is the height to which the water must be moved, $h=\answer{9}$.

You want to move a slice at height $y$ to a height of $9$. Letting
$d(y)$ represent the distance that a slice at height $y$ travels to
get to a height of $9$, we set $d(y) = \answer{9-y}$.

The cross-sectional area of this tank is:

\begin{multipleChoice}
\choice{is constant.}
\choice[correct]{varies depending on the height $y$ of the slice}.
\end{multipleChoice}

\begin{question}
We can find the radius of the slice in terms of $y$ by using similar triangles:

\begin{image}
\begin{tikzpicture}
\begin{axis}[
domain=-3:3,
xmin=-3.5, xmax=4.5,
xtick={-3,-2,-1,1,2,3},
ymin=-1, ymax=11,
axis lines =center,
xlabel=$x$, ylabel=$y$, every axis y label/.style={at=(current axis.above origin),anchor=south},
every axis x label/.style={at=(current axis.right of origin),anchor=west},
axis on top,
]

%top ellipse
\draw[penColor,very thick,smooth,fill=fill4] (axis cs: 0,9) ellipse (300 and 10);
\draw[penColor,very thick,smooth,fill=blue,opacity=0.25] (axis cs: 0,6) ellipse (200 and 6.67);

%middle
\draw[penColor,very thick,smooth] (axis cs:2,6) arc (360:180:200 and 6.67);
\draw[penColor,very thick,dashed] (axis cs:2,6) arc (0:180:200 and 6.67);

%slice
\draw[penColor2,very thick,smooth] (axis cs:1,3) arc (360:180:98 and 4);
\draw[penColor2,very thick,dashed] (axis cs:1,3) arc (0:180:98 and 4);

\addplot [penColor,very thick,smooth,domain=0:3]	{3*x};
\addplot [penColor,very thick,smooth,domain=-3:0]	{-3*x};

\addplot [name path=A,domain=0:3,draw=none] {3*x};   
\draw [name path=B,draw=none] (axis cs: 3,9) arc (0:180:300 and 10);
\addplot [fill4,opacity=0.5] fill between[of=A and B];

\addplot [name path=C,domain=-3:0.01,draw=none] {-3*x};   
\draw [name path=D,draw=none] (axis cs: -3,9) arc (180:360:300 and 10);
\addplot [fill4,opacity=0.5] fill between[of=C and D];

\addplot [name path=E,domain=-3:0.01,draw=none] {-3*x};   
\draw [name path=F,draw=none]  (axis cs:2,6) arc (0:180:200 and 6.67);
\addplot [blue,opacity=0.25] fill between[of=E and F];

\draw[decoration={brace,raise=.1cm},decorate,thin,penColor2] (axis cs:1,3) -- (axis cs:1,0);
\node[anchor=west,penColor2] at (axis cs:1.2,1.5) {$y$};
\node[anchor=west,penColor2] at (axis cs:.3,3.8) {$x$};
\node[anchor=west] at (axis cs:1,9.5) {$3$m};
\draw[decoration={brace,raise=.1cm},decorate,thin] (axis cs:3,9) -- (axis cs:3,0);
\node[anchor=west] at (axis cs:3.2,4.5) {$9$m};

%horizontal lines
\addplot[thick,penColor2] plot coordinates {(0,3) (1,3)};
\addplot[thick] plot coordinates {(0,9) (3,9)};

         
\end{axis}
\end{tikzpicture}
\end{image}

From similar triangles, we find $\frac{x}{y} = \answer{\frac{3}{9}}$, so $x= \answer{\frac{1}{3}y}$.

\begin{question}
The area is $A = \pi r^2$.  We see that $r$ is a horizontal distance, which we must express in terms of $y$.  Hence, $r=  \answer{\frac{1}{3} y}$ and $A=\answer{\pi \left(\frac{y}{3}\right)^2}$.

Now, substitute all of these relevant quantities into the integral.
\end{question}
\end{question}

\end{hint}

\end{exercise}
\end{document}
