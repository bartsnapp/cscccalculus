\documentclass{ximera}
%\usepackage{todonotes}

\newcommand{\todo}{}

\usepackage{esint} % for \oiint
\ifxake%%https://math.meta.stackexchange.com/questions/9973/how-do-you-render-a-closed-surface-double-integral
\renewcommand{\oiint}{{\large\bigcirc}\kern-1.56em\iint}
\fi


\graphicspath{
  {./}
  {ximeraTutorial/}
  {basicPhilosophy/}
  {functionsOfSeveralVariables/}
  {normalVectors/}
  {lagrangeMultipliers/}
  {vectorFields/}
  {greensTheorem/}
  {shapeOfThingsToCome/}
  {dotProducts/}
  {../productAndQuotientRules/exercises/}
  {../normalVectors/exercisesParametricPlots/}
  {../continuityOfFunctionsOfSeveralVariables/exercises/}
  {../partialDerivatives/exercises/}
  {../chainRuleForFunctionsOfSeveralVariables/exercises/}
  {../commonCoordinates/exercisesCylindricalCoordinates/}
  {../commonCoordinates/exercisesSphericalCoordinates/}
  {../greensTheorem/exercisesCurlAndLineIntegrals/}
  {../greensTheorem/exercisesDivergenceAndLineIntegrals/}
  {../shapeOfThingsToCome/exercisesDivergenceTheorem/}
  {../greensTheorem/}
  {../shapeOfThingsToCome/}
}

\newcommand{\mooculus}{\textsf{\textbf{MOOC}\textnormal{\textsf{ULUS}}}}

\usepackage{tkz-euclide}\usepackage{tikz}
\usepackage{tikz-cd}
\usetikzlibrary{arrows}
\tikzset{>=stealth,commutative diagrams/.cd,
  arrow style=tikz,diagrams={>=stealth}} %% cool arrow head
\tikzset{shorten <>/.style={ shorten >=#1, shorten <=#1 } } %% allows shorter vectors

\usetikzlibrary{backgrounds} %% for boxes around graphs
\usetikzlibrary{shapes,positioning}  %% Clouds and stars
\usetikzlibrary{matrix} %% for matrix
\usepgfplotslibrary{polar} %% for polar plots
\usepgfplotslibrary{fillbetween} %% to shade area between curves in TikZ
\usetkzobj{all}
%\usepackage[makeroom]{cancel} %% for strike outs
%\usepackage{mathtools} %% for pretty underbrace % Breaks Ximera
%\usepackage{multicol}
\usepackage{pgffor} %% required for integral for loops



%% http://tex.stackexchange.com/questions/66490/drawing-a-tikz-arc-specifying-the-center
%% Draws beach ball
\tikzset{pics/carc/.style args={#1:#2:#3}{code={\draw[pic actions] (#1:#3) arc(#1:#2:#3);}}}



\usepackage{array}
\setlength{\extrarowheight}{+.1cm}   
\newdimen\digitwidth
\settowidth\digitwidth{9}
\def\divrule#1#2{
\noalign{\moveright#1\digitwidth
\vbox{\hrule width#2\digitwidth}}}





\newcommand{\RR}{\mathbb R}
\newcommand{\R}{\mathbb R}
\newcommand{\N}{\mathbb N}
\newcommand{\Z}{\mathbb Z}

\newcommand{\sagemath}{\textsf{SageMath}}


%\renewcommand{\d}{\,d\!}
\renewcommand{\d}{\mathop{}\!d}
\newcommand{\dd}[2][]{\frac{\d #1}{\d #2}}
\newcommand{\pp}[2][]{\frac{\partial #1}{\partial #2}}
\renewcommand{\l}{\ell}
\newcommand{\ddx}{\frac{d}{\d x}}

\newcommand{\zeroOverZero}{\ensuremath{\boldsymbol{\tfrac{0}{0}}}}
\newcommand{\inftyOverInfty}{\ensuremath{\boldsymbol{\tfrac{\infty}{\infty}}}}
\newcommand{\zeroOverInfty}{\ensuremath{\boldsymbol{\tfrac{0}{\infty}}}}
\newcommand{\zeroTimesInfty}{\ensuremath{\small\boldsymbol{0\cdot \infty}}}
\newcommand{\inftyMinusInfty}{\ensuremath{\small\boldsymbol{\infty - \infty}}}
\newcommand{\oneToInfty}{\ensuremath{\boldsymbol{1^\infty}}}
\newcommand{\zeroToZero}{\ensuremath{\boldsymbol{0^0}}}
\newcommand{\inftyToZero}{\ensuremath{\boldsymbol{\infty^0}}}



\newcommand{\numOverZero}{\ensuremath{\boldsymbol{\tfrac{\#}{0}}}}
\newcommand{\dfn}{\textbf}
%\newcommand{\unit}{\,\mathrm}
\newcommand{\unit}{\mathop{}\!\mathrm}
\newcommand{\eval}[1]{\bigg[ #1 \bigg]}
\newcommand{\seq}[1]{\left( #1 \right)}
\renewcommand{\epsilon}{\varepsilon}
\renewcommand{\phi}{\varphi}


\renewcommand{\iff}{\Leftrightarrow}

\DeclareMathOperator{\arccot}{arccot}
\DeclareMathOperator{\arcsec}{arcsec}
\DeclareMathOperator{\arccsc}{arccsc}
\DeclareMathOperator{\si}{Si}
\DeclareMathOperator{\scal}{scal}
\DeclareMathOperator{\sign}{sign}


%% \newcommand{\tightoverset}[2]{% for arrow vec
%%   \mathop{#2}\limits^{\vbox to -.5ex{\kern-0.75ex\hbox{$#1$}\vss}}}
\newcommand{\arrowvec}[1]{{\overset{\rightharpoonup}{#1}}}
%\renewcommand{\vec}[1]{\arrowvec{\mathbf{#1}}}
\renewcommand{\vec}[1]{{\overset{\boldsymbol{\rightharpoonup}}{\mathbf{#1}}}}
\DeclareMathOperator{\proj}{\vec{proj}}
\newcommand{\veci}{{\boldsymbol{\hat{\imath}}}}
\newcommand{\vecj}{{\boldsymbol{\hat{\jmath}}}}
\newcommand{\veck}{{\boldsymbol{\hat{k}}}}
\newcommand{\vecl}{\vec{\boldsymbol{\l}}}
\newcommand{\uvec}[1]{\mathbf{\hat{#1}}}
\newcommand{\utan}{\mathbf{\hat{t}}}
\newcommand{\unormal}{\mathbf{\hat{n}}}
\newcommand{\ubinormal}{\mathbf{\hat{b}}}

\newcommand{\dotp}{\bullet}
\newcommand{\cross}{\boldsymbol\times}
\newcommand{\grad}{\boldsymbol\nabla}
\newcommand{\divergence}{\grad\dotp}
\newcommand{\curl}{\grad\cross}
%\DeclareMathOperator{\divergence}{divergence}
%\DeclareMathOperator{\curl}[1]{\grad\cross #1}
\newcommand{\lto}{\mathop{\longrightarrow\,}\limits}

\renewcommand{\bar}{\overline}

\colorlet{textColor}{black} 
\colorlet{background}{white}
\colorlet{penColor}{blue!50!black} % Color of a curve in a plot
\colorlet{penColor2}{red!50!black}% Color of a curve in a plot
\colorlet{penColor3}{red!50!blue} % Color of a curve in a plot
\colorlet{penColor4}{green!50!black} % Color of a curve in a plot
\colorlet{penColor5}{orange!80!black} % Color of a curve in a plot
\colorlet{penColor6}{yellow!70!black} % Color of a curve in a plot
\colorlet{fill1}{penColor!20} % Color of fill in a plot
\colorlet{fill2}{penColor2!20} % Color of fill in a plot
\colorlet{fillp}{fill1} % Color of positive area
\colorlet{filln}{penColor2!20} % Color of negative area
\colorlet{fill3}{penColor3!20} % Fill
\colorlet{fill4}{penColor4!20} % Fill
\colorlet{fill5}{penColor5!20} % Fill
\colorlet{gridColor}{gray!50} % Color of grid in a plot

\newcommand{\surfaceColor}{violet}
\newcommand{\surfaceColorTwo}{redyellow}
\newcommand{\sliceColor}{greenyellow}




\pgfmathdeclarefunction{gauss}{2}{% gives gaussian
  \pgfmathparse{1/(#2*sqrt(2*pi))*exp(-((x-#1)^2)/(2*#2^2))}%
}


%%%%%%%%%%%%%
%% Vectors
%%%%%%%%%%%%%

%% Simple horiz vectors
\renewcommand{\vector}[1]{\left\langle #1\right\rangle}


%% %% Complex Horiz Vectors with angle brackets
%% \makeatletter
%% \renewcommand{\vector}[2][ , ]{\left\langle%
%%   \def\nextitem{\def\nextitem{#1}}%
%%   \@for \el:=#2\do{\nextitem\el}\right\rangle%
%% }
%% \makeatother

%% %% Vertical Vectors
%% \def\vector#1{\begin{bmatrix}\vecListA#1,,\end{bmatrix}}
%% \def\vecListA#1,{\if,#1,\else #1\cr \expandafter \vecListA \fi}

%%%%%%%%%%%%%
%% End of vectors
%%%%%%%%%%%%%

%\newcommand{\fullwidth}{}
%\newcommand{\normalwidth}{}



%% makes a snazzy t-chart for evaluating functions
%\newenvironment{tchart}{\rowcolors{2}{}{background!90!textColor}\array}{\endarray}

%%This is to help with formatting on future title pages.
\newenvironment{sectionOutcomes}{}{} 



%% Flowchart stuff
%\tikzstyle{startstop} = [rectangle, rounded corners, minimum width=3cm, minimum height=1cm,text centered, draw=black]
%\tikzstyle{question} = [rectangle, minimum width=3cm, minimum height=1cm, text centered, draw=black]
%\tikzstyle{decision} = [trapezium, trapezium left angle=70, trapezium right angle=110, minimum width=3cm, minimum height=1cm, text centered, draw=black]
%\tikzstyle{question} = [rectangle, rounded corners, minimum width=3cm, minimum height=1cm,text centered, draw=black]
%\tikzstyle{process} = [rectangle, minimum width=3cm, minimum height=1cm, text centered, draw=black]
%\tikzstyle{decision} = [trapezium, trapezium left angle=70, trapezium right angle=110, minimum width=3cm, minimum height=1cm, text centered, draw=black]

\author{Jim Talamo and Bart Snapp}
\license{Creative Commons 3.0 By-NC}
\outcome{Find area as an integral with respect to $y$.}
\begin{document}

\begin{exercise}
The following is a guided exercise to find the area of the region bounded by the functions $y = x-3$, $y =
\sqrt{x-1}$ and the $x$-axis:
\begin{image}
\begin{tikzpicture}
	\begin{axis}[
            domain=0:5.5, ymax=2.8,xmax=5.5, ymin=0, xmin=0,
            axis lines =center, xlabel=$x$, ylabel=$y$,
            every axis y label/.style={at=(current axis.above origin),anchor=south},
            every axis x label/.style={at=(current axis.right of origin),anchor=west},
            axis on top,
          ]
          \addplot [ fill = fillp, smooth, samples=100, domain=(0:2)] ({1+x^2},{x}) \closedcycle;
          \addplot [draw=none,fill=background,domain=0:5.2] {x-3} \closedcycle;   
          \addplot [very thick, penColor2, smooth, samples=100, domain=(0:3)] ({1+x^2},{x});
          \addplot [draw=penColor,very thick,smooth] {x-3};
          
          \node at (axis cs:2,1.5) [penColor2] {$y=\sqrt{x-1}$};
          \node at (axis cs:4.5,0.7) [penColor] {$y=x-3$};
        \end{axis}
\end{tikzpicture}
\end{image}

\begin{exercise}
In order to express this area as a single integral, we should:

\begin{multipleChoice}
\choice{integrate with respect to $x$.}
\choice[correct]{integrate with respect to $y$.}
\end{multipleChoice}


\begin{exercise}
Since we are computing the area as an integral with respect to$x$ , we must use 
\begin{multipleChoice}
\choice{vertical slices.}
\choice[correct]{horizontal slices.}
\end{multipleChoice}


\begin{exercise}
Since we want to integrate with respect to $y$, we need to express the curves as functions of $y$, the limits of integration must be $y$ limits, and we must find $h$ in terms of $y$.

\begin{image}
\begin{tikzpicture}
	\begin{axis}[
            domain=0:5.5, ymax=2.5,xmax=5.5, ymin=0, xmin=0,
            axis lines =center, xlabel=$x$, ylabel=$y$,
            every axis y label/.style={at=(current axis.above origin),anchor=south},
            every axis x label/.style={at=(current axis.right of origin),anchor=west},
            axis on top,
          ]
          \addplot [ fill = fillp, smooth, samples=100, domain=(0:2)] ({1+x^2},{x}) \closedcycle;
          \addplot [draw=none,fill=background,domain=0:5.2] {x-3} \closedcycle;   
          \addplot [very thick, penColor2, smooth, samples=100, domain=(0:3)] ({1+x^2},{x});
          \addplot [draw=penColor,very thick,smooth] {x-3};
          
          \node at (axis cs:2,1.5) [penColor2] {$y=\sqrt{x-1}$};
          \node at (axis cs:4.5,0.7) [penColor] {$y=x-3$};

	  \addplot [draw=penColor, fill = gray!50] plot coordinates {(2,1) (2,1.1) (4, 1.1) (4,1) (2, 1)};

          \draw[decoration={brace,raise=.2cm},decorate,thin] (axis cs:2,1)--(axis cs:2,1.1);
          \node at (axis cs:1.5,1.05) {$\Delta y$};

          \draw[decoration={brace,mirror,raise=.2cm},decorate,thin] (axis cs:2,1)--(axis cs:4,1);
          \node at (axis cs:3,.7) {$h$};
        \end{axis}
\end{tikzpicture}
\end{image}

For the curve described by $y=\sqrt{x-1}$, $x= \begin{prompt} \answer{y^2+1} \end{prompt}$.

For the curve described by $y=x-3$, $x= \begin{prompt} \answer{y+3} \end{prompt}$.

The lower limit of integration is $y= \begin{prompt} \answer{0} \end{prompt}$ and the upper limit of integration is $y= \begin{prompt} \answer{2} \end{prompt}$


\begin{hint}
 The upper limit is the $y$-value where the curves intersect: 
 
   \begin{align*}
    y+3 &= y^2 +1\\
  y^2-y-2 &= 0\\
    y &= -1 \text{ or }\answer[given]{2}.
  \end{align*}
  From the picture, note that $y=-1$ is not relevant for this problem!
\end{hint}

We must now express $h$ in terms of the variable of integration!  Since $h$ is a horizontal distance: 

\begin{multipleChoice}
\choice{$h=y_{top}-y_{bot}$} 
\choice[correct]{$h=x_{right}-x_{left}$}.
\end{multipleChoice}

\begin{exercise}

The function used to determine the rightmost $x$-value, $x_{right}$ is:
\begin{multipleChoice}
\choice[correct]{$x_{right}=y+3$}
\choice{$x_{left}=y^2+1$}
\end{multipleChoice}

The function used to determine  the leftmost $x$-value, $x_{left}$ is:
\begin{multipleChoice}
\choice{$x_{right}=y+3$}
\choice[correct]{$x_{left}=y^2+1$}
\end{multipleChoice}

The height $h$ of the rectangle is thus $h=x_{right}-x_{left} = \answer{(y+3)-(y^2+1)}$.


\begin{exercise}

Thus, the area is given by:
  \[
 \int_{y=c}^{y=d} h \d y =  \int_{y=\answer{0}}^{y=\answer{2}} \answer{(y+3) - (y^2+1)} \d y = \answer{\frac{10}{3}}.
  \]
  \begin{hint}
    \begin{align*}
      \int_0^2 (y+3) - (y^2+1) \d y &= \int_0^2 -y^2+y+2 \d y\\
      &=\eval{\answer{\frac{-y^3}{3} + \frac{y^2}{2}+2y}}_0^2\\
      &=\answer{\frac{10}{3}}.
    \end{align*}
  \end{hint}

\end{exercise}
\end{exercise}
\end{exercise}
\end{exercise}
\end{exercise}
\end{exercise}
\end{document}