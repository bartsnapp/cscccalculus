\documentclass{ximera}

%\usepackage{todonotes}

\newcommand{\todo}{}

\usepackage{esint} % for \oiint
\ifxake%%https://math.meta.stackexchange.com/questions/9973/how-do-you-render-a-closed-surface-double-integral
\renewcommand{\oiint}{{\large\bigcirc}\kern-1.56em\iint}
\fi


\graphicspath{
  {./}
  {ximeraTutorial/}
  {basicPhilosophy/}
  {functionsOfSeveralVariables/}
  {normalVectors/}
  {lagrangeMultipliers/}
  {vectorFields/}
  {greensTheorem/}
  {shapeOfThingsToCome/}
  {dotProducts/}
  {../productAndQuotientRules/exercises/}
  {../normalVectors/exercisesParametricPlots/}
  {../continuityOfFunctionsOfSeveralVariables/exercises/}
  {../partialDerivatives/exercises/}
  {../chainRuleForFunctionsOfSeveralVariables/exercises/}
  {../commonCoordinates/exercisesCylindricalCoordinates/}
  {../commonCoordinates/exercisesSphericalCoordinates/}
  {../greensTheorem/exercisesCurlAndLineIntegrals/}
  {../greensTheorem/exercisesDivergenceAndLineIntegrals/}
  {../shapeOfThingsToCome/exercisesDivergenceTheorem/}
  {../greensTheorem/}
  {../shapeOfThingsToCome/}
}

\newcommand{\mooculus}{\textsf{\textbf{MOOC}\textnormal{\textsf{ULUS}}}}

\usepackage{tkz-euclide}\usepackage{tikz}
\usepackage{tikz-cd}
\usetikzlibrary{arrows}
\tikzset{>=stealth,commutative diagrams/.cd,
  arrow style=tikz,diagrams={>=stealth}} %% cool arrow head
\tikzset{shorten <>/.style={ shorten >=#1, shorten <=#1 } } %% allows shorter vectors

\usetikzlibrary{backgrounds} %% for boxes around graphs
\usetikzlibrary{shapes,positioning}  %% Clouds and stars
\usetikzlibrary{matrix} %% for matrix
\usepgfplotslibrary{polar} %% for polar plots
\usepgfplotslibrary{fillbetween} %% to shade area between curves in TikZ
\usetkzobj{all}
%\usepackage[makeroom]{cancel} %% for strike outs
%\usepackage{mathtools} %% for pretty underbrace % Breaks Ximera
%\usepackage{multicol}
\usepackage{pgffor} %% required for integral for loops



%% http://tex.stackexchange.com/questions/66490/drawing-a-tikz-arc-specifying-the-center
%% Draws beach ball
\tikzset{pics/carc/.style args={#1:#2:#3}{code={\draw[pic actions] (#1:#3) arc(#1:#2:#3);}}}



\usepackage{array}
\setlength{\extrarowheight}{+.1cm}   
\newdimen\digitwidth
\settowidth\digitwidth{9}
\def\divrule#1#2{
\noalign{\moveright#1\digitwidth
\vbox{\hrule width#2\digitwidth}}}





\newcommand{\RR}{\mathbb R}
\newcommand{\R}{\mathbb R}
\newcommand{\N}{\mathbb N}
\newcommand{\Z}{\mathbb Z}

\newcommand{\sagemath}{\textsf{SageMath}}


%\renewcommand{\d}{\,d\!}
\renewcommand{\d}{\mathop{}\!d}
\newcommand{\dd}[2][]{\frac{\d #1}{\d #2}}
\newcommand{\pp}[2][]{\frac{\partial #1}{\partial #2}}
\renewcommand{\l}{\ell}
\newcommand{\ddx}{\frac{d}{\d x}}

\newcommand{\zeroOverZero}{\ensuremath{\boldsymbol{\tfrac{0}{0}}}}
\newcommand{\inftyOverInfty}{\ensuremath{\boldsymbol{\tfrac{\infty}{\infty}}}}
\newcommand{\zeroOverInfty}{\ensuremath{\boldsymbol{\tfrac{0}{\infty}}}}
\newcommand{\zeroTimesInfty}{\ensuremath{\small\boldsymbol{0\cdot \infty}}}
\newcommand{\inftyMinusInfty}{\ensuremath{\small\boldsymbol{\infty - \infty}}}
\newcommand{\oneToInfty}{\ensuremath{\boldsymbol{1^\infty}}}
\newcommand{\zeroToZero}{\ensuremath{\boldsymbol{0^0}}}
\newcommand{\inftyToZero}{\ensuremath{\boldsymbol{\infty^0}}}



\newcommand{\numOverZero}{\ensuremath{\boldsymbol{\tfrac{\#}{0}}}}
\newcommand{\dfn}{\textbf}
%\newcommand{\unit}{\,\mathrm}
\newcommand{\unit}{\mathop{}\!\mathrm}
\newcommand{\eval}[1]{\bigg[ #1 \bigg]}
\newcommand{\seq}[1]{\left( #1 \right)}
\renewcommand{\epsilon}{\varepsilon}
\renewcommand{\phi}{\varphi}


\renewcommand{\iff}{\Leftrightarrow}

\DeclareMathOperator{\arccot}{arccot}
\DeclareMathOperator{\arcsec}{arcsec}
\DeclareMathOperator{\arccsc}{arccsc}
\DeclareMathOperator{\si}{Si}
\DeclareMathOperator{\scal}{scal}
\DeclareMathOperator{\sign}{sign}


%% \newcommand{\tightoverset}[2]{% for arrow vec
%%   \mathop{#2}\limits^{\vbox to -.5ex{\kern-0.75ex\hbox{$#1$}\vss}}}
\newcommand{\arrowvec}[1]{{\overset{\rightharpoonup}{#1}}}
%\renewcommand{\vec}[1]{\arrowvec{\mathbf{#1}}}
\renewcommand{\vec}[1]{{\overset{\boldsymbol{\rightharpoonup}}{\mathbf{#1}}}}
\DeclareMathOperator{\proj}{\vec{proj}}
\newcommand{\veci}{{\boldsymbol{\hat{\imath}}}}
\newcommand{\vecj}{{\boldsymbol{\hat{\jmath}}}}
\newcommand{\veck}{{\boldsymbol{\hat{k}}}}
\newcommand{\vecl}{\vec{\boldsymbol{\l}}}
\newcommand{\uvec}[1]{\mathbf{\hat{#1}}}
\newcommand{\utan}{\mathbf{\hat{t}}}
\newcommand{\unormal}{\mathbf{\hat{n}}}
\newcommand{\ubinormal}{\mathbf{\hat{b}}}

\newcommand{\dotp}{\bullet}
\newcommand{\cross}{\boldsymbol\times}
\newcommand{\grad}{\boldsymbol\nabla}
\newcommand{\divergence}{\grad\dotp}
\newcommand{\curl}{\grad\cross}
%\DeclareMathOperator{\divergence}{divergence}
%\DeclareMathOperator{\curl}[1]{\grad\cross #1}
\newcommand{\lto}{\mathop{\longrightarrow\,}\limits}

\renewcommand{\bar}{\overline}

\colorlet{textColor}{black} 
\colorlet{background}{white}
\colorlet{penColor}{blue!50!black} % Color of a curve in a plot
\colorlet{penColor2}{red!50!black}% Color of a curve in a plot
\colorlet{penColor3}{red!50!blue} % Color of a curve in a plot
\colorlet{penColor4}{green!50!black} % Color of a curve in a plot
\colorlet{penColor5}{orange!80!black} % Color of a curve in a plot
\colorlet{penColor6}{yellow!70!black} % Color of a curve in a plot
\colorlet{fill1}{penColor!20} % Color of fill in a plot
\colorlet{fill2}{penColor2!20} % Color of fill in a plot
\colorlet{fillp}{fill1} % Color of positive area
\colorlet{filln}{penColor2!20} % Color of negative area
\colorlet{fill3}{penColor3!20} % Fill
\colorlet{fill4}{penColor4!20} % Fill
\colorlet{fill5}{penColor5!20} % Fill
\colorlet{gridColor}{gray!50} % Color of grid in a plot

\newcommand{\surfaceColor}{violet}
\newcommand{\surfaceColorTwo}{redyellow}
\newcommand{\sliceColor}{greenyellow}




\pgfmathdeclarefunction{gauss}{2}{% gives gaussian
  \pgfmathparse{1/(#2*sqrt(2*pi))*exp(-((x-#1)^2)/(2*#2^2))}%
}


%%%%%%%%%%%%%
%% Vectors
%%%%%%%%%%%%%

%% Simple horiz vectors
\renewcommand{\vector}[1]{\left\langle #1\right\rangle}


%% %% Complex Horiz Vectors with angle brackets
%% \makeatletter
%% \renewcommand{\vector}[2][ , ]{\left\langle%
%%   \def\nextitem{\def\nextitem{#1}}%
%%   \@for \el:=#2\do{\nextitem\el}\right\rangle%
%% }
%% \makeatother

%% %% Vertical Vectors
%% \def\vector#1{\begin{bmatrix}\vecListA#1,,\end{bmatrix}}
%% \def\vecListA#1,{\if,#1,\else #1\cr \expandafter \vecListA \fi}

%%%%%%%%%%%%%
%% End of vectors
%%%%%%%%%%%%%

%\newcommand{\fullwidth}{}
%\newcommand{\normalwidth}{}



%% makes a snazzy t-chart for evaluating functions
%\newenvironment{tchart}{\rowcolors{2}{}{background!90!textColor}\array}{\endarray}

%%This is to help with formatting on future title pages.
\newenvironment{sectionOutcomes}{}{} 



%% Flowchart stuff
%\tikzstyle{startstop} = [rectangle, rounded corners, minimum width=3cm, minimum height=1cm,text centered, draw=black]
%\tikzstyle{question} = [rectangle, minimum width=3cm, minimum height=1cm, text centered, draw=black]
%\tikzstyle{decision} = [trapezium, trapezium left angle=70, trapezium right angle=110, minimum width=3cm, minimum height=1cm, text centered, draw=black]
%\tikzstyle{question} = [rectangle, rounded corners, minimum width=3cm, minimum height=1cm,text centered, draw=black]
%\tikzstyle{process} = [rectangle, minimum width=3cm, minimum height=1cm, text centered, draw=black]
%\tikzstyle{decision} = [trapezium, trapezium left angle=70, trapezium right angle=110, minimum width=3cm, minimum height=1cm, text centered, draw=black]

\author{Jim Talamo}

\outcome{Use the procedure of ``Slice, Approximate, Integrate'' to derive the washer method formula.}
\outcome{Compute volumes using the washer method.}
\outcome{Set up an integral or sum of integrals using the washer method.}

\title[Dig-In:]{The washer method}

\begin{document}
\begin{abstract}
We use the procedure of ``Slice, Approximate, Integrate" to develop the washer method to compute volumes of solids of revolution.
\end{abstract}
\maketitle


\section{The washer method}

We can slice a solid of revolution perpendicular to the axis of rotation.  We saw that we could generate the solid of revolution by considering the corresponding slices in the region of revolution in the $xy$-plane.  To illustrate the details, we start with a motivating example.  

\begin{model} Consider the region in the $xy$-plane bounded by $y=x^2-4$, $x=1$, and $y=5$, which is shown below. 

 \begin{image}
            \begin{tikzpicture}
            	\begin{axis}[
            		domain=-10:10, ymax=6.8,xmax=3.4, ymin=-.95, xmin=-3.4,
            		axis lines =center, xlabel=$x$, ylabel=$y$,
            		every axis y label/.style={at=(current axis.above origin),anchor=south},
            		every axis x label/.style={at=(current axis.right of origin),anchor=west},
            		axis on top,
            		]
                      
            	\addplot [draw=penColor,domain=0:3.1,very thick,smooth] {x^2-4};
	        \addplot [draw=penColor5,very thick,smooth] {5};
	        \addplot [draw=penColor2,very thick,smooth] coordinates {(1,-10)(1,10)};
            	\addplot [draw=penColor2,very thick,dashed] coordinates {(0,-10)(0,10)};
	                            
            	\addplot [name path=A,domain=1:3,draw=none] {5};   
            	\addplot [name path=B,domain=1:2,draw=none] {0};
	        \addplot [name path=C,domain=2:3,draw=none] {x^2-4};
            	\addplot [fillp] fill between[of=A and B];
	        \addplot [fillp] fill between[of=C and A];
	        
	         \addplot [draw=black,fill=gray!50,thick] coordinates {(1,2)(2.45,2)(2.51,2.3)(1,2.3)(1,2)};
	         
                                   
            	\node at (axis cs:2.5,6.3) [penColor] {$x=\sqrt{y+4}$};
            	\node at (axis cs:-1.5,5.5) [penColor5] {$y=5$};
		\node at (axis cs:1.55,5.5) [penColor2] {$x=1$};
	    
	      \end{axis}
            \end{tikzpicture}
            \end{image}

A solid of revolution is formed by revolving this region about the $y$-axis. 

 \begin{image}
            \begin{tikzpicture}
            	\begin{axis}[
            		domain=-10:10, ymax=6.8,xmax=3.4, ymin=-.95, xmin=-3.4,
            		axis lines =center, xlabel=$x$, ylabel=$y$,
            		every axis y label/.style={at=(current axis.above origin),anchor=south},
            		every axis x label/.style={at=(current axis.right of origin),anchor=west},
            		axis on top,
            		]
                      
            	\addplot [draw=penColor,domain=1.5:3.5,very thick,smooth] {x^2-4};
		\addplot [draw=penColor,domain=-3:-2,very thick,smooth] {x^2-4};
	         \addplot [draw=penColor2,very thick,dotted] coordinates {(1,0)(1,4.5)};
	          \addplot [draw=penColor2,very thick,dotted] coordinates {(-1,0)(-1,4.5)};
           	                            
            	%shades top
		\addplot [name path=A,domain=-2.99:2.99,draw=none,samples=100] {5+sqrt(.25- .25/9*x^2)};   
		\addplot [name path=B,domain=-2.999:2.999,draw=none,samples=100] {5-sqrt(.25- .25/9*x^2)};   
		\addplot [fillp!75] fill between[of=A and B];
		
		%shades outer part	
		\addplot [name path=C,domain=-2.999:2.999,draw=none,samples=100] {5-sqrt(.25- .25/9*x^2)};   
            	\addplot [name path=D,domain=-1.999:1.999,draw=none,samples=100] {-sqrt(.25- .25/4*x^2)};
	 	\addplot [name path=E,domain=2:3,draw=none] {x^2-4};
		\addplot [name path=F,domain=-3:-2,draw=none] {x^2-4};
	       	\addplot [fillp] fill between[of=C and D];
		\addplot [fillp] fill between[of=C and E];
		\addplot [fillp] fill between[of=C and F];
	     
                 %shades negative part	
                 \addplot [name path=G,domain=-.999:.999,draw=none,samples=100] {5+sqrt(.05- .05*x^2)};   
		\addplot [name path=H,domain=-.999:.999,draw=none,samples=100] {5-sqrt(.05- .05*x^2)};   
		\addplot [gray!20!fillp] fill between[of=G and H];
		\addplot [name path=I,domain=-.999:.999,draw=none,samples=100] {5-sqrt(.25- .25/9*x^2)};   
		\addplot [name path=J,domain=-.999:.999,draw=none,samples=100] {-sqrt(.05- .05*x^2)};   
		\addplot [gray!20!fillp] fill between[of=I and J];
		
                 %outer ellipses
                  \addplot [draw=penColor,domain=-2.999:2.999,very thick,smooth,samples=100] {5+sqrt(.25- .25/9*x^2)};
                  \addplot [draw=penColor,domain=-2.999:2.999,very thick,smooth,samples=100] {5-sqrt(.25- .25/9*x^2)};
                   \addplot [draw=penColor,domain=-1.999:1.999,very thick,smooth,samples=300,dashed] {sqrt(.25- .25/4*x^2)};
                  \addplot [draw=penColor,domain=-1.999:1.999,very thick,smooth,samples=300] {-sqrt(.25- .25/4*x^2)};
                  
                  %inner ellipses
                  \addplot [draw=penColor2,domain=-.999:.999,very thick,smooth,samples=100] {5+sqrt(.05- .05*x^2)};
                  \addplot [draw=penColor2,domain=-.999:.999,very thick,smooth,samples=100] {5-sqrt(.05- .05*x^2)};
                  \addplot [draw=penColor2,domain=-.999:.999,very thick,smooth,samples=100,dashed] {sqrt(.05- .05*x^2)};
                  \addplot [draw=penColor2,domain=-.999:.999,very thick,smooth,samples=100,dashed] {-sqrt(.05- .05*x^2)};
                  %%%%%%%%%%%%%%%%%%%%                 
            	\node at (axis cs:11,1.55) [penColor] {$x=4y^2$};
            	\node at (axis cs:15,-.9) [penColor2] {$x+4y=8$};
	    
	      \end{axis}
            \end{tikzpicture}
            \end{image}

Let's try to apply the ``Slice, Approximate, Integrate" procedure to find the volume of the solid.

\paragraph{Step 1: Slice}
The geometry of the base region suggests that it is advantageous to use horizontal slices, so we should integrate with respect to \wordChoice{\choice{$x$}\choice[correct]{$y$}}.

We draw a slice of thickness $\Delta y$ at a fixed, but unspecified $y$-value on the region of revolution.


\paragraph{Step 2: Approximate}
We approximate the slice in the region by a rectangle.

 \begin{image}
            \begin{tikzpicture}
            	\begin{axis}[
            		domain=-10:10, ymax=6.8,xmax=3.4, ymin=-.8, xmin=-3.4,
            		axis lines =center, xlabel=$x$, ylabel=$y$,
            		every axis y label/.style={at=(current axis.above origin),anchor=south},
            		every axis x label/.style={at=(current axis.right of origin),anchor=west},
            		axis on top,
            		]
                      
            	\addplot [draw=penColor,domain=0:3.1,very thick,smooth] {x^2-4};
	        \addplot [draw=penColor5,very thick,smooth] {5};
	        \addplot [draw=penColor2,very thick,smooth] coordinates {(1,-10)(1,10)};
            	\addplot [draw=penColor2,very thick,dashed] coordinates {(0,-10)(0,10)};
	                            
            	\addplot [name path=A,domain=1:3,draw=none] {5};   
            	\addplot [name path=B,domain=1:2,draw=none] {0};
	        \addplot [name path=C,domain=2:3,draw=none] {x^2-4};
            	\addplot [fillp] fill between[of=A and B];
	        \addplot [fillp] fill between[of=C and A];
	        
	         \addplot [draw=black,fill=gray!50,thick] coordinates {(1,2)(2.45,2)(2.45,2.3)(1,2.3)(1,2)};
	         
                 %draws Delta y
                 \node at (axis cs:2.8,2.1) [black] {$\Delta y$};
                 
                 %labels functions                  
            	\node at (axis cs:2.5,6.3) [penColor] {$x=\sqrt{y+4}$};
            	\node at (axis cs:-1.5,5.5) [penColor5] {$y=5$};
		\node at (axis cs:1.55,5.5) [penColor2] {$x=1$};
	    
	      \end{axis}
            \end{tikzpicture}
            \end{image}

The result of rotating the slice appears on the solid just as before.
\begin{image}
            \begin{tikzpicture}
            	\begin{axis}[
            		domain=-10:10, ymax=6.8,xmax=3.4, ymin=-.8, xmin=-3.4,
            		axis lines =center, xlabel=$x$, ylabel=$y$,
            		every axis y label/.style={at=(current axis.above origin),anchor=south},
            		every axis x label/.style={at=(current axis.right of origin),anchor=west},
            		axis on top,
            		]
                      
            	\addplot [draw=penColor,domain=2:3,very thick,smooth] {x^2-4};
		\addplot [draw=penColor,domain=-3:-2,very thick,smooth] {x^2-4};
	         \addplot [draw=penColor2,very thick,dotted] coordinates {(1,0)(1,4.5)};
	          \addplot [draw=penColor2,very thick,dotted] coordinates {(-1,0)(-1,4.5)};
           	                            
            	%shades top
		\addplot [name path=A,domain=-2.999:2.999,draw=none,samples=100] {5+sqrt(.25- .25/9*x^2)};   
		\addplot [name path=B,domain=-2.999:2.999,draw=none,samples=100] {5-sqrt(.25- .25/9*x^2)};   
		\addplot [fillp!75] fill between[of=A and B];
		
		%shades outer part	
		\addplot [name path=C,domain=-2.999:2.999,draw=none,samples=100] {5-sqrt(.25- .25/9*x^2)};   
            	\addplot [name path=D,domain=-1.999:1.999,draw=none,samples=100] {-sqrt(.25- .25/4*x^2)};
	 	\addplot [name path=E,domain=2:3,draw=none] {x^2-4};
		\addplot [name path=F,domain=-3:-2,draw=none] {x^2-4};
	       	\addplot [fillp] fill between[of=C and D];
		\addplot [fillp] fill between[of=C and E];
		\addplot [fillp] fill between[of=C and F];
	     
                 %shades negative part	
                 \addplot [name path=G,domain=-.99:.99,draw=none,samples=100] {5+sqrt(.05- .05*x^2)};   
		\addplot [name path=H,domain=-.999:.999,draw=none,samples=100] {5-sqrt(.05- .05*x^2)};   
		\addplot [gray!20!fillp] fill between[of=G and H];
		\addplot [name path=I,domain=-.999:.999,draw=none,samples=100] {5-sqrt(.25- .25/9*x^2)};   
		\addplot [name path=J,domain=-.999:.999,draw=none,samples=100] {-sqrt(.05- .05*x^2)};   
		\addplot [gray!20!fillp] fill between[of=I and J];
		
                 %outer ellipses
                  \addplot [draw=penColor,domain=-2.999:2.999,very thick,smooth,samples=100] {5+sqrt(.25- .25/9*x^2)};
                  \addplot [draw=penColor,domain=-2.999:2.999,very thick,smooth,samples=100] {5-sqrt(.25- .25/9*x^2)};
                   \addplot [draw=penColor,domain=-1.999:1.999,very thick,smooth,samples=300,dashed] {sqrt(.25- .25/4*x^2)};
                  \addplot [draw=penColor,domain=-1.999:1.999,very thick,smooth,samples=300] {-sqrt(.25- .25/4*x^2)};
                  
                  %inner ellipses
                  \addplot [draw=penColor2,domain=-.999:.999,very thick,smooth,samples=100] {5+sqrt(.05- .05*x^2)};
                  \addplot [draw=penColor2,domain=-.999:.999,very thick,smooth,samples=100] {5-sqrt(.05- .05*x^2)};
                  \addplot [draw=penColor2,domain=-.999:.999,very thick,smooth,samples=100,dashed] {sqrt(.05- .05*x^2)};
                  \addplot [draw=penColor2,domain=-.999:.999,very thick,smooth,samples=100,dashed] {-sqrt(.05- .05*x^2)};
                  %%%%%%%%%%%%%%%%%%%%
                  
                  %The revolved slice
                 %\addplot [draw=black,fill=gray!50,thick] coordinates {(1,2)(2.45,2)(2.45,2.3)(1,2.3)(1,2)};
                 \addplot [draw=black,fill=gray!50,thick] coordinates {(2.45,2)(2.45,2.3)};
                 \addplot [draw=black,fill=gray!50,thick] coordinates {(-2.45,2)(-2.45,2.3)};
                 \addplot [draw=black,domain=-2.449:2.449,thick,smooth,samples=100] {2.3+sqrt(.15- .15/6*x^2)};
                  \addplot [draw=black,domain=-2.449:2.449,thick,smooth,samples=300] {2-sqrt(.15- .15/6*x^2)};
                 \addplot [draw=black,domain=-2.449:2.449,thick,smooth,samples=100] {2.3-sqrt(.15- .15/6*x^2)};
                 \addplot [draw=black,domain=-.999:.999, thick,smooth,samples=300] {2.3+sqrt(.03- .03*x^2)};
                 \addplot [draw=black,domain=-.999:.999, thick,smooth,samples=300] {2.3-sqrt(.03- .03*x^2)}; 
                 
                 %shades edges of ellipses
                 \addplot [draw=black,domain=2.3:2.449,thick,smooth,samples=300] {2.3-sqrt(.15- .15/6*x^2)};
                 \addplot [draw=black,domain=2.3:2.449,thick,smooth,samples=300] {2.3+sqrt(.15- .15/6*x^2)};
                 \addplot [draw=black,domain=-2.449:-2.3,thick,smooth,samples=100] {2.3-sqrt(.15- .15/6*x^2)};
                 \addplot [draw=black,domain=-2.449:-2.3,thick,smooth,samples=100] {2.3+sqrt(.15- .15/6*x^2)};
                 
                   
                 %shades slice
                 %shades top
		\addplot [name path=K,domain=-2.449:2.449,draw=none,samples=100] {2.3-sqrt(.15- .15/6*x^2)};   
		\addplot [name path=L,domain=-2.449:2.449,draw=none,samples=100] {2-sqrt(.15- .15/6*x^2)};   
		\addplot [name path=M,domain=-2.449:2.449,draw=none,samples=100] {2.3+sqrt(.15- .15/6*x^2)};  
		\addplot [fill=gray!50] fill between[of=M and K];
		\addplot [fill=gray!70] fill between[of=K and L];
		
		%restores color of hole
		\addplot [name path=M,domain=-.999:.999,draw=none,samples=100] {2.3+sqrt(.05- .05*x^2)};   
		\addplot [name path=N,domain=-1:1,draw=none,samples=100] {2.3-sqrt(.05- .05*x^2)};   
		\addplot [gray!20!fillp] fill between[of=M and N];
		
		
		                    
            	\node at (axis cs:11,1.55) [penColor] {$x=4y^2$};
            	\node at (axis cs:15,-.9) [penColor2] {$x+4y=8$};
	    
	      \end{axis}
            \end{tikzpicture}
            \end{image}

Notice that the outer radius and inner radius are finite, but the thickness $\Delta y$ is thought of as quite small.  To find the volume of the hollow cylinder, recall

\begin{align*}
\left< \textrm{Volume of the hollow cylinder } \right>  &=   \\
 \left< \textrm{ Volume of the outer} \right. & \left. \textrm{cylinder } \right> - \left< \textrm{ Volume of the inner cylinder } \right>.
\end{align*}
 
The outer cylinder has radius $R(y)$ and its volume is $\Delta V_{outer} = \pi [R(y)]^2 \Delta y$, while the volume of the inner cylinder has radius $r(y)$, so its volume is $\Delta V_{inner} = \pi [r(y)]^2 \Delta$.  Here, we have explicitly noted that these radii will certainly depend at which $y$-value they are drawn, but to make the notation cleaner the rest of the way, we will only write $R$ and $r$ instead of $R(y)$ and $r(y)$.  The volume of our hollow cylinder is thus
 
\[
\Delta V = \pi R^2 \Delta y -\pi r^2 \Delta y=\pi \left(R^2-r^2\right) \Delta y 
\]

Our goal is now to express both $R$ and $r$ in terms of the unspecified $y$-value of the slice.  Using the picture is helpful to think about how to find $R$ and $r$.

\begin{itemize}
\item The outer radius $R$ is the distance from the axis of rotation to the \wordChoice{\choice[correct]{outer curve}\choice{inner curve}}.
\item The inner radius $r$ is the distance from the axis of rotation to the \wordChoice{\choice{outer curve}\choice[correct]{inner curve}}.
\end{itemize}

Also, both of these distances are \wordChoice{\choice[correct]{horizontal}\choice{vertical} distances.}

We now label these on the image of the region of rotation.

%%%%%%%%%%%%%%%%%%%%%
 \begin{image}
            \begin{tikzpicture}
            	\begin{axis}[
            		domain=-10:10, ymax=6.8,xmax=4.2, ymin=-.8, xmin=-.8,
            		axis lines =center, xlabel=$x$, ylabel=$y$,
            		every axis y label/.style={at=(current axis.above origin),anchor=south},
            		every axis x label/.style={at=(current axis.right of origin),anchor=west},
            		axis on top,
            		]
                      
            	\addplot [draw=penColor,domain=0:9,very thick,smooth] {x^2-4};
	        \addplot [draw=penColor5,very thick,smooth] {5};
	        \addplot [draw=penColor2,very thick,smooth] coordinates {(1,-10)(1,10)};
            	\addplot [draw=penColor2,very thick,dashed] coordinates {(0,-10)(0,10)};
	                            
            	\addplot [name path=A,domain=1:3,draw=none,samples=100] {5};   
            	\addplot [name path=B,domain=1:2,draw=none,samples=100] {0};
	        \addplot [name path=C,domain=2:3,draw=none] {x^2-4};
            	\addplot [fillp] fill between[of=A and B];
	        \addplot [fillp] fill between[of=C and A];
	        
	         \addplot [draw=black,fill=gray!50,thick] coordinates {(1,2)(2.45,2)(2.45,2.3)(1,2.3)(1,2)};
	         
                 %draws Delta y
                 \node at (axis cs:2.8,2.1) [black] {$\Delta y$};
                 
                 %Draws R and r
                  \addplot [draw=penColor,thick] coordinates {(0,3)(2.64,3)};
                  \node at (axis cs:1.8,3.3) [penColor] {$R$};
                  \addplot [draw=penColor,thick] coordinates {(0,2.5)(1,2.5)};
                   \node at (axis cs:1.5,2.7) [penColor] {$r$};
                   
                 %labels functions                  
            	\node at (axis cs:3,1) [penColor] {$x=\sqrt{y+4}$};
            	\node at (axis cs:3.5,4.6) [penColor5] {$y=5$};
		\node at (axis cs:1.5,5.4) [penColor2] {$x=1$};
	    
	      \end{axis}
            \end{tikzpicture}
            \end{image}
            
We can find both $R$ and $r$ now the way we always find horizontal distances.

For the outer radius $R$:

\begin{itemize}
\item The righthand curve is given by \wordChoice{\choice[correct]{$x_{right} = \sqrt{y+4}$}\choice{$x_{right} = 1$}\choice{$x_{right} = 0$}}
\item The lefthand curve is given by \wordChoice{\choice{$x_{left} = \sqrt{y+4}$}\choice{$x_{left} = 1$}\choice[correct]{$x_{left} = 0$}}
\end{itemize}

Thus $R = x_{right}-x_{left} = \answer[given]{\sqrt{y+4}-0}$.
            
Following similar logic for the inner radius $r$ gives $r = x_{right}-x_{left} = \answer[given]{1-0}$.   
   
The volume of our single approximate slice is thus

\[
\Delta V = \pi\left[\left(\sqrt{y+4}\right)^2-(1)^2\right]\Delta y = \pi(y+3)\Delta y.
\]   
   
and the approximate total volume using $n$ slices is found by adding up the volumes of each slice, so
   
\[
V = \sum_{k=1}^n \pi(y_k+3)\Delta y_k,
\]      

where $y_k$ is the $y$-value chosen for the $k$-th slice and $\Delta y_k$ is the thickness of the slice.
   
\paragraph{Step 3: Integrate}
In order to find the exact volume, we simultaneously must shrink the width of our slices while adding all of the volumes together.  As usual, the definite integral allows us to do this, and we may write

\[
V= \int_{y=0}^{y=5} \pi(y+3) \d y 
\]    
Evaluating this integral gives that the total volume is $\answer[given]{\frac{55}{2}\pi}$.   
\end{model}
%
%%%%%NEW SECTION%%%%%%%%%%
\section{The washer method formula}
Let's generalize the ideas in the above example.  First, note that we slice the region of revolution \emph{perpendicular} to the axis of revolution, and we approximate each slice by a rectangle.  We call the slice obtained this way a \emph{washer}.  If the washer is not hollow (i.e. $r=0$), it is sometimes referred to as a \emph{disk}.  Washers are characterized by finite inner and outer radii but infinitesimal height.  
We now summarize the results of the above argument.

\begin{formula}
Suppose that a region in the $xy$-plane has a continuous boundary and that a solid of revolution is formed by revolving the region about a vertical or horizontal line in the $xy$-plane that does not intersect the region.  

\begin{itemize}
\item If slices taken perpendicular to the axis of revolution are vertical, then the volume of the solid of revolution is given by

\[
V=\int_{x=a}^{x=b} \pi(R^2-r^2) \d x. 
\]
\item If slices taken perpendicular to the axis of revolution are horizontal, then the volume of the solid of revolution is given by

\[
 V=\int_{y=c}^{y=d} \pi(R^2-r^2) \d y. 
\]
\end{itemize}

In both cases, the outer radius $R$ is the distance from the axis of rotation to the outer curve and the inner radius $r$ is the distance from the axis of rotation to the inner curve.  These must be expressed with respect to the variable of integration.

\end{formula}   
   
At the risk of belaboring an important point, the variable of integration is determined by the requirement that the slices be perpendicular to the axis of rotation. Indeed, this requirement will allow us to determine if the slices are vertical or horizontal.  Just as in the previous applications, if we have vertical slices, we will integrate with respect to $x$, and if we have horizontal slices, we will integrate with respect to $y$.

Now, let's see how this formula works in action by considering an example where we take the same region and revolve it about different lines.

\begin{example}

Let $R$ be the region in the $xy$-plane bounded by $y=0$, $y=\sqrt{x}$, and $x=2$.  

\begin{question} Use the Washer Method to set up an integral that gives the volume of the solid of revolution when $R$ is revolved about the following line $x=4$.

\begin{explanation}
When we use the Washer Method, the slices are \wordChoice{\choice[correct]{perpendicular}\choice{parallel}} to the axis of rotation.  This means that the slices are horizontal and we must integrate with respect to $y$.  We draw and label a picture, making sure to describe all curves using functions of $y$.


            \begin{image}
            \begin{tikzpicture}
            	\begin{axis}[
            		domain=-.4:5.4, ymax=2.4,xmax=5.4, ymin=-0.4, xmin=-.4,
            		axis lines =center, xlabel=$x$, ylabel=$y$,
            		every axis y label/.style={at=(current axis.above origin),anchor=south},
            		every axis x label/.style={at=(current axis.right of origin),anchor=west},
            		axis on top,
            		]
                      
            	\addplot [draw=penColor,very thick,smooth] {0};
            	\addplot [draw=penColor2,very thick,smooth,samples=100] {sqrt(x)};
		\addplot [draw=penColor4,very thick,smooth] coordinates {(2,-4)(2,12)};
		\addplot [draw=penColor5,very thick,dotted] coordinates {(4,-4)(4,12)};
                       
            	\addplot [name path=A,domain=0:2,draw=none,samples=100] {sqrt(x)};   
            	\addplot [name path=B,domain=0:2,draw=none] {0};
            	\addplot [fillp] fill between[of=A and B];
	         
	         \node at (axis cs:1,1.4) [penColor2] {$x=y^2$};       
            	\node at (axis cs:4.5,1.2) [penColor5] {$x=4$};
		\node at (axis cs:3,.15) [penColor] {$y=0$};
		\node at (axis cs:1.4,2.3) [penColor4] {$x=2$};
		
		%Daw R, r
		\addplot [draw=penColor3,very thick,smooth] coordinates {(1,1)(4,1)};
		\addplot [draw=penColor3,very thick,smooth] coordinates {(2,.6)(4,.6)};
		\node at (axis cs:2.5,1.15) [penColor3] {$R$};
		\node at (axis cs:3,.7) [penColor3] {$r$};
		
            	\end{axis}
            \end{tikzpicture}
            \end{image}



Since we must integrate with respect to $y$, we will use the result

\[V = \int_{y=c}^{y=d} \pi\left(R^2-r^2\right) \d y \]

to set up the volume.  We must now find the limits of integration and express the outer radius $R$ and inner radius $r$ in terms of the variable of integration $y$. 

The limits of integration are: $c = \answer[given]{0}$ and $d = \answer[given]{\sqrt{2}}$. 

We see from the picture that both $R$ and $r$ are \wordChoice{\choice{vertical}\choice[correct]{horizontal}} distances an we can compute both by taking $x_{right}-x_{left}$.  This gives $R= \answer[given]{4-y^2}$ and $r= \answer[given]{4-2}$.

Thus, the integral that expresses the volume of the solid of revolution is
        
	\[
	V= \int_{y=\answer[given]{0}}^{y=\answer[given]{\sqrt{2}}}
	\answer[given]{\pi \left[(4-y^2)^2-(2)^2\right]}\d y.
	\]

\end{explanation}

\end{question}

\begin{question} Set up an integral that gives the volume when the region is revolved about the line $y=-1$.

\begin{explanation}
When we use the Washer Method, the slices are \wordChoice{\choice[correct]{perpendicular}\choice{parallel}} to the axis of rotation. This means that the slices are \wordChoice{\choice[correct]{vertical}\choice{horizontal}} and we should integrate with respect to \wordChoice{\choice[correct]{$x$}\choice{$y$}}.

We draw and label a picture, making sure to describe all curves by functions of $x$.

            \begin{image}
            \begin{tikzpicture}
            	\begin{axis}[
            		domain=-.4:5.4, ymax=2.4,xmax=3.4, ymin=-1.4, xmin=-.4,
            		axis lines =center, xlabel=$x$, ylabel=$y$,
            		every axis y label/.style={at=(current axis.above origin),anchor=south},
            		every axis x label/.style={at=(current axis.right of origin),anchor=west},
            		axis on top,
            		]
                      
            	\addplot [draw=penColor,very thick,smooth] {0};
            	\addplot [draw=penColor2,very thick,smooth,samples=100] {sqrt(x)};
		\addplot [draw=penColor4,very thick,smooth] coordinates {(2,-4)(2,12)};
	
                       
            	\addplot [name path=A,domain=0:2,draw=none,samples=100] {sqrt(x)};   
            	\addplot [name path=B,domain=0:2,draw=none] {0};
            	\addplot [fillp] fill between[of=A and B];
	         
	         \node at (axis cs:1,1.4) [penColor2] {$y=\sqrt{x}$};       
            	\node at (axis cs:3,.2) [penColor] {$y=0$};
		\node at (axis cs:1.6,2) [penColor4] {$x=2$};
		
		%axis of rotation and label
		\addplot [draw=penColor5,very thick,dotted] coordinates {(-10,-1)(10,-1)};
		\node at (axis cs:2.5,-1.2) [penColor5] {$y=-1$};
		
		%Daw R, r
		\addplot [draw=penColor3,very thick,smooth] coordinates {(1,-1)(1,1)};
		\addplot [draw=penColor3,very thick,smooth] coordinates {(.6,-1)(.6,0)};
		\node at (axis cs:1.2,.4) [penColor3] {$R$};
		\node at (axis cs:.7,-.5) [penColor3] {$r$};
		
            	\end{axis}
            \end{tikzpicture}
            \end{image}



Since we must integrate with respect to $x$, we will use the result

\[V = \int_{x=a}^{x=b} \pi\left(R^2-r^2\right) \d x. \]

We must now find the limits of integration as express the outer radius $R$ and the inner radius $r$ in terms of the variable of integration $x$. 

The limits of integration are: $a = \answer[given]{0}$ and $b = \answer[given]{2}$. 
            
 We see from the picture that both $R$ and $r$ are \wordChoice{\choice[correct]{vertical}\choice{horizontal}} distances, so we can compute both by taking $y_{top}-y_{bot}$.  
 
 Since the bottom $y$-values lie on the axis of rotation $y=-1$, $R= \answer[given]{\sqrt{x}-(-1)}$ and $r= \answer[given]{0-(-1)}$.

Thus, the volume integral to be evaluated is
            
	\[
	V= \int_{x=\answer[given]{0}}^{x=\answer[given]{2}} \answer[given]{\pi \left[\left(1+\sqrt{x}\right)^2-(1)^2\right]}\d x.
	\]


\end{explanation}
\end{question}

\end{example}   
   

\end{document}

