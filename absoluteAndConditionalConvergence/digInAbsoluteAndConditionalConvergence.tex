\documentclass{ximera}

%\usepackage{todonotes}

\newcommand{\todo}{}

\usepackage{esint} % for \oiint
\ifxake%%https://math.meta.stackexchange.com/questions/9973/how-do-you-render-a-closed-surface-double-integral
\renewcommand{\oiint}{{\large\bigcirc}\kern-1.56em\iint}
\fi


\graphicspath{
  {./}
  {ximeraTutorial/}
  {basicPhilosophy/}
  {functionsOfSeveralVariables/}
  {normalVectors/}
  {lagrangeMultipliers/}
  {vectorFields/}
  {greensTheorem/}
  {shapeOfThingsToCome/}
  {dotProducts/}
  {../productAndQuotientRules/exercises/}
  {../normalVectors/exercisesParametricPlots/}
  {../continuityOfFunctionsOfSeveralVariables/exercises/}
  {../partialDerivatives/exercises/}
  {../chainRuleForFunctionsOfSeveralVariables/exercises/}
  {../commonCoordinates/exercisesCylindricalCoordinates/}
  {../commonCoordinates/exercisesSphericalCoordinates/}
  {../greensTheorem/exercisesCurlAndLineIntegrals/}
  {../greensTheorem/exercisesDivergenceAndLineIntegrals/}
  {../shapeOfThingsToCome/exercisesDivergenceTheorem/}
  {../greensTheorem/}
  {../shapeOfThingsToCome/}
}

\newcommand{\mooculus}{\textsf{\textbf{MOOC}\textnormal{\textsf{ULUS}}}}

\usepackage{tkz-euclide}\usepackage{tikz}
\usepackage{tikz-cd}
\usetikzlibrary{arrows}
\tikzset{>=stealth,commutative diagrams/.cd,
  arrow style=tikz,diagrams={>=stealth}} %% cool arrow head
\tikzset{shorten <>/.style={ shorten >=#1, shorten <=#1 } } %% allows shorter vectors

\usetikzlibrary{backgrounds} %% for boxes around graphs
\usetikzlibrary{shapes,positioning}  %% Clouds and stars
\usetikzlibrary{matrix} %% for matrix
\usepgfplotslibrary{polar} %% for polar plots
\usepgfplotslibrary{fillbetween} %% to shade area between curves in TikZ
\usetkzobj{all}
%\usepackage[makeroom]{cancel} %% for strike outs
%\usepackage{mathtools} %% for pretty underbrace % Breaks Ximera
%\usepackage{multicol}
\usepackage{pgffor} %% required for integral for loops



%% http://tex.stackexchange.com/questions/66490/drawing-a-tikz-arc-specifying-the-center
%% Draws beach ball
\tikzset{pics/carc/.style args={#1:#2:#3}{code={\draw[pic actions] (#1:#3) arc(#1:#2:#3);}}}



\usepackage{array}
\setlength{\extrarowheight}{+.1cm}   
\newdimen\digitwidth
\settowidth\digitwidth{9}
\def\divrule#1#2{
\noalign{\moveright#1\digitwidth
\vbox{\hrule width#2\digitwidth}}}





\newcommand{\RR}{\mathbb R}
\newcommand{\R}{\mathbb R}
\newcommand{\N}{\mathbb N}
\newcommand{\Z}{\mathbb Z}

\newcommand{\sagemath}{\textsf{SageMath}}


%\renewcommand{\d}{\,d\!}
\renewcommand{\d}{\mathop{}\!d}
\newcommand{\dd}[2][]{\frac{\d #1}{\d #2}}
\newcommand{\pp}[2][]{\frac{\partial #1}{\partial #2}}
\renewcommand{\l}{\ell}
\newcommand{\ddx}{\frac{d}{\d x}}

\newcommand{\zeroOverZero}{\ensuremath{\boldsymbol{\tfrac{0}{0}}}}
\newcommand{\inftyOverInfty}{\ensuremath{\boldsymbol{\tfrac{\infty}{\infty}}}}
\newcommand{\zeroOverInfty}{\ensuremath{\boldsymbol{\tfrac{0}{\infty}}}}
\newcommand{\zeroTimesInfty}{\ensuremath{\small\boldsymbol{0\cdot \infty}}}
\newcommand{\inftyMinusInfty}{\ensuremath{\small\boldsymbol{\infty - \infty}}}
\newcommand{\oneToInfty}{\ensuremath{\boldsymbol{1^\infty}}}
\newcommand{\zeroToZero}{\ensuremath{\boldsymbol{0^0}}}
\newcommand{\inftyToZero}{\ensuremath{\boldsymbol{\infty^0}}}



\newcommand{\numOverZero}{\ensuremath{\boldsymbol{\tfrac{\#}{0}}}}
\newcommand{\dfn}{\textbf}
%\newcommand{\unit}{\,\mathrm}
\newcommand{\unit}{\mathop{}\!\mathrm}
\newcommand{\eval}[1]{\bigg[ #1 \bigg]}
\newcommand{\seq}[1]{\left( #1 \right)}
\renewcommand{\epsilon}{\varepsilon}
\renewcommand{\phi}{\varphi}


\renewcommand{\iff}{\Leftrightarrow}

\DeclareMathOperator{\arccot}{arccot}
\DeclareMathOperator{\arcsec}{arcsec}
\DeclareMathOperator{\arccsc}{arccsc}
\DeclareMathOperator{\si}{Si}
\DeclareMathOperator{\scal}{scal}
\DeclareMathOperator{\sign}{sign}


%% \newcommand{\tightoverset}[2]{% for arrow vec
%%   \mathop{#2}\limits^{\vbox to -.5ex{\kern-0.75ex\hbox{$#1$}\vss}}}
\newcommand{\arrowvec}[1]{{\overset{\rightharpoonup}{#1}}}
%\renewcommand{\vec}[1]{\arrowvec{\mathbf{#1}}}
\renewcommand{\vec}[1]{{\overset{\boldsymbol{\rightharpoonup}}{\mathbf{#1}}}}
\DeclareMathOperator{\proj}{\vec{proj}}
\newcommand{\veci}{{\boldsymbol{\hat{\imath}}}}
\newcommand{\vecj}{{\boldsymbol{\hat{\jmath}}}}
\newcommand{\veck}{{\boldsymbol{\hat{k}}}}
\newcommand{\vecl}{\vec{\boldsymbol{\l}}}
\newcommand{\uvec}[1]{\mathbf{\hat{#1}}}
\newcommand{\utan}{\mathbf{\hat{t}}}
\newcommand{\unormal}{\mathbf{\hat{n}}}
\newcommand{\ubinormal}{\mathbf{\hat{b}}}

\newcommand{\dotp}{\bullet}
\newcommand{\cross}{\boldsymbol\times}
\newcommand{\grad}{\boldsymbol\nabla}
\newcommand{\divergence}{\grad\dotp}
\newcommand{\curl}{\grad\cross}
%\DeclareMathOperator{\divergence}{divergence}
%\DeclareMathOperator{\curl}[1]{\grad\cross #1}
\newcommand{\lto}{\mathop{\longrightarrow\,}\limits}

\renewcommand{\bar}{\overline}

\colorlet{textColor}{black} 
\colorlet{background}{white}
\colorlet{penColor}{blue!50!black} % Color of a curve in a plot
\colorlet{penColor2}{red!50!black}% Color of a curve in a plot
\colorlet{penColor3}{red!50!blue} % Color of a curve in a plot
\colorlet{penColor4}{green!50!black} % Color of a curve in a plot
\colorlet{penColor5}{orange!80!black} % Color of a curve in a plot
\colorlet{penColor6}{yellow!70!black} % Color of a curve in a plot
\colorlet{fill1}{penColor!20} % Color of fill in a plot
\colorlet{fill2}{penColor2!20} % Color of fill in a plot
\colorlet{fillp}{fill1} % Color of positive area
\colorlet{filln}{penColor2!20} % Color of negative area
\colorlet{fill3}{penColor3!20} % Fill
\colorlet{fill4}{penColor4!20} % Fill
\colorlet{fill5}{penColor5!20} % Fill
\colorlet{gridColor}{gray!50} % Color of grid in a plot

\newcommand{\surfaceColor}{violet}
\newcommand{\surfaceColorTwo}{redyellow}
\newcommand{\sliceColor}{greenyellow}




\pgfmathdeclarefunction{gauss}{2}{% gives gaussian
  \pgfmathparse{1/(#2*sqrt(2*pi))*exp(-((x-#1)^2)/(2*#2^2))}%
}


%%%%%%%%%%%%%
%% Vectors
%%%%%%%%%%%%%

%% Simple horiz vectors
\renewcommand{\vector}[1]{\left\langle #1\right\rangle}


%% %% Complex Horiz Vectors with angle brackets
%% \makeatletter
%% \renewcommand{\vector}[2][ , ]{\left\langle%
%%   \def\nextitem{\def\nextitem{#1}}%
%%   \@for \el:=#2\do{\nextitem\el}\right\rangle%
%% }
%% \makeatother

%% %% Vertical Vectors
%% \def\vector#1{\begin{bmatrix}\vecListA#1,,\end{bmatrix}}
%% \def\vecListA#1,{\if,#1,\else #1\cr \expandafter \vecListA \fi}

%%%%%%%%%%%%%
%% End of vectors
%%%%%%%%%%%%%

%\newcommand{\fullwidth}{}
%\newcommand{\normalwidth}{}



%% makes a snazzy t-chart for evaluating functions
%\newenvironment{tchart}{\rowcolors{2}{}{background!90!textColor}\array}{\endarray}

%%This is to help with formatting on future title pages.
\newenvironment{sectionOutcomes}{}{} 



%% Flowchart stuff
%\tikzstyle{startstop} = [rectangle, rounded corners, minimum width=3cm, minimum height=1cm,text centered, draw=black]
%\tikzstyle{question} = [rectangle, minimum width=3cm, minimum height=1cm, text centered, draw=black]
%\tikzstyle{decision} = [trapezium, trapezium left angle=70, trapezium right angle=110, minimum width=3cm, minimum height=1cm, text centered, draw=black]
%\tikzstyle{question} = [rectangle, rounded corners, minimum width=3cm, minimum height=1cm,text centered, draw=black]
%\tikzstyle{process} = [rectangle, minimum width=3cm, minimum height=1cm, text centered, draw=black]
%\tikzstyle{decision} = [trapezium, trapezium left angle=70, trapezium right angle=110, minimum width=3cm, minimum height=1cm, text centered, draw=black]


\author{Tom Needham}
\outcome{Determine if a series converges absolutely.}
\outcome{Determine if a series converges conditionally.}

\title[Dig-In:]{Absolute and Conditional Convergence}

\begin{document}
\begin{abstract}
The basic question we wish to answer about a series is whether or not the series converges. If a series has both positive and negative terms, we can refine this question and ask whether or not the series converges when all terms are replaced by their absolute values. This is the distinction between absolute and conditional convergence, which we explore in this section.
\end{abstract}
\maketitle


Recall that the basic question about a series that we seek to answer is ``does the series converge?" It turns out that if the series contains negative terms, there is an interesting refinement of this question. This is illustrated by the following example. 

\begin{example}
We have seen that the alternating harmonic series 
$$
\sum_{n=1}^\infty \frac{(-1)^n}{n}
$$
converges. On the other hand, if we construct a new series by taking the absolute value of each term, we obtain
$$
\sum_{n=1}^\infty \left| \frac{(-1)^n}{n} \right| = \sum_{n=1}^\infty \frac{1}{n}.
$$
That is, we obtain the standard harmonic series, which is one of our favorite examples of a \emph{divergent} series.
\end{example}

This example shows that it is interesting to consider the role that negative terms play in the convergence of a series. 

\begin{definition}\index{absolutley convergent}\index{conditionally convergent}\hfil
\begin{itemize}
\item A series $\sum  a_n$ \dfn{converges absolutely} if $\sum  |a_n|$ converges.
\item A series $\sum  a_n$ \dfn{converges conditionally} if $\sum  a_n$ converges but $\sum  |a_n|$ diverges.
\end{itemize}
\end{definition}

We then refine the basic question about a series (``does the series converge or diverge?") to the following, more subtle, question: ``does the series converge absolutely, converge conditionally, or diverge?"

\begin{question}
  Does the series
  \[
  \sum_{n=1}^\infty (-1)^n\frac{n+3}{n^2+2n+5}
  \]
  converge absolutely, converge conditionally, or diverge?
  \begin{prompt}
    \begin{multipleChoice}
      \choice[correct]{The series converges conditionally.}
      \choice{The series converges absolutely.}
      \choice{The series diverges.}
    \end{multipleChoice}
    \begin{feedback}
      We can show the series
      \[
      \sum_{n=1}^\infty \left|(-1)^n\frac{n+3}{n^2+2n+5}\right|=
      \sum_{n=1}^\infty \frac{n+3}{n^2+2n+5}
      \]
      diverges using the limit comparison test, comparing with $1/n$.
      The series
      \[
      \sum_{n=1}^\infty (-1)^n\frac{n+3}{n^2+2n+5}
      \]
      converges using the alternating series test; we conclude the
      series converges conditionally.
    \end{feedback}
      \end{prompt}
    \end{question}
  
  \begin{question}
    Does the series
    \[
    \sum_{n=1}^\infty (-1)^n\frac{n^2+2n+5}{2^n}
    \]
    converge absolutely, converge conditionally, or diverge?
    \begin{prompt}
      \begin{multipleChoice}
        \choice{The series converges conditionally.}
        \choice[correct]{The series converges absolutely.}
        \choice{The series diverges.}
      \end{multipleChoice}
      \begin{feedback}
        We can show the series
        \[
        \sum_{n=1}^\infty \left|(-1)^n\frac{n^2+2n+5}{2^n}\right|=\sum_{n=1}^\infty \frac{n^2+2n+5}{2^n}
        \]
        converges using the ratio test.  Therefore we conclude
        \[
        \sum_{n=1}^\infty (-1)^n\frac{n^2+2n+5}{2^n}
        \]
        converges absolutely.
      \end{feedback}
    \end{prompt}
%    \begin{question}
%      Does the series
%      \[
%      \sum_{n=3}^\infty (-1)^n\frac{3n-3}{5n-10}
%      \]
%      converge absolutely, converge conditionally, or diverge?
%      \begin{prompt}
%        \begin{multipleChoice}
%          \choice{The series converges conditionally.}
%          \choice{The series converges absolutely.}
%          \choice[correct]{The series diverges.}
%        \end{multipleChoice}
%        \begin{feedback}
%	  The series
%          \[
%          \sum_{n=3}^\infty \left|(-1)^n\frac{3n-3}{5n-10}\right| =
%          \sum_{n=3}^\infty \frac{3n-3}{5n-10}
%          \]
%          diverges using the divergence test, so it does not converge
%          absolutely.  The series
%          \[
%          \sum_{n=3}^\infty (-1)^n\frac{3n-3}{5n-10}
%          \]
%          fails the conditions of the alternating series test as
%          $(3n-3)/(5n-10)$ does not approach $0$ as $n\to\infty$. We
%          can state further that this series diverges; as
%          $n\to\infty$, the series effectively adds and subtracts
%          $3/5$ over and over. This causes the sequence of partial
%          sums to oscillate and not converge.  Therefore the series
%          diverges.
%        \end{feedback}
%      \end{prompt}
%    \end{question}
  \end{question}

By definition, a series converges conditionally when $\sum a_n$ converges but $\sum |a_n|$ diverges. Conversely, one could ask whether it is possible for $\sum |a_n|$ to converge while $\sum a_n$ diverges. The following theorem shows that this is not possible.

\begin{theorem}[Absolute Convergence Theorem]
Every absolutely convergent series must converge.
\end{theorem}

Said differently, if a series $\sum |a_n|$ converges, then the series $\sum a_n$ must also converge. It is not hard to see why this is true. The terms of any sequence $\{a_n\}$ (possibly containing negative terms) satisfy the inequalities
$$
0 \leq a_n + |a_n| \leq 2|a_n|.
$$
If we assume that $\sum |a_n|$ converges, then $\sum (a_n + |a_n|)$ must also converge by the Comparison Test. But then the series $\sum a_n$ converges as well, as it is the difference of a pair of convergent series:
$$
\sum a_n = \sum (a_n + |a_n|) - \sum |a_n|.
$$

\begin{example}
Does the series $\sum_{n=1}^\infty \frac{\sin(n)}{n^2}$ converge?

The series contains both positive and negative terms, but it is not alternating. This makes it difficult to apply our standard tests to determine whether the series converges directly. On the other hand, consider the series 
$$
\sum_{n=1}^\infty  \left|\frac{\sin(n)}{n^2}\right|.
$$
By design, all of its terms are nonnegative. Moreover, since $|\sin(n)| \leq 1$, we have the comparison
$$
\left|\frac{\sin(n)}{n^2} \right| \leq \frac{1}{n^2}.
$$
It follows by the Comparison Test that $\sum \left|\frac{\sin (n)}{n^2}\right|$ converges. We conclude that $\sum \frac{\sin(n)}{n^2}$ converges absolutely, and the Absolute Convergence Theorem implies that it must therefore converge.
\end{example}

%Knowing that a series converges absolutely allows us to make two
%important statements. The first, given in the following theorem, is
%that absolute convergence is ``stronger'' than regular
%convergence. That is, just because
%\[
%\sum_{n=1}^\infty a_n
%\]
%converges, we cannot conclude that
%\[
%\sum_{n=1}^\infty |a_n|
%\]
%will converge, but knowing a series converges absolutely tells us that
%$\sum_{n=1}^\infty a_n$ will converge.
%
%One reason this is important is that our convergence tests all require
%that the underlying sequence of terms be positive. By taking the
%absolute value of the terms of a series where not all terms are
%positive, we are often able to apply an appropriate test and determine
%absolute convergence. This, in turn, determines that the series we are
%given also converges.
%
%The second statement relates to \dfn{rearrangements} of series. When
%dealing with a finite set of numbers, the sum of the numbers does not
%depend on the order in which they are added. (So $1+2+3 = 3+1+2$.) One
%may be surprised to find out that when dealing with an infinite set of
%numbers, the same statement does not always hold true: some infinite
%lists of numbers may be rearranged in different orders to achieve
%different sums. The theorem states that the terms of an absolutely
%convergent series can be rearranged in any way without affecting the
%sum.
%
%\begin{theorem}[Absolute Convergence]\index{absolute convergence theorem}
%  Let $\sum_{n=1}^\infty a_n$ be a series that converges absolutely.
%  Let $\seq{b_n}$ be any rearrangement of the sequence
%  $\seq{a_n}$. Then
%  \[
%  \sum_{n=1}^\infty b_n = \sum_{n=1}^\infty a_n.
%  \]
%\end{theorem}
%
%This theorem states that rearranging the terms of an absolutely
%convergent series does not affect its sum. Making such a statement implies that perhaps
%the sum of a conditionally convergent series can change based on the
%arrangement of terms. Indeed, it can. The \textit{Riemann
%  rearrangement theorem} (named after Bernhard Riemann) states that
%any conditionally convergent series can have its terms rearranged so
%that the sum is any desired value, including $\infty$!
%
%
%As an example, consider the alternating harmonic series once more. We
%have stated that
%\[
%\sum_{n=1}^\infty \frac{(-1)^{n+1}}{n} =1-\frac12+\frac13-\frac14+\frac15-\frac16+\frac17\cdots = \ln 2.
%\]
%Consider the rearrangement where every positive term is followed by two negative terms:
%\[
%1-\frac12-\frac14+\frac13-\frac16-\frac18+\frac15-\frac1{10}-\frac1{12}\cdots
%\]
%(Convince yourself that these are exactly the same numbers as appear
%in the alternating harmonic series, just in a different order.) Now
%group some terms and simplify:
%\begin{align*}
%\left(1-\frac12\right)-\frac14+\left(\frac13-\frac16\right)-\frac18+\left(\frac15-\frac1{10}\right)-\frac1{12}+\cdots &= \\
%\frac12-\frac14+\frac16-\frac18+\frac1{10}-\frac{1}{12}+\cdots &= \\
%\frac12\left(1-\frac12+\frac13-\frac14+\frac15-\frac16+\cdots\right) & = \frac12\ln 2.
%\end{align*}
%By rearranging the terms of the series, we have arrived at a different
%sum!
%\begin{quote}
%  One could \textit{try} to argue that the alternating harmonic series
%  does not actually converge to $\ln 2$, and here is an example of such 
%  an argument. According to the
%  alternating series test, we know that this series converges to some number $L$. 
%   If, as our intuition tells us should be true, the rearrangement does not change the sum,
%  then we have just seen that $L = L/2$.  The only possibility for $L$ is then $L=0$. But the alternating series
%  approximation theorem quickly shows that $L>0$. The only conclusion
%  is that the rearrangement \textit{did}, contrary to our intuition, change the sum.
%\end{quote}
%The fact that conditionally convergent series can be rearranged to
%equal any number is really an incredible result.
%
%
%
%While series are worthy of study in and of themselves, our ultimate
%goal within calculus is the study of \textit{power series}, which we
%will consider in the next section. We will use power series to create
%functions where the output is the result of an infinite summation.
%
%%A special type of power series is something called a Taylor Series; in the context of Taylor Series we will finally show the Alternating Harmonic Series converges to $\ln 2$.

\end{document}
