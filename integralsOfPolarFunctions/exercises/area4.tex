\documentclass{ximera}

%\usepackage{todonotes}

\newcommand{\todo}{}

\usepackage{esint} % for \oiint
\ifxake%%https://math.meta.stackexchange.com/questions/9973/how-do-you-render-a-closed-surface-double-integral
\renewcommand{\oiint}{{\large\bigcirc}\kern-1.56em\iint}
\fi


\graphicspath{
  {./}
  {ximeraTutorial/}
  {basicPhilosophy/}
  {functionsOfSeveralVariables/}
  {normalVectors/}
  {lagrangeMultipliers/}
  {vectorFields/}
  {greensTheorem/}
  {shapeOfThingsToCome/}
  {dotProducts/}
  {../productAndQuotientRules/exercises/}
  {../normalVectors/exercisesParametricPlots/}
  {../continuityOfFunctionsOfSeveralVariables/exercises/}
  {../partialDerivatives/exercises/}
  {../chainRuleForFunctionsOfSeveralVariables/exercises/}
  {../commonCoordinates/exercisesCylindricalCoordinates/}
  {../commonCoordinates/exercisesSphericalCoordinates/}
  {../greensTheorem/exercisesCurlAndLineIntegrals/}
  {../greensTheorem/exercisesDivergenceAndLineIntegrals/}
  {../shapeOfThingsToCome/exercisesDivergenceTheorem/}
  {../greensTheorem/}
  {../shapeOfThingsToCome/}
}

\newcommand{\mooculus}{\textsf{\textbf{MOOC}\textnormal{\textsf{ULUS}}}}

\usepackage{tkz-euclide}\usepackage{tikz}
\usepackage{tikz-cd}
\usetikzlibrary{arrows}
\tikzset{>=stealth,commutative diagrams/.cd,
  arrow style=tikz,diagrams={>=stealth}} %% cool arrow head
\tikzset{shorten <>/.style={ shorten >=#1, shorten <=#1 } } %% allows shorter vectors

\usetikzlibrary{backgrounds} %% for boxes around graphs
\usetikzlibrary{shapes,positioning}  %% Clouds and stars
\usetikzlibrary{matrix} %% for matrix
\usepgfplotslibrary{polar} %% for polar plots
\usepgfplotslibrary{fillbetween} %% to shade area between curves in TikZ
\usetkzobj{all}
%\usepackage[makeroom]{cancel} %% for strike outs
%\usepackage{mathtools} %% for pretty underbrace % Breaks Ximera
%\usepackage{multicol}
\usepackage{pgffor} %% required for integral for loops



%% http://tex.stackexchange.com/questions/66490/drawing-a-tikz-arc-specifying-the-center
%% Draws beach ball
\tikzset{pics/carc/.style args={#1:#2:#3}{code={\draw[pic actions] (#1:#3) arc(#1:#2:#3);}}}



\usepackage{array}
\setlength{\extrarowheight}{+.1cm}   
\newdimen\digitwidth
\settowidth\digitwidth{9}
\def\divrule#1#2{
\noalign{\moveright#1\digitwidth
\vbox{\hrule width#2\digitwidth}}}





\newcommand{\RR}{\mathbb R}
\newcommand{\R}{\mathbb R}
\newcommand{\N}{\mathbb N}
\newcommand{\Z}{\mathbb Z}

\newcommand{\sagemath}{\textsf{SageMath}}


%\renewcommand{\d}{\,d\!}
\renewcommand{\d}{\mathop{}\!d}
\newcommand{\dd}[2][]{\frac{\d #1}{\d #2}}
\newcommand{\pp}[2][]{\frac{\partial #1}{\partial #2}}
\renewcommand{\l}{\ell}
\newcommand{\ddx}{\frac{d}{\d x}}

\newcommand{\zeroOverZero}{\ensuremath{\boldsymbol{\tfrac{0}{0}}}}
\newcommand{\inftyOverInfty}{\ensuremath{\boldsymbol{\tfrac{\infty}{\infty}}}}
\newcommand{\zeroOverInfty}{\ensuremath{\boldsymbol{\tfrac{0}{\infty}}}}
\newcommand{\zeroTimesInfty}{\ensuremath{\small\boldsymbol{0\cdot \infty}}}
\newcommand{\inftyMinusInfty}{\ensuremath{\small\boldsymbol{\infty - \infty}}}
\newcommand{\oneToInfty}{\ensuremath{\boldsymbol{1^\infty}}}
\newcommand{\zeroToZero}{\ensuremath{\boldsymbol{0^0}}}
\newcommand{\inftyToZero}{\ensuremath{\boldsymbol{\infty^0}}}



\newcommand{\numOverZero}{\ensuremath{\boldsymbol{\tfrac{\#}{0}}}}
\newcommand{\dfn}{\textbf}
%\newcommand{\unit}{\,\mathrm}
\newcommand{\unit}{\mathop{}\!\mathrm}
\newcommand{\eval}[1]{\bigg[ #1 \bigg]}
\newcommand{\seq}[1]{\left( #1 \right)}
\renewcommand{\epsilon}{\varepsilon}
\renewcommand{\phi}{\varphi}


\renewcommand{\iff}{\Leftrightarrow}

\DeclareMathOperator{\arccot}{arccot}
\DeclareMathOperator{\arcsec}{arcsec}
\DeclareMathOperator{\arccsc}{arccsc}
\DeclareMathOperator{\si}{Si}
\DeclareMathOperator{\scal}{scal}
\DeclareMathOperator{\sign}{sign}


%% \newcommand{\tightoverset}[2]{% for arrow vec
%%   \mathop{#2}\limits^{\vbox to -.5ex{\kern-0.75ex\hbox{$#1$}\vss}}}
\newcommand{\arrowvec}[1]{{\overset{\rightharpoonup}{#1}}}
%\renewcommand{\vec}[1]{\arrowvec{\mathbf{#1}}}
\renewcommand{\vec}[1]{{\overset{\boldsymbol{\rightharpoonup}}{\mathbf{#1}}}}
\DeclareMathOperator{\proj}{\vec{proj}}
\newcommand{\veci}{{\boldsymbol{\hat{\imath}}}}
\newcommand{\vecj}{{\boldsymbol{\hat{\jmath}}}}
\newcommand{\veck}{{\boldsymbol{\hat{k}}}}
\newcommand{\vecl}{\vec{\boldsymbol{\l}}}
\newcommand{\uvec}[1]{\mathbf{\hat{#1}}}
\newcommand{\utan}{\mathbf{\hat{t}}}
\newcommand{\unormal}{\mathbf{\hat{n}}}
\newcommand{\ubinormal}{\mathbf{\hat{b}}}

\newcommand{\dotp}{\bullet}
\newcommand{\cross}{\boldsymbol\times}
\newcommand{\grad}{\boldsymbol\nabla}
\newcommand{\divergence}{\grad\dotp}
\newcommand{\curl}{\grad\cross}
%\DeclareMathOperator{\divergence}{divergence}
%\DeclareMathOperator{\curl}[1]{\grad\cross #1}
\newcommand{\lto}{\mathop{\longrightarrow\,}\limits}

\renewcommand{\bar}{\overline}

\colorlet{textColor}{black} 
\colorlet{background}{white}
\colorlet{penColor}{blue!50!black} % Color of a curve in a plot
\colorlet{penColor2}{red!50!black}% Color of a curve in a plot
\colorlet{penColor3}{red!50!blue} % Color of a curve in a plot
\colorlet{penColor4}{green!50!black} % Color of a curve in a plot
\colorlet{penColor5}{orange!80!black} % Color of a curve in a plot
\colorlet{penColor6}{yellow!70!black} % Color of a curve in a plot
\colorlet{fill1}{penColor!20} % Color of fill in a plot
\colorlet{fill2}{penColor2!20} % Color of fill in a plot
\colorlet{fillp}{fill1} % Color of positive area
\colorlet{filln}{penColor2!20} % Color of negative area
\colorlet{fill3}{penColor3!20} % Fill
\colorlet{fill4}{penColor4!20} % Fill
\colorlet{fill5}{penColor5!20} % Fill
\colorlet{gridColor}{gray!50} % Color of grid in a plot

\newcommand{\surfaceColor}{violet}
\newcommand{\surfaceColorTwo}{redyellow}
\newcommand{\sliceColor}{greenyellow}




\pgfmathdeclarefunction{gauss}{2}{% gives gaussian
  \pgfmathparse{1/(#2*sqrt(2*pi))*exp(-((x-#1)^2)/(2*#2^2))}%
}


%%%%%%%%%%%%%
%% Vectors
%%%%%%%%%%%%%

%% Simple horiz vectors
\renewcommand{\vector}[1]{\left\langle #1\right\rangle}


%% %% Complex Horiz Vectors with angle brackets
%% \makeatletter
%% \renewcommand{\vector}[2][ , ]{\left\langle%
%%   \def\nextitem{\def\nextitem{#1}}%
%%   \@for \el:=#2\do{\nextitem\el}\right\rangle%
%% }
%% \makeatother

%% %% Vertical Vectors
%% \def\vector#1{\begin{bmatrix}\vecListA#1,,\end{bmatrix}}
%% \def\vecListA#1,{\if,#1,\else #1\cr \expandafter \vecListA \fi}

%%%%%%%%%%%%%
%% End of vectors
%%%%%%%%%%%%%

%\newcommand{\fullwidth}{}
%\newcommand{\normalwidth}{}



%% makes a snazzy t-chart for evaluating functions
%\newenvironment{tchart}{\rowcolors{2}{}{background!90!textColor}\array}{\endarray}

%%This is to help with formatting on future title pages.
\newenvironment{sectionOutcomes}{}{} 



%% Flowchart stuff
%\tikzstyle{startstop} = [rectangle, rounded corners, minimum width=3cm, minimum height=1cm,text centered, draw=black]
%\tikzstyle{question} = [rectangle, minimum width=3cm, minimum height=1cm, text centered, draw=black]
%\tikzstyle{decision} = [trapezium, trapezium left angle=70, trapezium right angle=110, minimum width=3cm, minimum height=1cm, text centered, draw=black]
%\tikzstyle{question} = [rectangle, rounded corners, minimum width=3cm, minimum height=1cm,text centered, draw=black]
%\tikzstyle{process} = [rectangle, minimum width=3cm, minimum height=1cm, text centered, draw=black]
%\tikzstyle{decision} = [trapezium, trapezium left angle=70, trapezium right angle=110, minimum width=3cm, minimum height=1cm, text centered, draw=black]


\author{Jason Miller and Jim Talamo}
\license{Creative Commons 3.0 By-bC}


\outcome{}

\begin{document}
\begin{exercise}
%
%Two problems that require set up only, but each problem has a region where the inner and outer curve change and a region where there is an inner and outer curve (ex, r=2cos(2theta) and r=1, region 1 is region common to both curves, and region 2 is the region outside of r=1 but inside of r = 2 cos(2theta))

Consider the polar curves $r=2\sin(3\theta)$ (shown in blue) and $r=1$ (shown in red). 




\begin{image}  
  \begin{tikzpicture}  
    \begin{axis}[  
        xmin=-2.5,  
        xmax=3,  
        ymin=-2.5,  
        ymax=2.5,  
        axis lines=center,  
        xlabel=$x$,  
        ylabel=$y$,  
        every axis y label/.style={at=(current axis.above origin),anchor=south},  
        every axis x label/.style={at=(current axis.right of origin),anchor=west},axis on top
      ]  
      \addplot [data cs=polar, very thick, mark=none,fill=fill1,domain=10:50,samples=180,smooth] (x, {2*sin(3*x)});
      \addplot[data cs=polar, very thick, mark=none, fill=white,  domain=0:90, samples=180, smooth] (x, {1});
      \addplot[data cs=polar,penColor2,domain=0:360,samples=360,smooth, thick] (x,{2*sin(3*x)}) ;
      \addplot[data cs=polar, penColor, domain=0:360, samples=360, smooth, thick] (x, {1});      
            \end{axis}  
  \end{tikzpicture}  
\end{image} 

We want to set up an integral that expresses the area of the shaded region $S$ that lies inside the curve $r=2\sin(3\theta)$ but outside of the circle $r=1$ in the 1st quadrant. 





The area of the region $S$ is 

\[
\int_{\answer{\pi/18}}^{\answer{ 5\pi/18 } } \answer{ \frac{1}{2}\left[(2\sin(3\theta))^2-1\right]   } \d \theta=\answer{ \frac{\pi}{9}+\frac{\sqrt{3}}{6}}
\]





\begin{hint}

Consider a general polar curve $r=f(\theta)$ as below. 


\begin{image}
  \begin{tikzpicture}
\begin{axis}[
axis y line=middle,axis x line=middle,name=myplot,%
			%x=.37\marginparwidth,
			%y=.37\marginparwidth,
			%xtick={-1,1},
			%minor x tick num=1,% 
%			extra x ticks={.33},
%			extra x tick labels={$1/3$},
			%ytick={-1,1},
			%minor y tick num=1,%extra y ticks={-5,-3,...,7},%
			ymin=-.1,ymax=1.1,
			xmin=-.1,xmax=1.1, 
axis on top
]

\addplot [fill1,fill=fill1,area style, smooth,domain=18:72,samples=30] ({cos(x)*(1+.05*cos(9*x))},{sin(x)*(1+.05*cos(9*x))}) -- (axis cs:0,0) -- cycle;

\addplot [penColor2 ,thick, smooth,domain=0:90,samples=30] ({cos(x)*(1+.05*cos(9*x))},{sin(x)*(1+.05*cos(9*x))});


\addplot [fill1,fill=white ,area style, smooth,domain=12:64,samples=30] ({cos(x)*(.4+.05*cos(9*x))},{sin(x)*(.6+.05*cos(9*x))}) -- (axis cs:0,0) -- cycle;

\addplot [penColor,thick, smooth,domain=0:90,samples=30] ({cos(x)*(.4+.05*cos(9*x))},{sin(x)*(.6+.05*cos(9*x))});


\draw [thick,penColor,] (axis cs:0,0) -- (axis cs: 0.905831, 0.294322) node [pos=.8,below,rotate=18,black] { $\theta=\alpha$};

\draw [thick,penColor,] (axis cs:0,0) -- (axis cs:0.313792, 0.965751) node [pos=.8,above,rotate=72,black] { $\theta=\beta$};

\draw[very thick, orange] (axis cs:0,0) -- (axis cs:0.83, .65) node [pos=.8, above, rotate=40, orange] {$\theta=\theta_{0}$};


\draw (axis cs:.25, .37   ) node[penColor] {$ g(\theta)$};

\draw (axis cs:.8,.82) node[penColor2] { $f(\theta)$};


\end{axis}

\node [right] at (myplot.right of origin) { $\theta=0$};
\node [above] at (myplot.above origin) { $\theta=\pi/2$};
\end{tikzpicture}
\end{image}


The area enclosed by the polar curves $r=f(\theta)$ (in blue) and $r=g(\theta)$ (in red) from $\theta=\alpha$ to $\theta=\beta$ is given 

\[
\int_{\alpha}^{\beta} \frac{1}{2} (f(\theta)^2 -g(\theta)^2) \d \theta
\]

Consider a fixed angle $\theta=\theta_{0}$ (in orange). This ray extends from the origin outward and will intersect the two polar curves. Call the outermost curve that the ray intersects
$r_{outer}$ and call the innermost curve that the ray intersects $r_{inner}$. 

Then we express the area between two polar curves between $\alpha$ and $\beta$ as 

\[
\int_{\alpha}^{\beta} \frac{1}{2}( (r_{outer})^2 -(r_{inner})^2) \d \theta
\]


In our case (see below), we have $r_{outer}=2\sin(3\theta)$ and $r_{inner}=1$.




\begin{image}  
  \begin{tikzpicture}  
    \begin{axis}[  
        xmin=-2.5,  
        xmax=3,  
        ymin=-2.5,  
        ymax=2.5,  
        axis lines=center,  
        xlabel=$x$,  
        ylabel=$y$,  
        every axis y label/.style={at=(current axis.above origin),anchor=south},  
        every axis x label/.style={at=(current axis.right of origin),anchor=west},axis on top
      ]  
      \addplot [data cs=polar, very thick, mark=none,fill=fill1,domain=10:50,samples=180,smooth] (x, {2*sin(3*x)});
      \addplot[data cs=polar, very thick, mark=none, fill=white,  domain=0:90, samples=180, smooth] (x, {1});
      \addplot[data cs=polar,penColor2,domain=0:360,samples=360,smooth, thick] (x,{2*sin(3*x)}) ;
      \addplot[data cs=polar, penColor, domain=0:360, samples=360, smooth, thick] (x, {1});      
     \draw[very thick, orange] (axis cs:0,0) -- (axis cs:2, 1.55) node [pos=.9, above, rotate=40, orange] {$\theta=\theta_{0}$};
        \draw (axis cs:1.24, -1.47  ) node[penColor2] {$ r=2\sin(3\theta)$};
      \draw (axis cs:-1, -.89 ) node[penColor] {$r=1$};
            \end{axis}  
  \end{tikzpicture}  
\end{image} 


Now we need to identify the initial angle $\alpha$ and the final angle $\beta$ that bounds our region $S$. In order to determine these angles we need to think about how the curve is traced out as $\theta$ varies. 

Understanding how $r=1$ is traced out is the simplest. It is traced out in the standard fashion as $\theta$ varies. That is, for each value of $\theta$, the associated point on the curve is $(1, \theta)$ in polar coordinates. 

Let's graph $r=2\sin(3\theta)$ on $r$ and $\theta$ axes. 

\begin{image}  
  \begin{tikzpicture}  
    \begin{axis}[  
        xmin=-.5,  
        xmax=3.5,  
        ymin=-2.5,  
        ymax=2.5,  
        axis lines=center,  
        xlabel=$\theta$,  
        ylabel=$r$,  
        every axis y label/.style={at=(current axis.above origin),anchor=south},  
        every axis x label/.style={at=(current axis.right of origin),anchor=west},  
       xtick={ .523, 1.047, 1.57, 2.094, 2.618,  3.14   },
       xticklabels={ $\frac{\pi}{6}$, $\frac{\pi}{3}$, $\frac{\pi}{2}$, $\frac{2\pi}{3}$, $\frac{5\pi}{6}$, $\pi$ },
            ]  
      \addplot [ very thick, mark=none,domain=0:pi,smooth] {2*sin(deg(3*x))};
            \end{axis}  
  \end{tikzpicture}  
\end{image} 

Both curves can be traced out exactly once by letting $\theta$ vary from $0$ to $\pi$ (convince yourself of this!).

As $\theta$ goes from $0$ to $\frac{\pi}{3}$ we see that $r$ increase from $0$ to $2$ and then decreases from $2$ back down to $0$. This corresponds to the petal in the 1st quadrant. 

But we don't want the entire petal. So we need to find for which $\theta$ values the $r$ values of the two curves coincide. 

We set $r=2\sin(3\theta)$ and $r=1$ equal. 

Set $2\sin(3\theta)=1$.  Note that $\sin(\cdot) = 1/2$ when $(\cdot) = \frac{\pi}{6},\frac{5\pi}{6}, \ldots$  (meaning that we have to set the expression \emph{inside} the sine function equal to those values.  Doing this gives us that $3 \theta = \frac{\pi}{6},\frac{5\pi}{6}, \ldots$, so $\theta=\answer{\frac{\pi}{18}}$ and $\theta=\answer{\frac{5\pi}{18}}$ (list the $\theta$ values from smaller to larger. 


In order to evaluate the integral, recall the trig identity $\sin^2(\theta)=\frac{1-\cos(2\theta)}{2}$. 










\end{hint}

\end{exercise}
\end{document}
