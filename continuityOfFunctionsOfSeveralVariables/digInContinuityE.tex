\documentclass{ximera}

%\usepackage{todonotes}

\newcommand{\todo}{}

\usepackage{esint} % for \oiint
\ifxake%%https://math.meta.stackexchange.com/questions/9973/how-do-you-render-a-closed-surface-double-integral
\renewcommand{\oiint}{{\large\bigcirc}\kern-1.56em\iint}
\fi


\graphicspath{
  {./}
  {ximeraTutorial/}
  {basicPhilosophy/}
  {functionsOfSeveralVariables/}
  {normalVectors/}
  {lagrangeMultipliers/}
  {vectorFields/}
  {greensTheorem/}
  {shapeOfThingsToCome/}
  {dotProducts/}
  {../productAndQuotientRules/exercises/}
  {../normalVectors/exercisesParametricPlots/}
  {../continuityOfFunctionsOfSeveralVariables/exercises/}
  {../partialDerivatives/exercises/}
  {../chainRuleForFunctionsOfSeveralVariables/exercises/}
  {../commonCoordinates/exercisesCylindricalCoordinates/}
  {../commonCoordinates/exercisesSphericalCoordinates/}
  {../greensTheorem/exercisesCurlAndLineIntegrals/}
  {../greensTheorem/exercisesDivergenceAndLineIntegrals/}
  {../shapeOfThingsToCome/exercisesDivergenceTheorem/}
  {../greensTheorem/}
  {../shapeOfThingsToCome/}
}

\newcommand{\mooculus}{\textsf{\textbf{MOOC}\textnormal{\textsf{ULUS}}}}

\usepackage{tkz-euclide}\usepackage{tikz}
\usepackage{tikz-cd}
\usetikzlibrary{arrows}
\tikzset{>=stealth,commutative diagrams/.cd,
  arrow style=tikz,diagrams={>=stealth}} %% cool arrow head
\tikzset{shorten <>/.style={ shorten >=#1, shorten <=#1 } } %% allows shorter vectors

\usetikzlibrary{backgrounds} %% for boxes around graphs
\usetikzlibrary{shapes,positioning}  %% Clouds and stars
\usetikzlibrary{matrix} %% for matrix
\usepgfplotslibrary{polar} %% for polar plots
\usepgfplotslibrary{fillbetween} %% to shade area between curves in TikZ
\usetkzobj{all}
%\usepackage[makeroom]{cancel} %% for strike outs
%\usepackage{mathtools} %% for pretty underbrace % Breaks Ximera
%\usepackage{multicol}
\usepackage{pgffor} %% required for integral for loops



%% http://tex.stackexchange.com/questions/66490/drawing-a-tikz-arc-specifying-the-center
%% Draws beach ball
\tikzset{pics/carc/.style args={#1:#2:#3}{code={\draw[pic actions] (#1:#3) arc(#1:#2:#3);}}}



\usepackage{array}
\setlength{\extrarowheight}{+.1cm}   
\newdimen\digitwidth
\settowidth\digitwidth{9}
\def\divrule#1#2{
\noalign{\moveright#1\digitwidth
\vbox{\hrule width#2\digitwidth}}}





\newcommand{\RR}{\mathbb R}
\newcommand{\R}{\mathbb R}
\newcommand{\N}{\mathbb N}
\newcommand{\Z}{\mathbb Z}

\newcommand{\sagemath}{\textsf{SageMath}}


%\renewcommand{\d}{\,d\!}
\renewcommand{\d}{\mathop{}\!d}
\newcommand{\dd}[2][]{\frac{\d #1}{\d #2}}
\newcommand{\pp}[2][]{\frac{\partial #1}{\partial #2}}
\renewcommand{\l}{\ell}
\newcommand{\ddx}{\frac{d}{\d x}}

\newcommand{\zeroOverZero}{\ensuremath{\boldsymbol{\tfrac{0}{0}}}}
\newcommand{\inftyOverInfty}{\ensuremath{\boldsymbol{\tfrac{\infty}{\infty}}}}
\newcommand{\zeroOverInfty}{\ensuremath{\boldsymbol{\tfrac{0}{\infty}}}}
\newcommand{\zeroTimesInfty}{\ensuremath{\small\boldsymbol{0\cdot \infty}}}
\newcommand{\inftyMinusInfty}{\ensuremath{\small\boldsymbol{\infty - \infty}}}
\newcommand{\oneToInfty}{\ensuremath{\boldsymbol{1^\infty}}}
\newcommand{\zeroToZero}{\ensuremath{\boldsymbol{0^0}}}
\newcommand{\inftyToZero}{\ensuremath{\boldsymbol{\infty^0}}}



\newcommand{\numOverZero}{\ensuremath{\boldsymbol{\tfrac{\#}{0}}}}
\newcommand{\dfn}{\textbf}
%\newcommand{\unit}{\,\mathrm}
\newcommand{\unit}{\mathop{}\!\mathrm}
\newcommand{\eval}[1]{\bigg[ #1 \bigg]}
\newcommand{\seq}[1]{\left( #1 \right)}
\renewcommand{\epsilon}{\varepsilon}
\renewcommand{\phi}{\varphi}


\renewcommand{\iff}{\Leftrightarrow}

\DeclareMathOperator{\arccot}{arccot}
\DeclareMathOperator{\arcsec}{arcsec}
\DeclareMathOperator{\arccsc}{arccsc}
\DeclareMathOperator{\si}{Si}
\DeclareMathOperator{\scal}{scal}
\DeclareMathOperator{\sign}{sign}


%% \newcommand{\tightoverset}[2]{% for arrow vec
%%   \mathop{#2}\limits^{\vbox to -.5ex{\kern-0.75ex\hbox{$#1$}\vss}}}
\newcommand{\arrowvec}[1]{{\overset{\rightharpoonup}{#1}}}
%\renewcommand{\vec}[1]{\arrowvec{\mathbf{#1}}}
\renewcommand{\vec}[1]{{\overset{\boldsymbol{\rightharpoonup}}{\mathbf{#1}}}}
\DeclareMathOperator{\proj}{\vec{proj}}
\newcommand{\veci}{{\boldsymbol{\hat{\imath}}}}
\newcommand{\vecj}{{\boldsymbol{\hat{\jmath}}}}
\newcommand{\veck}{{\boldsymbol{\hat{k}}}}
\newcommand{\vecl}{\vec{\boldsymbol{\l}}}
\newcommand{\uvec}[1]{\mathbf{\hat{#1}}}
\newcommand{\utan}{\mathbf{\hat{t}}}
\newcommand{\unormal}{\mathbf{\hat{n}}}
\newcommand{\ubinormal}{\mathbf{\hat{b}}}

\newcommand{\dotp}{\bullet}
\newcommand{\cross}{\boldsymbol\times}
\newcommand{\grad}{\boldsymbol\nabla}
\newcommand{\divergence}{\grad\dotp}
\newcommand{\curl}{\grad\cross}
%\DeclareMathOperator{\divergence}{divergence}
%\DeclareMathOperator{\curl}[1]{\grad\cross #1}
\newcommand{\lto}{\mathop{\longrightarrow\,}\limits}

\renewcommand{\bar}{\overline}

\colorlet{textColor}{black} 
\colorlet{background}{white}
\colorlet{penColor}{blue!50!black} % Color of a curve in a plot
\colorlet{penColor2}{red!50!black}% Color of a curve in a plot
\colorlet{penColor3}{red!50!blue} % Color of a curve in a plot
\colorlet{penColor4}{green!50!black} % Color of a curve in a plot
\colorlet{penColor5}{orange!80!black} % Color of a curve in a plot
\colorlet{penColor6}{yellow!70!black} % Color of a curve in a plot
\colorlet{fill1}{penColor!20} % Color of fill in a plot
\colorlet{fill2}{penColor2!20} % Color of fill in a plot
\colorlet{fillp}{fill1} % Color of positive area
\colorlet{filln}{penColor2!20} % Color of negative area
\colorlet{fill3}{penColor3!20} % Fill
\colorlet{fill4}{penColor4!20} % Fill
\colorlet{fill5}{penColor5!20} % Fill
\colorlet{gridColor}{gray!50} % Color of grid in a plot

\newcommand{\surfaceColor}{violet}
\newcommand{\surfaceColorTwo}{redyellow}
\newcommand{\sliceColor}{greenyellow}




\pgfmathdeclarefunction{gauss}{2}{% gives gaussian
  \pgfmathparse{1/(#2*sqrt(2*pi))*exp(-((x-#1)^2)/(2*#2^2))}%
}


%%%%%%%%%%%%%
%% Vectors
%%%%%%%%%%%%%

%% Simple horiz vectors
\renewcommand{\vector}[1]{\left\langle #1\right\rangle}


%% %% Complex Horiz Vectors with angle brackets
%% \makeatletter
%% \renewcommand{\vector}[2][ , ]{\left\langle%
%%   \def\nextitem{\def\nextitem{#1}}%
%%   \@for \el:=#2\do{\nextitem\el}\right\rangle%
%% }
%% \makeatother

%% %% Vertical Vectors
%% \def\vector#1{\begin{bmatrix}\vecListA#1,,\end{bmatrix}}
%% \def\vecListA#1,{\if,#1,\else #1\cr \expandafter \vecListA \fi}

%%%%%%%%%%%%%
%% End of vectors
%%%%%%%%%%%%%

%\newcommand{\fullwidth}{}
%\newcommand{\normalwidth}{}



%% makes a snazzy t-chart for evaluating functions
%\newenvironment{tchart}{\rowcolors{2}{}{background!90!textColor}\array}{\endarray}

%%This is to help with formatting on future title pages.
\newenvironment{sectionOutcomes}{}{} 



%% Flowchart stuff
%\tikzstyle{startstop} = [rectangle, rounded corners, minimum width=3cm, minimum height=1cm,text centered, draw=black]
%\tikzstyle{question} = [rectangle, minimum width=3cm, minimum height=1cm, text centered, draw=black]
%\tikzstyle{decision} = [trapezium, trapezium left angle=70, trapezium right angle=110, minimum width=3cm, minimum height=1cm, text centered, draw=black]
%\tikzstyle{question} = [rectangle, rounded corners, minimum width=3cm, minimum height=1cm,text centered, draw=black]
%\tikzstyle{process} = [rectangle, minimum width=3cm, minimum height=1cm, text centered, draw=black]
%\tikzstyle{decision} = [trapezium, trapezium left angle=70, trapezium right angle=110, minimum width=3cm, minimum height=1cm, text centered, draw=black]


\author{Bart Snapp and Jim Talamo}

\outcome{Determine whether functions of several variables are continuous.}

\title[Dig-In:]{Continuity}

\begin{document}
\begin{abstract}
We investigate what continuity means for functions of several variables.
\end{abstract}
\maketitle

Now that we have defined limits, we can define continuity.

\begin{definition}
  Let $F:\R^n\to \R$ and $\pt{a}$ be an interior point of the domain of $F$. We say $F$ is \dfn{continuous} at $\pt{x}=\pt{a}$, if
  \begin{itemize}
  \item $F(\pt{a})$ exists.
  \item $\lim_{\pt{x}\to\pt{a}} F(\pt{x})$ exists.
  \item $\lim_{\pt{x}\to\pt{a}} F(\pt{x}) = F(\pt{a}).$
  \end{itemize}
  $F$ is \dfn{continuous on an open set} $S$ if $F$ is continuous at
  all $\pt{x} \in S$.
\end{definition}

The \textit{limit laws} can be used to write corresponding \textit{continuity laws}.

\begin{theorem}[Limit Laws]
  Let $F:\R^n\to \R$ and $G:\R^n\to \R$ be continuous functions of several
  variables, and $b$ be a real number.
  \begin{align*}
    \pt{x} &= \point{x_1,x_2,\dots,x_n}\\ \pt{a} &=
    \point{a_1,a_2,\dots,a_n},
  \end{align*}
  where
  \[
  \lim_{\pt{x}\to\pt{a}}F(\pt{x}) = L \quad \text{and}\quad \lim_{\pt{x}\to\pt{a}} G(\pt{x}) = M.
  \]
\begin{description}
\item[Constant Law] $F(\pt{x}) = b$ is continuous.
\item[Identity Law] $F(\pt{x}) = x_i$ is continuous.
\item[Sum/Difference Law] $F(\pt{x})\pm G(\pt{x})$ is continuous.
\item[Scalar Multiple Law] $b\cdot F(\pt{x})$ is continuous.
\item[Product Law] $F(\pt{x})\cdot G(\pt{x})$ is continuous.
\item[Quotient Law] $\frac{F(\pt{x})}{G(\pt{x})}$ is continuous where  $G(\pt{x}) \neq 0$.
\end{description}
\end{theorem}

\begin{question}
  True or false: If $F:\R^2\to\R$ and $G:\R^2\to\R$ are continuous
  functions on an open disk $B$, then $F\pm G$ is continuous on $B$.
  \begin{prompt}
    \begin{multipleChoice}
      \choice[correct]{True}
      \choice{False}
  \end{multipleChoice}
  \end{prompt}
\end{question}

\begin{question}
  True or false: If $F:\R^2\to\R$ and $G:\R^2\to\R$ are continuous
  functions on an open disk $B$, then $F/G$ is continuous on $B$.
  \begin{prompt}
    \begin{multipleChoice}
      \choice{True}
      \choice[correct]{False}
    \end{multipleChoice}
    \begin{feedback}
      The function $F/G$ may or may not be continuous, it depends on
      whether $G(x,y)=0$. If $G(x,y)=0$, then $F/G$ not continuous at that point.
    \end{feedback}
  \end{prompt}
\end{question}


\begin{theorem}[Composition Limit Law]
  Let $f:\R\to\R$ be a continuous function on an interval $I$. Let
  $G:\R^n\to \R$ be a function whose range is contained in (or equal
  to) $I$, Then
  \[
  \lim_{\pt{x}\to\pt{a}} f( G(\pt{x})) = f(\lim_{\pt{x}\to\pt{a}}G(\pt{x}))
  \]
\end{theorem}


\begin{corollary}[Composition of Composite Functions]
  Let $G:\R^n\to \R$ be continuous on an open disk $B$, where the
  range of $G$ on $B$ is $I$, and let $f$ be a single variable
  function that is continuous on $I$. Then
  \[
  f\circ G(\pt{x}) =f(G(\pt{x})),
  \]
  is continuous on $B$.
\end{corollary}



\begin{example}
  Show that the function
  \[
  F(x,y) = \sin(x^2\cos(y))
  \]
  is continuous for all points in $\R^2$.
  \begin{explanation}
    Let
    \[
    F_1(x,y) = x^2.
    \]
    Since $y$ is not actually used in the function, and polynomials
    \wordChoice{\choice[correct]{are continuous}\choice{are not
        continuous}}, we conclude $F_1$ is continuous everywhere. A
    similar statement can be made about
    \[
    F_2(x,y) = \cos(y).
    \]
    Setting
    \[
    F_3=F_1\cdot F_2
    \]
    we obtain a continuous function from $\R^2\to \R$. Since sine \wordChoice{\choice[correct]{is
    continuous}\choice{is not continuous}} for all real values, the composition of sine with $F_3$
    is continuous. Hence, $\sin (F_3(x,y)) = \sin(x^2\cos y)$ is
    continuous everywhere.
    \begin{onlineOnly}
      We finish by presenting you with a plot of $F$:
      \begin{center}
        \geogebra{TNETssA9}{800}{600} %https://ggbm.at/TNETssA9
      \end{center}
    \end{onlineOnly}
  \end{explanation}
\end{example}


\begin{example}
  Let
  \[
  F(x,y) = \begin{cases}
    \frac{\cos(y)\sin(x)}{x} & x\neq 0 \\
    \cos(y) & x=0
  \end{cases}
  \]
  Is $F$ continuous at $(0,0)$? Is $F$ continuous everywhere?
  \begin{explanation}
    To determine if $F$ is continuous at $(0,0)$, we need to compare
    \[
    \lim_{\point{x,y}\to\point{0,0}} F(x,y)\quad\text{to}\quad F(0,0).
    \]
    Applying the definition of $F$, we see that:
    \[
    F(0,0) = \answer[given]{1}
    \]
    We now consider the limit
    \[
    \lim_{\point{x,y}\to\point{0,0}}F(x,y).
    \]
    Substituting $0$ for $x$ and $y$ in $(\cos(y)\sin(x))/x$ returns the
    indeterminate form \zeroOverZero, so we need to do more work to
    evaluate this limit.
    
    Consider two related limits:
    \begin{align*}
      \lim_{\point{x,y}\to\point{0,0}} \cos(y)\\
      \lim_{\point{x,y}\to\point{0,0}} \frac{\sin(x)}{x}.
    \end{align*}
    The first limit does not contain $x$, and since $\cos(y)$ is
    continuous,
    \begin{align*}
    \lim_{\point{x,y}\to\point{0,0}} \cos(y) &=\lim_{y\to 0} \cos(y) \\
    &=\answer[given]{1}
    \end{align*}
    The second limit does not contain $y$. But we know
    \begin{align*}
      \lim_{\point{x,y}\to\point{0,0}} \frac{\sin(x)}{x} &= \lim_{x\to 0} \frac{\sin(x)}{x} \\
      &= \answer[given]{1}.
    \end{align*}
    Finally, we know that we can combine these two limits so that 
    $\lim_{\point{x,y}\to\point{0,0}} \frac{\cos(y)\sin(x)}{x}$
    \begin{align*}
      &= \lim_{\point{x,y}\to\point{0,0}} (\cos(y))\left(\frac{\sin(x)}{x}\right) \\ 
      &=\left(\lim_{\point{x,y}\to\point{0,0}} \cos(y)\right)\left(\lim_{\point{x,y}\to\point{0,0}} \frac{\sin(x)}{x}\right) \\
            &=\answer[given]{1}\cdot \answer[given]{1}.
    \end{align*}
    We have found that $\lim_{\point{x,y}\to\point{0,0}} \frac{\cos(y)\sin(x)}{x} =
    F(0,0)$, so $F$ is continuous at $(0,0)$.

    A similar analysis shows that $F$ is continuous at all points in
    $\mathbb{R}^2$. As long as $x\neq0$, we can evaluate the limit
    directly; when $x=0$, a similar analysis shows that the limit is $\cos
    y$. Thus we can say that $F$ is continuous everywhere.
    \begin{onlineOnly}
      We finish by presenting you with a plot of $F$:
      \begin{center}
        \geogebra{VK6thpMa}{800}{600} %https://ggbm.at/VK6thpMa
      \end{center}
    \end{onlineOnly}
  \end{explanation}
\end{example}


\end{document}
