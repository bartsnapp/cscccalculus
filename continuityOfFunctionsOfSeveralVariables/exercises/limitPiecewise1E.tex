\documentclass{ximera}

\author{Jim Talamo}

%\usepackage{todonotes}

\newcommand{\todo}{}

\usepackage{esint} % for \oiint
\ifxake%%https://math.meta.stackexchange.com/questions/9973/how-do-you-render-a-closed-surface-double-integral
\renewcommand{\oiint}{{\large\bigcirc}\kern-1.56em\iint}
\fi


\graphicspath{
  {./}
  {ximeraTutorial/}
  {basicPhilosophy/}
  {functionsOfSeveralVariables/}
  {normalVectors/}
  {lagrangeMultipliers/}
  {vectorFields/}
  {greensTheorem/}
  {shapeOfThingsToCome/}
  {dotProducts/}
  {../productAndQuotientRules/exercises/}
  {../normalVectors/exercisesParametricPlots/}
  {../continuityOfFunctionsOfSeveralVariables/exercises/}
  {../partialDerivatives/exercises/}
  {../chainRuleForFunctionsOfSeveralVariables/exercises/}
  {../commonCoordinates/exercisesCylindricalCoordinates/}
  {../commonCoordinates/exercisesSphericalCoordinates/}
  {../greensTheorem/exercisesCurlAndLineIntegrals/}
  {../greensTheorem/exercisesDivergenceAndLineIntegrals/}
  {../shapeOfThingsToCome/exercisesDivergenceTheorem/}
  {../greensTheorem/}
  {../shapeOfThingsToCome/}
}

\newcommand{\mooculus}{\textsf{\textbf{MOOC}\textnormal{\textsf{ULUS}}}}

\usepackage{tkz-euclide}\usepackage{tikz}
\usepackage{tikz-cd}
\usetikzlibrary{arrows}
\tikzset{>=stealth,commutative diagrams/.cd,
  arrow style=tikz,diagrams={>=stealth}} %% cool arrow head
\tikzset{shorten <>/.style={ shorten >=#1, shorten <=#1 } } %% allows shorter vectors

\usetikzlibrary{backgrounds} %% for boxes around graphs
\usetikzlibrary{shapes,positioning}  %% Clouds and stars
\usetikzlibrary{matrix} %% for matrix
\usepgfplotslibrary{polar} %% for polar plots
\usepgfplotslibrary{fillbetween} %% to shade area between curves in TikZ
\usetkzobj{all}
%\usepackage[makeroom]{cancel} %% for strike outs
%\usepackage{mathtools} %% for pretty underbrace % Breaks Ximera
%\usepackage{multicol}
\usepackage{pgffor} %% required for integral for loops



%% http://tex.stackexchange.com/questions/66490/drawing-a-tikz-arc-specifying-the-center
%% Draws beach ball
\tikzset{pics/carc/.style args={#1:#2:#3}{code={\draw[pic actions] (#1:#3) arc(#1:#2:#3);}}}



\usepackage{array}
\setlength{\extrarowheight}{+.1cm}   
\newdimen\digitwidth
\settowidth\digitwidth{9}
\def\divrule#1#2{
\noalign{\moveright#1\digitwidth
\vbox{\hrule width#2\digitwidth}}}





\newcommand{\RR}{\mathbb R}
\newcommand{\R}{\mathbb R}
\newcommand{\N}{\mathbb N}
\newcommand{\Z}{\mathbb Z}

\newcommand{\sagemath}{\textsf{SageMath}}


%\renewcommand{\d}{\,d\!}
\renewcommand{\d}{\mathop{}\!d}
\newcommand{\dd}[2][]{\frac{\d #1}{\d #2}}
\newcommand{\pp}[2][]{\frac{\partial #1}{\partial #2}}
\renewcommand{\l}{\ell}
\newcommand{\ddx}{\frac{d}{\d x}}

\newcommand{\zeroOverZero}{\ensuremath{\boldsymbol{\tfrac{0}{0}}}}
\newcommand{\inftyOverInfty}{\ensuremath{\boldsymbol{\tfrac{\infty}{\infty}}}}
\newcommand{\zeroOverInfty}{\ensuremath{\boldsymbol{\tfrac{0}{\infty}}}}
\newcommand{\zeroTimesInfty}{\ensuremath{\small\boldsymbol{0\cdot \infty}}}
\newcommand{\inftyMinusInfty}{\ensuremath{\small\boldsymbol{\infty - \infty}}}
\newcommand{\oneToInfty}{\ensuremath{\boldsymbol{1^\infty}}}
\newcommand{\zeroToZero}{\ensuremath{\boldsymbol{0^0}}}
\newcommand{\inftyToZero}{\ensuremath{\boldsymbol{\infty^0}}}



\newcommand{\numOverZero}{\ensuremath{\boldsymbol{\tfrac{\#}{0}}}}
\newcommand{\dfn}{\textbf}
%\newcommand{\unit}{\,\mathrm}
\newcommand{\unit}{\mathop{}\!\mathrm}
\newcommand{\eval}[1]{\bigg[ #1 \bigg]}
\newcommand{\seq}[1]{\left( #1 \right)}
\renewcommand{\epsilon}{\varepsilon}
\renewcommand{\phi}{\varphi}


\renewcommand{\iff}{\Leftrightarrow}

\DeclareMathOperator{\arccot}{arccot}
\DeclareMathOperator{\arcsec}{arcsec}
\DeclareMathOperator{\arccsc}{arccsc}
\DeclareMathOperator{\si}{Si}
\DeclareMathOperator{\scal}{scal}
\DeclareMathOperator{\sign}{sign}


%% \newcommand{\tightoverset}[2]{% for arrow vec
%%   \mathop{#2}\limits^{\vbox to -.5ex{\kern-0.75ex\hbox{$#1$}\vss}}}
\newcommand{\arrowvec}[1]{{\overset{\rightharpoonup}{#1}}}
%\renewcommand{\vec}[1]{\arrowvec{\mathbf{#1}}}
\renewcommand{\vec}[1]{{\overset{\boldsymbol{\rightharpoonup}}{\mathbf{#1}}}}
\DeclareMathOperator{\proj}{\vec{proj}}
\newcommand{\veci}{{\boldsymbol{\hat{\imath}}}}
\newcommand{\vecj}{{\boldsymbol{\hat{\jmath}}}}
\newcommand{\veck}{{\boldsymbol{\hat{k}}}}
\newcommand{\vecl}{\vec{\boldsymbol{\l}}}
\newcommand{\uvec}[1]{\mathbf{\hat{#1}}}
\newcommand{\utan}{\mathbf{\hat{t}}}
\newcommand{\unormal}{\mathbf{\hat{n}}}
\newcommand{\ubinormal}{\mathbf{\hat{b}}}

\newcommand{\dotp}{\bullet}
\newcommand{\cross}{\boldsymbol\times}
\newcommand{\grad}{\boldsymbol\nabla}
\newcommand{\divergence}{\grad\dotp}
\newcommand{\curl}{\grad\cross}
%\DeclareMathOperator{\divergence}{divergence}
%\DeclareMathOperator{\curl}[1]{\grad\cross #1}
\newcommand{\lto}{\mathop{\longrightarrow\,}\limits}

\renewcommand{\bar}{\overline}

\colorlet{textColor}{black} 
\colorlet{background}{white}
\colorlet{penColor}{blue!50!black} % Color of a curve in a plot
\colorlet{penColor2}{red!50!black}% Color of a curve in a plot
\colorlet{penColor3}{red!50!blue} % Color of a curve in a plot
\colorlet{penColor4}{green!50!black} % Color of a curve in a plot
\colorlet{penColor5}{orange!80!black} % Color of a curve in a plot
\colorlet{penColor6}{yellow!70!black} % Color of a curve in a plot
\colorlet{fill1}{penColor!20} % Color of fill in a plot
\colorlet{fill2}{penColor2!20} % Color of fill in a plot
\colorlet{fillp}{fill1} % Color of positive area
\colorlet{filln}{penColor2!20} % Color of negative area
\colorlet{fill3}{penColor3!20} % Fill
\colorlet{fill4}{penColor4!20} % Fill
\colorlet{fill5}{penColor5!20} % Fill
\colorlet{gridColor}{gray!50} % Color of grid in a plot

\newcommand{\surfaceColor}{violet}
\newcommand{\surfaceColorTwo}{redyellow}
\newcommand{\sliceColor}{greenyellow}




\pgfmathdeclarefunction{gauss}{2}{% gives gaussian
  \pgfmathparse{1/(#2*sqrt(2*pi))*exp(-((x-#1)^2)/(2*#2^2))}%
}


%%%%%%%%%%%%%
%% Vectors
%%%%%%%%%%%%%

%% Simple horiz vectors
\renewcommand{\vector}[1]{\left\langle #1\right\rangle}


%% %% Complex Horiz Vectors with angle brackets
%% \makeatletter
%% \renewcommand{\vector}[2][ , ]{\left\langle%
%%   \def\nextitem{\def\nextitem{#1}}%
%%   \@for \el:=#2\do{\nextitem\el}\right\rangle%
%% }
%% \makeatother

%% %% Vertical Vectors
%% \def\vector#1{\begin{bmatrix}\vecListA#1,,\end{bmatrix}}
%% \def\vecListA#1,{\if,#1,\else #1\cr \expandafter \vecListA \fi}

%%%%%%%%%%%%%
%% End of vectors
%%%%%%%%%%%%%

%\newcommand{\fullwidth}{}
%\newcommand{\normalwidth}{}



%% makes a snazzy t-chart for evaluating functions
%\newenvironment{tchart}{\rowcolors{2}{}{background!90!textColor}\array}{\endarray}

%%This is to help with formatting on future title pages.
\newenvironment{sectionOutcomes}{}{} 



%% Flowchart stuff
%\tikzstyle{startstop} = [rectangle, rounded corners, minimum width=3cm, minimum height=1cm,text centered, draw=black]
%\tikzstyle{question} = [rectangle, minimum width=3cm, minimum height=1cm, text centered, draw=black]
%\tikzstyle{decision} = [trapezium, trapezium left angle=70, trapezium right angle=110, minimum width=3cm, minimum height=1cm, text centered, draw=black]
%\tikzstyle{question} = [rectangle, rounded corners, minimum width=3cm, minimum height=1cm,text centered, draw=black]
%\tikzstyle{process} = [rectangle, minimum width=3cm, minimum height=1cm, text centered, draw=black]
%\tikzstyle{decision} = [trapezium, trapezium left angle=70, trapezium right angle=110, minimum width=3cm, minimum height=1cm, text centered, draw=black]


\outcome{Evaluate limits of functions of several variables.}

\begin{document}
\begin{exercise}

Suppose that $F(x,y) = \begin{cases}  4x^2y , & x+y > 1 \\ 2x, & x+y \leq 1 \end{cases}$.  

Answer the following questions.

\begin{exercise}
True or False:  

$\Lim{x,y}{-1,1} F(x,y)= \Lim{x,y}{-1,1} 2x = -2$.   \wordChoice{\choice[correct]{True}\choice{False}}

\begin{feedback}[correct]
Near the point $(-1,1)$, the function $F(x,y)$ is defined by $F(x,y) = 2x$ for all $(x,y)$.  

\begin{image}
\begin{tikzpicture}

\begin{axis}
	[
	domain=-3:5, ymax=2.9,xmax=3.9, ymin=-2.9, xmin=-2.9,
	axis lines=center, xlabel=$x$, ylabel=$y$,
	every axis y label/.style={at=(current axis.above origin),anchor=south},
	every axis x label/.style={at=(current axis.right of origin),anchor=west},
	axis on top,
	typeset ticklabels with strut,
	]

	\addplot [draw=penColor,very thick, dashed] {1.05-x};
\addplot [draw=penColor2,very thick, smooth] {1-x};
	
	\addplot [name path=A,domain=-3:4,draw=none] {1-x};   
	\addplot [name path=B,domain=-3:4,draw=none] {10};
	\addplot [name path=C,domain=-3:4,draw=none] {-10};
	\addplot [fill=penColor!40] fill between[of=A and B];
	\addplot [fill=penColor2!40] fill between[of=A and C];
	
	\node at (axis cs:2,1.6) [penColor] {\small $F(x,y) = 4x^2y$};
	\node at (axis cs:2,1.2) [penColor] {\small for all $(x,y)$ here.};
	\node at (axis cs:-1.5,-1.2) [penColor2] {\small $F(x,y) = 2x$};
	\node at (axis cs:-1.5,-1.6) [penColor2] {\small for all $(x,y)$ here.};
	
	\addplot[color=penColor2,fill=penColor2,only marks,mark=*] coordinates{(-1,1)};
\end{axis}
\end{tikzpicture}
\end{image}
\begin{center}
The domain of the function, $\R^2$, is shown as well as the 

formula used to find $F(x,y)$ for each $(x,y)$ in the domain.
\end{center}

More precisely, this means that we can squeeze an entire open ball centered at $(-1,1)$ in which $F(x,y) = 2x$ for any $(x,y)$ in the open ball.  To construct the open ball, we just need to use a radius so the ball does not intersect the line $x+y=1$.  From the image provided, it is clear that we can find such a ball.
\end{feedback}

\end{exercise}

\begin{exercise}
True or False:  

$F(-1,2) = 2(-1)+2$, so $\Lim{x,y}{-1,2} F(x,y)= \Lim{x,y}{-1,2} 2x = -2$.   \wordChoice{\choice{True}\choice[correct]{False}}

\begin{feedback}[correct]
Near the point $(-1,2)$, the function $F(x,y)$ is not defined by $F(x,y) = 2x$ for all $(x,y)$; at any point lower than and/or left of $(-1,2)$, we have $F(x,y) = 2x$, but for any point higher than and/or right of $(-1,2)$, we have that $F(x,y) =4x^2y$.    

\begin{image}
\begin{tikzpicture}

\begin{axis}
	[
	domain=-3:5, ymax=2.9,xmax=3.9, ymin=-2.9, xmin=-2.9,
	axis lines=center, xlabel=$x$, ylabel=$y$,
	every axis y label/.style={at=(current axis.above origin),anchor=south},
	every axis x label/.style={at=(current axis.right of origin),anchor=west},
	axis on top,
	typeset ticklabels with strut,
	]

	\addplot [draw=penColor, thick, dashed] {1.03-x};
\addplot [draw=penColor2, thick, smooth] {1-x};
	
	\addplot [name path=A,domain=-3:4,draw=none] {1-x};   
	\addplot [name path=B,domain=-3:4,draw=none] {10};
	\addplot [name path=C,domain=-3:4,draw=none] {-10};
	\addplot [fill=penColor!40] fill between[of=A and B];
	\addplot [fill=penColor2!40] fill between[of=A and C];
	
	\node at (axis cs:2,1.6) [penColor] {\small $F(x,y) = 4x^2y$};
	\node at (axis cs:2,1.2) [penColor] {\small for all $(x,y)$ here.};
	\node at (axis cs:-1.5,-1.2) [penColor2] {\small $F(x,y) = 2x$};
	\node at (axis cs:-1.5,-1.6) [penColor2] {\small for all $(x,y)$ here.};
	
	\addplot[color=black,fill=black,only marks,mark=*] coordinates{(-1,2)};
\end{axis}
\end{tikzpicture}
\end{image}
\begin{center}
The domain of the function, $\R^2$, is shown as well as the 

formula used to find $F(x,y)$ for each $(x,y)$ in the domain.
\end{center}

More precisely, this means that no open ball centered at $(-1,2)$ contains points for $F(x,y) = 2x$ for any $(x,y)$ in the open ball. From the image provided, it is clear that no matter how small the radius of a ball centered at $(-1,2)$ may be, it will always contain points in both the blue and red regions.
\end{feedback}

\begin{exercise}
Compute $\Lim{x,y}{-1,2} F(x,y)$ or conclude that it does not exist by following the steps below.

\begin{itemize}
\item If we approach $(-1,2)$ along the line $x=-1$ and take $y<2$, $F(x,y) = \answer{2x}$.  Thus $F(x,y) \to \answer{-2}$ as $\point{x,y} \to \point{-1,2}$ along this path.
\item If we approach $(-1,2)$ along the line $x=-1$ and take $y>2$, $F(x,y) = \answer{4x^2y}$.  Thus $F(x,y) \to \answer{8}$ as $\point{x,y} \to \point{-1,2}$ along this path.
\end{itemize}

Does $\Lim{x,y}{-1,2} F(x,y)$ exist? \wordChoice{\choice{Yes}\choice[correct]{No}}
\end{exercise}
\end{exercise}

\begin{exercise}
Is the function continuous at any point $(x,y)$ for which $x+y \neq 1$? \wordChoice{\choice[correct]{Yes}\choice{No}}

Is the function discontinuous at any point for which $x+y =1$? \wordChoice{\choice{Yes}\choice[correct]{No}}

\begin{feedback}[correct]
As long as $x+y \neq 1$, the function is defined by the same formula for all $(x,y)$ in a suitable open ball centered at $(x,y)$, so $\Lim{x,y}{a,b}F(x,y) = F(a,b)$.  Since each piece is continuous, this means $F(x,y)$ is continuous at all $(a,b)$ for which $a+b \neq 1$.

Along the line $x+y=1$, we have to check whether the pieces ``line up'' to give the same $z$-value.
\end{feedback}
\begin{exercise}
If $x+y=1$, how many ordered pair $(x,y)$ are there for which $F(x,y)$ is continuous?

\begin{multipleChoice}
\choice{None}
\choice[correct]{One}
\choice{Two}
\choice{Three}
\choice{More than three, but finitely many}
\choice{Infinitely many}
\end{multipleChoice}

To find these points, we must have 

\[
4x^2y=2x.
\]
and noting that along the line $x+y=1$, we have $y=1-x$ and thus must find all $x$ so

\[
4x^2(1-x)=2x.
\]

Simplifying and rearranging this gives

\[
\answer{-4}x^3+\answer{4}x^2-\answer{2}x = 0.
\]
(don't just cancel out the $x$ on each side; doing so eliminates one of the solutions!)

This means either $x=0$ or $\answer{-4x^2+4x}-2=0$. 

\begin{itemize}
\item For $x=0$, we find that $y=\answer{1}$, so the function is continuous at $\left(0,\answer{1}\right)$.
\item The quadratic equation can be used to show that there are \wordChoice{\choice[correct]{no}\choice{one}\choice{two}} real solutions to the quadratic equation $-4x^2+4x-2=0$.
\end{itemize}

\end{exercise}
\end{exercise}
 \end{exercise}
\end{document}
