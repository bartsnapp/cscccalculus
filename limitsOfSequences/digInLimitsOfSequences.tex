\documentclass{ximera}

%\usepackage{todonotes}

\newcommand{\todo}{}

\usepackage{esint} % for \oiint
\ifxake%%https://math.meta.stackexchange.com/questions/9973/how-do-you-render-a-closed-surface-double-integral
\renewcommand{\oiint}{{\large\bigcirc}\kern-1.56em\iint}
\fi


\graphicspath{
  {./}
  {ximeraTutorial/}
  {basicPhilosophy/}
  {functionsOfSeveralVariables/}
  {normalVectors/}
  {lagrangeMultipliers/}
  {vectorFields/}
  {greensTheorem/}
  {shapeOfThingsToCome/}
  {dotProducts/}
  {../productAndQuotientRules/exercises/}
  {../normalVectors/exercisesParametricPlots/}
  {../continuityOfFunctionsOfSeveralVariables/exercises/}
  {../partialDerivatives/exercises/}
  {../chainRuleForFunctionsOfSeveralVariables/exercises/}
  {../commonCoordinates/exercisesCylindricalCoordinates/}
  {../commonCoordinates/exercisesSphericalCoordinates/}
  {../greensTheorem/exercisesCurlAndLineIntegrals/}
  {../greensTheorem/exercisesDivergenceAndLineIntegrals/}
  {../shapeOfThingsToCome/exercisesDivergenceTheorem/}
  {../greensTheorem/}
  {../shapeOfThingsToCome/}
}

\newcommand{\mooculus}{\textsf{\textbf{MOOC}\textnormal{\textsf{ULUS}}}}

\usepackage{tkz-euclide}\usepackage{tikz}
\usepackage{tikz-cd}
\usetikzlibrary{arrows}
\tikzset{>=stealth,commutative diagrams/.cd,
  arrow style=tikz,diagrams={>=stealth}} %% cool arrow head
\tikzset{shorten <>/.style={ shorten >=#1, shorten <=#1 } } %% allows shorter vectors

\usetikzlibrary{backgrounds} %% for boxes around graphs
\usetikzlibrary{shapes,positioning}  %% Clouds and stars
\usetikzlibrary{matrix} %% for matrix
\usepgfplotslibrary{polar} %% for polar plots
\usepgfplotslibrary{fillbetween} %% to shade area between curves in TikZ
\usetkzobj{all}
%\usepackage[makeroom]{cancel} %% for strike outs
%\usepackage{mathtools} %% for pretty underbrace % Breaks Ximera
%\usepackage{multicol}
\usepackage{pgffor} %% required for integral for loops



%% http://tex.stackexchange.com/questions/66490/drawing-a-tikz-arc-specifying-the-center
%% Draws beach ball
\tikzset{pics/carc/.style args={#1:#2:#3}{code={\draw[pic actions] (#1:#3) arc(#1:#2:#3);}}}



\usepackage{array}
\setlength{\extrarowheight}{+.1cm}   
\newdimen\digitwidth
\settowidth\digitwidth{9}
\def\divrule#1#2{
\noalign{\moveright#1\digitwidth
\vbox{\hrule width#2\digitwidth}}}





\newcommand{\RR}{\mathbb R}
\newcommand{\R}{\mathbb R}
\newcommand{\N}{\mathbb N}
\newcommand{\Z}{\mathbb Z}

\newcommand{\sagemath}{\textsf{SageMath}}


%\renewcommand{\d}{\,d\!}
\renewcommand{\d}{\mathop{}\!d}
\newcommand{\dd}[2][]{\frac{\d #1}{\d #2}}
\newcommand{\pp}[2][]{\frac{\partial #1}{\partial #2}}
\renewcommand{\l}{\ell}
\newcommand{\ddx}{\frac{d}{\d x}}

\newcommand{\zeroOverZero}{\ensuremath{\boldsymbol{\tfrac{0}{0}}}}
\newcommand{\inftyOverInfty}{\ensuremath{\boldsymbol{\tfrac{\infty}{\infty}}}}
\newcommand{\zeroOverInfty}{\ensuremath{\boldsymbol{\tfrac{0}{\infty}}}}
\newcommand{\zeroTimesInfty}{\ensuremath{\small\boldsymbol{0\cdot \infty}}}
\newcommand{\inftyMinusInfty}{\ensuremath{\small\boldsymbol{\infty - \infty}}}
\newcommand{\oneToInfty}{\ensuremath{\boldsymbol{1^\infty}}}
\newcommand{\zeroToZero}{\ensuremath{\boldsymbol{0^0}}}
\newcommand{\inftyToZero}{\ensuremath{\boldsymbol{\infty^0}}}



\newcommand{\numOverZero}{\ensuremath{\boldsymbol{\tfrac{\#}{0}}}}
\newcommand{\dfn}{\textbf}
%\newcommand{\unit}{\,\mathrm}
\newcommand{\unit}{\mathop{}\!\mathrm}
\newcommand{\eval}[1]{\bigg[ #1 \bigg]}
\newcommand{\seq}[1]{\left( #1 \right)}
\renewcommand{\epsilon}{\varepsilon}
\renewcommand{\phi}{\varphi}


\renewcommand{\iff}{\Leftrightarrow}

\DeclareMathOperator{\arccot}{arccot}
\DeclareMathOperator{\arcsec}{arcsec}
\DeclareMathOperator{\arccsc}{arccsc}
\DeclareMathOperator{\si}{Si}
\DeclareMathOperator{\scal}{scal}
\DeclareMathOperator{\sign}{sign}


%% \newcommand{\tightoverset}[2]{% for arrow vec
%%   \mathop{#2}\limits^{\vbox to -.5ex{\kern-0.75ex\hbox{$#1$}\vss}}}
\newcommand{\arrowvec}[1]{{\overset{\rightharpoonup}{#1}}}
%\renewcommand{\vec}[1]{\arrowvec{\mathbf{#1}}}
\renewcommand{\vec}[1]{{\overset{\boldsymbol{\rightharpoonup}}{\mathbf{#1}}}}
\DeclareMathOperator{\proj}{\vec{proj}}
\newcommand{\veci}{{\boldsymbol{\hat{\imath}}}}
\newcommand{\vecj}{{\boldsymbol{\hat{\jmath}}}}
\newcommand{\veck}{{\boldsymbol{\hat{k}}}}
\newcommand{\vecl}{\vec{\boldsymbol{\l}}}
\newcommand{\uvec}[1]{\mathbf{\hat{#1}}}
\newcommand{\utan}{\mathbf{\hat{t}}}
\newcommand{\unormal}{\mathbf{\hat{n}}}
\newcommand{\ubinormal}{\mathbf{\hat{b}}}

\newcommand{\dotp}{\bullet}
\newcommand{\cross}{\boldsymbol\times}
\newcommand{\grad}{\boldsymbol\nabla}
\newcommand{\divergence}{\grad\dotp}
\newcommand{\curl}{\grad\cross}
%\DeclareMathOperator{\divergence}{divergence}
%\DeclareMathOperator{\curl}[1]{\grad\cross #1}
\newcommand{\lto}{\mathop{\longrightarrow\,}\limits}

\renewcommand{\bar}{\overline}

\colorlet{textColor}{black} 
\colorlet{background}{white}
\colorlet{penColor}{blue!50!black} % Color of a curve in a plot
\colorlet{penColor2}{red!50!black}% Color of a curve in a plot
\colorlet{penColor3}{red!50!blue} % Color of a curve in a plot
\colorlet{penColor4}{green!50!black} % Color of a curve in a plot
\colorlet{penColor5}{orange!80!black} % Color of a curve in a plot
\colorlet{penColor6}{yellow!70!black} % Color of a curve in a plot
\colorlet{fill1}{penColor!20} % Color of fill in a plot
\colorlet{fill2}{penColor2!20} % Color of fill in a plot
\colorlet{fillp}{fill1} % Color of positive area
\colorlet{filln}{penColor2!20} % Color of negative area
\colorlet{fill3}{penColor3!20} % Fill
\colorlet{fill4}{penColor4!20} % Fill
\colorlet{fill5}{penColor5!20} % Fill
\colorlet{gridColor}{gray!50} % Color of grid in a plot

\newcommand{\surfaceColor}{violet}
\newcommand{\surfaceColorTwo}{redyellow}
\newcommand{\sliceColor}{greenyellow}




\pgfmathdeclarefunction{gauss}{2}{% gives gaussian
  \pgfmathparse{1/(#2*sqrt(2*pi))*exp(-((x-#1)^2)/(2*#2^2))}%
}


%%%%%%%%%%%%%
%% Vectors
%%%%%%%%%%%%%

%% Simple horiz vectors
\renewcommand{\vector}[1]{\left\langle #1\right\rangle}


%% %% Complex Horiz Vectors with angle brackets
%% \makeatletter
%% \renewcommand{\vector}[2][ , ]{\left\langle%
%%   \def\nextitem{\def\nextitem{#1}}%
%%   \@for \el:=#2\do{\nextitem\el}\right\rangle%
%% }
%% \makeatother

%% %% Vertical Vectors
%% \def\vector#1{\begin{bmatrix}\vecListA#1,,\end{bmatrix}}
%% \def\vecListA#1,{\if,#1,\else #1\cr \expandafter \vecListA \fi}

%%%%%%%%%%%%%
%% End of vectors
%%%%%%%%%%%%%

%\newcommand{\fullwidth}{}
%\newcommand{\normalwidth}{}



%% makes a snazzy t-chart for evaluating functions
%\newenvironment{tchart}{\rowcolors{2}{}{background!90!textColor}\array}{\endarray}

%%This is to help with formatting on future title pages.
\newenvironment{sectionOutcomes}{}{} 



%% Flowchart stuff
%\tikzstyle{startstop} = [rectangle, rounded corners, minimum width=3cm, minimum height=1cm,text centered, draw=black]
%\tikzstyle{question} = [rectangle, minimum width=3cm, minimum height=1cm, text centered, draw=black]
%\tikzstyle{decision} = [trapezium, trapezium left angle=70, trapezium right angle=110, minimum width=3cm, minimum height=1cm, text centered, draw=black]
%\tikzstyle{question} = [rectangle, rounded corners, minimum width=3cm, minimum height=1cm,text centered, draw=black]
%\tikzstyle{process} = [rectangle, minimum width=3cm, minimum height=1cm, text centered, draw=black]
%\tikzstyle{decision} = [trapezium, trapezium left angle=70, trapezium right angle=110, minimum width=3cm, minimum height=1cm, text centered, draw=black]

\author{Bart Snapp and Jim Talamo}

\outcome{Recognize sequences can be generated by functions.}
\outcome{Compute limits of sequences.}
\outcome{Understand growth rates of basic sequences.}
\outcome{Introduce important terminology for sequences.}
\outcome{Apply the monotone convergence theorem.}

\title[Dig-In:]{Limits of sequences}

\begin{document}
\begin{abstract}
There are two ways to establish whether a sequence has a limit.
\end{abstract}
\maketitle

In the previous section, we defined a sequence as a function defined on a subset of the natural numbers, and we discussed how we can represent this by an ordered list.  We chose the notation $\{a_n\}_{n=1}$ to denote the list below.

\[
a_1, a_2, a_3 , \ldots
\]

In the previous section, we found many ways to generate this list.  Regardless of how we obtain it, there are two fundamental questions we can ask.
\begin{itemize}
\item[1.] Do the numbers in the list approach a finite value?
\item[2.] Can I add all of the numbers in the list together and obtain a finite result?
\end{itemize}

As it turns out, the second question will be more important for us.  However, as we will see in a future section, we can reduce the second question to the first one.  As such, we should examine the first question in detail.  We begin by giving an intuitive definition.

\begin{definition}\index{limit of a sequence}
  Given a sequence $\{a_n\}_{n =n_0}$, we say that the \dfn{limit} of the sequence is $L$ if, as $n$ grows arbitrarily large, $a_n$ becomes arbitrarily close to $L$. 
  
If $\lim_{n\to\infty}a_n=L$ we say that the sequence
\dfn{converges}\index{convergent sequence}\index{sequence!convergent}.
If there is no finite value $L$ so that $\lim_{n\to\infty}a_n = L$,
then we say that the limit \dfn{does not exist}, or equivalently that
the sequence \dfn{diverges}\index{divergent
  sequence}\index{sequence!divergent}.
\end{definition}

This intuitive definition of a limit can be made more precise as follows.

\begin{definition}\index{formal limit of a sequence}
\label{definition:limit-of-a-sequence}
Suppose that $\{a_n\}_{n=n_0}$ is a sequence.  We say that
$\lim_{n\to \infty}a_n=L$ if for every $\epsilon>0$, there exists an integer $N$, such that $|a_n-L|<\epsilon$ for any $n \geq N$.
\end{definition}

This precise definition captures the same idea as the intuitive definition but makes it more precise.  The quantity $\epsilon$ measures how close the terms in the sequence are to the limit $L$.  We say that the limit exists and is $L$ if we can choose how far we want the terms to be from $L$ and we know the terms in the sequence eventually become and stay that close to $L$.  

\begin{remark}
The precise definition is extremely important to establish the theoretical foundations of sequences and is used frequently in more theoretically-oriented courses.  For our purposes, however, the intuitive definition will be sufficient.
\end{remark}

\begin{question}
Suppose that $\{a_n\}_{n=1}$ is a sequence and that $\lim_{n \to \infty} a_n = L$.  Intuitively, what can we say about $\lim_{n \to \infty} a_{n+1}$? \wordChoice{\choice{$\lim_{n \to \infty} a_{n+1}$ exists, but we do not know what its value is.} \choice[correct]{$\lim_{n \to \infty} a_{n+1}$ exists, and $\lim_{n \to \infty} a_{n+1}=L$ exists.} \choice{$\lim_{n \to \infty} a_{n+1}$ may or may not exist.}}

One way to think about this is by noting that the sequence $\{a_n\}_{n=1}$ is represented by the list

\[
a_1,a_2,a_3, \ldots ,
\]

while the sequence $\{a_{n+1}\}_{n=1}$ is represented by the list below.

\[
a_2,a_3,a_4, \ldots
\]

If the first sequence tends to $L$, the second sequence must also tend to $L$.

\end{question}





%\begin{question}
%  To say that the sequence $a_n$ converges to $L$ means what?  In
%  other words, what is the definition of the statement
%  $\lim_{n\to\infty} a_n = L$?
%  \begin{hint}
%    We are trying to make precise the idea that, eventually, all the elements of the sequence $a_n$ are as close as we want to $L$.
%  \end{hint}
%  \begin{hint}
%    To measure closeness to $L$, we will use a positive real number $\epsilon$.
%  \end{hint}
%  \begin{hint}
%    We must achieve any desired degree of closeness, so we will make a statement which is true for any positive real number $\epsilon$.
%  \end{hint}
%  \begin{hint}
%    In other words, the definition will begin ``For every positive real number $\epsilon > 0$\ldots''.
%  \end{hint}
%  \begin{hint}
%    We now must make precise the idea of ``eventually'' close.
%  \end{hint}
%  \begin{hint}
%    We use a whole number $N$ to capture the idea of ``sufficiently large'' values of $n$.
%  \end{hint}
%  \begin{hint}
%    Specifically, the definition will begin ``For every positive real number $\epsilon > 0$, there exists an $N \in \mathbb{N}$\ldots''
%  \end{hint}
%  \begin{hint}
%    The ``sufficiently large'' value of $n$ is any value which is at least as large as $N$.
%  \end{hint}
%  \begin{hint}
%    So, we will only consider those $n$ for which $n \ge N$.
%  \end{hint}
%    \begin{hint}
%      Thus the definition goes ``For every positive real number $\epsilon > 0$, there exists an $N \in \mathbb{N}$ so that whenever $n \ge N$\ldots''
%    \end{hint}
%    \begin{hint}
%      What happens ``eventually'' is that elements of the sequence are close to $L$.  How close?  Within $\epsilon$.
%    \end{hint}
%    \begin{hint}
%      The quantity $|a_n - L|$ is the distance between $a_n$ and $L$.
%    \end{hint}
%    \begin{hint}
%      To say that $a_n$ is within $\epsilon$ of $L$ is to say that $|a_n - L| < \epsilon$.
%    \end{hint}
%    \begin{hint}
%      Therefore the definition is ``For every positive real number $\epsilon > 0$ there exists an $N \in \mathbb{N}$ so that whenever $n \ge N$, we have $ |a_n - L| < \epsilon $.''
%    \end{hint}
%
%    \begin{multipleChoice}
%      \choice[correct]{For every positive real number $\epsilon > 0$ there exists an $N \in \mathbb{N}$ so that whenever $n \ge N$, we have $ |a_n - L| < \epsilon $.}
%      \choice{For every real number $\epsilon > 0$ there exists an $N \in \mathbb{N}$ so that $ |a_N - L| < \epsilon $.}
%      \choice{For every real number $\epsilon \in \mathbb{R}$ there exists an $N \in \mathbb{N}$ so that whenever $n \ge N$, we have $ |a_n - L| < \epsilon $.}
%      \choice{For every whole number $N > 0$ there exists a positive real number $\epsilon > 0$ so that whenever $n \ge N$, we have $ |a_n - L| < \epsilon $.}
%      \choice{For every whole number $N > 0$ there exists a real number $\epsilon \in \mathbb{R}$ so that whenever $n \ge N$, we have $ |a_n - L| < \epsilon $.}
%    \end{multipleChoice}
%    
%
%  The definition of limit can be written as if it were poetry with
%  line breaks and all.  Like the best of poems, it deserves to be
%  memorized, performed, and internalized.  Humanity struggled for millennia
%  to find the wisdom contained in this definition.
%\end{question}

\begin{warning}
  In the case that $\lim_{n \to \infty} a_n = \pm\infty$, we say that
  $\{a_n\}$ diverges.  The only time we say that a sequence converges
    is when the limit exists and is equal to a \textit{finite} value.
\end{warning}

%\youtube{https://www.youtube.com/watch?v=0UCRZAsIkXM}


\section{Connections to real-valued functions}
Since sequences are functions defined on the integers, the notion of a ``limit at a specific $n$'' is not very interesting since we can explicitly find $a_n$ for a given $n$. However, limits at \textit{infinity} are a different story.  An important question can now be asked; given a sequence, how do we determine if it has a limit?  

There are several techniques that allow us to find limits of real-valued functions, and we have seen that if we have a sequence, we can often find a real-valued function that agrees with it on their common domains.   Suppose that we have found a real-valued function $f(x)$ that agrees with $a_n$ on their common domains, i.e. that $f(n)=a_n$.  If we know  $\lim_{x\to\infty} f(x)$, can we use this to conclude something about $\lim_{n \to \infty} a_n$?  

Before answering this question, consider the following cautionary example.

\begin{example}
  Let $a_n = \sin(n\pi)$ and $f(x) = \sin(\pi x)$. Show that
  \[
  \lim_{n\to\infty} a_n \ne \lim_{x\to \infty}f(x).
  \]
  \begin{explanation}
  The sequence $\{a_n\}_{n=1}^{\infty}$ is represented by the ordered  list of numbers below.
  \[
  \sin(0\pi),\, \sin(1\pi),\, \sin(2\pi),\,\sin(3\pi),\,\ldots
  \]
Since $\sin(n\pi)=0$, this list is actually a list of zeroes.
  \[
  0,\, 0 , \, 0 ,\,0,\,\ldots
  \]
Since every term in the sequence is $0$, we have
\[
\lim_{n\to\infty} a_n = 0. 
\]
But $\lim_{x\to\infty}f(x)$, when $x$ is real, does not exist; as $x$
becomes arbitrarily large, the values $\sin(x\pi)$ do not get closer and
closer to a single value, but instead oscillate between $-1$ and $1$.

This is shown graphically below.

\begin{image}
\begin{tikzpicture}
	\begin{axis}[
            domain=.8:6.5,xmin=0,xmax=6.5,ymin=-1.5,ymax=1.5,
            width=4in,
            height=2in,
            axis lines =middle, xlabel=$x$, ylabel=$y$,
            xtick={1,2,...,6},
            ytick={-1,1},
            every axis y label/.style={at=(current axis.above origin),anchor=south},
            every axis x label/.style={at=(current axis.right of origin),anchor=west},
            clip=false,
            %axis on top,
          ]
       
          \addplot [penColor5,very thick,smooth,domain=.2:6.2,samples=300]{sin(deg(pi*x)))};
          \addplot[color=penColor,fill=penColor,only marks,mark=*,ultra thick] coordinates{(1,0)};  %% closed hole        
          \addplot[color=penColor,fill=penColor,only marks,mark=*,ultra thick] coordinates{(2,0)};  %% closed hole        
          \addplot[color=penColor,fill=penColor,only marks,mark=*,ultra thick] coordinates{(3,0)};  %% closed hole        
          \addplot[color=penColor,fill=penColor,only marks,mark=*,ultra thick] coordinates{(4,0)};  %% closed hole        
          \addplot[color=penColor,fill=penColor,only marks,mark=*,ultra thick] coordinates{(5,0)};  %% closed hole    
          \addplot[color=penColor,fill=penColor,only marks,mark=*,ultra thick] coordinates{(6,0)};  %% closed hole         
        \end{axis}
\end{tikzpicture}
\end{image}


  \end{explanation}
\end{example}

What can we conclude from the above example?

\begin{multipleChoice}
\choice{If $\lim_{n \to \infty} a_n$ exists, then $\lim_{x \to \infty} f(x)$ exists.}
\choice{If  $\lim_{x \to \infty} f(x)$ does not exist, then $\lim_{n \to \infty} a_n$ does not exist.}
\choice[correct]{If $\lim_{x \to \infty} f(x)$ does not exist, $\lim_{n \to \infty} a_n$ may still exist.}
\end{multipleChoice}

This might lead us to believe that we need to develop a whole new arsenal of techniques in order to determine if limits of sequences exist, but there is good news.

\section{Calculating limits of sequences}

\begin{theorem}
  Let $\{a_n\}$ be a sequence and suppose that $f(x)$ is a real-valued function for which $f(n) = a_n$ for all integers $n$.  If
  \[
  \lim_{x\to\infty}f(x)=L,
  \]
  then $\lim_{n\to\infty} a_n=L$ as well.
\end{theorem}

If we think about the theorem a bit further, the conclusion  of the theorem and the content of the preceding example should seem reasonable.  If the values of $f(x)$ become arbitrarily close to a number $L$ for \emph{all} arbitrarily large $x$-values, then the result should still hold when we only consider \emph{some} of these values.  However, if we only know what happens for \emph{some} arbitrarily large $x$-values, we cannot say what happens for \emph{all} of them!

%
%Here is some general advice. If you want to know $\lim_{n\to\infty}
%a_n$, you might first think of a function $f(x)$ where $a_n = f(n)$,
%and then attempt to compute $\lim_{x\to\infty}f(x)$.  If the limit of
%the function exists, then it is equal to the limit of the sequence.
%But, if for some reason $\lim_{x\to\infty}f(x)$ does not exist, it may
%nevertheless still be the case that $\lim_{n\to\infty}a_n$ exists,
%you'll just have to figure out another way to compute it.

%%%%%%%%%%%%%THE COMMENTED MATERIAL BELOW IS DANGEROUS; AN IMPORTANT POINT IN THE PRECEDING SECTION IS THAT THERE IS A DIFFERENCE BETWEEN THE LIST AND THE RULE THAT GENERATES IT.  WE SHOULD NOT BE PURPOSEFULLY ENCOURAGING STUDENTS TO CONFLATE CLASSES OF OBJECTS IN A CHAPTER THAT REALLY REQUIRES THEM TO CRYSTALLIZE DEFINITIONS AND CONCEPTS%%%%%%%%%%%%%%%%%%%

%Let's summarize the preceding section in the following formal definition.
%\begin{definition}
%  A \dfn{sequence} $\{a_n\}_{n=N}$ is an order list that is generated by a real-valued
%  function $f:D \to \R$, where $D = \{ N, N+1, N+2, \ldots \}$ is the domain of the function
%  \]
%\end{definition}
%
%Stated more humbly, a sequence assigns a unique real number to each of the integers
%starting with an index $N$.
%
%When thought of as a function, the ``outputs'' of a sequence are the
%\dfn{elements} of the sequence; the ``$n$th element'' is the real
%number that the sequence associates to the natural number $n$, and is
%usually written $a_n$. \index{sequence!element} The $n$ in the phrase
%``$n$th element'' is called an \dfn{index}\index{sequence!index}; the
%plural of index is either indices or indexes, depending on who you
%ask.  The first index $N$ is called the \dfn{initial index}. 


%%%%%%%%%%%%%%%%%%%%%%%%%%%%%%%%%%%%%%%%%%%%%%%%%%%%



%\section{Plotting sequences}<------ARITHMETIC TO EXERCISES?
%
%First, we plot sequences as points. Later, we will see another
%interpretation.
%
%

\begin{example}
Let $a_n = \frac{5n+1}{6n+7}$.  Determine if the sequence $\{a_n\}_{n=1}^{\infty}$ has a limit.

\begin{explanation}
A function that can be used to generate the sequence is $f(x) = \frac{5x+1}{6x+7}$.  Since $\lim_{x \to \infty} \frac{5x+1}{6x+7} = \answer[given]{\frac{5}{6}}$, $\lim_{n \to \infty} a_n = \answer[given]{\frac{5}{6}}$.
\end{explanation}

\end{example}

\begin{warning}
Remember that the converse of this theorem is not
true.  In the example preceding this theorem, we have an explicit example of a function $f(x)$ and a sequence $(a_n)$ where $a_n
=f(n)$ and
\[
\lim_{x\to\infty}f(x)=\text{DNE} \quad\text{but} \quad \lim_{n\to\infty} a_n = 0.
\]
\end{warning}

In practice, we use the above theorem to compute limits without explicitly exhibiting the function of a real variable from which the limit is derived.  

\section{Computing limits of sequences using dominant term analysis}
The last example shows us that for many sequences, we can employ the same techniques that we used to compute limits previously.  While algebraic techniques and L'Hopital's rule are useful, in many of the following sections, being able to determine limits quickly is an important skill.    

\begin{example}
Let $a_n = \frac{n^3+4n^2-1}{2-4n^4}$.  Determine if the limit of the sequence $\{a_n\}_{n=1}^{\infty}$ exists.

\begin{explanation}
The highest degree term in the numerator is $n^3$, while the largest term in the denominator is $-4n^4$.  We can factor out the largest terms from both the numerator and denominator and do a little algebra.

\[
\frac{n^3+4n^2-1}{2-4n^4} = \frac{n^3\left(1+\frac{4}{n}-\frac{1}{n^3}\right)}{-4n^4\left(-\frac{1/2}{n^4}+1\right)} = \frac{n^3}{-4n^4} \cdot  \frac{1+\frac{4}{n}-\frac{1}{n^3}}{-\frac{1/2}{n^4}+1}
\]
The second term becomes arbitrarily close to $1$ as $n$ grows larger and larger, so the limit of the sequence is completely determined by the ratio of the highest degree term in the numerator to the highest degree term in the denominator.  In this case, that ratio is $\frac{n^3}{-4n^4} = \frac{1}{-4n}$, so $\lim_{n \to \infty} a_n = 0$.

\end{explanation}
\end{example}

In the preceding example, we say that the \emph{dominant term} in the numerator is $n^3$ and that the \emph{dominant term} in the denominator is $-4n^4$ because these terms are the only ones that are relevant when finding the limit.

\begin{remark}
The reader may notice that the last example is a special case of the Rational Function Theorem.  However, the name given to this result is not as important as the idea it captures.  When finding limits of functions, it is only necessary to consider the dominant term.  When treating quotients of functions, we only need to consider the dominant terms in the numerator and denominator.
\end{remark}

Sometimes, this technique can be used to find limits where L'Hopital's rule or an algebraic approach would be complicated.

\begin{example}
Let $a_n = \frac{n^2(2n+1)(5-3n)}{(1+2n)^4}$.  Determine if the limit of the sequence $\{a_n\}_{n=1}^{\infty}$ exists.

\begin{explanation}
While we could perform the multiplication in both the numerator and denominator explicitly, we can spot the dominant term more efficiently.  
\begin{itemize}
\item The highest degree term in the numerator is $n^2 \cdot 2n \cdot (-3n) = \answer{-6n^4}$.
\item The dominant term in the denominator is $(2n)^4 = \answer{16}n^4$.  
\end{itemize}

By noting that 

\[ \lim_{n \to \infty} \frac{n^2(2n+1)(5-3n)}{(1+2n)^4} = \lim_{n \to \infty} \frac{-6n^4}{16n^4} = \answer{-\frac{6}{16}}, \]
we find $\lim_{n \to \infty} a_n = \answer{-\frac{6}{16}}$.
\end{explanation}
\end{example}
 
\subsection{Growth rates}
The preceding examples illustrate that higher positive powers of $n$ grow more quickly than lower positive powers of $n$.  We can introduce a little notation that captures the rate at which terms in a sequence grow in a succinct way.

\begin{definition}
  Given two sequences $\{a_n\}$ and $\{b_n\}$, the notation $a_n \ll
  b_n$ means that
  \[
  \lim_{n\to\infty} \frac{a_n}{b_n} =
  0\qquad\text{and}\qquad\lim_{n\to\infty} \frac{b_n}{a_n} =\infty.
  \]
\end{definition}

In essence, writing $a_n \ll b_n$ says that the sequence $(b_n)$ grows
much faster than $(a_n)$.

\begin{example}
Suppose that $a_n = 4n^2+3n$ and $b_n = 5n^{3/2}+2n$.  Then, we can compute $\lim_{n \to \infty} \frac{a_n}{b_n} = \answer{\infty}$ and $\lim_{n \to \infty} \frac{b_n}{a_n} = \answer{0}$.  Using the notation we just introduced, we have that \wordChoice{\choice{$a_n  \ll b_n$}\choice[correct]{$b_n  \ll a_n$}}
\end{example}

Many sequences of interest involve terms other than powers of $n$.  It is often useful to understand how different \emph{types} of functions grow relative to each other.

\begin{theorem}[Growth rates of sequences]
  Let $p,q$ be positive real numbers, and let $b> 1$. We have the
  following relationships.
  \[
  \ln^p(n)\ll n^q \ll b^n \ll n! \ll n^n
  \]
\end{theorem}


%\textbf{PICTURES TO FOLLOW}

The first inequality in this theorem essentially guarantees that \emph{any} power of $\ln(n)$ grows more slowly than \emph{any} power of $n$.  For example: 

\begin{example}
  Let $a_n  = \frac{\ln^{9}(n)}{n^{1/2}}$.  What is $\lim_{n \to \infty} a_n$?
  
  \begin{explanation}
  The notation  $\ln^p(n)\ll n^q$ means that $\lim_{n \to \infty} \frac{\ln^p(n)}{n^q} = 0$ for any positive numbers $p$ and $q$.  In this example, $p=9$ and $q=1/2$, so by the growth rates result, $\lim_{n\to\infty}a_n =\answer[given]{0}$.  
  
  \end{explanation}
\end{example}

This allows us to extend the \emph{dominant term} idea to more complicated expressions.

\begin{example}
  Let $a_n  = \frac{n^{100} + n^n}{n!+5^n}$.  What is $\lim_{n \to \infty} a_n$?
  
  \begin{explanation}
  By growth rates, the dominant term in the numerator is $n^n$, and the dominant term in the denominator is $n!$.  We thus will know if $\lim_{n \to \infty} a_n$ exists by considering $\lim_{n \to \infty} \frac{n^n}{n!}$.  By growth rates, this limit is infinite, so $\lim_{n \to \infty} a_n = \infty$.
  
This can be made more explicit by the following computation, which shows exactly how the growth rates results are used.  As with a previous example, it relies on factoring the dominant term in the numerator and the denominator.  These terms are determined by the growth rates results.

\[
\lim_{n \to \infty} \frac{n^{100} + n^n}{n!+5^n} = \lim_{n \to \infty} \frac{n^n \left(\frac{n^{100}}{n^n} + 1\right)}{n!\left(1+\frac{5^n}{n!}\right)} 
\]

By the growth rates results, $\lim_{n \to \infty} \frac{n^{100}}{n^n} =0$ and  $\lim_{n \to \infty}\frac{5^n}{n!} =0$, so we have: 
  
 \[  \lim_{n \to \infty} \frac{n^n \left(\frac{n^{100}}{n^n} + 1\right)}{n!\left(1+\frac{5^n}{n!}\right)} =   \lim_{n \to \infty} \frac{n^n \left(0 + 1\right)}{n!\left(1+0\right)} = \lim_{n \to \infty} \frac{n^n}{n!}.  \]
   
   
  \end{explanation}
\end{example}

%%%%%%%%%%%%%%%%%%%%%%%%%%%%%%%%%%%%%%%%
\subsection{The squeeze theorem}

%
%\textbf{A PICTURE IS FORTHCOMING}

Previously, when considering limits, one of our techniques was to replace 
complicated functions by simpler functions. The \textit{Squeeze Theorem}
tells us one situation where this is possible.

\begin{theorem}[Squeeze Theorem]\index{Squeeze Theorem}
  Suppose that $(a_n)$, $(b_n)$, and $(c_n)$ are sequences with
  \[
  a_n \le b_n \le c_n
  \]
  for all $n$ greater than some index $N$. If
  \[
  \lim_{n\to\infty} a_n = L = \lim_{n\to\infty} c_n,
  \] 
  then $\lim_{n\to\infty} b_n = L$.
\end{theorem}

Let's see an example.

\begin{example}
  Consider the sequence $(b_n)_{n=1}^{\infty}$ where $b_n =
  \left(\frac{-1}{2}\right)^n$. Compute:
  \[
  \lim_{n\to\infty}b_n
  \]
  \begin{explanation}
    To compute this limit we will use the squeeze theorem. Write with
    me:
    \[
    -\left(\frac{1}{2}\right)^n\le \left(\frac{-1}{2}\right)^n \le \left(\frac{1}{2}\right)^n
    \]
    but we know
    \[
    \lim_{n\to \infty}\left(-\left(\frac{1}{2}\right)^n\right) = \answer[given]{0}
    \]
    and
    \[
    \lim_{n\to \infty}\left(\frac{1}{2}\right)^n =\answer[given]{0}.
    \]
    Hence by the squeeze theorem, $\lim_{n\to\infty} b_n = \answer[given]{0}$.
  \end{explanation}
\end{example}

%
%\textbf{WHAT DO WE WANT TO DO HERE?  ADD MORE PICTURES AS EVIDENCE?}

The squeeze theorem is helpful in establishing a more general result about \emph{geometric} sequences.
\begin{theorem}
  Given a geometric sequence $\{a_n\}_{n=n_0}$ where $a_n = a \cdot r^{n}$,
  \[
  \lim_{n\to\infty} a_n =
  \begin{cases}
    0 &\text{if $|r|<1$,}\\
    1 &\text{if $r=1$,}\\
    \text{DNE} &\text{if $|r|>1$ or $r=-1$.}
  \end{cases}
  \]
\end{theorem}
Of course, when $r$ is positive, the squeeze theorem is not necessary, but it is useful when establishing the convergence results for $1<r<0$ as in the preceding example.
%%%%%%%%%%%%%%%%%%%%%%%%%%%%%%%%%%%%%%%%

\section{Existence results for limits}

Perhaps the best way to determine whether the limit of a sequence exists is to compute it.  Even though we've been working with sequences that are generated by an explicit formula in this section thus far, not all sequences are defined this way.  Sometimes, we'll only have a recursive description of a sequence rather than an explicit one, and sometimes we will have neither.  In many of the coming sections, we will only have a recursive description of a sequence, so we want to determine a good approach for determining whether a limit exists without having to compute it directly.  To do this, we introduce some terminology focused on the relationships between the terms of a sequence.

\begin{definition}
  A sequence is called
  \begin{itemize}
    \item \dfn{increasing} if $a_n<a_{n+1}$ for all $n$,
    \item \dfn{nondecreasing} if $a_n\le a_{n+1}$ for all $n$,
    \item \dfn{decreasing} if $a_n>a_{n+1}$ for all $n$,
    \item \dfn{nonincreasing} if $a_n\ge a_{n+1}$ for all $n$.
  \end{itemize}
\end{definition}

Lots of facts are true for sequences which are either increasing or
decreasing; to talk about this situation without constantly saying
``either increasing or decreasing,'' we can introduce a single word to
cover both cases.
\begin{definition}
  If a sequence is always increasing, or  always nondecreasing, or always decreasing, or always nonincreasing, it is said to be \dfn{monotonic}\index{sequence!monotonic}.
\end{definition}


\begin{example}
If $a_n = 2n^2+1$, then $\{a_n\}_{n=1}^{\infty}$ is \wordChoice{\choice[correct]{increasing}\choice{decreasing}}, so it is
      \wordChoice{\choice[correct]{monotonic}\choice{not monotonic}}
\end{example}

\begin{question}
  If an arithmetic sequence $a_n = m\cdot n + b$ is monotonic, what
  must be true about $m$ and $b$?
  \begin{prompt}
    \begin{quote}
      The sign of $m$ \wordChoice{\choice{is positive}\choice{is
          negative}\choice[correct]{does not matter}}, and the sign of $b$ is
      \wordChoice{\choice{is positive}\choice{is negative}\choice[correct]{does
          not matter}}
    \end{quote}
  \end{prompt}
  \begin{feedback}
   We can model an arithmetic sequence $a_n = m\cdot n + b$ with the line $f(x) = mx+b$.  Can a line ever increase then decrease or vice-versa? 
  \end{feedback}
\end{question}

\begin{question}
  If a geometric sequence $a_n = a_1 \cdot r^{n-1}$ is monotonic, what
  must be true about $a_1$ and $r$?
  \begin{prompt}
    \begin{quote}
      The sign of $a_1$ \wordChoice{\choice{is positive}\choice{is
          negative}\choice[correct]{does not matter}}, and the sign of
      $r$ is \wordChoice{\choice[correct]{is positive}\choice{is
          negative}\choice{does not matter}}
    \end{quote}
  \end{prompt}
    \begin{feedback}
    From our examples earlier in the section, a geometric sequence $a_n = a_1 \cdot r^{n-1}$ can be modeled by an exponential function (which is always increasing or always decreasing) if the sign of $r$ is positive.  If $r$ is negative, the signs of each successive term is different from the last.
  \end{feedback}
\end{question}

%Let's see some examples:
%\begin{example}
%  Describe the growth of $a_n = \frac{2^n-1}{2^n}$ for
%  $n=1,2,3,\dots$.
%  \begin{explanation}
%    To do this, check out the difference between two sequential terms,
%    write with me:
%    \begin{align*}
%    a_{n+1} - a_n &= \answer[given]{\frac{2^{n+1}-1}{2^{n+1}}} - \frac{2^n-1}{2^n}\\
%    &=\frac{2^{n+1}-1 - 2\left(\answer[given]{2^n-1}\right)}{2^{n+1}}\\
%    &=\frac{2^{n+1}-1 - 2^{n+1}+2}{2^{n+1}}\\
%    &=\frac{\answer[given]{1}}{2^{n+1}}.
%    \end{align*}
%    Since $\frac{1}{2^{n+1}}$ is always
%    \wordChoice{\choice[correct]{positive}\choice{negative}\choice{zero}},
%    we see that $(a_n)$ is
%    \wordChoice{\choice[correct]{increasing}\choice{nondecreasing}\choice{decreasing}\choice{nonincreasing}}, and hence is \wordChoice{\choice[correct]{monotonic}\choice{not monotonic}}.
%  \end{explanation}
%\end{example}
%
%\begin{example}
%  Describe the growth of $a_n = \frac{n+1}{n}$ for $n=1,2,3,\dots$.
%  \begin{explanation}
%    To do this, check out the difference between two sequential terms.
%    Write with me.
%    \begin{align*}
%    a_{n+1} - a_n &= \answer[given]{\frac{(n+1)+1}{n+1}} - \frac{n+1}{n}\\
%    &=\frac{n(n+2) - (n+1)(\answer[given]{n+1})}{n+1}\\
%    &=\frac{n^2+2n - n^2 -2n-1}{n+1}\\
%    &=\frac{\answer[given]{-1}}{n+1}.
%    \end{align*}
%    Since $\frac{-1}{n+1}$ is always
%    \wordChoice{\choice{positive}\choice[correct]{negative}\choice{zero}} for the values of $n$ we are considering,
%    we see that $(a_n)$ is
%    \wordChoice{\choice{increasing}\choice{nondecreasing}\choice[correct]{decreasing}\choice{nonincreasing}}, and hence is \wordChoice{\choice[correct]{monotonic}\choice{not monotonic}}.
%  \end{explanation}
%\end{example}

Sometimes we want to classify sequences for which the terms do not get too big or too
small.  

\begin{definition}
  \label{definition:sequence-bounded}
  A sequence $\{a_n\}$ is:
  
  \begin{itemize}
  \item \dfn{bounded above} if there is some number $M$ so
  that for all $n$, we have $a_n\le M$.
    \item \dfn{bounded below} if there is some number $m$ so
  that for all $n$, we have $a_n\ge m$.
  \item \dfn{bounded} if it is both bounded above and bounded below.
  \end{itemize}
\end{definition}

So what does this definition actually say? Essentially, we say that a sequence is bounded above if its terms cannot become too large, bounded below if its terms cannot become too large and negative, and bounded if the terms cannot become too large and positive or too large and negative.

\begin{question}
  True or False: If a sequence $(a_n)_{n=0}^\infty$ is nondecreasing
  it is bounded below by $a_0$.
  \begin{prompt}
    \begin{multipleChoice}
    \choice[correct]{True}
    \choice{False}
    \end{multipleChoice}
  \end{prompt}
  \begin{feedback}
    If a sequence is nondecreasing, then its smallest value is its
    first element.
  \end{feedback}
\end{question}


\begin{question}
  True or False: If a sequence $(a_n)_{n=0}^\infty$ is nonincreasing
  it is bounded above by $a_0$.
  \begin{prompt}
    \begin{multipleChoice}
    \choice[correct]{True}
    \choice{False}
    \end{multipleChoice}
  \end{prompt}
  \begin{feedback}
    If a sequence is nonincreasing, then its largest value is its
    first element.
  \end{feedback}
\end{question}

So, what do these previous definitions have to do with the idea of a limit?  Essentially, there are three reasons that a sequence may diverge:

\begin{itemize}
\item the terms eventually are either always positive or always negative but become arbitrarily large in magnitude.
\item the terms are never eventually monotonic.
\item the terms are never eventually monotonic \emph{and} become arbitrarily large in magnitude.
\end{itemize}

Let's think about the terminology we introduced.

\begin{question}
Think about the following statements and choose the correct option.
\begin{itemize}
\item If we know that a sequence is monotonic and its limit does not exist, then  \wordChoice{\choice[correct]{the terms become too large in magnitude}\choice{the terms are never eventually monotonic}\choice{the terms  are never eventually monotonic and become arbitrarily large in magnitude}}.

\item If we know that a sequence is bounded and its limit does not exist, then  \wordChoice{\choice{the terms become arbitrarily large in magnitude.}\choice[correct]{the terms  are never eventually monotonic}\choice{the terms  are never eventually monotonic and become arbitrarily large in magnitude}}.

\end{itemize}
\end{question}

We can now state an important theorem:

\begin{theorem}[Bounded-monotone convergence theorem]
  If the sequence $a_n$ is bounded and monotonic, then $\lim_{n \to  \infty} a_n $ exists.
\end{theorem}

To think about the statement of the theorem, if we have a sequence that is bounded, the only way it could diverge is if the terms are never eventually monotonic.  However, if we know the sequence is also monotonic, this cannot happen!  Thus, the series cannot diverge, so it must have a limit.

%\begin{question}
%  Consider the sequence $a_{n}$.  Suppose you know that for all $n >
%  1$,
%  \[
%  -6 \le a_{n} \le 0
%  \]
%  and $a_{1} = 2$, and $a_{2} = -1$, and that the sequence is
%  nonincreasing.  Does the sequence converge?
%  \begin{hint}
%    Since the sequence is nonincreasing, the sequence is monotone.
%  \end{hint}
%  \begin{hint}
%    Since for all $n \ge 1$, we have $a_{n} \ge -6$, the sequence is
%    bounded below.
%  \end{hint}
%  \begin{hint}
%    So by the Monotone Convergence Theorem, the sequence converges to
%    some value; let us call it $L$.
%  \end{hint}
%  \begin{hint}
%    Now consider the direction in which the sequence is heading.
%  \end{hint}
%  \begin{hint}
%    Since the sequence is nonincreasing, for all $n \ge 2$, we have
%    $-6 \le a_{n} \le -1$.
%  \end{hint}
%  \begin{hint}
%    The limit $L$ must be in that interval as well.
%  \end{hint}
%  \begin{hint}
%    Therefore the sequence converges to a value $L$ so that $-6 \le L
%    \le -1$.
%  \end{hint}
%  \begin{prompt}
%  \begin{multipleChoice}
%    \choice[correct]{Yes, with limit between $-6$ and $-1$.}
%    \choice{No, the sequence does not converge.}
%    \choice{Yes, with limit between $-1$ and $0$.}
%  \end{multipleChoice}
%  \end{prompt}
%\end{question}

In short, bounded monotonic sequences always converge, though we can't
necessarily describe the number to which they converge.  Let's try
some examples.

\begin{example}
  Given the sequence $a_n=\frac{2^n-1}{2^n}$ for $n=1,2,3,\dots$,
  explain how you know that $\lim_{n\to\infty} a_n$ converges to a
  finite value without computing its limit.
  \begin{explanation}
    To start, note that $a_n=\frac{2^n-1}{2^n} = \frac{2^n}{2^n} - \frac{1}{2^n} = 1 - \frac{1}{2^n} $.  Thus, $\{a_n\}$ is
    \wordChoice{\choice[correct]{monotonic}\choice{not monotonic}}.
    Moreover, all of the elements $a_n = (2^n-1)/2^n$ are less than
    $2$  and greater than zero.   So, the sequence is bounded above by $2$ and below by $0$, so it is bounded.  By the bounded-monotone
    convergence theorem, $\lim_{n\to\infty} a_n$ must converge to a finite value.
  \end{explanation}
\end{example}

%\begin{example}
%  Given the sequence $a_n=\frac{n+1}{n}$ for $n=1,2,3,\dots$,
%  explain how you know that $\lim_{n\to\infty} a_n$ converges to a
%  finite value.
%  \begin{explanation}
%    To start, from previous work we know that $a_n$ is
%    \wordChoice{\choice[correct]{monotonic}\choice{not monotonic}}.
%    Moreover, all of the elements $a_n = (n+1)/n$ are less than $2$,
%    and greater than zero. Hence by the bounded-monotone convergence
%    theorem, $\lim_{n\to\infty} a_n$ converges to a finite value.
%  \end{explanation}
%\end{example}

\begin{remark}
We don't actually need to know that a sequence is always monotonic to apply
the bounded-monotone convergence theorem. It is enough to know that
the sequence is eventually monotonic.  More formally, this means that there is 
some integer $N$ for which the sequence $\{a_n\}_{n=N}$ 
is always increasing or always decreasing.
\end{remark}

%\begin{example}
%Show that the sequence $(a_n)$ given by $a_n = n^{1/n}$ converges.
%\begin{explanation}
%  We might first show that this sequence is decreasing, that is, we show
%  that for all $n$,
%  \[
%  n^{1/n} > (n+1)^{1/(n+1)}.
%  \]
%  But this isn't true!  Take a look
%  \begin{align*}
%    a_1 &= 1, \\
%    a_2 &= \sqrt{2} \approx 1.4142, \\
%    a_3 &= \sqrt[3]{3} \approx 1.4422, \\
%    a_4 &= \sqrt[4]{4} \approx 1.4142, \\
%    a_5 &= \sqrt[5]{5} \approx 1.3797, \\
%    a_6 &= \sqrt[6]{6} \approx 1.3480, \\
%    a_7 &= \sqrt[7]{7} \approx 1.3205, \\
%    a_8 &= \sqrt[8]{8} \approx 1.2968, \\
%    a_9 &= \sqrt[9]{9} \approx 1.2765.
%  \end{align*}
%  But it does seem that this sequence perhaps is decreasing after the
%  first few terms.  Can we justify this?
%
%  Yes!  Consider the real function $f(x)=x^{1/x}$ when $x\ge 1$.  We
%  compute the derivative, perhaps via \index{logarithmic
%    differentiation}logarithmic differentiation, to find
%  \[
%  f'(x)=\answer[given]{\frac{x^{1/x} (1-\ln(x))}{x^2}}.
%  \]
%  Note that when $x\ge 3$, the derivative $f'(x)$ is negative.  Since
%  the function $f$ is decreasing, we can conclude that the sequence is
%  decreasing---well, at least for $n \geq 3$.
%
%  Since all terms of the sequence are positive, the sequence is
%  decreasing and bounded $0\le a_n \le a_3$ when $n \ge 3$, and so the
%  sequence converges.
%\end{explanation}
%\end{example}

In the previous examples, we could write down a function $f(x)$ corresponding to each series and apply the theorem from earlier in the section.  However, this is not always possible.

\begin{example}
Suppose that $a_n = 2 + \sin(n)$, and let $s_n = \sum_{k=1}^{n} a_k$.  Determine whether the sequences $\{a_n\}$ and $\{s_n\}$ are bounded or monotonic, and explain whether either has a limit.

\begin{explanation}
The sequence $a_n$ is certainly not monotonic, but it  is bounded since $1 \geq a_n \geq 3$ for all $n$.  We also can see that $\lim_{n \to \infty} a_n$ does not exist since the terms oscillate.

For $s_n$, note that since $a_n \geq 1$ for all $n$, each term $a_n$ is positive.  Thus, $s_n$ is increasing and hence monotonic.  

However, since $a_n\geq1$ for all $n$, we have the following inequality.

\[s_n = \sum_{k=1}^n a_n \geq \sum_{k=1}^n 1 = n\]

Hence, $\{s_n\}$ is not bounded and $\lim_{n \to \infty} s_n$ does not exist.
\end{explanation}
\end{example}

\begin{remark}
We had a way to analyze $\{a_n\}$ in the above example because we had an explicit formula for $a_n$, but how would we find such a formula for $s_n$?  As it turned out, we didn't have to do so in order to determine that $\lim_{n \to \infty} s_n$ does not exist.  We will make arguments in the coming sections that allow us to determine whether limits of sequences exist without relying on having explicit formulas for the sequences.  
\end{remark}



\end{document}
